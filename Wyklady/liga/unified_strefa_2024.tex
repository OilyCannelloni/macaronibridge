\documentclass[12pt, a4paper]{article}
\usepackage{import}
\import{../../lib}{bridge.sty}

\title{\vspace{-2.5cm}Basic Strefa 2024}
\author{}
\date{}

\begin{document}
\maketitle

\section*{Co ustalamy z partnerem przed grą}
\begin{itemize}
    \item Podniesienia po otwarciach 1\hearts, 1\spades \vimp
    \item 2\spades i 2\nt po otwarciu 1\nt
    \item Co po 2/1 (tradycyjne | non-serious) 
    \item Czy i jaki lebensohl po wejściu na 1\nt
    \br
    \item Silne przyjęcia transferów po otwarciu 2\nt \imp
    \item Co po 2\clubs Acol
    \item Czy otwarcie 2\hearts może być dwukolorowe z \spades
    \item Czy pokazujemy króle razem z Q atu
    \item Czym wistujemy z AK i jakie dokładamy sygnały
    \item Czy i jaki lebensohl po \dbl na blok przeciwnika
    \br
    \item Czy na pewno gramy (1\clubs) 2\diams = stare \exq
    \item Co przeciwko Multi 
    \item Poziom 3 po otwarciu 1\nt
    \item Czy gramy pasem forsującym
    \item Jak gramy do \dbl wywoławczej partnera (1M \nf | \fonce\!)
    \item Czy gramy 2\clubs Drury do wejścia i \textbf{czy po odzywce 3 gracza też}
    \item Licytacja po \dbl wywoławczej przeciwnika na 1\major
    \item Skok w 2\nt do wejścia partnera (3 ręka licytuje?)
    \item Jak agresywnie blokujemy w danej pozycji i założeniach
    \br 
    \item Czy nowy kolor po wejściu 2\minor forsuje? \qq
    \item Czy nowy kolor na poziomie 2 po wejściu partnera 1x forsuje?
\end{itemize}

\pagebreak
\section*{1\clubs}
\sequence{{1\clubs}}
\begin{options}[2]
    \item[1\diams] dowolne 0-6 \orr 6+\diams 7-12 \orr 5+/4+\minor 7-12
    \item[1\hearts] 4+ \fonce
    \item[1\spades] 4+ \fonce  
    \item[1\nt] 7-10 \bal (może zawierać 5\diams) 
    \item[2\clubs] \gf (jeśli zawiera 4\major, zawiera też 5\clubs)
    \item[2\diams] \gf z dobrymi 5+\diams
    \item[2\hearts] 5+\spades 4+\hearts 6-9
    \item[2\spades] 11+ \bal, bez 4\major
    \item[2\nt] 11-12 \bal
    \item[3\clubs] 6+\clubs, 6-9   
\end{options}

\subsection*{Negat}

\sequence{{1\clubs}{1\diams}}
\begin{options}[1]
    \item[1\hearts] 3+\hearts słaba ręka
    \item[1\spades] (3)4+\spades słaba ręka
    \item[1\nt] 18-20 \bal
    \item[2\clubs] 6+\clubs słaba ręka
    \item[2\diams] rewers, \fonce (slowdown 2\nt \nf, 2M = 5+)
    \item[2\hearts] rewers, \fonce
    \item[2\spades] rewers, \fonce
\end{options}

\sequence{{1\clubs}{1\diams}{1\major}}
\begin{options}[2]
    \item[1\spades] \nf
    \item[1\nt] \nf
    \item[2\clubs] \nf
    \item[2\diams] 6+\diams, 0-9
    \item[2\major] góra negatu z fitem
    \item[2\twosuit{\spades}{\hearts}] 5+/4+ \minor 7-9 (2\nt ask)
    \item[2NT] 5+/4+ \minor 10-12 \imp
    \item[3\diams] 6+\diams, 10-12 
\end{options}

\subsection*{Rebid po 1/1}

\sequence{{1\clubs}{1\hearts}}
\begin{options}[1]
    \item[1\spades] 4+\spades \nf
    \item[1\nt] 12-14 \bal
    \item[2\clubs] 6+\clubs 12-15 lub =1345 
    \item[2\diams] rewers, \fonce
    \item[2\hearts] 4+\hearts, \nf
    \item[2\spades] jump shift, \gf
    \item[2\nt] \gf
    \item[3\clubs] 6+\clubs 15-18 
    \item[3\diams] \gf z długimi \clubs (bez nadzieji na grę w stary partnera)  \vimp
    \item[3\hearts] 4+\hearts, 15-17 
\end{options}

\sequence{{1\clubs}{1\spades}}
\begin{options}[1]
    \item[1\nt] 12-14 \bal
    \item[2\clubs] 6+\clubs 12-15 lub 1345\clubs 
    \item[2\diams] rewers, \fonce
    \item[2\hearts] rewers, \fonce
    \item[2\spades] 4+\spades, 12-14
    \item[2\nt] \gf
    \item[3\clubs] 6+\clubs 15-18 
    \item[3\diams] \gf z długimi \clubs (bez nadzieji na grę w stary partnera) \vimp
    \item[3\spades] 4+\spades, 15-17 
\end{options}

\pagebreak
\subsection*{Rebid 1\spades}
\sequence{{1\clubs}{1\hearts}{1\spades}}
\begin{options}[2]
    \item[\pass] OK
    \item[1\nt] śmietnik, w tym ręce z 3\spades dające szanse na końcówkę do 17PC
    \item[2\clubs] magister
    \item[2\diams] magister
    \item[2\hearts] \soff
    \item[2\nt] transfer na 3\clubs \vimp
    \item[3\clubs] 5\hearts 5\clubs \gf \imp
    \item[3\diams] 5\hearts 5\diams \gf \imp
\end{options}

\subsection*{Rebid 1NT}
\sequence{{1\clubs}{1\hearts}{1N}}
\begin{options}[2]
    \item[2\clubs] magister
    \item[2\diams] magister
    \item[2\hearts] \soff
    \item[2\nt] transfer na 3\clubs \vimp
    \item[3\clubs] 5\hearts 5\clubs \gf \imp
    \item[3\diams] 5\hearts 5\diams \gf \imp
\end{options}

\sequence{{1\clubs}{1\hearts}{1N}{2\clubs}{2\diams}}
\begin{options}[2]
    \item[2\hearts] 5\hearts \inv
    \item[2\nt] \bal \inv 
    \item[3\clubs] 5\hearts 5\clubs \inv
    \item[3\diams] 5\hearts 5\diams \inv
    \item[3\nt] (5\hearts 332) wybór końcówki \vimp
\end{options}

\pagebreak
\subsection*{Rebid 2\nt}

\sequence{{1\clubs}{1\hearts}{2\ntx}}
\begin{options}[2]
    \item[3\clubs] \textbf{Sztuczne pytanie} \imp
    \item[3\diams] 4+\diams 
    \item[3\hearts] 6+\hearts, ustala kolor
    \item[3\spades] 4\hearts, 4\spades 
\end{options}

\sequence{{1\clubs}{1\hearts}{2\ntx}{3\clubs}}
\begin{options}[1]
    \item[3\diams] 4+\clubs, brak fitu \hearts
    \item[3\hearts] 3-\clubs, 3\hearts
    \item[3\spades] 4+\clubs, 3\hearts 
    \item[3\nt] Nic z powyższych
    \item[4\clubs] 4\hearts  
\end{options}

\sequence{{1\clubs}{1\spades}{2\ntx}}
\begin{options}[2]
    \item[3\clubs] \textbf{Sztuczne pytanie} \imp
    \item[3\diams] 4+\diams 
    \item[3\hearts] 5\spades, 4\hearts 
    \item[3\spades] 6+\spades, ustala kolor
\end{options}

\sequence{{1\clubs}{1\spades}{2\ntx}{3\clubs}}
\begin{options}[1]
    \item[3\diams] 4+\clubs, brak fitu \spades
    \item[3\hearts] 4+\clubs, 4\spades
    \item[3\spades] 3-\clubs, 4\spades 
    \item[3\nt] Brak trefli i potencjalnego fitu \spades
    \item[4\clubs] 4\spades  
\end{options}





\pagebreak
\subsection*{Rebid 2\clubs po 1\clubs}

\sequence{{1\clubs}{1\hearts}{2\clubs}}
\begin{options}[2]
    \item[2\diams] sztuczny \gf, pytanie o skład \vimp
    \item[2\hearts] \soff
    \item[2\spades] \gf, pytanie o trzymania
    \item[2\nt] \inv 
    \item[3\clubs] \inv 
\end{options}

\sequence{{1\clubs}{1\spades}{2\clubs}}
\begin{options}[2]
    \item[2\diams] sztuczny \gf, pytanie o skład \vimp
    \item[2\hearts] 5\spades, 4\hearts 10-11 \inv
    \item[2\spades] \soff
    \item[2\nt] \inv
    \item[3\clubs] \inv 
\end{options}

\subsection*{Inne rebidy po 1\clubs}

\sequence{{1\clubs}{1\hearts}{3\clubs}}
\begin{options}[2]
    \item[3\diams] \textbf{Sztuczne pytanie}, zazwyczaj o 3 kiery
    \item[3\hearts] 6+\hearts  
\end{options}

\sequence{{1\clubs}{1\spades}{3\clubs}}
\begin{options}[2]
    \item[3\diams] \textbf{Sztuczne pytanie}, zazwyczaj o 3 piki
    \item[3\hearts] 5/5 \twosuit{\hearts}{\spades}  
\end{options}


\sequence{{1\clubs}{1\hearts}{3\diams}}
\begin{options}[2]
    \item[3\hearts] 6+\hearts
    \item[3\spades] transfer na 3\nt lub silne ustalenie \clubs cue-bidem 
\end{options}

\pagebreak

\section*{1\diams}
\sequence{{1\diams}}
\begin{options}[2]
    \item[1\hearts] 4+\hearts, 4+PC
    \item[1\spades] 4+\spades, 4+PC
    \item[1\nt] 7-10 PC
    \item[2\clubs] \gf, jeśli zawiera starszą czwórkę, zawiera też 5\clubs
    \item[2\diams] 10+PC, forsuje do 3\diams
    \item[2\hearts] 5+\spades, 4+\hearts 6-9PC
    \item[2\spades] 11+ \bal 
    \item[2\nt] 11-12
    \item[3\clubs] Inwit z \clubs
    \item[3\diams] 4+\diams, 6-9
    \item[3\hearts, 3\spades] Splinter          
\end{options}


\subsection*{Rebid po 1/1}
\sequence{{1\diams}{1\hearts}}
\begin{options}[1]
    \item[2\clubs] 4+\clubs, 12-17
    \item[2\diams] 6+\diams, 12-14
    \item[2\spades] 4+\spades, \gf
    \item[2\nt] 18-20 \bal \orr \gf na długich \diams
    \item[3\clubs] co najmniej 5\diams4\clubs \gf
    \item[3\diams] 15-17, 6+\diams    
\end{options}

\subsection*{Rebid 2\nt}
\sequence{{1\diams}{1\hearts}{2\ntx}}
\begin{options}[2]
    \item[3\clubs] \textbf{Sztuczne pytanie}, często automatyczne \vimp
    \item[3\diams] Ustalenie koloru
    \item[3\hearts] 6+\hearts (uwaga!)
\end{options}

\sequence{{1\diams}{1\hearts}{2\ntx}{3\clubs}}
\begin{options}[1]
    \item[3\diams] Nie mam jednak 18-20 \bal, mam nierówny \gf na \diams \vimp
    \item[3\hearts] 3\hearts (to ma priorytet)
\end{options}

Po 1\spades analogicznie.

\pagebreak

\subsection*{1\diams\ --- 2\diams Inverted}
\sequence{{1\diams}{2\diams}}
\begin{options}[1]
    \item[2\hearts] trzymam \hearts, ale brakuje mi innego
    \item[2\spades] trzymam \spades, ale nie \hearts
    \item[2\nt] trzymam wszystko
    \item[3\hearts] krótkość
    \item[3\spades] krótkość  
\end{options}
Ten, kto przekracza 3\diams, obiecuje przyjęcie inwitu.

\subsection*{Rebid 2\diams po otwarciu 1\diams}
\sequence{{1\diams}{1\hearts}{2\diams}}
\begin{options}
    \item[2\hearts] \soff
    \item[2\spades] trzeci kolor, \gf raczej w stronę sprawdzenia trzymań
    \item[2\nt] \gf, pytanie o skład
    \item[3\diams] inwit   
\end{options}

\sequence{{1\diams}{1\spades}{2\diams}}
\begin{options}
    \item[2\hearts] 5\spades, 4\hearts \fonce
    \item[2\spades] \soff
    \item[2\nt] \gf, pytanie o skład
    \item[3\clubs] trzeci kolor, \gf raczej w stronę sprawdzenia trzymań 
    \item[3\diams] inwit 
\end{options}

\subsection*{Inne rebidy}
\sequence{{1\diams}{1\hearts}{3\diams}}
\begin{options}[2]
    \item[3\hearts] 5+\hearts \gf \imp
\end{options}

\sequence{{1\diams}{1\spades}{3\diams}}
\begin{options}[2]
    \item[3\hearts] Sztuczne pytanie
    \item[3\spades] 6+\spades 
\end{options}

\pagebreak
\section*{1\hearts}

\sequence{{1\hearts}}
\begin{options}[2]
    \item[1\spades] 4+\spades \orr 5+\spades, 3+\hearts  \gf
    \item[1\nt] \nf śmietnik \orr 4-6PC, 3+\hearts
    \item[2\hearts] 7-10PC, 3+\hearts
    \item[2\spades] Mini-Splinter 9-11PC, 4+\hearts i krótkość \br
    \item[2\nt] Inwit raczej z 3-kartowym fitem  
    \item[3\clubs] Silny Bergen: 10-11PC, 4+\hearts, bez krótkości
    \item[3\diams] Słaby Bergen: 6-9PC, 4+\hearts (może mieć krótkość) \br
    \item[3\hearts] Syf 0-5 4+\hearts taktyczne
    \item[3\spades] Splinter 12-15PC
    \item[3\nt] Splinter \diams 12-15PC   
    \item[4\clubs] Splinter 12-15PC
    \item[4\diams] Słabej jakości 11-12 z fitem  
\end{options}

\section*{1\spades}
\sequence{{1\spades}}
\begin{options}[2]
    \item[1\nt] \nf śmietnik \orr 4-6PC, 3+\spades
    \item[2\spades] 7-10PC, 3+\spades
    \item[2\nt] Mini-Splinter 9-11PC, 4+\spades i krótkość \br
    \item[3\clubs] Silny Bergen: 10-11PC, 4+\hearts, bez krótkości
    \item[3\diams] Słaby Bergen: 6-9PC, 4+\hearts (może mieć krótkość)
    \item[3\hearts] Inwit raczej z 3-kartowym fitem \spades \br
    \item[3\spades] Syf 0-5 4+\spades taktyczne
    \item[3\nt] Splinter \hearts 12-15PC  
    \item[4\clubs] Splinter
    \item[4\diams] Splinter
    \item[4\hearts] Słabej jakości 11-12 z fitem    
\end{options}


\pagebreak
\section*{1\ntx}
\sequence{{1N}}
\begin{options}[2]
    \item[2\spades] transfer na 6+\clubs \orr \inv \bal
    \item[2\nt] transfer na 6+\diams
    \item[3\clubs] Puppet
    \item[3\diams] 5/5 \twosuit{\clubs}{\diams}
    \item[3\hearts] 3\spades1\hearts(54)    
    \item[3\spades] 1\spades3\hearts(54) 
    \item[4\clubs] 5/5 \twosuit{\hearts}{\spades} 
\end{options}

\subsection*{Stayman}
\sequence{{1N}{2\clubs}{2\diams}}
\begin{options}[2]
    \item[2\hearts] 4-5\hearts, 4\spades do gry
    \item[2\spades] 5\spades, 4\hearts do gry
    \item[3\hearts] Smolen: 5\spades, 4\hearts \gf
    \item[3\spades] Smolen: 5\hearts, 4\spades \gf 
\end{options}

\sequence{{1N}{2\clubs}{2\hearts}}
\begin{options}[2]
    \item[2\spades] Silne ustalenie \hearts \imp
    \item[3\spades+] Splinter 
\end{options}

\sequence{{1N}{2\clubs}{2\spades}}
\begin{options}[2]
    \item[3\hearts] Silne ustalenie \spades \imp
    \item[4\clubs+] Splinter 
\end{options}


\subsection*{Transfery}
\sequence{{1N}{2\diams}{2\hearts}}
\begin{options}[2]
    \item[2\spades] Inwit z 5\hearts4\spades
    \item[3\spades+] Splinter 
    \item[4\hearts] Inwit do szlemika z 6\xhearts322  
\end{options}

\sequence{{1N}{2\hearts}{2\spades}}
\begin{options}[2]
    \item[3\hearts] Inwit z 5\spades4\hearts
    \item[4\clubs+] Splinter 
    \item[4\spades] Inwit do szlemika z 6\xspades322  
\end{options}

\subsection*{Transfery na młode}

\sequence{{1N}{2\spades}}
\begin{options}[1]
    \item[2\nt] Odrzucenie inwitu
    \item[3\clubs] Przyjęcie inwitu  
\end{options}

\sequence{{1N}{2\spades}{2N}}
\begin{options}[2]
    \item[3\clubs] \soff
    \item[3\diams] krótkość
    \item[3\hearts] krótkość
    \item[3\spades] krótkość   
\end{options}


\sequence{{1N}{2N}}
\begin{options}[1]
    \item[3\clubs] Góra bilansowa \orr zachęta do 3\nt w oparciu o fit karo
    \item[3\diams] Syf 
\end{options}

\sequence{{1N}{2N}{3\clubs}}
\begin{options}[2]
    \item[3\diams] \soff
    \item[3\hearts] krótkość
    \item[3\spades] krótkość   
    \item[4\clubs] krótkość 
\end{options}


\pagebreak
\section*{2\clubs Acol}

\sequence{{2\clubs}}
\begin{options}[2]
    \item[2\diams] Co najmniej król lub 2 damy \gf
    \item[2\hearts] Gorsze
\end{options}

\sequence{{2\clubs}{2\diams}}
\begin{options}[1]
    \item[2\hearts] 5+\hearts unbalanced
    \item[2\spades] 5+\spades unbalanced
    \item[2\nt] 23+ \bal
    \item[3\clubs] 6+\clubs unbalanced
    \item[3\diams] 6+\diams unbalanced bez 4M
    \item[3\hearts+] samoustalenie
\end{options}

\sequence{{2\clubs}{2\diams}{2\hearts}}
\begin{options}[2]
    \item[2\spades] \textbf{Sztuczne pytanie}, raczej automat. Można ustalić odp trsf.
\end{options}

\sequence{{2\clubs}{2\diams}{2\spades}}
\begin{options}[2]
    \item[2\nt] \textbf{Sztuczne pytanie}, raczej automat
\end{options}

\sequence{{2\clubs}{2\diams}{3\clubs}}
\begin{options}[2]
    \item[3\diams] \textbf{Sztuczne pytanie}, odp. 3M \textbf{z krótkości} (z 6/4 1\clubs\then2\hearts)
    \item[3\hearts] 5+ niezłych \hearts
    \item[3\spades] 5+ niezłych \spades  
\end{options}

\sequence{{2\clubs}{2\diams}{3\diams}}
\begin{options}[2]
    \item[3\hearts] 5+ niezłych \hearts
    \item[3\spades] 5+ niezłych \spades  
\end{options}

\compsequence{{2\clubs}{\textbf{coś}}}
\begin{compoptions}[3]
    \item[\pass] dobra ręka ok 5+PC
    \item[\dbl] słaba ręka
    \item[Kolor] naturalnie, raczej 6+  
\end{compoptions}


\pagebreak

\section*{2\diams Multi}
\sequence{{2\diams}}
\begin{options}[2]
    \item[2\hearts] Do koloru, może zawierać dobrą kartę do pików
    \item[2\spades] Z pikami zostaw, z kierami \textbf{przyjmij/odrzuć inwit} \imp
    \item[2\nt] Pytanie o kolor i siłę, \invp \imp
    \item[3\hearts] Blok z fitami do obydwu kolorów
    \item[4\diams] Blok z fitami do obydwu kolorów \vimp
    \item[4\major] Do gry z własnego
\end{options}

\sequence{{2\diams}{2\ntx}}
\begin{options}[1] 
    \item[3\clubs] Minimum \vimp
    \item[3\diams] Maximum z \hearts
    \item[3\hearts] Maximum z \spades  
\end{options}

\sequence{{2\diams}{2\ntx}{3\clubs}}
\begin{options}[2]
    \item[3\diams] Pokaż kolor \textbf{odwrotnie}, \gf \vimp
    \item[3\hearts] Pasuj / popraw
    \item[3\spades] Pasuj z pikami, daj 4\hearts z kierami 
\end{options}

\sequence{{2\diams}{2\ntx}{3\diams}}
\begin{options}[2]
    \item[3\hearts] Próba szlemikowa, pytanie o krótkość (3\nt brak) \vimp
    \item[Kolor] Próba szlemikowa, cue-bid z krótkości
    \item[3\nt] Do gry 
    \item[4\hearts] Do gry 
\end{options}

\compsequence{{2\diams}{\dbl}}
\begin{compoptions}[3]
    \item[\pass] Chcę grać 2\diams\!\!\dbl
    \item[\rdbl] \textbf{Do koloru}
    \item[2\hearts] 6+ własnych dobrych \hearts, \nf
    \item[2\spades] 6+ własnych dobrych \spades, \nf
\end{compoptions}

\compsequence{{2\diams}{2\major}}
\begin{compoptions}[3]
    \item[\pass] Ręka niechętna do walki
    \item[\dbl] \textbf{Do koloru}
\end{compoptions}
    
\compsequence{{2\diams}{2\ntx}}
\begin{compoptions}[3]
    \item[\dbl] Karna
\end{compoptions}

\compsequence{{2\diams}{3\minor}}
\begin{compoptions}[3]
    \item[\dbl] Karna
\end{compoptions}


\pagebreak
\section*{Rewersy}
\begin{itemize}
    \item Minimum na rewers 15PC
    \item Niższa z 2\nt i 4 koloru = \textbf{Slowdown Bid}, zwalnia z \gf
    \item Ten kto przekracza Slowdown nie licytując go ustawia \gf
\end{itemize}

\sequence{{1\clubs}{1\hearts}{2\diams}}
\begin{options}[2]
    \item[2\hearts] 5+\hearts \fonce
    \item[2\spades] \textbf{Slowdown}
    \item[2\nt] Pytanie o skład \gf  
\end{options}

\sequence{{1\clubs}{1\hearts}{2\diams}{2\spades}}
\begin{options}[1]
    \item[2\nt] \nf
    \item[3\clubs] 6+\clubs \nf
\end{options}

\sequence{{1\clubs}{1\spades}{2\diams}{2\hearts}}
\begin{options}[1]
    \item[2\spades] 3\spades \nf
    \item[2\nt] 2-\spades \nf
    \item[3\clubs] 6+\clubs \nf 
\end{options}

\sequence{{1\diams}{1\spades}{2\hearts}}
\begin{options}[2]
    \item[2\spades] 5+\spades \fonce
    \item[2\nt] \textbf{Slowdown}
\end{options}

\sequence{{1\diams}{1\spades}{2\hearts}{2N}}
\begin{options}[1]
    \item[3\clubs] 3\clubs, \nf
    \item[3\diams] 2-\clubs, \nf  
\end{options}


\pagebreak
\section*{GF po ustaleniu fitu / Pytanie o dubla}
\begin{itemize}
    \item Sekwencje: 1M -- 2M, 1m -- 1M --- 2M
    \item +1 = sztuczne pytanie (2\spades na \hearts, 2\nt na \spades)
    \item Na \hearts 2\nt = inwit przez \spades, 
    \item Odpowiedzi: 
    \subitem Powrót na kolor: 4333
    \subitem Bez przeskoku: dubel licytowany (odpowiedź 2\nt na \hearts = dubel \spades)
    \subitem Z przeskokiem: krótkość licytowana
    \subitem 3\nt\!: Najwyższa krótkość
\end{itemize}

Przykład:
\sequence{{1\clubs}{1\hearts}{2\hearts}}
\begin{options}[2]
    \item[2\spades] Pytanie o dubla \gf
    \item[2\nt] Inwit przez piki
    \item[3\clubs] Inwit przez trefle
    \item[3\nt] Propozycja gry z 4333    
\end{options}

\sequence{{1\clubs}{1\hearts}{2\hearts}{2\spades}}
\begin{options}[1]
    \item[2\nt] Dubel \spades
    \item[3\clubs] Dubel \clubs
    \item[3\diams] Dubel \diams
    \item[3\hearts] 4333
    \item[3\spades] krótkość \spades
    \item[3\nt] krótkość \diams
    \item[4\clubs] krótkość \clubs (4\diams Last Train)
\end{options}

\end{document}