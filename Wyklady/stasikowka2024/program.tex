\documentclass[12pt, a4paper]{article}
\usepackage{import}
\import{../../lib}{bridge.sty}

\setlength\extrarowheight{8pt}

\title{\vspace{-2cm}Stasikówka 2024 - Grupa średnia}
\author{}
\date{}

\begin{document}
\maketitle
\section*{Program}
\begin{table}[h]
    \centering
    \begin{tabular}{rp{7.5cm}p{2.5cm}}
        Środa, $15^{00}$ &  \textbf{Odpowiedzialne blokowanie} \newline 
        \emph{Znaczenie założeń i pozycji. \newline
        Kiedy otwierać końcówką? \newline
        Powtórka licytacji po 2\ding{169} Multi.} & Ocena karty \newline Taktyka \\

        Czwartek, $11^{00}$ & \textbf{Wejścia dwukolorówką} \newline
        \emph{Jak, po co i kiedy pokazywać 10 kart jedną odzywką.} & System \newline Ocena karty \\

        Sobota, $11^{00}$ & \textbf{Schematy obkładania na wiście} \newline 
        \emph{Jak wyczuć, że partner chce przebitkę? \newline
        Jak wziąć na drugą damę atu? \newline 
        Skrót, czyli lewy atutowe znikąd} & Obrona \\

        Sobota, $15^{00}$ & \textbf{Schematy obkładania na wiście, cz.2} \newline
        \emph{Zabójczy wist w atuta. \newline Mini-turniej z motywami} & Obrona  \\

        Niedziela, $11^{00}$ & \textbf{Bridge master na kacu} \newline
        \emph{Macie dość, ja wiem. \newline
        Ale widzenie łatwych motywów przy dużym zmęczeniu to ważna umiejętność.} & Rozgrywka \\
    \end{tabular}
\end{table}

\section*{Materiały}
Wszystkie materiały znajdują się pod linkiem
\begin{center}
    tinyurl.com/stasikowka24bartek
\end{center}
Link będzie też na FB. Zachęcam wszystkie grupy do przeglądania!
\end{document}