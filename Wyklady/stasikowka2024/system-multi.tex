\documentclass[12pt, a4paper]{article}
\usepackage{import}
\import{../../lib}{bridge.sty}

\begin{document}
\sequence{{2\diams}}
\begin{options}[2]
    \item[2\hearts] Do koloru, może zawierać dobrą kartę do pików
    \item[2\spades] Z pikami zostaw, z kierami \textbf{przyjmij/odrzuć inwit} \imp
    \item[2\nt] Pytanie o kolor i siłę, \invp \imp
    \item[3\hearts] Blok z fitami do obydwu kolorów
    \item[4\diams] Blok z fitami do obydwu kolorów \vimp
    \item[4\major] Do gry z własnego
\end{options}

\sequence{{2\diams}{2\ntx}}
\begin{options}[1] 
    \item[3\clubs] Minimum \vimp
    \item[3\diams] Maximum z \hearts
    \item[3\hearts] Maximum z \spades  
\end{options}

\sequence{{2\diams}{2\ntx}{3\clubs}}
\begin{options}[2]
    \item[3\diams] Pokaż kolor \textbf{odwrotnie}, \gf \vimp
    \item[3\hearts] Pasuj / popraw
    \item[3\spades] Pasuj z pikami, daj 4\hearts z kierami 
\end{options}

\sequence{{2\diams}{2\ntx}{3\diams}}
\begin{options}[2]
    \item[3\hearts] Próba szlemikowa, pytanie o krótkość (3\nt brak) \vimp
    \item[Kolor] Próba szlemikowa, cue-bid z krótkości
    \item[3\nt] Do gry 
    \item[4\hearts] Do gry 
\end{options}

\compsequence{{2\diams}{\dbl}}
\begin{compoptions}[3]
    \item[\pass] Chcę grać 2\diams\!\!\dbl
    \item[\rdbl] \textbf{Do koloru}
    \item[2\hearts] 6+ własnych dobrych \hearts, \nf
    \item[2\spades] 6+ własnych dobrych \spades, \nf
\end{compoptions}

\compsequence{{2\diams}{2\major}}
\begin{compoptions}[3]
    \item[\pass] Ręka niechętna do walki
    \item[\dbl] \textbf{Do koloru}
\end{compoptions}
    
\compsequence{{2\diams}{2\ntx}}
\begin{compoptions}[3]
    \item[\dbl] Karna
\end{compoptions}

\compsequence{{2\diams}{3\minor}}
\begin{compoptions}[3]
    \item[\dbl] Karna
\end{compoptions}

\end{document}