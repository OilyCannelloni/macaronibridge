\documentclass[12pt, a4paper]{article}
\usepackage{import}

\import{../lib/}{bridge.sty}
\setmainlanguage{polish}

\author{}
\date{}
\title{Prosty Acol bez kontroli}

\begin{document}
\maketitle
\section{Granie bez kontroli}
\begin{itemize}
    \item System jest prosty i przejrzysty, nie wymaga szczegółowych ustaleń
    \item Miejsce na pokazanie składu, brak rozjazdów bez fitu na poziomie 5
    \item Silne ręce odpowiadającego i tak da się rozlicytować
\end{itemize}

\section{Otwieranie acola}
Ręka musi spełniać jeden z dwóch warunków:
\begin{itemize}
    \item 23+ PC
    \item 9 lew w ręce lub $\leq 4$ przegrywające
\end{itemize}

Musimy posiadać także około 3 \textbf{lewy w obronie}. Jeśli przeciwnicy wejdą, partner musi 
na jakiejś podstawie podjąć decyzję, czy kontrować. Np rękę
\begin{center}
    \hhand{AKQJ8753}{-}{QJ}{KQ5}
\end{center}
należy otworzyć 1\spades, a na 1\nt\ zalicytować np jakieś 4\hearts. Ryzyko że 1\spades\ zbiegnie jest minimalne,
bo jak partner nie ma nic, to przeciwnikom wychodzi 4\hearts.

\section{Odpowiedzi na acola}
\begin{itemize}
    \item 2\diams\ - dowolna sensowna ręka
    \item 2\hearts\ - supernegat, 0-3\hcp, z K dajemy 2\diams. \textbf{Zwalnia z \gf}!
    \item (smaczek dla lubiących ustalenia): 2\spades, 2\nt, 3\clubs, 3\diams\ = naturalne z piątki,
    co najmniej 4 honory (z AKQJT) w kolorze (lub coś co gra jak 4 honory). 2\nt\ = \hearts
\end{itemize}


\pagebreak
\section{2\clubs\ - 2\diams\ --- ?}
\begin{itemize}
    \item 2\hearts, 2\spades\ - naturalne, ręka raczej niezrównoważona. Chyba, że gramy
    \emph{Kokish Relay}, wtedy 2\hearts\ zawiera również 25+ BAL.
    \item 2\nt\ - 23-24 PC, może być ze 5\major\ i 4\minor, czasami też z singlem
    \item 3\clubs\ - (5)6+\clubs
    \item 3\diams\ - 6+\diams, raczej tylko długie kara, bo jest problem z pokazywaniem dwukolorówek. 
    Z rękami dwukolorowymi na karach - 1\diams.
    \item 3\hearts, 3\spades, 4\clubs, 4\diams\ - samoustalenia
    \item 3\nt\ - 25+ BAL, ale jest ryzyko, że się pogubimy! Zwykle \textbf{wystarczy dać 2\ntx}, 
    nawet z silną kartą, bo i tak mamy \gf!
\end{itemize}

\section{2\clubs\ - 2\hearts\ --- ?}
\begin{itemize}
    \item 2\spades\ = \fonce
    \item 2\nt, 3\clubs, 3\diams\ = \nf
    \item 3\hearts\ = \gf
\end{itemize}
Ze słabą ręką kierową można spasować na 2\hearts.

\section{Licytacja relayowa odpowiadającego}
\raggedright
Odpowiadający w swojej drugiej odzywce może zalicytować odzywkę o 1 wyższą, by zapytać się OTW o skład.
Przykładowo: \\[1em]
\webidding{
    2\clubs & 2\diams \\
    2\hearts & \conventional{2\spades}
}
\webidding{
    2\clubs & 2\diams \\
    2\spades & \conventional{2\ntx}
} \\[1em]
\raggedright
Te odzywki nie są naturalne i tylko pytają o skład.

\begin{formal}
    W sekwencji 2\clubs\ - 2\diams\ --- 3\diams\ relay \textbf{nie występuje}. 3\hearts\ i 3\spades\ są naturalne,
    z piątki. Dlatego należy uważać z acolem na karach i ręce z 6\diams\ i 4\major\ otwierać 1\diams.
\end{formal}

Kiedy nie dawać relay'a? Kolor 5+ z czterama z pięciu honorów mogliśmy pokazać w pierwszym kółku.
W drugim kółku - pokazujemy kolor naturalnie, jeśli ma \textbf{dokładnie 3 honory}. No i oczywiście
\diams QJTxx nie jest warte pokazywania - dajmy OTW się odlicytować!

\pagebreak
\section{Smaczki dla zaawansowanych}
\raggedright
Pojawia sie problem w sekwencjach typu: \\[1em]

\webidding{
    2\clubs\ & 2\diams \\
    2\spades\ & 2\nt \\
    3\spades\ & ?
} \\[1em] \raggedright

Jak tu silnie ustalić piki? Rozwiązanie - transfery po pytającym relay'u!

\subsection*{2\clubs\ --- 2\diams\ \\ 2\hearts\ --- 2\spades\ \\ ?}
\begin{itemize}
    \item 2\nt\ = 5\hearts, 4+\clubs
    \item 3\clubs\ = 5\hearts, 4+\diams
    \item 3\diams\ = 6+\hearts
    \item 3\hearts\ = 5\hearts, 4\spades
\end{itemize}

Takie transfery przyjmujemy \textbf{tylko z fitem!} Bez fitu licytujemy naturalnie. Np:

\webidding{
    2\clubs\ & 2\diams \\
    2\hearts\ & 2\spades \\
    \conventional{3\diams} & 3\hearts
} \\[1em] \raggedright

3\diams\ pokazało szóstego kiera, więc 3\hearts\ ustala kiery silnie! Bez fitu dajemy 3\nt.

\raggedright
Pozwala to nam też pokazywać acola na 5-5 w sensowny sposób: \\[1em]
\webidding{
    2\clubs\ & 2\diams \\
    2\spades\ & 2\nt \\
    \conventional{3\spades} & 3\nt\ \\
    4\clubs\
} \\[1em] \raggedright

W takich sekwencjach 4\spades\ i 4\nt\ powinny być do gry, a 4\diams\ - ustalać trefle silnie.

\end{document}