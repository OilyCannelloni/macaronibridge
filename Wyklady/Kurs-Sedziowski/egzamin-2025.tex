\documentclass[12pt, a4paper]{article}
\usepackage{import}
\usepackage{geometry}
\geometry{
	left=20mm,
	right=25mm,
	top=20mm,
	bottom=20mm
}

\import{../../lib/}{bridge.sty}
\setmainlanguage{polish}

\title{Kurs sędziowski AGH - Egzamin końcowy}
\date{11.06.2025}
\author{Opracowanie: Bartek Słupik}

\newtheorem{pyt}{}

\begin{document}
\maketitle

\begin{itemize}
	\item Czas przeznaczony na egzamin to 90 minut.
	\item Odpowiedzi zamieszczamy w Google Forms
	\item Można korzystać z dowolnej wersji Prawa Brydżowego oraz wszystkich materiałów z kursu.
	\item Bardzo proszę o niekonsultowanie odpowiedzi z innymi. Egzamin ma Wam pozwolić sprawdzić swoje umiejętności, które i tak zweryfikuje praktyka!
	\item Egzamin składa się z:
	\subitem 9 pytań prawda/fałsz (0.5 pkt za odpowiedź + 0.5 za przepis, jeśli odpowiedź jest poprawna)
	\subitem 7 pytań ABCD - dokładnie 1 jest poprawna (0.5 pkt za odpowiedź + 0.5 za przepis, jeśli odpowiedź jest poprawna)
	\subitem 4 pytań otwartych (2, 4, 4, 4 pkt)
	\item Do każdej odpowiedzi należy \textbf{podać przepis} z którego korzystamy!
	\item Przykład: E jest Dealerem, S pasuje poza kolejnością, pas nie zostaje zaakceptowany. E otwiera 1\clubs. S musi spasować. \\
			Odpowiedź: \textbf{P, 30A}
\end{itemize}

\pagebreak

\section*{Pytania Prawda/Fałsz}

\begin{pyt} 
\end{pyt}
Rozgrywający prosi o zagranie “ósemki kier” ze stołu, ale jej tam nie ma. Jedyna ósemka
na stole to \xclubs 8 - dziadek musi ją zagrać.

\begin{pyt} 
\end{pyt}
Obrońca zagrał kartę z ręki i położył na stole. Nie może jej zmienić, nawet jeżeli zrobi to
bez przerwy na namysł, a zamiar innego zagrania jest bezsporny.


\begin{pyt} 
\end{pyt}
Wist blotką w kolor \xdiams AQJ106 w stole, do \xdiams 74 w ręku. Rozgrywający dokłada ze stołu
\xdiams 10 i bierze lewę. W dalszym ciągu rozdania wraca do ręki i wychodzi w \diams. Po dołożeniu
przez LHO mówi „małe”. Obrońcy uważają, że zagraną kartą jest \xdiams 6. Sędzia wyda jednak
inny werdykt.


\begin{pyt} 
\end{pyt}
Wist \xdiams Q w kolor \xdiams Kxxx w stole, do \xdiams Axxx w ręku. Rozgrywający mówi: “wysokie”. RHO
dokłada do koloru, a rozgrywający kładzie na stół asa. W tym momencie dziadek, który
nie zdążył sięgnąć po króla, pyta rozgrywającego co dołożyć.
\begin{enumerate}[label=\alph*.]
\item Rozgrywający musi dołożyć \xdiams K ze stołu.
\item Rozgrywający musi zagrać \xdiams A z ręki.
\end{enumerate}

\begin{pyt} 
\end{pyt}
E jest Dealerem, S otwiera poza kolejnością 1\hearts. Otwarcie nie zostaje zaakceptowane.
\begin{enumerate}[label=\alph*.]
\item E otwiera 1\clubs. S musi wejść 1\hearts.
\item E pasuje. S musi otworzyć 1\hearts.
\end{enumerate}


\begin{pyt} 
\end{pyt}
W jest dealerem i otworzył 2\nt. N nie zauważywszy 2\nt daje 2\hearts (blok na kierach).
Zapowiedź nie została zaakceptowana. Po wezwaniu sędziego N zmienia odzywkę na
3\hearts (naturalne wejście do licytacji na poziomie 3). S musi pasować do końca licytacji.


\begin{pyt} 
\end{pyt}
Grany jest kontrakt 7\clubs\dbl. Obrona ma asa atu. W czwartej lewie obrońca zrobił fałszywy
renons, który zauważono dopiero w ostatniej lewie. Ostatecznie wyszło 7\clubs\dbl-1.
Niezależnie od wcześniejszego i dalszego przebiegu rozgrywki należy zapisać 7\clubs\dbl=.



\pagebreak
\section*{Pytania ABCD}

\begin{pyt} 
\end{pyt}
Zawodnik słyszy, że jego partner tłumaczy odzywkę niezgodnie z ustaleniami. Czy powinien sprostować tłumaczenie, jeśli będzie obrońcą?
\begin{enumerate}[label=\alph*.]
\item Nie ma takiego obowiązku
\item Tak, zaraz po usłyszeniu błędnego wyjaśnienia
\item Tak, po zakończeniu licytacji
\item Tak, po zakończeniu rozgrywki
\end{enumerate}


\begin{pyt} 
\end{pyt}
Rozgrywający wziął poprzednią lewę \xdiams A na stole i mówi „zagraj waleta”. \\ Na stole są: \xclubs J, \xdiams J i \xspades J.
\begin{enumerate}[label=\alph*.]
\item Rozgrywający musi zagrać \xdiams J.
\item Rozgrywający musi zagrać waleta, ale może sprecyzować, którego z nich.
\item Rozgrywający może zagrać dowolną kartę.
\end{enumerate}


\begin{pyt} 
\end{pyt}
Gracz N otworzył poza kolejnością 2\diams Multi. E nie
zaakceptował i otworzył 1\nt. W zalicytował 3\nt, co skończyło licytację. Co może zrobić rozgrywający przed wistem?
\begin{enumerate}[label=\alph*.]
\item Zakazać S wistu w pika lub kiera, ale w karo nie.
\item Nakazać lub zakazać S wistu w pika lub kiera, ale w karo nie.
\item Zakazać S wistu w karo.
\item Zakazać wistu w dowolny kolor.
\end{enumerate}


\begin{pyt} 
\end{pyt}
S rozgrywa 3\nt. E ma normalną kartę przygwożdżoną, \xdiams Q. W dochodzi do lewy i:\\
- „Nakazuję wyjście w karo.” \\
- „Ale ja nie mam kara.” \\
- „W takim razie nakazuję wyjście w kiera.” \\
\begin{enumerate}[label=\alph*.]
\item W wychodzi w dowolny kolor, \xdiams Q pozostaje przygwożdżona
\item W wychodzi w dowolny kolor, \xdiams Q zostaje podjęta
\item W musi wyjść w kiera, \xdiams Q pozostaje przygwożdżona
\item W musi wyjść w kiera, \xdiams Q zostaje podjęta
\end{enumerate}

\pagebreak
\begin{pyt} 
\end{pyt}
S rozgrywa 2 trefl. Popełnia fałszywy renons, co dostrzega tylko dziadek i:

\begin{enumerate}[label=\alph*.]
\item ma obowiązek niezwłocznie zwrócić na to uwagę, by nie dopuścić do uprawomocnienia
\item ma obowiązek zwrócić na to uwagę natychmiast po uprawomocnieniu
\item ma obowiązek zwrócić na to uwagę natychmiast po skończeniu rozdania
\item nie ma obowiązku komukolwiek o tym mówić 
\end{enumerate}


\begin{pyt} 
\end{pyt}
Rozgrywający zagrywa z ręki \xspades6. LHO posiada w ręce \xspades AK1072 i przed zagraniem karty
upuszcza niechcący \xspades 2. LHO do bieżącej lewy dokłada:
\begin{enumerate}[label=\alph*.]
\item Co chce.
\item \xspades 2
\item \xspades A, \xspades K, \xspades 10 lub \xspades 2
\item \xspades 2 lub \xspades 7
\end{enumerate}

\begin{pyt} 
\end{pyt}
N jest otwierającym. W licytuje 1\diams, a następnie N licytuje 1\hearts. N wyjaśnia sędziemu, że chciał
otworzyć 1\hearts i nie widział odzywki 1\diams u przeciwnika.
\begin{enumerate}[label=\alph*.]
\item Otwarcie poza kolejnością 1\diams zostało zaakceptowane.
\item N otworzył 1\hearts. Jeśli para NS będzie rozgrywała, odzywka 1\diams może spowodować ograniczenia wistowe
\item N otworzył 1\hearts. Jeśli para NS będzie rozgrywała, nie będzie automatycznych konsekwencji wynikających z odzywki 1\diams
\end{enumerate}

\section*{Zadania otwarte}
\textbf{Pamiętaj o podaniu wszystkich przepisów, z których korzystasz!}
\begin{pyt} (2 pkt)
\end{pyt}
W otworzył poza kolejnością 2\clubs jako blok na obu kolorach starszych. Nie zaakceptowano tej
odzywki, licytacja wróciła do S, właściwego otwierającego, i potoczyła się następująco:

\allbidding{
	W & N & E & S \\
	\textst{2\clubs} & - & - & 1\diams \\
	1\spades & 3\nt & \pass & \pass \\
	\pass
}

Opisz, jakie ograniczenia wistowe może nałożyć N na E.
\pagebreak


\begin{pyt} (4 pkt)
\end{pyt}

S rozgrywa 3\nt i wziął już 7 lew. W końcówce rozdania wychodzi ze stołu.

\handdiagramv{\vhand{QT6}{-}{4}{-}}{\vhand{J2}{-}{K7}{}}{\vhand{-}{-}{Q}{T87}}{\vhand{-}{T}{-}{J92}}

``Zagram karo ze stołu, wpuszczając E. Po wzięciu dwóch kar zagra mi w pika do QT, więc wezmę
w końcówce 2 lewy. Swoje.''

Przeciwnicy zgadzają się, zapisują wynik. Kilka minut później, podczas
rozgrywki kolejnego rozdania tej rundy W pyta: „A co jeśli mój partner puściłby karo?”. Panie sędzio!


\begin{pyt} (4 pkt) \end{pyt}
Zawodnik N otworzył poza kolejnością 1\spades. Sędzia! Odzywka
nie została zaakceptowana. \textbf{Opisz Twoją interwencję przy stole.} Licytację kontynuowano: 

\allbidding{
	\vul{W} & \nvul{N} & \vul{E} & \nvul{S} \\
	- & \textst{1\spades} & - & -\\
	1\hearts & 1\spades & 4\xhearts* & \pass \\
	\pass & \pass
}

* - konstruktywne, z bilansu.

W przegrał 4\hearts bez jednej. Wzywa sędziego, argumentując, że gdyby wykroczenie nie miało miejsca, S poszedłby w obronę 4\spades\dbl bez 1 lub 2 (50\%-50\% szans).
Jaki orzekamy werdykt?

\begin{pyt} (4 pkt) \end{pyt}
Licytacja przebiegła następująco:
\webidding{
	1\hearts & ...2\nt\alrt \\
	4\hearts
}

E wyraźnie wypadł z tempa przy 2\nt oznaczjącym inwit z fitem. NS wezwali sędziego twierdząc, że 4\hearts padło pod wpływem namysłu. Wynik: 4\xhearts=.

Ręka N: \hhand{JT5}{AQT92}{KQT7}{8}.

Jeśli zrobimy panel, spośród 5 osób: 2 dadzą 4\hearts, 2 waha się, ale raczej 4\hearts, 1 licytuje 3\hearts.





\end{document}