\documentclass[12pt, a4paper]{article}
\usepackage{../lib/bridgetex2}
\usepackage{polyglossia}
\usepackage{enumitem}
\setmainlanguage{polish}


\title{\vspace{-2cm}SAYCo jajco}
\author{}
\date{}

\begin{document}
\maketitle
\section{Założenia}
\begin{itemize}
    \item Otwarcia kolorem młodszym Walsh Style: otwieramy dłuższy młodszy, z 3-3 w młodych - 1\clubs, a z 4-4 i
    5-5 - 1\diams. Rebid otwierającego w kolor na poziomie 1 pokazuje rękę niezrównoważoną, zazwyczaj 5m-4M.
    \item Preferencja kolorów starszych w słabych rękach - pomijamy 1\diams, jeśli mamy starszą czwórkę i rękę maksymalnie \inv.
    \item Zachowana jest strefowa struktura podniesień 1\major\ - nie gramy Jacoby 2NT
    \item Bezrewersowe 2/1 z niepoważnym 3\nt
    \item Podniesienia z trójki w licytacji jednostronnej i dwustronnej
\end{itemize}

\pagebreak

\section{Otwarcie 1\minor}
\subsection{1\clubs\ --- ?}
\begin{itemize}
    \item 1\diams\ - ręka \gf\ z 5\diams\ lub ręka maksymalnie \inv\ z 5\diams\ \textbf{bez starszych czwórek}.
    \item 1\nt\ - 6-11 PC bez starszych czwórek.
    \item 2\clubs\ - 5+\clubs, 10+PC, bez starszych czwórek, \textbf{forsuje do 3\clubs} (Inverted Minors)
        Otwierający licytuje trzymania ekonomicznie,
        gracz, który przekroczy swoją odzywką 3\clubs\ forsuje do końcówki.
    \item 2\diams\ - Słabe Multi 3-5 PC
    \item 2\hearts\ - 5\spades, 4+\hearts, 4-8 PC
    \item 2\spades\ - transfer na \nt, ręka co najmniej \inv. Jest to jedyna droga do zainwitowania 3\nt
        z ręką zrównoważoną. Przyjęcie z bilansu.
    \item 2\nt\ - 13-15 lub 18+ \gf, ręka zrównoważona, może zawierać starsze czwórki
    \subitem 3\clubs\ - Stayman
    \subitem 3\diams\ - Rebid w kolor otwarcia (6+\minor)
    \subitem 3\major\ - splinter, co najmniej 5/4 w \minor
    \item 3\clubs\ - \mixed\ z 5+\clubs\
    \item 3\nt\ - 15-17, ręka zrównoważona
\end{itemize}

\subsection{1\diams\ --- ?}
\begin{itemize}
    \item 1\nt\ - 6-11 PC bez starszych czwórek.
    \item 2\clubs\ - \gf\ na \textbf{ściśle} 5+\clubs
    \item 2\diams\ - 4+\diams, 10+PC, \textbf{forsuje do 3\diams} (Inverted Minors)
    \item 2\hearts\ - 5\spades, 4+\hearts, 4-8 PC
    \item 2\spades\ - transfer na \nt, ręka co najmniej \inv.
    \item 2\nt\ - \gf\ BAL (jak wyżej)
    \item 3\clubs\ - \inv\ z 6+\clubs, taki bardzo treflowy
\end{itemize}

\pagebreak


\section{Otwarcie 1\major}
\subsection{Na 1 i 2 ręce}
\begin{itemize}
    \item 1\nt: nieforsujące, może zawierać 3-6 z fitem
    \item Bergeny:
    \subitem 3\clubs\ - silny: \hhand{Axx}{Jxxx}{Kx}{Qxxx}
    \subitem 3\diams\ - mixed: \hhand{Axx}{xxxx}{Kx}{Jxxx}
    \subitem 3M - blok: \hhand{Qxx}{xxxx}{x}{Jxxxx}
    \item Inwit z krótkością (2M+1): \hhand{Kxx}{xxxx}{x}{KJTxx}
    \subitem +1 pytanie, odp. niska -- średnia -- wyskoka
    \item Inwit siłowy (1\hearts\ --- 2\nt, 1\spades\ --- 3\hearts): \hhand{Axx}{Kxx}{Kx}{Jxxxx}
    \item Bilansowe podniesienie - między silnym Bergenem a \gf\ (4M-1): \\ \hhand{KQ}{xxxx}{KJx}{Qxxx}
\end{itemize}

\subsection{Na 3 i 4 ręce}
\begin{itemize}
    \item 1\nt: nieforsujące, może zawierać 3-6 z fitem
    \item 2\clubs\ - Drury. Używamy agresywnie w sile Mixed Raise+ \\
        \hhand{KTxx}{x}{KTxxx}{xxx} to takie minimum
    \subitem 2\diams\ - inwit (jakieś 13-16), również 1\spades\ - 2\hearts\ = NAT 
    \subitem 2\major\ - poniżej otwarcia (8-12)
    \subitem Wszystko powyżej = 17+ i ustawia \gf.
    \item 2\major\ - 4-7(8), śmieciowe
    \item 2M+1 - MAX z dokładnie singlem, +1 pytanie, odp. niski -- średni -- wyskoki
    \item Splintery jak na pierwszej, tyle, że 10-11 \textbf{z renonsu}
    \item 3\clubs, 3\diams, 1\spades\ - 3\hearts\ - Kolor+fit
\end{itemize}


\pagebreak


\section{Licytacja po rebidzie 1\ntx\ (12-14) otwierającego}
\begin{itemize}
    \item 2\clubs\ - dowolny \inv, \soff\ na \diams, lub propozycja gry w 3\nt\ z 5\major
    \subitem 2\clubs\ - 2\diams\ --- 3\nt\ - 5332 w licytowanym \major, propozycja kontraktu
    \item 2\diams\ - dowolny \gf\, który nie jest zawarty w innej odzywce.
    Priorytet pokazywania:
    \subitem - 4 w drugim starym
    \subitem - 3 w licytowanym przez partnera starym
    \subitem - młodsza piątka
    \subitem - nic z powyższych
    \item 2\nt\ - słaby transfer na 3\clubs
    \item 3\minor\ - 5\major, 5\minor\ \gf\
    \item 3\hearts\ (w nasz kolor) - ustalenie koloru licytowanego, próba szlemikowa
        (\textbf{nie \inv\ z 6} - to przez 2\clubs\ i skok na 3\major)
    \item 3\spades\ i wyżej - splintery
    \item 3\nt\ - do gry
\end{itemize}



\pagebreak
\section{Otwarcie 1\ntx}
\subsection{1\ntx\ --- ?}
\begin{itemize}
    \item 2\spades\ - transfer na \clubs\ lub inwit do 2\nt
    \subitem 2\nt\ - minimum bilansowe
    \subitem 3\clubs\ - maksimum bilansowe
    \item 2\nt\ - transfer na \diams\ (może być jakieś 5/5 \minor)
    \subitem 3\clubs\ - wolę trefle od kar
    \subitem 3\diams\ - wolę kara od trefli
    \item 3\clubs\ - Krótkość \clubs, 5+\diams
    \item 3\diams\ - Krótkość \diams, 5+\clubs
    \item 3\hearts, 3\spades\ - młode co najmniej 5/4, krótkość
    \item 4\clubs\ - 5/5 \major\ bez ambicji (4\diams\ nie istnieje)
\end{itemize}

\subsection{1\ntx\ --- 2\clubs \\ 2\diams\ --- ?}
\begin{itemize}
    \item 2\hearts, 2\spades\ - do pasa!
    \item 3\clubs\ - \textbf{Pytanie o skład na młodych}
    \subitem 3\diams\ - 4+/4+ \minor, 3\major\ uzgadnia odpowiednio młody
    \subitem 3\hearts, 3\spades\ = 5+\clubs/\diams
    \subitem 3\nt\ - 4333 (można dać z gównianym 4/4\minor)
    \item 3\diams\ - \textbf{Pytanie o starszą trójkę}
    \subitem 4\clubs\ - obie trójki, następnie 4\diams\hearts\ \textbf{transfer}
\end{itemize}

\subsection{Szczególne sekwencje}
\begin{itemize}
    \item 1\nt\ - 2\hearts\ --- 2\spades\ - 3\hearts\ = \inv\ na 5\spades 4\hearts
    \item Po transferze nie ma ustalenia forsującego do szlemika - używamy dużego transferu i poziomu 5!
\end{itemize}



\pagebreak
\section{Licytacja po rebidzie 2\ntx\ (18-19) otwierającego}
\subsection{Ogólna teoria}
\begin{itemize}
    \item 2\nt\ \textbf{nie forsuje!}
    \item W dowolnej sekwencji 1X - 1Y --- 2\nt, odzywki 3\clubs, 3\diams, 3\hearts, 3\spades\
    są \textbf{transferami} na kolor wyżej. Transfery pokazują \textbf{tyle kart w kolorze, ile trzeba}. (xd)
    \item Transfery sign-off'owe to 3\clubs\ (na \diams) i transfer na kolor odpowiadającego.
    Otwierający \textbf{musi} je przyjąć bezpośrednio, bo odpowiadający może chcieć uciec z NT na 3 w kolor
    (ale może mieć silniejszą rękę). Np:
    \begin{center}
        \webidding{ 
            1\diams\ & 1\spades \\
            2\nt\ & 3\clubs \\
            3\diams & (\pass?)
        } \qquad\qquad
        \webidding{
            1\clubs\ & 1\spades \\
            2\nt\ & 3\hearts \\
            3\spades & (\pass?)
        }
    \end{center}
    \item Transfery \gf\ to wszystkie inne. Przyjmujemy je bezpośrednio tylko z fitem. Np:
    \begin{center}
        \webidding{ 
            1\diams\ & 1\spades \\
            2\nt\ & 3\diams \\
            3\hearts (fit \hearts) & 7\hearts 
        } \qquad\qquad
        \webidding{
            1\clubs\ & 1\spades \\
            2\nt\ & 3\diams(4\hearts5\spades) \\
            3\spades (fit \spades) & 7\spades
        } \\[0.7cm]
        \webidding{
            1\clubs\ & 1\hearts \\
            2\nt\ & 3\hearts*\\
            3\nt\ & 6\nt
        } \qquad\qquad
        \webidding{
            1\diams\ & 1\hearts \\
            2\nt\ & 3\spades\\
            3\nt** & \pass
        }     
        \\[0.5cm]
        \raggedright
        * - pokazuje 4\hearts4\spades, bo piątego kiera pokażemy przez 3\diams. \\
        ** - Partnerze, mam w dupie twoje trefle.

    \end{center}
\end{itemize}

\subsection{Niemożliwe 4\diams}
Zalicytowanie 4\diams, nie wykonując uprzednio transferu na \diams, jest odzywką niemożliwą, i ustala
ten kolor, \textbf{którego ustalenie ma sens}. Stąd wniosek, że transfer na \diams\ trzeba wykonywać zawsze, 
jeśli się go ma.

\subsection{Dodatkowe ustalenia}
\begin{itemize}
    \item Skok przez odpowiadającego na 4\clubs\ jest pytaniem o asy w kolorze otwarcia,
    a skok na 4\diams\ jest pytaniem o asy w kolorze odpowiedzi
\end{itemize}

% **** %
% Acol %
% **** %
\pagebreak
\section{Licytacja po otwarciu 2\clubs}
\subsection{Założenia}
\begin{itemize}
    \item Odpowiedzi kontrolami 0-1 --- 2 --- 3+, po 2\major\ jest \gf
\end{itemize}

\subsection{2\clubs\ - 2\diams\ --- ?}
\begin{itemize}
    \item 2\hearts\ - relay do 2\spades, kiery lub 24-25 na równym
    \subitem 2\nt\ - 24-25 na równym
    \subitem 3\clubs\ - 5\hearts, 4+\diams
    \subitem 3\diams\ - 6+\hearts\
    \subitem 3\hearts\ - 5\hearts\ + 4\spades, 4\clubs\ ustala \hearts
    \subitem 3\spades\ - 5\hearts\ + 4\clubs, 4\diams\ ustala \hearts
    \subitem 3\nt\ - 6\hearts 332, dobra karta do 3\nt
    \item 3\diams\ - 6+\diams, bez starszych czwórek
    \item 3\hearts\ = 5+\diams, 4\hearts
    \item 3\spades\ = 5+\diams, 4\spades
\end{itemize}

\subsection{Po wejściu przeciwnika}
\begin{itemize}
    \item 2\clubs\ - \enemy{\dbl}: \rdbl\ = 0/1K, \pass\ = 2+K, 2\diams\hearts\spades\ = 2+K, kolor 5+ z 3H
    \item 2\clubs\ - \enemy{2\diams}: \dbl\ = 0/1K, \pass\ = 2+K, 2\hearts\spades\ = 2+K, kolor 5+ z 3H
    \item 2\clubs\ - \enemy{2\hearts+}: \dbl\ = słaba ręka, \pass\ = dobra ręka, 2\spades\ = 5+, 3H
    \item 2\clubs\ - \enemy{2X}\ - \pass\ - \enemy{\pass}: \dbl\ = T/O
    \item 2\clubs\ - \enemy{\pass}\ - 2\diams\ - \enemy{2X}: \dbl\ = T/O, \pass\ = może trap, ODP \dbl\ jest T/O 
    \item 2\clubs\ - \enemy{\pass}\ - 2\major\ - \enemy{2X}: \dbl\ = karna
\end{itemize}


% ************************* %
% Licytacja po naszym bloku %
% ************************* %
\pagebreak
\section{Licytacja po blokach 2\diams, 2\hearts, 2\spades}
\subsection{Styl bloków}
\begin{itemize}
    \item Na poziomie 2
    \begin{itemize}
        \item Na pierwszej i drugiej ręce \nvul{średnie}\ / \vul{dobre}
        \item Na trzeciej ręce \nvul{bardzo agresywne}\ / \vul{bardzo agresywne} 
    \end{itemize}
    
    \item 3\clubs\diams
    \begin{itemize}
        \item Na pierwszej i drugiej ręce \nvul{średnie}\ / \vul{solidne}
        \item Na trzeciej ręce \nvul{bardzo agresywne}\ / \vul{bardzo agresywne} 
    \end{itemize}

    \item 3\hearts\spades
    \begin{itemize}
        \item Na pierwszej i drugiej ręce \nvul{agresywne}\ / \vul{średnie}
        \item Na trzeciej ręce \nvul{bardzo agresywne}\ / \vul{bardzo agresywne} 
    \end{itemize}
\end{itemize}

\subsection{Dodatkowe ustalenia}
\begin{itemize}
    \item 2\nt\ = pytanie o wartość
    \item 2\diams\ - 3\hearts, 2\hearts\ - 3\spades, 2\spades\ - 4\clubs\ = Keycard.
    \item 2\diams\hearts\spades\ - 2\nt\ --- 3X - 4\clubs\ = Keycard.
    Odpowiedzi 0/1/1Q/2/2Q.
\end{itemize}




% ********************* %
% Podniesienia z trójki %
% ********************* %
\pagebreak
\section{Podniesienia z trójki}
W licytacji 1X - 1M możemy podnieść kolor partnera z trójki, jeśli mamy fajną kartę.
Streszczenie kontynuacji w 3 zdania:
\begin{enumerate}[label=\textbf{\arabic*.}]
    \item 2\nt\ i 3 w kolor otwarcia to zawsze \inv\ z czwórki, bez lub z fitem \minor\ 
    \item Najniższa odzywka niewymieniona w kroku \textbf{1} to pytanie o krótkość z piątki, \inv+
    \item Najniższa odzywka z tego, co zostało po \textbf{1 i 2}, to \gf\ z czwórki
\end{enumerate}

\subsection{1\clubs\ --- 1\hearts \\ 2\hearts\ --- ?}
\begin{itemize}
    \item 2\spades\ - pytanie o krótkość, 5+\spades, \inv+:
    \begin{itemize}
        \item 2\nt\ - brak krótkości
        \item 3\clubs\ - krótkość niska
        \item 3\diams\ - krótkość wysoka
    \end{itemize}
    \item 2\nt\ - \inv, pytanie o fit \spades, bez fitu w kol. otwarcia. Dalej nat.
    \item 3\clubs\ (kol. otwarcia) - \inv\ z 4\spades\ i fitem 4+ w kolorze otwarcia
    (zauważmy, że tu jest gwarancja jednego z fitów - bo nie podnosimy z trójki z 4333)
    \item 3\diams\ - \gf\ z 4\spades\ (z pięcioma i slam try - pyt. o krótkość!)
\end{itemize}

\subsection{1\clubs\ --- 1\spades \\ 2\spades\ --- ?}
\begin{itemize}
    \item 2\nt\ - pytanie o fit, \inv+
    \item 3\clubs\ - \inv\ z 4\spades\ i fitem w kolorze otwarcia
    \item 3\diams\ - pytanie o krótkość*, 5+\spades
    \item 3\hearts\ - \gf\ z 4\spades
\end{itemize}
* tutaj pokazanie krótkości wysokiej (\hearts) jest \gf\ NO I HUJ MORZE PUJDZIE

\begin{formal}
    Po odzywce "GF z czwórki" i sfitowaniu koloru 4-4 \textbf{nie gramy} non-serious - 3\nt\ jest propozycyjne.
\end{formal}

\subsection{Podnoszenie w licytacji dwustronnej \\
            1\clubs\ - (1\hearts) - 1\spades\ - (\passx) \\
            2\spades\ - (\passx) - ?}

\begin{itemize}
    \item 2\nt\ - sztuczne pytanie o długość (3\spades\ = 4, inne = 3)
    \item 3\clubs\ - inwit z 4+\clubs
    \item 3\diams\ - pytanie o krótkość
    \item 3\hearts\ - GF z czwórką
\end{itemize}


% ******* %
% Gazilli %
% ******* %
\pagebreak
\section{Gazilli}
W sekwencjach 1\hearts\ --- 1\spades, 1\hearts\ --- 1\nt\ oraz 1\spades\ --- 1\nt, rebid 2\clubs\
jest naturalny, lub z dowolną ręką 16+. Odpowiedź 2\diams\ na to 2\clubs\ jest sztuczna i pokazuje 8+ oraz ustawia \gf.

\subsection{1\major\ - 1X --- 2\clubs\ - ?}
\begin{itemize} 
    \item 2\diams\ - \textbf{dowolne 8+}
    \subitem 2\major\ - "Moje trefle były naturalne i mam jednak 11-15"
    \subitem Inne - 16+, naturalne
    \subitem Bezpośredni splinter jest silniejszy niż przez Gazilli
    \subitem Bezpośredni skok kolorem na 3x pokazuje 5/5, przez Gazilli 5/4
    \item 2\major\ - 5-7, negatywny powrót na kolor
    \item 2\spades\ (po 1\hearts) - 5/5\minor (bo jest niemożliwe), 
    generalnie wszystko co przechodzi 2\major\ pokazuje max singla w nim
    \item 2\nt\ - 5-7, ręka z singlem w kolorze otwarcia, która nie ma lepszej odzywki
    \item 3 w kolor - 5-7, naturalne z 6.
    \item \textbf{Wyjątek} - 1\hearts\ --- 1\spades\ / 2\clubs\ --- 3\spades\ = 10-11 \inv\ z 6+\spades
\end{itemize}

\subsection{1\hearts\ - 1\spades\ --- 2\clubs\ - 2\diams\ --- ?}
\begin{itemize}
    \item 2\spades\ - 5+\hearts, \textbf{3\spades}, 16+
    \item 3\hearts\ - 6+\hearts, bez 3 pików, 16+
    \item 3\spades\ - silne ustalenie \spades\ bez krótkości
\end{itemize}

\subsection{Inne ustalenia}
\begin{itemize}
    \item 1\major\ - 1X --- 2\nt\ = silne 6\major\ + 5\minor. \\ 3\clubs\ pytanie, 
    3\diams\ = \clubs, 3\hearts\ = \diams, 3\spades\ = 6\spades5\hearts\ (tylko po 1\spades)
\end{itemize}




% **************** %
% Kontra bilansowa %
% **************** %
\pagebreak
\section{Kontra bilansowa}
Kontra fit jest narzędziem wsadzonym do większości systemów przez wujotowych \emph{impostorów}. Wyrwijmy tę zarazę z korzeniami!
Szczególnie, że potrzebujemy narzędzia do odlicytowania kart w średnim przedziale siły.
\subsection{Rebid otwierającego}
1X - 1Y, przeciwnik z prawej wchodzi do licytacji.
Ogólna zasada:
\begin{itemize}
    \item \dbl\ lub \rdbl\ - (14)15+PC, oraz albo fit \textbf{3-kartowy}, albo brak \textbf{trzymania},
    albo potężna \textbf{układówka}. Innymi słowy karta,
    która chce sforsować, ale nie wie jeszcze, gdzie jej miejsce.

    \item \textbf{Podniesienie} koloru partnera - 12-14, z czwórki lub z trójki, jeśli z trójki to nie na syfie
    \item 2\nt\ - \inv\ z \textbf{6+ w kolorze otwarcia}
    \item 3 w kolor otwarcia - 6+, słabe
    \item 3 w inny kolor - słabe, dwukolorowe
    \item \textbf{Kolor przeciwnika} - \gf\ z \textbf{fitem} w kolorze partnera
\end{itemize}
Przykład:
\allbidding{
    1\diams & \enemy{\pass} & 1\spades & \enemy{2\hearts} \\
    ?
}
\begin{itemize}
    \item \dbl\ - 14(15+) łorewa
    \item 2\spades\ - 3-4\spades
    \item 2\nt\ - 6+\diams, \inv
    \item 3\clubs\ - 5\diams, 5\clubs
    \item 3\diams\ - 6+\diams, słabe
    \item 3\hearts\ - \gf\ z 4\spades
\end{itemize}

\subsection{Ustalenia związane z kontrą bilansową}
\begin{itemize}
    \item Jeśli kontra bilansowa sforsowała nas do 3\nt, forsuje tylko do 4\minor\, ale \textbf{ustawia pas forsujący}.
\end{itemize}



%*********************%
% Obrona na Michaelsa %
%*********************%
\pagebreak
\section{Obrona przeciwko Michaelsowi}

\subsection{Michaelsy określone}
Mamy dwa cuebidy i dwa określone grywalne kolory. Stosujemy zasadę \textbf{lower for lower},
czyli niższy cuebid odpowiada niższemu kolorowi. Kontra jest 10+ bez fitu i ustawia sytuację ukarniania.

\subsubsection{1\clubs\ --- (2\clubs) --- ?}
\begin{itemize}
    \item 2\diams\ - Naturalne, do pasa
    \item 2\hearts\ - co najmniej \inv\ na 5+\clubs
    \item 2\spades\ - \gf\ na 5+\diams
\end{itemize}

\subsubsection{1\hearts\ --- (2\ntx) --- ?}
\begin{itemize}
    \item 3\clubs\ - co najmniej \inv\ z fitem \hearts
    \item 3\diams\ - \gf\ z 5+\spades
    \item 3\hearts\ - 3+\hearts, 6-10 PC
    \item 3\spades\ - 6+\spades, słabe
\end{itemize}
Po otwarciu 1\spades, 3\clubs\ to \gf\ na \hearts, a 3\diams\ to \inv+ na \spades.
Cuebid z przeskokiem to splinter.

\subsection{Michaelsy nieokreślone}
Mając tyko jeden cuebid i nie znając grywalnych kolorów, trzeba odpalić lepsze narzędzia.
Kontra nadal karna i ustawia sytuację ukarniania.
Cuebid z przeskokiem to splinter.
Inne kolory z przeskokiem to kolor+fit.
\subsubsection{1\hearts\ --- (2\hearts) --- ?}
\begin{itemize}
    \item 2\spades\ - \inv+ z fitem \hearts
    \item 2\nt\ - transfer na 3\clubs
    \item 3\clubs\ - transfer na 3\diams
    \item 3\diams\ - transfer na nasz kolor - \mixed
    \item 3\hearts\ - 6-10 z fitem \hearts
\end{itemize}
\subsubsection{1\spades\ --- (2\spades) --- ?}
\begin{itemize}
    \item 2\nt\ - transfer na 3\clubs
    \item 3\clubs\ - transfer na 3\diams
    \item 3\diams\ - transfer na nasz kolor przez cuebid - \mixed
    \item 3\hearts\ - \inv+ z fitem \spades
    \item 3\spades\ - 6-10 z fitem \spades
\end{itemize}

\pagebreak


\section{Licytacja po wejściu}

\subsection{Styl wejść}
\begin{table}[h!]
    \centering
    \begin{tabular}{ccc}
        \textbf{Kolor} & \nvul{Zielone} & \vul{Czerwone} \\
        1\diams\hearts\spades & \textls{AQTxx} & Bardziej od 9PC \\
        2\minor & \textbf{6-kart} i 12PC & \textbf{6-kart} i 12PC \\
        2\major & 5-kart i 12PC & 5-kart i ładne 12PC \\
    \end{tabular}
    \caption{Minima wejść}
\end{table}

\subsection{Po wejściu kolorem na poziomie 1}
\begin{itemize}
    \item 1\major\ - \nf\ z czwórki
    \item 1\nt\ - 8-12, konstruktywne
    \item 2\nt\ - 13-15
    \item 2\clubs\ po naszym pasie i wejściu 1\major\ - \textbf{Drury}
    \item \textbf{Cue} - ???
    \item \textbf{Cue z przeskokiem} - \mixed
    \item Nowy kolor z przeskokiem = \gf, po pasie = K+F
\end{itemize}

\subsection{Po wejściu kolorem na poziomie 2}
\begin{itemize}
    \item 2\major\ - \fonce\ z piątki
    \item 2\nt\ - 10-12
\end{itemize}


\subsection{Drugi przeciwnik licytuje \dbl\ negatywną na wejście 1\major}
\begin{itemize}
    \item \rdbl\ - Dokładnie dubel w kolorze wejścia i (9)10+
    \item 1X - NAT z czwórki
    \item Między 1\nt\ a 2W - transfery:
    \subitem Transfer na cuebid = \inv\ z fitem
    \subitem Transfer na nasz kolor = słabe z fitem
    \subitem 2\major\ = konstruktywne z fitem
    \item 2\nt - co najmniej inwit z 4-kartowym fitem. To obowiązuje też po zalicytowaniu koloru 
    przez drugiego przeciwnika
\end{itemize}


\pagebreak

\section{Licytacja po bloku przeciwnika}

\pagebreak


\section{Licytacja po wejściu 1\ntx}

\pagebreak


\section{Licytacja przeciwko 1\ntx\ przeciwnika}

\pagebreak


% ***************************** %
% Kontry karne i pasy forsujące %
% ***************************** %
\section{Sekwencje karne}
\subsection{Ustalone sekwencje kontry karnej}
\begin{table}[h!]
    \centering
    \begin{tabular}{cccc}
        1\nt & \enemy{\pass} & 2\diams/2\hearts & \enemy{3\clubs} \\
        \dbl
    \end{tabular}
    \caption*{
        Kontra wywoławcza oznaczałaby brak fitu (bo czemu nie 3\spades?), 
        a co za tym idzie 3-4 trefle, partner nie może mieć kierów skoro nie dał 2\clubs. 
    }
\end{table}

\begin{table}[h!]
    \centering
    \begin{tabular}{cccc}
        2\clubs & \enemy{\pass} & 2\hearts/2\spades & \enemy{2X} \\
        \dbl
    \end{tabular}
    \caption*{
        Trzeba przeciwnika trzymać za pysk żeby nie wchodził tak 
    }
\end{table}

\begin{table}[h!]
    \centering
    \begin{tabular}{cccc}
        \enemy{X} & Y & \enemy{\dbl} & \pass \\
        \enemy{Z} & \pass & \enemy{\pass} & \dbl
    \end{tabular}
    \caption*{
        To nie jest wznówka, bo z dublem pik damy \rdbl\ w pierwszym kółku.
    }
\end{table}

\subsection{Ustalone sekwencje pasa forsującego}
\begin{itemize}
    \item Jeśli mamy \gf\ oraz fit, a przeciwnik wchodzi, \pass\ pokazuje krótkość lub wyłączenie w kolorze wejścia.
    Powrót na kolor jest odzywką podstawową.
    \item Jeśli pójdzie blok na trzeciej ręce, nasze wejście, oraz ten blok zostanie podniesiony powyżej
    naszej końcówki, \pass\ jest forsujący
\end{itemize}

\subsection{Inne pecjalne kontry}
\begin{itemize}
    \item Na Splintera:
    \subitem w \nvul{korzystnych}\ - propozycja opłacalnej obrony
    \subitem w \textbf{równych} lub \vul{niekorzystnych}\ - wskazuje wist w \textbf{starszy} z pozostałych kolorów
    \item Na NT lub szlemika, jeśli dziadek licytował kolor - prosi o wist w kolor dziadka
    \item Na NT lub szlemika, jeśli nie licytowano kolorów, a kontrakt wydaje się nadwyżkowy - ...
      
\end{itemize}

\end{document} 