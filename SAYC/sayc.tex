\documentclass[12pt, a4paper]{article}
\usepackage{../lib/bridgetex2}
\usepackage{polyglossia}
\usepackage{enumitem}
\setmainlanguage{polish}


\title{\vspace{-2cm}SAYCo jajco}
\author{}
\date{}

\begin{document}
\maketitle
\section{Założenia}
\begin{itemize}
    \item Otwarcia kolorem młodszym Walsh Style: otwieramy dłuższy młodszy, z 3-3 w młodych - 1\clubs, a z 4-4 i
    5-5 - 1\diams. Rebid otwierającego w kolor na poziomie 1 pokazuje rękę niezrównoważoną, zazwyczaj 5m-4M.
    \item Preferencja kolorów starszych w słabych rękach - pomijamy 1\diams, jeśli mamy starszą czwórkę i rękę maksymalnie \inv.
    \item Zachowana jest strefowa struktura podniesień 1\major\ - nie gramy Jacoby 2NT
    \item Bezrewersowe 2/1 z niepoważnym 3\nt
\end{itemize}

\pagebreak

\section{Otwarcie 1\minor}
\subsection{1\clubs\ --- ?}
\begin{itemize}
    \item 1\diams\ - ręka \gf\ z 5\diams\ lub ręka maksymalnie \inv\ z 5\diams\ \textbf{bez starszych czwórek}.
    \item 1\nt\ - 6-11 PC bez starszych czwórek.
    \item 2\clubs\ - 4+\clubs, 10+PC, bez starszych czwórek, \textbf{forsuje do 3\clubs} (Inverted Minors)
        Otwierający licytuje trzymania ekonomicznie,
        gracz, który przekroczy swoją odzywką 3\clubs\ forsuje do końcówki.
    \item 2\diams\ - ??? może \diams+\major?
    \item 2\hearts\ - 5\spades, 4+\hearts, 4-8 PC
    \item 2\spades\ - transfer na \nt, ręka co najmniej \inv. Jest to jedyna droga do zainwitowania 3\nt
        z ręką zrównoważoną. Przyjęcie z bilansu.
    \item 2\nt\ - 13-15 lub 18+ \gf, ręka zrównoważona, może zawierać starsze czwórki
    \subitem 3\clubs\ - Stayman
    \subitem 3\diams\ - Rebid w kolor otwarcia (6+\minor)
    \subitem 3\major\ - splinter, co najmniej 5/4 w \minor
    \item 3\clubs\ - \mixed\ z 5+\clubs\
    \item 3\nt\ - 15-17, ręka zrównoważona
\end{itemize}

\subsection{1\diams\ --- ?}
\begin{itemize}
    \item 1\nt\ - 6-11 PC bez starszych czwórek.
    \item 2\clubs\ - \gf\ na \textbf{ściśle} 5+\clubs
    \item 2\diams\ - 4+\diams, 10+PC, \textbf{forsuje do 3\diams} (Inverted Minors)
    \item 2\hearts\ - 5\spades, 4+\hearts, 4-8 PC
    \item 2\spades\ - transfer na \nt, ręka co najmniej \inv.
    \item 2\nt\ - \gf\ BAL (jak wyżej)
    \item 3\clubs\ - \inv\ z 6+\clubs, taki bardzo treflowy
\end{itemize}

\pagebreak


\section{Otwarcie 1\major}
\subsection{1\hearts\ --- ?}
\begin{itemize}
    \item 1\nt\ - semi-forsujące, może zawierać 3-6 z fitem lub zwykły inwit z fitem
    \item 2\hearts\ - 7-10, konstruktywne
    \item 2\nt\ - \textbf{Jacoby}, 4+\hearts, \gf. Wyklucza range silnych splinterów.
    \item 3\clubs\ - 10-11
    \item 3\diams\ - 7-9
    \item 3\hearts\ - 0-6
\end{itemize}

\section{Licytacja po rebidzie 1\ntx\ (12-14) otwierającego}
\begin{itemize}
    \item 2\clubs\ - dowolny \inv, \soff\ na \diams, lub propozycja gry w 3\nt\ z 5\major
    \subitem 2\clubs\ - 2\diams\ --- 3\nt\ - 5332 w licytowanym \major, propozycja kontraktu
    \item 2\diams\ - dowolny \gf\, który nie jest zawarty w innej odzywce.
    Priorytet pokazywania:
    \subitem - 4 w drugim starym
    \subitem - 3 w licytowanym przez partnera starym
    \subitem - młodsza piątka
    \subitem - nic z powyższych
    \item 2\nt\ - słaby transfer na 3\clubs
    \item 3\minor\ - 5\major, 5\minor\ \gf\
    \item 3\hearts\ (w nasz kolor) - ustalenie koloru licytowanego, próba szlemikowa
        (\textbf{nie \inv\ z 6} - to przez 2\clubs\ i skok na 3\major)
    \item 3\spades\ i wyżej - splintery
    \item 3\nt\ - do gry
\end{itemize}



\pagebreak
\section{Otwarcie 1\ntx}
\subsection{1\ntx\ --- ?}
\begin{itemize}
    \item 2\spades\ - transfer na \clubs\ lub inwit do 2\nt
    \subitem 2\nt\ - minimum bilansowe
    \subitem 3\clubs\ - maksimum bilansowe
    \item 2\nt\ - transfer na \diams\
    \subitem 3\clubs\ - nie lubię kar i minimum
    \subitem 3\diams\ - lubię kara lub maximum
    \item 3\clubs\ - Puppet
    \item 3\diams\ - 22(5/4)
    \item 3\hearts, 3\spades\ - młode co najmniej 5/4, krótkość
    \item 4\clubs\ - 5/5 \major\
\end{itemize}



\pagebreak
\section{Licytacja po rebidzie 2\ntx\ (18-19) otwierającego}
\subsection{Ogólna teoria}
\begin{itemize}
    \item 2\nt\ \textbf{nie forsuje!}
    \item W dowolnej sekwencji 1X - 1Y --- 2\nt, odzywki 3\clubs, 3\diams, 3\hearts, 3\spades\
    są \textbf{transferami} na kolor wyżej. Transfery pokazują \textbf{tyle kart w kolorze, ile trzeba}. (xd)
    \item Transfery sign-off'owe to 3\clubs\ (na \diams) i transfer na kolor odpowiadającego.
    Otwierający \textbf{musi} je przyjąć bezpośrednio, bo odpowiadający może chcieć uciec z NT na 3 w kolor
    (ale może mieć silniejszą rękę). Np:
    \begin{center}
        \webidding{ 
            1\diams\ & 1\spades \\
            2\nt\ & 3\clubs \\
            3\diams & (\pass?)
        } \qquad\qquad
        \webidding{
            1\clubs\ & 1\spades \\
            2\nt\ & 3\hearts \\
            3\spades & (\pass?)
        }
    \end{center}
    \item Transfery \gf\ to wszystkie inne. Przyjmujemy je bezpośrednio tylko z fitem. Np:
    \begin{center}
        \webidding{ 
            1\diams\ & 1\spades \\
            2\nt\ & 3\diams \\
            3\hearts (fit \hearts) & 7\hearts 
        } \qquad\qquad
        \webidding{
            1\clubs\ & 1\spades \\
            2\nt\ & 3\diams(4\hearts5\spades) \\
            3\spades (fit \spades) & 7\spades
        } \\[0.7cm]
        \webidding{
            1\clubs\ & 1\hearts \\
            2\nt\ & 3\hearts*\\
            3\nt\ & 6\nt
        } \qquad\qquad
        \webidding{
            1\diams\ & 1\hearts \\
            2\nt\ & 3\spades\\
            3\nt** & \pass
        }     
        \\[0.5cm]
        \raggedright
        * - pokazuje 4\hearts4\spades, bo piątego kiera pokażemy przez 3\diams. \\
        ** - Partnerze, mam w dupie twoje trefle.

    \end{center}
\end{itemize}

\subsection{Niemożliwe 4\diams}
Zalicytowanie 4\diams, nie wykonując uprzednio transferu na \diams, jest odzywką niemożliwą, i ustala
ten kolor, \textbf{którego ustalenie ma sens}. Stąd wniosek, że transfer na \diams trzeba wykonywać zawsze, 
jeśli się go ma.

\subsection{Dodatkowe ustalenia}
\begin{itemize}
    \item Skok przez odpowiadającego na 4\clubs\ jest pytaniem o asy w kolorze otwarcia,
    a skok na 4\diams\ jest pytaniem o asy w kolorze odpowiedzi
\end{itemize}

% **** %
% Acol %
% **** %
\pagebreak
\section{Licytacja po otwarciu 2\clubs}
TODO


% ************************* %
% Licytacja po naszym bloku %
% ************************* %
\pagebreak
\section{Licytacja po blokach 2\diams, 2\hearts, 2\spades}
\subsection{Styl bloków}
\begin{itemize}
    \item Na poziomie 2
    \begin{itemize}
        \item Na pierwszej i drugiej ręce \nvul{średnie}\ / \vul{dobre}
        \item Na trzeciej ręce \nvul{bardzo agresywne}\ / \vul{bardzo agresywne} 
    \end{itemize}
    
    \item 3\clubs
    \begin{itemize}
        \item Na pierwszej i drugiej ręce \nvul{średnie}\ / \vul{solidne}
        \item Na trzeciej ręce \nvul{bardzo agresywne}\ / \vul{bardzo agresywne} 
    \end{itemize}

    \item 3\diams\hearts\spades
    \begin{itemize}
        \item Na pierwszej i drugiej ręce \nvul{agresywne}\ / \vul{średnie}
        \item Na trzeciej ręce \nvul{bardzo agresywne}\ / \vul{bardzo agresywne} 
    \end{itemize}
\end{itemize}

\subsection{2\diams\ --- 2\ntx \\ ?}
\inv+, z fitem lub bez.
\begin{itemize}
    \item 3\clubs\ - minimum, słaby kolor
    \item 3\diams\ - maximum, słaby kolor
    \item 3\hearts\ - minimum, dobry kolor
    \item 3\spades\ - maximum, dobry kolor
\end{itemize}

\subsection{Definicje range'u i jakości}
\begin{itemize}
    \item Dobry kolor to co najmniej AQT na pierwszej i drugiej oraz co najmniej KJT na trzeciej
    \item Maksimum to 8+PC lub wyrobialny kolor + dojście bokiem, np \\ \hhand{Kxx}{x}{KJT9xx}{xxx}
\end{itemize}


% ********************* %
% Podniesienia z trójki %
% ********************* %
\pagebreak
\section{Podniesienia z trójki}
W licytacji 1X - 1M możemy podnieść kolor partnera z trójki, jeśli mamy fajną kartę.
Streszczenie kontynuacji w 3 zdania:
\begin{enumerate}[label=\textbf{\arabic*.}]
    \item 2\nt\ i 3 w kolor otwarcia to zawsze \inv\ z czwórki, bez lub z fitem \minor\ 
    \item Najniższa odzywka niewymieniona w kroku \textbf{1} to pytanie o krótkość z piątki, \inv+
    \item Najniższa odzywka z tego, co zostało po \textbf{1 i 2}, to \gf\ z czwórki
\end{enumerate}

\subsection{1\clubs\ --- 1\hearts \\ 2\hearts\ --- ?}
\begin{itemize}
    \item 2\spades\ - pytanie o krótkość, 5+\spades, \inv+:
    \begin{itemize}
        \item 2\nt\ - brak krótkości
        \item 3\clubs\ - krótkość niska
        \item 3\diams\ - krótkość wysoka
    \end{itemize}
    \item 2\nt\ - \inv, pytanie o fit \spades, bez fitu w kol. otwarcia. Dalej nat.
    \item 3\clubs\ (kol. otwarcia) - \inv\ z 4\spades\ i fitem 4+ w kolorze otwarcia
    (zauważmy, że tu jest gwarancja jednego z fitów - bo nie podnosimy z trójki z 4333)
    \item 3\diams\ - \gf\ z 4\spades\ (z pięcioma i slam try - pyt. o krótkość!)
\end{itemize}

\subsection{1\clubs\ --- 1\spades \\ 2\spades\ --- ?}
\begin{itemize}
    \item 2\nt\ - pytanie o fit, \inv+
    \item 3\clubs\ - \inv\ z 4\spades\ i fitem w kolorze otwarcia
    \item 3\diams\ - pytanie o krótkość*, 5+\spades
    \item 3\hearts\ - \gf\ z 4\spades
\end{itemize}
* tutaj pokazanie krótkości wysokiej (\hearts) jest \gf\ NO I HUJ



% **************** %
% Kontra bilansowa %
% **************** %
\pagebreak
\section{Kontra bilansowa}
Kontra fit jest narzędziem wsadzonym do większości systemów przez wujotowych \emph{impostorów}. Wyrwijmy tę zarazę z korzeniami!
Szczególnie, że potrzebujemy narzędzia do odlicytowania kart w średnim przedziale siły.
\subsection{Rebid otwierającego}
1X - 1Y, przeciwnik z prawej wchodzi do licytacji.
Ogólna zasada:
\begin{itemize}
    \item \dbl\ lub \rdbl\ - (14)15+PC, oraz albo fit \textbf{3-kartowy}, albo brak \textbf{trzymania},
    albo potężna \textbf{układówka}. Innymi słowy karta,
    która chce sforsować, ale nie wie jeszcze, gdzie jej miejsce.

    \item \textbf{Podniesienie} koloru partnera - 12-14, z czwórki lub z trójki, jeśli z trójki to nie na syfie
    \item 2\nt\ - \inv\ z \textbf{6+ w kolorze otwarcia}
    \item 3 w kolor otwarcia - 6+, słabe
    \item 3 w inny kolor - słabe, dwukolorowe
    \item \textbf{Kolor przeciwnika} - \gf\ z \textbf{fitem} w kolorze partnera
\end{itemize}
Przykład:
\allbidding{
    1\diams & \enemy{\pass} & 1\spades & \enemy{2\hearts} \\
    ?
}
\begin{itemize}
    \item \dbl\ - 14(15+), albo 3\spades, albo równa karta bez trzymania, albo układ
    \item 2\spades\ - 3-4\spades
    \item 2\nt\ - 6+\diams, \inv
    \item 3\clubs\ - 5\diams, 5\clubs
    \item 3\diams\ - 6+\diams, słabe
    \item 3\hearts\ - \gf\ z 4\spades
\end{itemize}

\subsection{Ustalenia związane z kontrą bilansową}
\begin{itemize}
    \item Jeśli kontra bilansowa sforsowała nas do 3\nt, forsuje tylko do 4\minor\, ale \textbf{ustawia pas forsujący}.
\end{itemize}



%*********************%
% Obrona na Michaelsa %
%*********************%
\pagebreak
\section{Obrona przeciwko Michaelsowi}

\subsection{Michaelsy określone}
Mamy dwa cuebidy i dwa określone grywalne kolory. Stosujemy zasadę \textbf{lower for lower},
czyli niższy cuebid odpowiada niższemu kolorowi. Kontra jest 10+ bez fitu i ustawia sytuację ukarniania.

\subsubsection{1\clubs\ --- (2\clubs) --- ?}
\begin{itemize}
    \item 2\diams\ - Naturalne, do pasa
    \item 2\hearts\ - co najmniej \inv\ na 5+\clubs
    \item 2\spades\ - \gf\ na 5+\diams
\end{itemize}

\subsubsection{1\hearts\ --- (2\ntx) --- ?}
\begin{itemize}
    \item 3\clubs\ - co najmniej \inv\ z fitem \hearts
    \item 3\diams\ - \gf\ z 5+\spades
    \item 3\hearts\ - 3+\hearts, 6-10 PC
    \item 3\spades\ - 6+\spades, słabe
\end{itemize}
Po otwarciu 1\spades, 3\clubs\ to \gf\ na \hearts, a 3\diams\ to \inv+ na \spades.
Cuebid z przeskokiem to splinter.

\subsection{Michaelsy nieokreślone}
Mając tyko jeden cuebid i nie znając grywalnych kolorów, trzeba odpalić lepsze narzędzia.
Kontra nadal karna i ustawia sytuację ukarniania.
Cuebid z przeskokiem to splinter.
Inne kolory z przeskokiem to kolor+fit.
\subsubsection{1\hearts\ --- (2\hearts) --- ?}
\begin{itemize}
    \item 2\spades\ - \inv+ z fitem \hearts
    \item 2\nt\ - transfer na 3\clubs
    \item 3\clubs\ - transfer na 3\diams
    \item 3\diams\ - transfer na nasz kolor - \mixed
    \item 3\hearts\ - 6-10 z fitem \hearts
\end{itemize}
\subsubsection{1\spades\ --- (2\spades) --- ?}
\begin{itemize}
    \item 2\nt\ - transfer na 3\clubs
    \item 3\clubs\ - transfer na 3\diams
    \item 3\diams\ - transfer na nasz kolor przez cuebid - \mixed
    \item 3\hearts\ - \inv+ z fitem \spades
    \item 3\spades\ - 6-10 z fitem \spades
\end{itemize}

\subsection{Przeciwko multi i dwukoloruwkom po otwarciu 1\clubs}
TODO: Naprawić to
\subsubsection{1\clubs\ --- (2\diams\ Multi) --- ?}
\begin{itemize}
    \item 2\hearts, 2\spades\ - słabe naturalne
    \item 2\nt\ coś jakby trefle
    \item 3\clubs\ coś jakby dzwonki
    \item 3\diams\ - \inv+ na \hearts
    \item 3\hearts\ - \inv+ na \spades
\end{itemize}
\dots




% ***************************** %
% Kontry karne i pasy forsujące %
% ***************************** %
\pagebreak
\section{Sekwencje karne}
\subsection{Ustalone sekwencje kontry karnej}
\begin{table}[h!]
    \centering
    \begin{tabular}{cccc}
        1\nt & \enemy{\pass} & 2\diams/2\hearts & \enemy{3\clubs} \\
        \dbl
    \end{tabular}
    \caption*{
        Kontra wywoławcza oznaczałaby brak fitu (bo czemu nie 3\spades?), 
        a co za tym idzie 3-4 trefle, partner nie może mieć kierów skoro nie dał 2\clubs. 
    }
\end{table}



\end{document} 