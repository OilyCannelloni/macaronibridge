\documentclass[12pt, a4paper]{article}
\usepackage{../lib/bridgetex2}
\usepackage{polyglossia}
\setmainlanguage{polish}


\title{\vspace{-2cm}SAYCo jajco}
\author{}
\date{}

\begin{document}
\maketitle
\section{Założenia}
\begin{itemize}
    \item Otwarcia kolorem młodszym Walsh Style: otwieramy dłuższy młodszy, z 3-3 w młodych - 1\clubs, a z 4-4 i
    5-5 - 1\diams. Rebid otwierającego w kolor na poziomie 1 pokazuje rękę niezrównoważoną, zazwyczaj 5m-4M.
    \item Preferencja kolorów starszych w słabych rękach - pomijamy 1\diams, jeśli mamy starszą czwórkę i rękę maksymalnie \inv.
    \item Zachowana jest strefowa struktura podniesień 1\major\ - nie gramy Jacoby 2NT
    \item Bezrewersowe 2/1 z niepoważnym 3\nt
\end{itemize}

\pagebreak

\section{Otwarcie 1\minor}
\subsection{1\clubs\ --- ?}
\begin{itemize}
    \item 1\diams\ - ręka \gf\ z 5\diams\ lub ręka maksymalnie \inv\ z 5\diams\ \textbf{bez starszych czwórek}.
    \item 1\nt\ - 6-11 PC bez starszych czwórek.
    \item 2\clubs\ - 4+\clubs, 10+PC, bez starszych czwórek, \textbf{forsuje do 3\clubs} (Inverted Minors)
        Otwierający licytuje trzymania ekonomicznie,
        gracz, który przekroczy swoją odzywką 3\clubs\ forsuje do końcówki.
    \item 2\diams\ - ??? może \diams+\major?
    \item 2\hearts\ - 5\spades, 4+\hearts, 4-8 PC
    \item 2\spades\ - transfer na \nt, ręka co najmniej \inv. Jest to jedyna droga do zainwitowania 3\nt
        z ręką zrównoważoną. Przyjęcie z bilansu.
    \item 2\nt\ - 13-15 lub 18+ \gf, ręka zrównoważona, może zawierać starsze czwórki
    \subitem 3\clubs\ - Stayman
    \subitem 3\diams\ - Rebid w kolor otwarcia (6+\minor)
    \subitem 3\major\ - splinter, co najmniej 5/4 w \minor
    \item 3\clubs\ - \mixed\ z 5+\clubs\
    \item 3\nt\ - 15-17, ręka zrównoważona
\end{itemize}

\subsection{1\diams\ --- ?}
\begin{itemize}
    \item 1\nt\ - 6-11 PC bez starszych czwórek.
    \item 2\clubs\ - \gf\ na \textbf{ściśle} 5+\clubs
    \item 2\diams\ - 4+\diams, 10+PC, \textbf{forsuje do 3\diams} (Inverted Minors)
    \item 2\hearts\ - 5\spades, 4+\hearts, 4-8 PC
    \item 2\spades\ - transfer na \nt, ręka co najmniej \inv.
    \item 2\nt\ - \gf\ BAL (jak wyżej)
    \item 3\clubs\ - \mixed\ z 5+\clubs\
\end{itemize}

\pagebreak

\section{Licytacja po rebidzie 1\ntx\ (12-14) otwierającego}
\begin{itemize}
    \item 2\clubs\ - dowolny \inv, \soff\ na \diams, lub propozycja gry w 3\nt\ z 5\major
    \subitem 2\clubs\ - 2\diams\ --- 3\nt\ - 5332 w licytowanym \major, propozycja kontraktu
    \item 2\diams\ - dowolny \gf\, który nie jest zawarty w innej odzywce.
    Priorytet pokazywania:
    \subitem - 4 w drugim starym
    \subitem - 3 w licytowanym przez partnera starym
    \subitem - młodsza piątka
    \subitem - nic z powyższych
    \item 2\nt\ - słaby transfer na 3\clubs
    \item 3\minor\ - 5\major, 5\minor\ \gf\
    \item 3\hearts\ (w nasz kolor) - ustalenie koloru licytowanego, próba szlemikowa
        (\textbf{nie \inv\ z 6} - to przez 2\clubs\ i skok na 3\major)
    \item 3\spades\ i wyżej - splintery
    \item 3\nt\ - do gry
\end{itemize}

\pagebreak

\section{Licytacja po rebidzie 2\ntx\ (18-19) otwierającego}
\begin{itemize}
    \item 2\nt\ \textbf{nie forsuje!}
    \item W dowolnej sekwencji 1X - 1Y --- 2\nt, odzywki 3\clubs, 3\diams, 3\hearts, 3\spades\
    są \textbf{transferami} na kolor wyżej. Transfery pokazują \textbf{tyle kart w kolorze, ile trzeba}. (xd)
    \item Transfery sign-off'owe to 3\clubs\ (na \diams) i transfer na kolor odpowiadającego.
    Otwierający \textbf{musi} je przyjąć bezpośrednio, bo odpowiadający może chcieć uciec z NT na 3 w kolor
    (ale może mieć silniejszą rękę). Np:
    \begin{center}
        \webidding{ 
            1\diams\ & 1\spades \\
            2\nt\ & 3\clubs \\
            3\diams & (\pass?)
        } \qquad\qquad
        \webidding{
            1\clubs\ & 1\spades \\
            2\nt\ & 3\hearts \\
            3\spades & (\pass?)
        }
    \end{center}
    \item Transfery \gf\ to wszystkie inne. Przyjmujemy je bezpośrednio tylko z fitem. Np:
    \begin{center}
        \webidding{ 
            1\diams\ & 1\spades \\
            2\nt\ & 3\diams \\
            3\hearts (fit \hearts) & 7\hearts 
        } \qquad\qquad
        \webidding{
            1\clubs\ & 1\spades \\
            2\nt\ & 3\diams(4\hearts5\spades) \\
            3\spades (fit \spades) & 7\spades
        } \\[0.7cm]
        \webidding{
            1\clubs\ & 1\hearts \\
            2\nt\ & 3\hearts*\\
            3\nt\ & 6\nt
        } \qquad\qquad
        \webidding{
            1\diams\ & 1\hearts \\
            2\nt\ & 3\spades\\
            3\nt** & \pass
        }     
        \\[0.5cm]
        \raggedright
        * - pokazuje 4\hearts4\spades, bo piątego kiera pokażemy przez 3\diams. \\
        ** - Partnerze, mam w dupie twoje trefle.

    \end{center}
\end{itemize}
    
\pagebreak

% \section{Licytacja po otwarciu 2\clubs}
% \begin{itemize}
%     \item Nie gramy na kontrole, bo to ma być maksymalnie intuicyjny system bez pamięciówy.
%     \item Po objaśnieniu acola na niesamoustalającym kolorze, kolejna, wyższa odzywka \textbf{jest sztuczna}
%     i relay'owo pyta o dalszy skład.
%     \item Ze składem 5431 z młodszą piątką \textbf{rebidujemy \ntx!}
% \end{itemize}


% \subsection{2\clubs\ --- ?}
% \begin{itemize}
%     \item 2\diams\ - 4+PC. Dajemy to również z samotnym Królem, jest to nielimitowane od góry. \gf\
%     \item 2\hearts\ - 0-3PC, supernegat. Po rebidzie 2\nt\ otwierającego można spasować.
%     \item Wyższe odzywki - naturalny kolor 5(6)+ kartowy, zawierający \textbf{ściśle co najmniej KQ}. 2\nt\ jest na kierach.
% \end{itemize}

% \subsection{2\clubs\ --- 2\diams \\ ?}
% \begin{itemize}
%     \item 2\hearts\ - Kiery lub silne \nt. Wymusza automat 2\spades:
%     \subitem 2\nt\ - 24-25 BAL
%     \subitem 3\clubs\ - 6+\hearts. Zamienione, żeby się to dało sfitować przez 3\hearts.
%     \subitem 3\diams\ - 5\hearts, 4+\diams
%     \subitem 3\hearts\ - 5\hearts, 4+\clubs
%     \subitem 3\spades\ - 5\hearts, 4\spades
%     \subitem 3\nt\ - 26+ BAL

%     \item 2\spades\ - Naturalne, 5+, ręka niezrównoważona. (odp. 2\nt\ pyta)
%     \item 2\nt\ - 22-23 BAL
%     \item 3\clubs\ - 6+\clubs, naturalne (odp. 3\diams pyta, 3\major\ z piątki)
%     \item 3\diams\ - 6+\diams, brak starszej czwórki. (odp. 3\major\ z piątki)
%     \item 3\hearts, 3\spades\ - samoustalające
%     \item 3\nt\ - 6\diams\ + 4\major\ (jedyna pamięciówa, ale tego się nie da inaczej)
% \end{itemize}


%*********************%
% Obrona na Michaelsa %
%*********************%
\pagebreak
\section{Obrona przeciwko Michaelsowi}

\subsection{Michaelsy określone}
Mamy dwa cuebidy i dwa określone grywalne kolory. Stosujemy zasadę \textbf{lower for lower},
czyli niższy cuebid odpowiada niższemu kolorowi. Kontra jest 10+ bez fitu i ustawia sytuację ukarniania.

\subsubsection{1\clubs\ --- (2\clubs) --- ?}
\begin{itemize}
    \item 2\diams\ - Naturalne, do pasa
    \item 2\hearts\ - co najmniej \inv\ na 5+\clubs
    \item 2\spades\ - \gf\ na 5+\diams
\end{itemize}

\subsubsection{1\hearts\ --- (2\ntx) --- ?}
\begin{itemize}
    \item 3\clubs\ - co najmniej \inv\ z fitem \hearts
    \item 3\diams\ - \gf\ z 5+\spades
    \item 3\hearts\ - 3+\hearts, 6-10 PC
    \item 3\spades\ - 6+\spades, słabe
\end{itemize}
Po otwarciu 1\spades, 3\clubs\ to \gf\ na \hearts, a 3\diams\ to \inv+ na \spades.
Cuebid z przeskokiem to splinter.

\subsection{Michaelsy nieokreślone}
Mając tyko jeden cuebid i nie znając grywalnych kolorów, trzeba odpalić lepsze narzędzia.
Kontra nadal karna i ustawia sytuację ukarniania.
Cuebid z przeskokiem to splinter.
Inne kolory z przeskokiem to kolor+fit.
\subsubsection{1\hearts\ --- (2\hearts) --- ?}
\begin{itemize}
    \item 2\spades\ - \inv+ z fitem \hearts
    \item 2\nt\ - transfer na 3\clubs
    \item 3\clubs\ - transfer na 3\diams
    \item 3\diams\ - transfer na nasz kolor - \mixed
    \item 3\hearts\ - 6-10 z fitem \hearts
\end{itemize}
\subsubsection{1\spades\ --- (2\spades) --- ?}
\begin{itemize}
    \item 2\nt\ - transfer na 3\clubs
    \item 3\clubs\ - transfer na 3\diams
    \item 3\diams\ - transfer na nasz kolor przez cuebid - \mixed
    \item 3\hearts\ - \inv+ z fitem \spades
    \item 3\spades\ - 6-10 z fitem \spades
\end{itemize}

\subsection{Przeciwko multi i dwukoloruwkom po otwarciu 1\clubs}
TODO: Naprawić to
\subsubsection{1\clubs\ --- (2\diams\ Multi) --- ?}
\begin{itemize}
    \item 2\hearts, 2\spades\ - słabe naturalne
    \item 2\nt\ coś jakby trefle
    \item 3\clubs\ coś jakby dzwonki
    \item 3\diams\ - \inv+ na \hearts
    \item 3\hearts\ - \inv+ na \spades
\end{itemize}
\dots



\end{document} 