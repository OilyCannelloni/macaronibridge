\documentclass[12pt, a4paper]{article}
\usepackage{../lib/bridgetex2} 
\usepackage{polyglossia}
\usepackage{caption}
\setmainlanguage{polish}

\title{Lebensohl Basic (po 1\nt)}
\author{Bartek Słupik}

\captionsetup[table]{
    name=Ręka
}

\begin{document}
    \section{Założenia}
    Lebensohlem gramy, kiedy przeciwnik wchodzi kolorem na nasze 1\nt.
    \begin{itemize}
        \item Kolory na poziomie 2 są słabe i do pasa
        \item Kolory na poziomie 3 są 5+ i \gf
        \item Duże transfery grają normalnie jeśli są z przeskokiem \br
        \item \dbl\ jest karna
        \item 2\nt\ to \emph{Lebensohl} - wymusza automat 3\clubs\ u partnera,
        poprawienie na inny kolor jest do pasa \br
        \item Kolor przeciwnika to \textbf{Stayman} bez trzymania, 3\nt\ jest do gry \textbf{bez trzymania}
        \item Lebensohl, a potem kolor przeciwnika to \textbf{Stayman} z trzymaniem, 3\nt\ jest do gry \textbf{z trzymaniem}
        \item Lebensohl, a potem kolor, który mógł być zalicytowany na poziomie 2 to inwit z co najmniej ładną piątką.
    \end{itemize}

    \section{Kiedy gramy Lebensohlem?}
    To jest wersja Basic, więc maksymalnie upraszczamy radzenie sobie ze wszystkim.
    Zastrzegam, że jest to lekko nieoptymalne i wymaga późniejszej zmiany jeśli poczujecie się komfortowo
    \begin{itemize}
        \item \textbf{Po wejściu 2\clubs\ naturalnym} - \dbl\ to Stayman, reszta systemowo. 
        \item \textbf{Po wejściu 2\clubs\ sztucznym} - zawsze, niezależnie co to. \dbl\ jest karna punktowa, nie ma koloru przeciwnika 
        (2\clubs\ zwykle oznacza 5/4 \major, więc po co nam stayman skoro nie chcemy grać w \major?).
        \item \textbf{Po wejściu 2\diams} - zawsze, niezależnie co to. \dbl\ jest karna punktowa, 3\diams\
        \textbf{jest zawsze} kolorem przeciwnika, nawet jeśli 2\diams\ było multi - na multi ze słabymi karami pasujemy.
        To sztuczne 3\diams\ na multi \textbf{alertujemy!!!}
        \item \textbf{Po wejściu 2\major} - zawsze, niezależnie co to. \dbl\ jest karna punktowa.
    \end{itemize}

    Na wejścia na wysokości 3 - \textbf{\dbl\ jest negatywna!!!}, kolory są \gf

    \pagebreak

    \section{Przykłady licytacji}
    \begin{table}[h!]
        \centering
        \begin{tabular}{cccc}
            W & N & E & S \\
            1\nt & 2\hearts & 3\spades & \pass \\
            4\spades
        \end{tabular}
        \caption{3\spades\ - GF z 5\spades. 4\spades\ - do gry, również czasem z dublem pik bez trzymania}
    \end{table}

    \begin{table}[h!]
        \centering
        \begin{tabular}{cccc}
            W & N & E & S \\
            1\nt & 2\spades & 2\nt & \pass \\
            3\clubs & \pass & 3\diams
        \end{tabular}
        \caption{Słaba ręka z karami - zwykle 6+, w zielonych można z 5}
    \end{table}

    \begin{table}[h!]
        \centering
        \begin{tabular}{cccc}
            W & N & E & S \\
            1\nt & 2\hearts & 2\nt & \pass \\
            3\clubs & \pass & 3\hearts
        \end{tabular}
        \caption{4 piki GF z trzymaniem (slow shows)}
    \end{table}

    \begin{table}[h!]
        \centering
        \begin{tabular}{cccc}
            W & N & E & S \\
            1\nt & 2\spades & 3\spades & \pass \\
            3\nt 
        \end{tabular}
        \caption{4 kiery GF bez trzymania. Z ręką W bez trzymania należy powiedzieć 4 w dłuższy młody}
    \end{table}

    \begin{table}[h!]
        \centering
        \begin{tabular}{cccc}
            W & N & E & S \\
            1\nt & 2\diams* & 2\nt & \pass \\
            3\clubs & \pass & 3\diams* & \pass \\
            3\hearts & \pass & 4\hearts 
        \end{tabular}
        \caption{3\diams\ Stayman z trzymaniem przeciw 2\diams\ multi}
    \end{table}
    
    \begin{table}[h!]
        \centering
        \begin{tabular}{cccc}
            W & N & E & S \\
            1\nt & 2\hearts & 2\nt & \pass \\
            3\clubs & \pass & 3\spades & \pass \\
            4\spades 
        \end{tabular}
        \caption{Inwit z pikami - do pasa i GF z pikami można było zalicytować inaczej}
    \end{table}

    \begin{table}
        \centering
        \begin{tabular}{cccc}
            W & N & E & S \\
            1\nt & 2\hearts & 3\clubs & \pass \\
            3\hearts & \pass & 3\nt \\
        \end{tabular}
        \caption{GF na treflach, 3\hearts\ pytanie o trzymanie}
    \end{table}

    \begin{table}
        \centering
        \begin{tabular}{cccc}
            W & N & E & S \\
            1\nt & 2\hearts & \dbl & \pass \\
            2\spades & \pass & 3\nt
        \end{tabular}
        \caption{\dbl\ karną można wynieść z ofensywną ręką}
    \end{table}

    \begin{table}
        \centering
        \begin{tabular}{cccc}
            W & N & E & S \\
            1\nt & 2\hearts* & 2\nt & 3\clubs \\
            \pass & \pass & 3\diams
        \end{tabular}
        \caption{Kiedy drugi przeciwnik podnosi, systemowo}
    \end{table}

    \begin{table}
        \centering
        \begin{tabular}{cccc}
            W & N & E & S \\
            1\nt & 2\hearts* & 2\nt & 3\hearts \\
            \pass & \pass & \dbl & \pass \\
            3\nt
        \end{tabular}
        \caption{Kiedy drugi przeciwnik podnosi, X = negatywna}
    \end{table}

\end{document}