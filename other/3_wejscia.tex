\documentclass[12pt, a4paper]{article}
\usepackage{../lib/bridgetex2}
\usepackage{polyglossia}
\setmainlanguage{polish}

\title{\vspace{-2cm}Wchodzenie do licytacji przeciwnika}
\author{}
\date{}

\begin{document}
    \maketitle
    \section{Streszczenie}
    Kiedy przeciwnik otwiera licytację, dynamika rozdania zmienia się natychmiastowo.
    Średnio to przeciwnicy będą mieli teraz \textbf{przewagę siły}, w dużej części przypadków
    wychodzi im końcówka. Musimy temu przeciwdziałać i dać sobie możliwie największe szanse
    znalezieniu lepszej częściówki lub obłożenia przeciwnika. 

    \subsection{Po co wchodzimy do licytacji?}
    \begin{itemize}
        \item W celu \textbf{znalezienia własnego fitu} i zagraniu zwykle częściówki, choć końcówki też się zdarzają.
        Na przykład po otwarciu przeciwnika 1\hearts, z ręką
        \begin{center}
            \hhand{QJ852}{3}{J832}{AQJ}
        \end{center}
        zalicytujemy 1\spades, by spróbować później przebić 2\hearts\ do 2\spades, jeśli znajdziemy fit.
        Czasami zdarzy się też, że stwierdzimy, że lepiej będzie rozgrywać 4\spades\dbl-2 za -300, niż bronić
        łatwego 4\hearts= przeciwnika za -620. Na przykład, gdy do powyższej karty partner dokłada
        \begin{center}
            \hhand{T9764}{952}{A}{7642}
        \end{center}
        4\spades\ jest na impasie, a 5\hearts\ nie idzie, gdyż odbieramy 2 asy i przebitkę (a może i dwie) karo.
        Przeciwnik ma 25 punktów i stoi przed beznadziejnym wyborem, czy \emph{poświęcić} 4\spades\ i
        w 50\% przypadków postawić 590, czy powiedzieć 5\hearts\ i przegrać zawsze.
        \item Aby zablokować i \textbf{utrudnić licytację} przeciwnikowi, co czasami spowoduje rozgrywanie przez niego złego kontraktu.
        Na otwarcie 1\diams\ z ręką
        \begin{center}
            \hhand{95}{KQ9863}{3}{7632}
        \end{center}
        wejdziemy 2\hearts, co postawi drugiego przeciwnika w trudnej sytuacji, w której czasami nie będzie mógł
        znaleźć odpowiedniego bilansu na końcówkę. Dajmy mu rękę
        \begin{center}
            \hhand{AK6}{7542}{Q8}{Q854}
        \end{center}
        To pachnie jak 3\nt, ale nawet nie wiadomo, czy trzymamy kiery... A bilansu sprawdzić się nie da.

        \item Żeby wskazać partnerowi \textbf{bezpieczny kolor wistu}, który pomoże obłożyć kontrakt przeciwników.
        Czasami na poprawne wejście wystarczy dość mocny kolor:
        \begin{center}
            \hhand{AQT85}{62}{974}{J43}
        \end{center}
        Powiedzmy, że licytacja potoczyła się jak poniżej, a my trzymamy rękę N:
        \begin{table}[h!]
            \centering
            \begin{tabular}{cccc}
                \vul{W} & \nvul{N} & \vul{E} & \nvul{S} \\
                1\clubs & 1\spades & 3\nt & \pass \\
                \pass & \pass
            \end{tabular}
        \end{table}

        a partner trzyma poniższą kartę i wistuje w \spades6.
        \begin{center}
            \hhand{J62}{A573}{T942}{865}
        \end{center}
        Dopóki rozgrywający nie trzyma czwartego \spades K, będzie musiał zabić, a S po dojściu \hearts A zagra znów w pika
        i weźmiemy 5 lew, tym samym obkładając 28-punktową końcówkę.
    \end{itemize}

    \pagebreak
    \section{Wejście kolorem}
    Jest na to potrzebny \textbf{niepusty} pięciokart oraz teoretycznie 8-17 PC. Dlaczego teoretycznie?
    Porównajmy następujące ręce:
    \begin{enumerate}
        \item \hhand{AJT96}{K97}{6432}{9}
        \item \hhand{J7643}{K97}{KQJT}{9}
    \end{enumerate}
    Mimo, że pierwsza ma tylko 8 PC, wejście jest automatyczne. Wist pikowy może być jedynym dobrym!
    Za to druga, mimo że silniejsza, może wypuszczać kontrakt, np. jeśli partner zawistuje spod \spades K.
    Zatem podsumowując:

    \begin{formal}
        Im słabsza ręka, tym lepszej jakości musi być kolor. Przy minimum siły
        potrzeba coś koło AQTxx, przy maksimum może być z Jxxxx.
    \end{formal}
    
    \subsection{Licytacja na przeciw wejścia}
    Zacznijmy do podstawowej zasady licytacji dwustronnej:
    \begin{formal}
        W licytacji dwustronnej podnoszenie bezpośrednio koloru partnera jest \textbf{słabe}
        i \textbf{przepychowe}. Skoki w nasz kolor są \textbf{blokujące}.
    \end{formal}
    
    \begin{table}[h!]
        \centering
        \begin{tabular}{cccc}
            \nvul{W} & \nvul{N} & \nvul{E} & \nvul{S} \\
            1\diams & 1\spades & \pass & ?
        \end{tabular} \\
        \raggedright
        2\spades\ = 6-11, 3+\spades. Zwróćmy uwagę na bardzo szeroki zakres - partner może mieć tylko 8 punktów!
    \end{table}
    \begin{table}[h!]
        \centering
        \begin{tabular}{cccc}
            \nvul{W} & \nvul{N} & \nvul{E} & \nvul{S} \\
            1\diams & 1\spades & 1\nt & ?
        \end{tabular} \\
        \raggedright
        2\spades\ = mniej więcej 5-11 w korzystnych, 7-11 w niekorzystnych - przeciwnicy mają co najmniej połowę talii!
    \end{table}
    \begin{table}[h!]
        \centering
        \begin{tabular}{cccc}
            \nvul{W} & \nvul{N} & \nvul{E} & \nvul{S} \\
            1\diams & 1\spades & 2\diams & ?
        \end{tabular} \\
        \raggedright
        2\spades\ = 4-9! Przeciwnicy mają fit i my też, co oznacza że rozdanie jest układowe, punkty się mniej liczą,
        a partner musi wiedzieć o ficie, żeby przebić przeciwników, jeśli uzna to za stosowne.
    \end{table}

    \subsection{Jak pokazać silniejszą rękę?}
    Mamy następującą rękę po licytacji:
    \begin{table}[h!]
        \centering
        \begin{tabular}{cccc}
            \nvul{W} & \nvul{N} & \nvul{E} & \nvul{S} \\
            1\clubs & 1\spades & \pass & ?
        \end{tabular} \\[0.5em]
        \hhand{QJ83}{83}{A632}{KJ7}
    \end{table}

    Nie potrzebujemy wiele, by ugrać końcówkę - wystarczy \hhand{Kxxxx}{Axx}{x}{Qxxx}. Chcielibyśmy zatem 
    pokazać większą siłę, niż 2\spades. Jednak nie możemy podnieść pików bezpośrednio. Inne kolory oraz \nt\ są naturalne.
    Co pozostaje? \textbf{Trefle!} Licytowanie trefli nie może oznaczać trefli, bo \textbf{ma je przeciwnik}.
     Licytujemy 2\clubs, i przyjmujemy zasadę:

    \begin{formal}
        W licytacji dwustronnej, gdy partner wszedł lub otworzył kolorem, zalicytowanie \textbf{koloru przeciwnika}
        pokazuje \textbf{dobrą rękę z fitem} - co najmniej na inwit.
    \end{formal}

    \subsection{Jeśli nie mamy fitu...}
    \begin{itemize}
        \item Odpowiedzi w \nt\ są naturalne - 1\nt\ = 8-12 (poparcie), 2\nt\ = 13-14 (inwit), 3\nt\ = 15+ (do gry)
        \item Odpowiedzi kolorami są naturalne z piątki i \textbf{nie forsują} - jeśli partner ma 7-punktowe wejście, zwykle spasuje
        lub wróci na kolor, którym wszedł.
        \item Odpowiedzi kolorami z przeskokiem są \gf\ z piątki.
    \end{itemize}

    \subsection{Wejście na wysokości 2 bez przeskoku}
    Tu zasady są trochę bardziej rygorystyczne:
    \begin{itemize}
        \item Potrzebujemy ręki w sile otwarcia (11+) z 
        \textbf{sześciokartowym kolorem!!!} - z wyjątkiem wejścia 2\hearts\ po otwarciu 1\spades, które może być z piątki.
        \item Odpowiedzi są podobne, z jedną zmianą - nowe kolory forsują na jedno okrążenie.
    \end{itemize} 

    
    \pagebreak
    \section{Wejście kontrą}
    Zaznaczę to wyraźnie i od razu, zanim będzie za późno. Warto na start przyjąć proste, a skuteczne ustalenie:
    \begin{formal}
        \begin{itemize}
            \item Kontry na częściówki są \textbf{wywoławcze} (czyli \textbf{nie karne}),
            chyba że zostało ustalone explicite inaczej.
            \item Kontry na końcówki są \textbf{karne}.
        \end{itemize}
    \end{formal}

    \subsection{A co to ta kontra wywoławcza?}
    Przeciwnik otwiera 1\diams, a my trzymamy:
    \begin{center}
        \hhand{AJ93}{KJ87}{5}{KJT3}
    \end{center}
    Bardzo chcielibyśmy wejść do licytacji, ale nie mamy pięciokartu! Pozostaje nam zatem tylko jedna odzywka
    - tzw. czerwony kartonik.
    \begin{formal}
        Wejście do licytacji kontrą pokazuje rękę z siłą co najmniej \textbf{na otwarcie} (12+), która:
        \begin{itemize}
            \item Nie ma pięciokartu, którym może wejść
            \item Na kolor starszy, ma czwórkę w drugim kolorze starszym
            \item Na kolor młodszy, ma co najmniej 4-3 w kolorach starszych
            \item \textbf{Toleruje grę w każdy inny kolor} niż kolor otwarcia.
        \end{itemize}
    \end{formal}

    \subsection{Z czym kontrować?}
    Przeciwnik otwiera 1\hearts.
    \begin{table}[h!]
        \centering
        \setlength{\extrarowheight}{3pt}
        \begin{tabular}{rc}
            \multicolumn{1}{c}{Ręka} & \multicolumn{1}{c}{Licytuj} \\
            \hhand{5432}{63}{AKQ8}{AJ3} & \dbl \\
            \hhand{AJ98}{6}{Q8632}{AJ3} & \dbl, \textbf{nie wolno} wejść 2\diams \\
            \hhand{AJ98}{863}{AQ8}{KJ9} & \dbl, skład nadrabia siłą \\
            \hhand{QT932}{-}{AJ86}{AK63} & 1\spades, \dbl\ nie ze starszą 5tką
        \end{tabular}
    \end{table}

    \pagebreak
    Przeciwnik otwiera 1\clubs.
    \begin{table}[h!]
        \centering
        \setlength{\extrarowheight}{3pt}
        \begin{tabular}{rc}
            \multicolumn{1}{c}{Ręka} & \multicolumn{1}{c}{Licytuj} \\
            \hhand{5432}{AKJ7}{QJ87}{3} & \dbl, modelowa mimo 11 PC \\
            \hhand{AJ42}{Q863}{Q83}{Q6} & \pass, za słaby skład \\
            \hhand{AJ98}{AK3}{Q873}{542} & \dbl, skład nadrabia siłą \\
            \hhand{A763}{AQ8}{983}{Q63} & \pass, 4-3 w starych na minus.
        \end{tabular}
    \end{table}

    \subsection{Licytacja na przeciw kontry}
    Jest 100\% naturalna i z bilansu. Czyli (powiedzmy, że przeciwnik otwiera 1\diams):
    \begin{itemize}
        \item 1\hearts, 1\spades\ - 0-6 PC. \textbf{Musimy licytować!} Chyba nie chcemy grać 1\diams\dbl!
        \item 2\hearts, 2\spades, 1\nt\ - 7-10 (poparcie)
        \item 3\hearts, 3\spades, 2\nt\ - 11-12 (inwit)
        \item 4\hearts, 4\spades, 3\nt\ - do gry
    \end{itemize} 
    Trudno to rozpisać, ale najniższa odzywka w każdy kolor jest \textbf{od zera}. Np. po otwarciu 1\hearts\ 
    przeciwnika i \dbl\ partnera, 1\spades\ jest 0-6, a 2\clubs\ i 2\diams\ są 0-9(10).
    1\nt\ jest zawsze 7-9. Jeszcze jedna uwaga:
    \begin{formal}
        Kolor zalicytowany przez partnera kontrującego uznaje się za ustalony. Jeśli kontrujący go zmieni,
        pokazuje to duże nadwyżki (18+), gdyż jest to wtedy tzw. \emph{kontra objaśniająca}.
    \end{formal}
    Przykładowo w tej sekwencji
    \begin{table}[h!]
        \centering
        \begin{tabular}{cccc}
            1\spades\ & \dbl & \pass & 2\clubs \\
            \pass & 2\diams
        \end{tabular}
    \end{table}
    kontrujący pokazał dowolną rękę 18+ na 5+\diams, ale o tym kiedy indziej. Na razie mamy inne problemy :)
    
    \subsection{Pewien problem}
    1\hearts\ przeciwnika, \dbl\ partnera, \pass\ z prawej. Trzymamy:
    \begin{center}
        \hhand{97}{75432}{532}{965}
    \end{center}
    Co tu zrobić? Dowód zostawiam czytelnikowi, bo sam nie wiem. 

\end{document}