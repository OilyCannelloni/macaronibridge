\documentclass[12pt, a4paper]{article}
\usepackage{../lib/bridgetex2}
\usepackage{polyglossia}
\setmainlanguage{polish}
\newcommand{\cy}{\cellcolor{Yellow!25}}
\newcommand{\co}{\cellcolor{Orange!25}}
\newcommand{\cx}{\cellcolor{Red!25}}


\title{\vspace{-2cm}Granie w karty wytłumaczone w 2 strony}
\author{}
\date{}

\begin{document}
    \maketitle
    \section{Punktacja}
    Cała taktyka gry w brydża jest w zasadzie pochodną punktacji za kontrakty.
    Za wylicytowanie i ugranie kontraktów otrzymuje się punkty za lewy oraz premie:
    \begin{table}[h!]
        \centering
        \begin{tabular}{cccccc}
            \nvul{NS} & \clubs & \diams & \hearts & \spades & \nt \\
            1 & 70 & 70 & 80 & 80 & 90 \\
            2 & 90 & 90 & 110 & 110 & 120 \\
            3 & 110 & 110 & 140 & 140 & \cy400 \\
            4 & 130 & 130 & \cy420 & \cy420 & \cy430 \\
            5 & \cy400 & \cy400 & \cy450 & \cy450 & \cy460 \\
            6 & \co920 & \co920 & \co980 & \co980 & \co990 \\
            7 & \cx1440 & \cx1440 & \cx1520 & \cx1520 & \cx1530 \\
        \end{tabular}
        \caption{Punktacja}
    \end{table} 

    Zwróćmy uwagę na następujące rzeczy:
    \begin{itemize}
        \item Kontrakty na wysokosci 6 i 7 przychodzą rzadko - dlatego najwięcej zysku
        jest z grania w strefie żółtej, tzw. końcówek
        \item Nie opłaca się grać w \clubs\ i \diams, bo żeby dostać premię, trzeba ugrać kontrakt na
        wysokości 5! Dlatego często nawet mając fit w treflach lub karach, będziemy chcieć grać 3\nt.
    \end{itemize}

    Z tych powodów wszystkie systemy licytacji w brydżu mają na celu przede wszystkim sprawdzenie,
    czy jesteśmy wystarczająco silni na końcówkę, oraz priorytetyzują grę w kolory starsze i \nt.

    Na razie pominiemy też szukanie w licytacji szlemików i szlemów - zysk z tego jest niewspółmiernie
    mały do zysku z prawidłowego znajdowania końcówek.

    \pagebreak
    \section{Bilans i fit}

    Statystyka mówi, że jedną z najlepszych metod oceny siły kart jest skala punktowa,
    gdzie za Asa dajemy 4, Króla 3, Damę 2 i Waleta 1 punkt. 
    
    Wówczas można dowieść, że jeśli na obu rękach w parze jest w sumie \textbf{co najmniej 25} punktów,
    opłacalna jest próba zagrania \textbf{końcówki}. Dlatego w licytacji będziemy
    się skupiać na sprawdzeniu tego warunku wystarczającego.

    \subsection{W kolor czy \ntx?}
    Generalnie granie w kolor jest lepsze, gdyż granie w \nt\ niesie ze sobą dużo zagrożeń.
    Nawet mimo to, że w \hearts\ lub \spades\ trzeba wziąć 10 lew na końcówkę, a w \nt\ 9,
    \textbf{wolimy grać w kolor} jeśli mamy fit, czyli co najmniej 8 kart w tym kolorze.

    \subsection{Otwarcie}
    Połowa z 25 to 12, więc otwieramy, jeśli mamy na ręce \textbf{co najmniej 12 punktów},
    lub rękę, która nadrabia brakujące punkty układem.

    \subsection{Strefy}
    Jako odpowiadający możemy wpaść w cztery strefy:
    \begin{itemize}
        \item Nic (0-6 PC) - żeby tu szła końcówka, partner musi mieć $\geq$ 19 punktów,
        na co szansa jest mała - \textbf{pasujemy}
        \item Poparcie (7-10 PC) - tu wystarczy 15-18 punktów u partnera, trzeba \textbf{podtrzymać licytację}!
        \item Inwit (11-12 PC) - Do końcówki brakuje nam minimalnej nadwyżki u partnera, chcemy żeby dołożył
        mając 13-14. Licytujemy \textbf{wyżej}, ale jeszcze nie końcówkę
        \item Forsing (13+ PC) - Wiemy, że mamy co najmniej 25 - \textbf{sami wrzucamy końcówkę}
    \end{itemize}

    \pagebreak
    \section{Podstawy systemu}
    \subsection{Otwarcie kolorem starszym}
    1\hearts\ i 1\spades\ otwieramy z piątki w sile 12+ PC. Odpowiedzi w poniższej tabeli:
    \begin{table}[h!]
        \centering
        \begin{tabular}{ccc}
        1\hearts\ --- ? & Z fitem & Bez fitu \\
        Poparcie & 2\hearts & 1\nt \\
        Inwit & 3\hearts & 2\nt \\
        Forsing & 4\hearts & 3\nt
        \end{tabular}
    \end{table}

    \subsubsection{Przykłady licytacji}
    Partner otwiera 1\hearts. * - wyjaśnione niżej.
    \begin{table}[h!]
        \centering
        \setlength{\extrarowheight}{3pt}
        \begin{tabular}{rc}
        \multicolumn{1}{c}{Karta} & Licytuj \\
        \hhand{A875}{K64}{KJT43}{2} & 3\hearts \\
        \hhand{AQ75}{K64}{KJT43}{2} & 4\hearts \\
        \hhand{AK3}{62}{QJ654}{KJ2} & 3\nt \\
        \hhand{98}{J3}{Q87}{AQJ42} & 1\nt \\
        \hhand{Q764}{2}{AJ65}{8742} & 1\spades*        
        \end{tabular}
    \end{table} 

    Otworzyliśmy 1\spades, partner odpowiedział 2\nt.
    \begin{table}[h!]
        \centering
        \setlength{\extrarowheight}{3pt}
        \begin{tabular}{rc}
        \multicolumn{1}{c}{Karta} & Licytuj \\
        \hhand{KJT98}{K64}{KQT4}{2} & \pass \\
        \hhand{KJT98}{KQ4}{KQT4}{2} & 3\nt \\
        \hhand{AKJ983}{K4}{KT94}{2} & 4\spades
        \end{tabular}
    \end{table} 

    Otworzyliśmy 1\hearts, partner odpowiedział 2\hearts.
    \begin{table}[h!]
        \centering
        \setlength{\extrarowheight}{3pt}
        \begin{tabular}{rc}
        \multicolumn{1}{c}{Karta} & Licytuj \\
        \hhand{K65}{AJ963}{KQT4}{2} & \pass \\
        \hhand{K65}{AQ963}{AQJ4}{2} & 3\hearts \\
        \hhand{K65}{AQJ63}{AKJ4}{2} & 4\hearts
        \end{tabular}
    \end{table} 

    \pagebreak
    \subsection{Otwarcie kolorem młodszym}
    Nie chcemy grać w kolor młodszy, dlatego zawsze staramy się znaleźć fit w starszym.
    \begin{itemize}
        \item 1\clubs\ będziemy otwierać z 5+\clubs, lub ręką równą, która nie ma pięciokartu w sile 12-14 lub 18-20
        (15-17 otworzymy 1\nt)
        \item 1\diams\ będziemy otwierać z 5+\diams, lub mając rękę w składzie 4441.
    \end{itemize}
    Odpowiedzi:
    \begin{table}[h!]
        \centering
        \begin{tabular}{cl}
            1\clubs\ --- 1\diams & \textbf{słaba ręka} 0-6 PC, \pass\ nie, bo partner może nie mieć trefli! \\
            1\hearts, 1\spades & z co najmniej \textbf{czwórki}, siła co najmniej poparcie (7+) \\
            2\clubs, 2\diams, 2\hearts, 2\spades & z co najmniej  piątki, siła \gf\ (13+)\\
            1\diams\ --- 2\diams & 7-9 z 4+\diams \\
            1\nt, 2\nt, 3\nt & z bilansu punktowego, jak po otwarciu 1\hearts\ i 1\spades.
        \end{tabular}
        \caption{Odpowiedzi na otwarcia 1\clubs\ i 1\diams}
    \end{table}
    \subsubsection{Przykłady licytacji}
    Partner otwiera 1\clubs.
    \begin{table}[h!]
        \centering
        \setlength{\extrarowheight}{3pt}
        \begin{tabular}{rc}
        \multicolumn{1}{c}{Karta} & Licytuj \\
        \hhand{KQ98}{6543}{AJ8}{32} & 1\hearts \\
        \hhand{J8743}{AKQ43}{K87}{2} & 2\spades \\
        \hhand{KJ8}{K46}{AQ94}{2} & 3\nt \\
        \hhand{T98}{K46}{AQ94}{2} & 1\nt
        \end{tabular}
    \end{table} 
    
    Otwarliśmy 1\clubs.
    Partner odpowiada 1\spades.
    Tutaj podniesienie koloru z fitem z bilansu, 1\nt\ = 12-14, 2\nt\ = 18-20.
    \begin{table}[h!]
        \centering
        \setlength{\extrarowheight}{3pt}
        \begin{tabular}{rc}
        \multicolumn{1}{c}{Karta} & Licytuj \\
        \hhand{A54}{7654}{KQJ8}{K2} & 1\nt \\
        \hhand{A543}{765}{KQJ8}{K2} & 2\spades \\
        \hhand{AK43}{7}{KQ8}{A9874} & 3\spades \\
        \hhand{AKJ3}{765}{AQJ8}{K2} & 4\spades \\
        \hhand{Q86}{AK8}{KQJ3}{A74} & 2\nt 
        \end{tabular}
    \end{table} 

    \pagebreak
    \subsection{Otwarcie 1\ntx}
    1\nt\ = 15-17 PC, skład zbalansowany, czyli co najmniej 2 w każdym kolorze.

    \subsubsection{Z czym otwierać 1\ntx?}
    \begin{table}[h!]
        \centering
        \setlength{\extrarowheight}{3pt}
        \begin{tabular}{rc}
        \multicolumn{1}{c}{Karta} & Licytuj \\
        \hhand{AKQ8}{54}{KQJ4}{432} & 1\nt \\
        \hhand{AK8}{Q4}{AQJ87}{42} & 1\nt \\
        \hhand{98}{AQJ52}{A94}{KJ5} & 1\nt \\
        \end{tabular}
    \end{table} 

    Jak widać, 1\nt\ otwieramy z pięciokartami w składzie 5332. Dlaczego? Bo jeśli 
    z ręką nr 3 otworzylibyśmy 1\hearts, a partner powie 1\spades, to nie mamy odzywki 
    (na rebid 1\nt\ jesteśmy za silni). Lepiej jest ukryć czasami pięciokart,
    a w przyszłości poznacie narzędzia do ujawnienia go później.

    Mówiąc 1\nt, otwierający bardzo szczegółowo określa swoją siłę. Dlatego to odpowiadający będzie ciągnął licytację.
    \subsubsection{Szukanie fitu 4-4 w kolorze starszym}
    Odpowiedź 2\clubs\ (tzw. Stayman) to pytanie o starszą czwórkę. Siła co najmniej na inwit (tutaj 8+, bo partner ma 15-17!).
    Na 2\clubs\ licytujemy posiadaną starszą czwórkę, a jak nie mamy, to 2\diams. Dalej licytacja jest naturalna z bilansu.
    
    Otwieramy 1\nt, partner odpowiada 2\clubs.
    \begin{table}[h!]
        \centering
        \setlength{\extrarowheight}{3pt}
        \begin{tabular}{rc}
        \multicolumn{1}{c}{Karta} & Licytuj \\
        \hhand{KQ98}{A54}{K8}{A832} & 2\spades \\
        \hhand{KQ9}{A54}{K8}{A8432} & 2\diams \\
        \hhand{KQ98}{A654}{K8}{A84} & 2\hearts \\
        \end{tabular}
    \end{table} 

    Partner otwiera 1\nt, a na staymana 2\clubs\ odpowiada 2\hearts.
    \begin{table}[h!]
        \centering
        \setlength{\extrarowheight}{3pt}
        \begin{tabular}{rc}
        \multicolumn{1}{c}{Karta} & Licytuj \\
        \hhand{KQ9}{8743}{A5}{J876} & 2\clubs, potem 4\hearts \\
        \hhand{KQ9}{8743}{K5}{J876} & 2\clubs, potem 3\hearts \\
        \hhand{KQ9}{8743}{85}{J876} & \pass \\
        \hhand{KQ97}{K46}{Q943}{52} & 2\clubs, potem 3\nt \\
        \hhand{KQ97}{Q46}{Q943}{52} & 2\clubs, potem 2\nt
        \end{tabular}
    \end{table} 

    \pagebreak
    \subsubsection{Transfery}
    Odpowiedzi 2\diams\ i 2\hearts\ oznaczają odpowiednio 5+\hearts\ i 5+\spades. Dlaczego?
    \begin{itemize}
        \item Chcemy podtrzymać licytację, by odpowiadający mógł się odezwać jeszcze raz
        \item Lepiej, żeby rozgrywała ręka silniejsza, bo:
        \item Figury są ukryte i przeciwnicy ich nie widzą
        \item Wist jest w kierunku silnej ręki i czasami da nam to dodatkową lewę, np jak mamy AQ w kolorze wistu.
        \item Z kartą typu \hhand{J9854}{7}{652}{9754} lepiej grać 2\spades\ na 7 atutach niż 1\nt.
    \end{itemize}

    Otwierający musi zaakceptować transfer licytując kolor o 1 wyżej. Zauważmy, że jeśli mamy 6+
    kart w kolorze, na który transferujemy, to mamy fit, bo partner ma co najmniej dwie. Jeśli mamy dokładnie 5, należy
    zaproponować grę w \nt, którą otwierający może poprawić na kolorową jeśli ma fit.

    W poniższych przykładach partner otwiera 1\nt, a następnie akceptuje transfer na piki przez 2\spades.
    \begin{table}[h!]
        \centering
        \setlength{\extrarowheight}{3pt}
        \begin{tabular}{rc}
        \multicolumn{1}{c}{Karta} & Licytuj \\
        \hhand{65432}{432}{432}{32} & 2\hearts, potem \pass \\
        \hhand{KJ9752}{43}{AQ3}{98} & 2\hearts, potem 4\spades \\
        \hhand{KJ975}{432}{AQ3}{98} & 2\hearts, potem 3\nt \\
        \hhand{KJ9752}{43}{A83}{98} & 2\hearts, potem 3\spades \\
        \hhand{KJ975}{432}{AJ3}{98} & 2\hearts, potem 2\nt
        \end{tabular}
    \end{table} 

    \subsubsection{Ani Stayman, ani transfer}
    Licytujemy \nt\ z bilansu. Poniżej partner otwiera 1\nt.
    \begin{table}[h!]
        \centering
        \setlength{\extrarowheight}{3pt}
        \begin{tabular}{rc}
        \multicolumn{1}{c}{Karta} & Licytuj \\
        \hhand{KQ9}{654}{AJ84}{432} & 3\nt \\
        \hhand{KQ9}{654}{A984}{432} & 2\nt \\
        \hhand{KQ9}{654}{Q984}{432} & \pass \\
        \hhand{KQ9}{AJ6}{AK98}{J83} & 6\nt
        \end{tabular}
    \end{table} 


\end{document}