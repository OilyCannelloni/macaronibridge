\documentclass[12pt, a4paper]{article}
\usepackage{import}

\import{../lib/}{bridge.sty}
\setmainlanguage{polish}
\newtheorem{problem}[table]{Problem}

\begin{document}

\begin{problem}
    Otwarcie.
\end{problem}
\chhand{AKT973}{2}{KT86}{54}

\begin{problem}
    Otwarcie.
\end{problem}
\chhand{AQ7}{632}{A4}{AKT86}

\begin{problem}
    Otwarcie.
\end{problem}
\chhand{Q}{J732}{KJ64}{AJ53}

\begin{problem}
    Partner otwiera \emph{1\hearts}.
\end{problem}
\chhand{AKQ98}{532}{6}{Q742}

\begin{problem}
    Partner otwiera \emph{1\hearts}
\end{problem}
\chhand{-}{K9842}{QJT86}{643}

\begin{problem}
    Partner otwiera \emph{1\nt}
\end{problem}
\chhand{AQ83}{2}{Q986}{T984}

\begin{problem}
    Otwieramy \emph{1\hearts}, partner licytuje \emph{2\hearts}
\end{problem}
\chhand{7}{AKQ63}{87}{AT432}

\begin{problem}
    Otwieramy \emph{1\hearts}, partner licytuje \emph{2\nt}
\end{problem}
\chhand{8}{Q96532}{AK8}{Q43}

\begin{problem}
    Otwieramy \emph{1\hearts}, partner licytuje \emph{1\nt}
\end{problem}
\chhand{8}{KJ952}{A8}{AQ932}

\begin{problem}
    Otwieramy \emph{1\diams}, partner licytuje \emph{1\nt}
\end{problem}
\chhand{9873}{3}{AJT3}{AQ98}

\pagebreak

\section*{Odpowiedzi}
\subsection*{Problem 1}
Otwarcie.
\chhand{AKT973}{2}{KT86}{54}
Odpowiedzi: 1\spades\ (10p), 2\spades\ (3p) \\
Mimo 10 punktów mamy układ 6-4, dobrze położone figury i wysokie blotki. Minimum, które w tym
składzie należy otworzyć na wysokości 1 to mniej więcej \chhand{AJT932}{2}{KJT6}{54}

\subsection*{Problem 2}
Otwarcie.
\chhand{AQ7}{632}{A4}{AKT86}
Odpowiedzi: 1\clubs\ (10p), 1\nt\ (5p) \\
Ta ręka jest za silna na 1\nt\ mimo 17 punktów. Zauważmy, że otworzymy 1\clubs\ jako 18-20 z dużo gorszą ręką
\chhand{AQ7}{J32}{AQJ}{KJ32}

\subsection*{Problem 3}
Otwarcie.
\chhand{Q}{J732}{KJ64}{AJ53}
Odpowiedzi: 1\diams\ (10p), \pass\ (5p), 1\clubs\ (2p) \\
Nie liczymy singlowej damy negatywnie. Jest bardzo duża szansa, że partner ma piki i dama będzie grała!
Mamy 12 punktów więc otweiramy. Nie 1\clubs, bo po odpowiedzi 1\spades\ nie mamy co powiedzieć. 1\nt\ wskazałoby
skład zrównoważony, a 2\diams\ - około 16+ punktów. Po otwarciu 1\diams\ mamy komfortowy rebid 2\clubs.


\subsection*{Problem 4}
Partner otwiera 1\hearts.
\chhand{AKQ98}{532}{6}{Q742}
Odpowiedzi: 3\hearts\ (10p), 4\hearts\ (8p), 1\spades\ (3p), 2\hearts\ (0p) \\
Mimo solidnych pików, należy pokazać fit. Partner pewnie ma singla pik i nie ma co kombinować,
szczególnie, że nie mamy systemu do pokazania najpierw \spades, a potem fitu \hearts.
Należy zainwitować końcówkę, choć karta jest bliska forsingu. Jeśli mielibyśmy czwartego kiera,
należy dołożyć końcówkę, np:
\chhand{AKQ98}{6542}{6}{Q74}

\subsection*{Problem 5}
Partner otwiera 1\hearts.
\chhand{-}{K9842}{QJT86}{643}
Odpowiedzi: 4\hearts\ (10p), 3\hearts\ (3p) \\
Wyobraźmy sobie typową słabą rękę partnera:
\chhand{K983}{AT763}{82}{AJ}
Rozliczając rozgrywkę z naszej ręki, oddajemy dwa kara i trefla, bo trzeciego przebijemy.
Nawet, jeśli przeciwnicy znajdą przebitkę karo na wiście, kolor uda nam się wyrobić bez oddawania lewy 
i powyrzucać trefle. Końcówka jest świetna nawet mimo zmarnowanego \spades K.

\subsection*{Problem 6}
Partner otwiera 1\nt.
\chhand{AQ83}{2}{Q986}{T984}
Odpowiedzi: 2\clubs\ i potem 2\nt, ale jak partner pokaże piki to 4\spades\ (10p), 2\clubs\ i potem inwit (6p),
2\clubs\ i potem końcówka (3p) \\
Mamy 8 punktów i do gry w 3\nt\ potrzebujemy maksimum partnera (17) - to jasne. Natomiast jeśli 
mamy fit pikowy, wartość naszej karty drastycznie wzrasta - jest duża szansa, że weźmiemy nawet dwie przebitki kier.

\subsection*{Problem 7}
Otwieramy 1\hearts, partner licytuje 2\hearts.
\chhand{7}{AKQ63}{87}{AT432}
Odpowiedzi: 3\clubs\ (10p), 3\hearts (7p) 4\hearts\ (5p) \\
5-5 i świetne figury podnoszą wartość karty do inwitu. Popatrzmy jednak na możliwe ręce partnera:
\chhand{KJ8}{T97}{KQ32}{986} \vspace{-5mm}
\chhand{J632}{J42}{963}{KQ9}
W pierwszym przypadku końcówka nie ma szans, mimo że partner ma 10 PC, w drugim jest po górze mimo, że ma 7.
Odzywka 3\clubs\ nie zamyka licytacji mimo znanego fitu - oznacza, że jeszcze nie wiemy co grać.
A sens mają tylko dwa kontrakty: 3\hearts\ i 4\hearts. Zatem 3\clubs\ jest inwitem pokazującym trefle naturalnie.
Partner weźmie to pod uwagę przy swojej decyzji.

\subsection*{Problem 8}
Otwieramy 1\hearts, partner licytuje 2\nt.
\chhand{8}{Q96532}{AK8}{Q43}
Odpowiedzi: 3\hearts\ (10p), \pass\ (5p) \\
Oczywiście nie przyjmujemy inwitu, mamy tylko 11 PC. Natomiast lepiej grać w kiery - w NT 
nasza ręka daje ok 2,5 lewy, a w 3\hearts, nawet do singla partnera, może łatwo dać 5 lew.
A czasami trafimy do dubla! 

\subsection*{Problem 9}
Otwieramy 1\hearts, partner licytuje 1\nt.
\chhand{8}{KJ952}{A8}{AQ932}
Odpowiedzi: 2\clubs\ (10p), \pass\ (3p) \\
Nie mamy na końcówkę, ale szansa na fit treflowy jest bardzo duża, a lepiej grać w kolor szczególnie z takim składem!

\subsection*{Problem 10}
Otwieramy 1\diams, partner licytuje 1\nt.
\chhand{9873}{3}{AJT3}{AQ98}
Odpowiedzi: 2\clubs\ (10p), \pass\ (3p) \\
Partner nie ma starszej czwórki, oraz nie podniósł naszych kar. Ma zatem maksymalnie skład 333x.
x = 13-9 = co najmniej 4 trefle. Mamy gwarancję fitu \clubs!
\end{document}