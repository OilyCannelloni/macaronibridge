\documentclass[12pt, a4paper]{article}
\usepackage{../lib/bridgetex2}
\usepackage{polyglossia}
\setmainlanguage{polish}
\newtheorem{problem}[table]{Problem}

\begin{document}

\begin{problem}
    Otwieramy \emph{1\spades}, partner odpowiada \emph{1\nt}.
\end{problem}
\chhand{AKT93}{KQ98}{A5}{K4}


\begin{problem}
    Partner otwiera \emph{1\diams}. Odpowiadamy \emph{1\hearts}, a partner rebiduje \emph{1\spades}.
\end{problem}
\chhand{832}{K97432}{A54}{4}


\begin{problem}
    Partner otwiera \emph{2\nt}.
\end{problem}
\chhand{832}{T987432}{65}{4}


\begin{problem}
    Jesteśmy odpowiadającym w następującej licytacji jednostronnej:
\end{problem}
\begin{center}
    \webidding{
        1\diams & 1\spades \\
        1\nt & \conventional{2\diams} \\
        2\hearts & ?
    }
\end{center}
\chhand{KJ832}{AK8}{Q542}{4}


\begin{problem}
    Partner otwiera \emph{1\hearts}. Odpowiadamy \emph{1\spades}, a partner rebiduje \emph{1\nt}.
\end{problem}
\chhand{K975}{Q2}{T8}{Q9843}


\begin{problem}
    Partner otwiera \emph{1\diams}. Odpowiadamy \emph{1\spades}, a partner rebiduje \emph{2\nt}.
\end{problem}
\chhand{A9632}{K9432}{74}{4}


\begin{problem}
    Partner otwiera \emph{1\diams}. Odpowiadamy \emph{1\hearts}, a partner rebiduje \emph{1\spades}.
\end{problem}
\chhand{832}{K9743}{4}{A542}


\pagebreak

\section*{Odpowiedzi}
\subsection*{Problem 1}
Otwieramy 1\spades, partner odpowiada 1\nt.
\chhand{AKT93}{KQ98}{A5}{K4}
Odpowiedzi: 2\nt\ (10p), 3\hearts\ (7p) \\
Partner nadal może mieć 4 kiery! Musimy dać mu szansę je pokazać. Po rebidzie 2\nt\ spodziewamy się
3\clubs\ pytającego nas o kiery (3\hearts\ byłoby z pięciu)


\subsection*{Problem 2}
Partner otwiera 1\diams. Odpowiadamy 1\hearts, a partner rebiduje 1\spades.
\chhand{832}{K97432}{A54}{4}
Odpowiedzi: 2\clubs\ (10p), \pass\ (8p), 2\hearts\ (3p)\\
Partner zrebidował 1\spades, więc nadal może mieć rękę niezrównoważoną z renonsem kier - nie
chcemy fosować gry w tak słaby kolor! Szczególnie, kiedy wiemy o ficie \diams\ i mamy sposób, 
żeby zagrać dobre 2\diams. \pass\ wchodzi również w grę - zagramy na 7 pikach, ale na poziomie 1.

Jeśli partner zrebidowałby 1\nt, 2\hearts\ jest jak najbardziej uzasadnione.


\subsection*{Problem 3}
Partner otwiera 2\nt.
\chhand{832}{T987432}{65}{4}
Odpowiedzi: 3\diams, potem 4\hearts\ (10p), 3\diams\ i \pass (4p) \\
Na pewno trzeba wykonać transfer. Natomiast zauważmy, że mimo naszej długości partner ma co najmniej dwa kiery.
Jeśli ma drugiego Asa, a kolor dzieli się 2-2, oddamy tylko jedną lewę kierową!

\pagebreak
\subsection*{Problem 4}
Jesteśmy odpowiadającym w następującej licytacji jednostronnej:
\begin{center}
    \webidding{
        1\diams & 1\spades \\
        1\nt & \conventional{2\diams} \\
        2\hearts & ?
    }
\end{center}
\chhand{KJ832}{AK8}{Q542}{4}
Odpowiedzi: 2\spades\ (10p), 3\nt\ (3p) \\
Partner pokazał 4 kiery, bo taka była najniższa odzywka, która opisywała jego rękę.
Ale nadal może mieć 3 piki! Dlatego należy pokazać piątego naturalnie.


\subsection*{Problem 5}
Partner otwiera 1\hearts. Odpowiadamy 1\spades, a partner licytuje 1\nt.
\chhand{K985}{Q2}{T8}{Q9843}
Odpowiedzi: 2\hearts\ (10p), \pass\ (5p) \\
Lepiej grać w kolor, nawet na ficie 5-2. Szczególnie, że prawdopodobnie mamy dziurę karową.


\subsection*{Problem 6}
Partner otwiera 1\diams. Odpowiadamy 1\spades, a partner licytuje 2\nt.
\chhand{A9632}{K9432}{74}{4}
Odpowiedzi: 3\hearts\ (10p) \\
Na początku zobaczmy, dlaczego konieczne jest pokazywanie \textbf{starszego} z kolorów mając 5-5.
Teraz pokazujemy kiery i po 3\hearts\ pokazaliśmy cały nasz skład. Partner nadal ma odzywkę 3\spades,
żeby sfitować piki.
Jeśli pokazalibyśmy najpierw kiery, odzywka 3\spades\ mogłaby paść również z układem 4\spades5\hearts, a 
ponadto zajęłaby więcej przestrzeni.
Dlaczego 3\hearts\ pokazało 5-5? Bo mając 5\spades4\hearts\ użylibyśmy pytania 3\clubs. Pokazawszy piątego
kiera wskazujemy też piątego pika, bo zaczęliśmy od 1\spades!


\subsection*{Problem 7}
Partner otwiera 1\clubs. Odpowiadamy 1\hearts, a partner rebiduje 1\spades.
\chhand{832}{K9743}{4}{A542}
Odpowiedzi: \pass\ (10p), 1\nt\ (7p) \\
Możemy spokojnie grać 1\spades\ na 7 atutach. Mamy fajną krótkość karo, która wyprodukuje przebitkę.
Na pewno nie należy mówić 2\hearts! Możemy trafić do renonsu. Na 2\hearts\ potrzebowalibyśmy koloru w okolicy
\hearts \textls{KQT953}.

\end{document}

