\documentclass[12pt, a4paper]{article}
\usepackage{import}

\import{../lib/}{bridge.sty}
\setmainlanguage{polish}


\title{\vspace{-2cm}System Zgrubsza}
\author{}
\date{}

\begin{document}
\section{Opis systemu}
\begin{itemize}
    \item Licytacja przebiega anty-naturalnie, tzn. licytuje się to, czego się z grubsza nie ma.
    \item Licytacja zostawia szerokie pole do improwizacji i własnej interpretacji odzywek z grubsza.
\end{itemize}
W jednostronnej licytacji naturalnej, czyli takiej, która nie wpadnie w relay'e, występują 3 fazy:
\begin{enumerate}[label=\textbf{\arabic*.}]
    \item \textbf{Faza otwarcia} - na pierwszej odzywce pokazujemy czy mamy układ i jaką siłę
    \item \textbf{Faza krótkości} - odpowiedź i rebid otwierającego pokazują krótkości lub ich brak.
    Rebid w kolor po otwarciu zrównoważonym pokazuje dubla.
    \item \textbf{Faza wyboru kontraktu} - Rebid odpowiadającego może przyjąć dwie formy:
    \subitem Odzywka inna niż 3\diams\ bez przeskoku - jest \textbf{bezwzględnie do pasa}
    \subitem Odzywka inna niż 3\diams\ z przeskokiem - pokazuje drugą krótkość i zwykle skład 5/5,
    oraz forsuje do końcówki
    \subitem Odzywka 3\diams\ - \textbf{wysoce sztuczna} - opisana poniżej
\end{enumerate}
\begin{formal}
    \textbf{Odzywka 3\diams} jako rebid odpowiadającego w licytacji \textbf{jednostronnej} ustawia \gf\ 
    i pyta partnera "Czy podoba ci się dotychczasowa licytacja?"
    Odpowiedzi:
    \begin{itemize}
        \item 3\hearts\ - Nie, rozjechaliśmy się lub mamy misfit, albo pokazuje dół otwarcia
        \item 3\spades\ - Tak, raczej mamy jakiś fit lub pokazuje górę otwarcia 
    \end{itemize}
    Następnie licytujący 3\diams\ ma do wyboru trzy opcje:
    \begin{itemize}
        \item Zalicytować jakąś końcówkę - jest bezwzględnie do pasa, nawet jeśli jest w krótkość partnera
        \item Zalicytować 4\clubs, ustalając silnie niższy z kolorów, które ma sens
        \item Zalicytować 4\diams, ustalając silnie wyższy z kolorów, które mają sens
    \end{itemize}
\end{formal}

\section{Otwarcia}
\begin{itemize}
    \item \pass\ - Z grubsza mało figur, skład równy lub z grubsza brak figur
    \item 1\clubs\ - Z grubsza mało figur, skład niezrównoważony
    \item 1\diams\ - Z grubsza średnio figur, skład równy
    \item 1\hearts\ - Z grubsza średnio figur, skład niezrównoważony
    \item 1\spades\ - Z grubsza dużo figur, skład równy, \fonce
    \item 1\nt\ - Z grubsza dużo figur, skład niezrównoważony, \fonce
    \br
    \item 2\clubs, 2\diams, 2\hearts, 2\spades\ - Z grubsza 4441, albo coś co przypomina 4441
    \item 2\nt\ - Z grubsza dwa-trzy renonse
\end{itemize}

\pagebreak
\section{Otwarcia niezrównoważone}
\subsection{Licytacja po otwarciu 1\clubs}
\begin{formal}
    1\clubs\ - Z grubsza mało figur, skład niezrównoważony
\end{formal}
\begin{itemize}
    \item \pass\ - Z grubsza nic w karcie, przydałoby się 5+\clubs, ale nie musi być, szczególnie w zielonych
    \item 1\diams\ - pokaż swoje kolory CRASHem lub zalicytuj monokolor
    \subitem 1\hearts\ - Czarne lub czerwone
    \subitem 1\spades\ - Młode lub stare
    \subitem 1\nt\ - Ostre lub okrągłe
    \subitem 2\clubs, 2\diams, 2\hearts, 2\spades\ - Z grubsza monokolorowa ręka
    \item 1\hearts\ - Ręka z grubsza na częściówkę bez \hearts
    \item 1\spades\ - Ręka z grubsza na częściówkę bez \spades
    \item 1\nt\ - Ręka z grubsza na częściówkę z co najmniej dublem w każdym kolorze
    \item 2\clubs, 2\diams, 2\hearts, 2\spades\ - \gf, krótkość licytowana
    \item 2\nt\ - \gf\ bez krótkości
    \subitem 3\clubs\ - pytanie o dubla (3\nt\ = dubel \clubs)
    \subitem 3\diams\ - stayman (3\nt\ = brak starszych czwórek)
\end{itemize}


\pagebreak
\subsection{Licytacja po otwarciu 1\hearts}
\begin{formal}
    1\hearts\ - Z grubsza średnio figur, skład niezrównoważony
\end{formal}
\begin{itemize}
    \item \pass\ - Z grubsza nic w karcie, przydałoby się 5+\hearts, ale nie musi być, szczególnie w zielonych
    \item 1\spades\ - pokaż swoje kolory CRASHem lub zalicytuj monokolor
    \subitem 1\nt\ - Czarne lub czerwone
    \subitem 2\clubs\ - Młode lub stare
    \subitem 2\diams\ - Ostre lub okrągłe
    \subitem 2\hearts, 2\spades, 3\clubs, 3\diams\ - Z grubsza monokolorowa ręka
    \item 1\nt\ - Ręka z grubsza na częściówkę z co najmniej dublem w każdym kolorze
    \item 2\clubs, 2\diams, 2\hearts, 2\spades\ - \gf, krótkość licytowana
    \item 2\nt\ - \gf\ bez krótkości
    \subitem 3\clubs\ - pytanie o dubla (3\nt\ = dubel \clubs)
    \subitem 3\diams\ - stayman (3\nt\ = brak starszych czwórek)
\end{itemize}

\pagebreak
\subsection{Licytacja po otwarciu 1\ntx}
\begin{formal}
    1\nt\ - Z grubsza dużo figur, skład niezrównoważony
\end{formal}
\begin{itemize}
    \item 2\clubs\ - Z grubsza nic w karcie, \nf.
    \subitem Na poziomie 2 lub \pass\ naturalnie z 5, minimum
    \subitem 2\nt\ - Z grubsza acolowy monokolor (3\clubs\ pytanie)
    \subitem 3\clubs, 3\diams, 3\hearts\ - z grubsza acolowa dwukolorówka CRASHem
    \item 2\diams\ - pokaż swoje kolory CRASHem lub zalicytuj monokolor, \gf
    \subitem 2\hearts\ - Czarne lub czerwone
    \subitem 2\spades\ - Młode lub stare
    \subitem 2\nt\ - Ostre lub okrągłe
    \subitem 3\clubs, 3\diams, 3\hearts, 3\spades\ - Z grubsza monokolorowa ręka
    \item 2\hearts\ - Z grubsza \gf\ z krótkością \hearts
    \item 2\spades\ - Z grubsza \gf\ z krótkością \spades
    \item 2\nt\ - \gf\ na 4333
    \item 3\clubs\ - Z grubsza \gf\ z krótkością \clubs
    \item 3\diams\ - Z grubsza \gf\ z krótkością \diams 
\end{itemize}

\pagebreak
\section{Otwarcia zrównoważone}
\subsection{Licytacja po otwarciu 1\diams}
\begin{formal}
    1\diams\ - Z grubsza średnio figur, skład zrównoważony
\end{formal}
\begin{itemize}
    \item 1\hearts\ - krótkość \hearts, \fonce
    \item 1\spades\ - krótkość \spades, \fonce
    \item 1\nt\ - Do gry
    \item 2\clubs\ - krótkość \clubs, pytanie o stare
    \subitem 2\diams\ - z grubsza minimum bez starych
    \subitem 2\nt\ - z grubsza maksimum bez starych
    \item 2\diams\ - krótkość \diams, pytanie o stare
    \subitem 2\nt\ - z grubsza minimum bez starych
    \subitem 3\clubs\ - z grubsza maksimum bez starych
    \item 2\hearts\ - pytanie o dubla, zwykle stosowane ze starszą piątką
    \item 2\spades\ - transfer na \nt
    \item 2\nt\ - \gf\ na równym
    \item 3\clubs\ - \gf\ transfer na 6+\hearts\ (+1 silne przyjęcie)
    \item 3\diams\ - \gf\ transfer na 6+\spades\ (+1 silne przyjęcie)
\end{itemize}

\pagebreak
\subsection{Licytacja po otwarciu 1\spades}
\begin{formal}
    1\spades\ - Z grubsza dużo figur, skład zrównoważony
\end{formal}
\begin{itemize}
    \item 1\nt\ - Syf w karcie, do gry.
    \item 2\clubs\ - krótkość \clubs, pytanie o stare
    \subitem 2\diams\ - z grubsza minimum bez starych
    \subitem 2\nt\ - z grubsza maksimum bez starych
    \item 2\diams\ - krótkość \diams, pytanie o stare
    \subitem 2\nt\ - z grubsza minimum bez starych
    \subitem 3\clubs\ - z grubsza maksimum bez starych
    \item 2\hearts\ - pytanie o dubla, \gf
    \item 2\spades\ - krótkość \spades, \gf\ (może mieć 4\hearts)
    \item 2\nt\ - \gf\ na równym
    \item 3\clubs\ - \gf\ transfer na 6+\hearts\ (+1 silne przyjęcie)
    \item 3\diams\ - \gf\ transfer na 6+\spades\ (+1 silne przyjęcie)
    \item 3\hearts\ - krótkość \hearts, \gf\ (może mieć 4\spades)
    \item 3\spades\ - transfer na \nt
\end{itemize}

\pagebreak
\section{Wejście przeciwnika}
\subsection{Na poziomie 1}
\begin{itemize}
    \item \dbl\ - z grubsza niezła ręka bez krótkości.
    \subitem Po otwarciu zrównoważonym - wysoki element karności
    \subitem Po otwarciu niezrównoważonym - pokazuje 3-kartowy fit w każdym innym kolorze
    \item 1\nt\ - krótkość w kolorze przeciwnika i ręka częściówkowa
    \item \textbf{kolor przeciwnika} - krótkość w kolorze przeciwnika i \gf
    \item 2\nt\ - \gf\ z niezłym trzymaniem w kolorze przeciwnika
    \item Kolory bez przeskoku - naturalne, do pasa
    \item Kolory z przeskokiem - \gf\ z krótkości
\end{itemize}

\subsection{Na poziomie 2}
\begin{itemize}
    \item \dbl\ - z grubsza niezła ręka bez krótkości.
    \subitem Po otwarciu zrównoważonym - wysoki element karności.
    \subitem Po otwarciu niezrównoważonym - pokazuje 3-kartowy fit w każdym innym kolorze.
    \item \textbf{+1 - krótkość w kolorze przeciwnika} i słaba ręka
    \item \textbf{kolor przeciwnika} - krótkość w kolorze przeciwnika i \gf
    \item 2\nt:
    \subitem Na otwarcia zrównoważone - Lebensohl
    \subitem Na otwarcia niezrównoważone - \gf\ z trzymaniem w kolorze wejścia
    \item Kolory na poziomie 3 - \gf\ z krótkości
    \item Inne kolory na poziomie 2 - naturalne, do pasa
\end{itemize}

\subsection{Wejście kontrą}
\begin{itemize}
    \item \rdbl\ - karna
    \item Inne system ON
\end{itemize}


\pagebreak
\section{Wejście do licytacji}
\begin{itemize}
    \item Na otwarcia zrównoważone (1\clubs, 1\nt) wchodzimy CRASHem od \dbl, powyżej CRASHa z krótkości lub jej braku
    \item Na otwarcia w kolor (1\diams, 1\hearts, 1\spades) wchodzimy:
    \subitem \dbl\ - z grubsza co najmniej średnio figur, bez krótkości
    \subitem Kolorem bez przeskoku z co najmniej 4/4 w pozostałych dwóch
    \subitem Kolorem z przeskokiem pokazując silną rękę z co najmniej 5/4 w pozostałych dwóch
    \subitem \textbf{Kolorem przeciwnika}, pokazując pozostałe kolory (4441+)
    \subitem 1\nt, pokazując jakiś długi kolor 
    \item Po dwóch odzywkach przeciwnika wchodzimy jak w normalnym systemie.
\end{itemize}

\section{Odzywka 3\diams\ w licytacji dwustronnej}
Odzywka ta może mieć kilka znaczeń w zależności od przebiegu wcześniejszej licytacji,
jednak zawsze ma dwie odpowiedzi: Negatywne 3\hearts\ i Pozytywne 3\spades.
\begin{itemize}
    \item Jeśli jest licytowana bezpośrednio do odzywki zrównoważonej, pyta o trzymanie w kolorze przeciwnika
    \item Jeśli jest licytowana bezpośrednio do odzywki niezrównoważonej, w której partner pokazał kolor starszy,
    pyta o piątą kartę w tym kolorze
    \item Jeśli jest licytowana w dalszym okrążeniu i nie może być naturalna, wyraża wątpliwość co do
    ogarnięcia o co chodzi w rozdaniu.
\end{itemize}

\end{document}