\documentclass[12pt, a4paper]{article}
\usepackage{import}

\import{../lib/}{bridge.sty}
\setmainlanguage{polish}

\title{Układy w Mahjongu}
\newcommand{\conc}[1]{\textbf{\color{BrickRed}Zakryta \color{Black} #1}\\}
\newcommand{\han}[1]{\\ \textbf{\color{BrickRed}\link{#1 }\color{Black}Han}}
\newcommand{\chan}[2]{\\ \textbf{\color{PineGreen}\link{#1}\color{Black}/\color{BrickRed}#2 \color{Black}Han}}

\setlength{\parindent}{0pt}
\setlength{\parskip}{\baselineskip}%

\begin{document}
    \section*{Najczęstsze}
    \conc{Riichi}
    Deklarujesz bycie 1 od wygranej. Od tej pory możesz odrzucać tylko właśnie dobrany kafelek. \han{1}

    \vspace*{1cm}

    \textbf{Smoczy Set\\}
    Układ zawiera set smoków dowolnego koloru, set wiatru rundy, lub set wiatru pozycji gracza. \han{1}

    \textbf{Tanyao\\}
    Układ składa się tylko z kafelków liczbowych od 2 do 8. \han{1}

    \conc{Pinfu}
    Układ zawiera 4 strity oraz blotkową parę. Wygrana musi nastąpić przez uzupełnienie 
    strita, do którego mogliśmy dobrać od góry lub od dołu. \han{1}

    \vspace*{1cm}

    \conc{7 par}
    Układ wyjątkowo zamiast 33332 zawiera 7 par. \han{2}

    \textbf{Outside Hand\\}
    Każdy element układu zawiera 1, 9, lub honor. \chan{1}{2}

    \textbf{Half Flush\\}
    Ręka zawiera tylko 1 kolor i honory. \chan{2}{3}

    \textbf{Terminale\\}
    Każdy element układu zawiera 1 lub 9. \chan{2}{3}


    \pagebreak
    \section*{Inne}

    \conc{Pure Double Chi}
    Układ zawiera dwa razy identyczny strit. \han{1}

    \textbf{Triple Chi \\}
    Układ zawiera po jednym stricie o tych samych wartościach w każdym kolorze. \chan{1}{2}

    \textbf{Pure Straight \\}
    Układ zawiera 123, 456 i 789 jednego koloru. \chan{1}{2}

    \conc{Double Pure Double Chi}
    Układ zawiera dwa razy po dwa identyczne strity. \han{3}

    \textbf{Triple Pung \\}
    Układ zawiera set o identycznej wartości w każdym kolorze. \han{2}

    \textbf{Triple Concealed Pung \\}
    Układ zawiera trzy zamknięte sety. Czwarty element może być otwarty. \han{2}

    \textbf{All Pung \\}
    Układ zawiera cztery sety. \han{2}

    \textbf{Little Three Dragons \\}
    Układ zawiera dwa sety smoków i parę smoków. \han{2}

    \textbf{Full Flush \\}
    Układ zawiera tylko jeden kolor, bez honorów. \chan{4}{5}


    \pagebreak
    \section*{Dodatkowe}
    \begin{table}[h!]
        \centering
        \begin{tabular}{p{11cm}p{2cm}}
            Wygrana na ostatnim możliwym do dobrania kafelku lub ostatniej zrzutce & \textbf{1 Han} \\[2em]
            Wygrana na dodatkowym kafelku dodanym po zgłoszeniu czwórki & \textbf{1 Han} \\[2em]
            Wygrana na kafelku dobranym zaraz po ulepszeniu trójki do czwórki & \textbf{1 Han} \\[2em]
            Czerwone kafelki i kafelki premiowe danej rundy, za każdy & \textbf{1 Han} \\[2em]
        \end{tabular}
    \end{table}

    \section*{Mini Punkty}
    \begin{table}[h!]
        \centering
        \begin{tabular}{rc}
            Wygrana & 20 \\
            Wygrana przy dobraniu z muru & +2 \\
            Zakryta ręka & +10 \\
            7 par & 25 \\[7mm]
            Strit & 0 \\
            Set blotek & +2/+4 \\
            Set terminali lub honorów & +4/+8 \\
            Czwórka blotek & +8/+16 \\
            Czwórka terminali lub honorów & +16/+32 \\
            Para smoków, wiatru gracza lub rundy & +2 \\[7mm]
            Oczekiwanie na 12\textbf{(3)} lub \textbf{(7)}89 & +2 \\
            Oczekiwanie na środkowy & +2 \\
            Oczekiwanie na parę & +2 \\
        \end{tabular}
    \end{table}
\end{document}