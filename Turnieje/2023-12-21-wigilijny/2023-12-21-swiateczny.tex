\documentclass[12pt, a4paper]{article}
\usepackage{import}

\import{../lib/}{bridge.sty}
\setmainlanguage{polish}

\title{\vspace{-3cm}Turniej wigilijny 2023}
\date{21.12.2023}

\begin{document}
\subsection*{Rozdanie 1}
    \begin{center}
        \hspace*{-12mm}%
        \vhand{QT62}{8}{AK863}{J43} \\
        \vhand{85}{T93}{T7}{AK9762}%
        \begin{minipage}{3cm}%
            \centering
            \vspace{-5mm}
            \nvul{N} \\[4mm]
            \nvul{W} \ \ \ \textbf{\large1} \ \ \ \nvul{E} \\[4mm]
            \nvul{S}
        \end{minipage}%
        \vhand{94}{KQ642}{Q954}{T5} \\
        \hspace*{-7mm}%
        \vhand{AKJ73}{AJ75}{J2}{Q8}
    \end{center}

    \subsubsection*{Proponowana licytacja}
    \begin{table}[h!]
        \centering
        \begin{tabular}{cccc}
            \nvul{W} & \nvul{N} & \nvul {E} & \nvul{S} \\
            - & - & - & 1\spades \\
            \pass & 3\spades & \pass & 4\spades \\
            \pass & \pass & \pass
        \end{tabular}
    \end{table}

    \subsubsection*{Analiza}
    S otwiera 1\spades. Z pozycji N należy się inwit do końcówki - karta ma wprawdzie tylko 10 punktów,
    jednak mamy 4 piki i singla - będzie łatwo brać przebitki.

    W wistuje w \clubs A. E powinien dołożyć \clubs 5, aby wskazać partnerowi,
    że chce kontynuacji trefli. Możemy łatwo wziąć przebitkę, jeśli rozgrywający ma 3 trefle!
    
    Rozgrywający bije trzeciego trefla atutem. Liczymy przegrywające:
    \spades: 0, \hearts: 3, \diams: 0, \clubs: 2. Musimy pozbyć się jak największej ilości.
    Próba przebicia wszystkich 3 kierów wymaga jednak powrotu do naszej ręki 3 razy przed ściągnięciem atutów.
    Do komunikacji możemy użyć kar, ale trzeba pamiętać, żeby przebić trzecie wysoko!
    Zatem:

    (1-3) \clubs A, K, x przebita (4) \hearts A, (5) \hearts 5 przebita,
    (6-7) \diams A, K, (8) \diams 3 przebita \spades J, (9) \hearts 7 przebita,
    (10) \diams 6 przebita \spades K, (11) \hearts J przebity \spades T, (12) \spades Q, (13) \spades A.
    Wynik: +1. 



    \pagebreak
    \subsection*{Rozdanie 2}

    \begin{center}
        \hspace*{-12mm}%
        \vhand{-}{QJT98}{AT8}{KT943} \\
        \vhand{QJT65}{K54}{J}{8765}%
        \begin{minipage}{3cm}%
            \centering
            \vspace{-5mm}
            \vul{N} \\[4mm]
            \nvul{W} \ \ \ \textbf{\large2} \ \ \ \nvul{E} \\[4mm]
            \vul{S}
        \end{minipage}%
        \vhand{AK84}{76}{KQ9}{AQJ2} \\
        \hspace*{-7mm}%
        \vhand{9732}{A32}{765432}{-}
    \end{center}

    \subsubsection*{Proponowana licytacja}
    \begin{table}[h!]
        \centering
        \begin{tabular}{cccc}
            \nvul{W} & \vul{N} & \nvul {E} & \vul{S} \\
            - & - & 1\clubs & \pass \\
            1\spades & 2\hearts & 4\spades & \pass \\
            \pass & \pass &  &
        \end{tabular}
    \end{table}

    \subsubsection*{Analiza}
    Wistujemy w \hearts Q. Oczywiście w otwarte karty widać, że treflowy jest lepszy,
    ale raczej trudno jest na to wpaść. 
    S widzi, że wistujący nie ma \hearts K. Zatem ma go W. Jeśli rozgrywający weźmie tą lewę,
    kontrakt jest już niemalże pewny. Patrząc na dziadka, odda \hearts A, \diams A i zapisze +1.
    Musimy poszukać szybkiej szansy przebitkowej.

    \hearts Q należy zabić asem i odegrać w karo - \diams A u partnera jest pewny, on 
    coś na wejście musi mieć, a z resztą to jedyna szansa na dojście. Ale skąd partner ma wiedzieć,
    żeby grać w trefle? Należy wyjść \diams 2, wskazując najmłodszy kolor!

    Jeśli nie weźmiemy przebitki, rozgrywający ma 11 lew, gdyż trefle wyrzuci na wyrobione \diams KQ.
    Na maksy to różnica rzędu kilkudziesięciu procent!



    \pagebreak
    \subsection*{Rozdanie 3}

    \begin{center}
        \hspace*{-12mm}%
        \vhand{T982}{AQJ3}{Q75}{AQ} \\
        \vhand{Q65}{8542}{T93}{JT2}%
        \begin{minipage}{3cm}%
            \centering
            \vspace{-5mm}
            \nvul{N} \\[4mm]
            \vul{W} \ \ \ \textbf{\large3} \ \ \ \vul{E} \\[4mm]
            \nvul{S}
        \end{minipage}%
        \vhand{AKJ7}{76}{J86}{K976} \\
        \hspace*{-7mm}%
        \vhand{43}{KT9}{AK42}{8543}
    \end{center}

    \subsubsection*{Proponowana licytacja}
    \begin{table}[h!]
        \centering
        \begin{tabular}{cccc}
            \vul{W} & \nvul{N} & \vul {E} & \nvul{S} \\
            - & - & - & \pass \\
            \pass & 1\nt & \pass & 3\nt \\
            \pass & \pass & \pass 
        \end{tabular}
    \end{table}

    \subsubsection*{Analiza}
    Wistujemy w \spades A. Partner, posiadając figurę, powinien nam dorzucić \spades 5! 
    Po odczytaniu zrzutki należy kontynuować \textbf{małym} pikiem. W ten sposób odblokujemy kolor i weźmiemy 4 lewy.

    Rozgrywający liczy lewy: \spades: 0, \hearts: 4, \diams: 3. \clubs: 1. Brakuje nam zatem jednej.
    Jakie mamy szanse? Impas trefl lub kara 3-3. Należy najpierw sprawdzić kara, gdyż
    daje nam to możliwość sprawdzenia obydwu szans, jeśli pierwsza nie zachodzi.



    \pagebreak
    \subsection*{Rozdanie 4}

    \begin{center}
        \hspace*{-12mm}%
        \vhand{7}{KQ986}{A643}{973} \\
        \vhand{AKQ3}{T74}{KJT9}{A6}%
        \begin{minipage}{3cm}%
            \centering
            \vspace{-5mm}
            \vul{N} \\[4mm]
            \vul{W} \ \ \ \textbf{\large4} \ \ \ \vul{E} \\[4mm]
            \vul{S}
        \end{minipage}%
        \vhand{T964}{32}{Q5}{KQJT8} \\
        \hspace*{-7mm}%
        \vhand{J852}{AJ5}{872}{542}
    \end{center}

    \subsubsection*{Proponowana licytacja}
    \begin{table}[h!]
        \centering
        \begin{tabular}{cccc}
            \vul{W} & \vul{N} & \vul {E} & \vul{S} \\
            1\nt & \pass & 2\clubs & \pass \\
            2\spades\ & \pass & 3\spades* & \pass \\
            4\spades & \pass &\pass & \pass
        \end{tabular}
    \end{table}

    * Tu można się pokusić o 4\spades od razu jeśli wierzymy w umiejętności rozgrywkowe partnera.

    \subsubsection*{Analiza}
    Wistujemy w \hearts K. Partner powinien dołożyć \hearts 5. Kontynuujemy kiera, 
    a za trzecim razem rozgrywający przebija.
    Wydaje się, że kontrakt jest z góry. Ściągamy atuty, N nie dokłada jednak do drugiego pika.
    Musimy zatem wyimpasować waleta u S.
    
    Jeśli jednak zrobimy to natychmiast - przegramy, gdyż pozbędziemy się atutów,
    a musimy oddać jeszcze \diams A, po czym obrońcy ściągną wszystkie kiery.
    
    Zachodzi zatem konieczność zachowania tempa - musimy najpierw oddać karo, 
    grając \diams 9 do \diams Q, a dopiero potem ściągnąć atuty.



    \pagebreak
    \subsection*{Rozdanie 5}

    \begin{center}
        \hspace*{-12mm}%
        \vhand{KQJ784}{AKQ}{Q97}{T} \\
        \vhand{T2}{9}{A32}{Q986542}%
        \begin{minipage}{3cm}%
            \centering
            \vspace{-5mm}
            \vul{N} \\[4mm]
            \nvul{W} \ \ \ \textbf{\large5} \ \ \ \nvul{E} \\[4mm]
            \vul{S}
        \end{minipage}%
        \vhand{A6}{JT654}{KJT8}{J3} \\
        \hspace*{-7mm}%
        \vhand{953}{8732}{654}{AK7}
    \end{center}

    \subsubsection*{Proponowana licytacja}
    \begin{table}[h!]
        \centering
        \begin{tabular}{cccc}
            \nvul{W} & \vul{N} & \nvul {E} & \vul{S} \\
            -    & 1\spades & \pass & 2\spades \\
            \pass & 4\spades & \pass & \pass \\
            \pass &  &  & 
        \end{tabular}
    \end{table}

    \subsubsection*{Analiza}
    Wistujemy w \hearts J. Wist treflowy jest słabą alternatywą,
    gdyż mamy dużo punktów - partner raczej nie ma dobrych trefli.
    Partner dokłada \hearts 9, a rozgrywający bije asem. 
    Co się dzieje w kierach? Brakuje nam A, K, Q, 9. Jeśli partner miałby figurę,
    powinien przejąć waleta. Zatem z dużym prawdopodobieństwem 9 jest singlowa!

    Rozgrywający liczy przegrywające. \spades: 1, \hearts: 0, \diams: 3, \clubs: 0.
    Jedno z kar możemy wyrzucić na \clubs AK i należy zrobić to natychmiast,
    zanim obrona ściągnie karo! Wydaje się, że kontrakt jest górny.

    Po dojściu asem pik E odchodzi do przebitki, technicznie \hearts 10,
    wskazując \diams. W ściąga asa i gra karo do króla. Wynik: -1.


    \pagebreak
    \subsection*{Rozdanie 6}

    \begin{center}
        \hspace*{-12mm}%
        \vhand{A}{Q86}{KQT953}{KT5} \\
        \vhand{KJ}{KJ7}{A86}{AQ976}%
        \begin{minipage}{3cm}%
            \centering
            \vspace{-5mm}
            \nvul{N} \\[4mm]
            \vul{W} \ \ \ \textbf{\large6} \ \ \ \vul{E} \\[4mm]
            \nvul{S}
        \end{minipage}%
        \vhand{QT9753}{AT93}{J7}{8} \\
        \hspace*{-7mm}%
        \vhand{8642}{542}{42}{J432}
    \end{center}

    \subsubsection*{Proponowana licytacja}
    \begin{table}[h!]
        \centering
        \begin{tabular}{cccc}
            \vul{W} & \nvul{N} & \vul {E} & \nvul{S} \\
            -    & -     & \pass & \pass \\
            1\clubs & 1\diams & 1\spades & \pass \\
            2\nt & \pass & 4\spades & \pass \\
            \pass & \pass
        \end{tabular}
    \end{table}

    \subsubsection*{Analiza}
    W otwiera 1\clubs\ z zamiarem rebidu 2\nt, pokazującego 18-20 PC.
    Po 2\nt, E powinien od razu skoczyć w 4\spades. 2\nt\ forsuje do
    końcówki, a gdy jesteśmy sforsowani, ze słabą kartą licytujemy szybko.

    S wistuje w kolor partnera \diams 2. E liczy przegrywające:
    \spades: 1, \hearts: 1, \diams: 1, \clubs: 0.
    Aby pozbyć się niektórych, można trafić impas kier, lub jeśli impas 
    trefl wychodzi, wyrzucić karo. Druga opcja jest jednak mało prawdopodobna, 
    gdyż N licytował. Na początku ściągamy atuty.

    N bije asem i gra \diams K, \diams 3. Tu W powinien przebić dziewiątką! Po wiście w dwójkę 
    widać co się dzieje w rozdaniu.

    Następnie rozgrywający ściąga atuty i impasuje damę kier u N. Wynik: swoje.



    \pagebreak
    \subsection*{Rozdanie 7}

    \begin{center}
        \hspace*{-12mm}%
        \vhand{Q54}{AKQ6}{74}{AQ83} \\
        \vhand{J763}{T932}{K92}{76}%
        \begin{minipage}{3cm}%
            \centering
            \vspace{-5mm}
            \vul{N} \\[4mm]
            \vul{W} \ \ \ \textbf{\large7} \ \ \ \vul{E} \\[4mm]
            \vul{S}
        \end{minipage}%
        \vhand{982}{754}{Q63}{T942} \\
        \hspace*{-7mm}%
        \vhand{AKT}{J8}{AJT85}{KJ5}
    \end{center}

    \subsubsection*{Proponowana licytacja}
    \begin{table}[h!]
        \centering
        \begin{tabular}{cccc}
            \vul{W} & \vul{N} & \vul {E} & \vul{S} \\
            -    & -     & -    & 1\nt \\
            \pass & 2\clubs & \pass & 2\diams \\
            \pass & 6\nt & \pass & \pass \\
            \pass & & &
        \end{tabular}
    \end{table}

    \subsubsection*{Analiza}
    Czasami dostajemy na linii bardzo dużo punktów. Partner otwiera 1\nt, a my mamy aż 17!
    W sumie z partnerem - między 32 a 34. Wykonajmy proste obliczenia:
    W talii jest 40 punktów i 13 lew - średnio 3 punkty na lewę! Zatem z naszych punktów 
    wychodzi nam 11 lew.

    Jednak lewę może wziąć też na przykład \hearts 6, albo \clubs 3. Te kolory to 
    \textbf{źródła lew}, dzięki którym mozemy dodać sobie lewę do naszego bilansu. Wychodzi nam
    12 lew, zatem 6\nt!

    Na turnieju często widziałem licytacię 3\clubs\ od N - próba znalezienia fitu trefl to świetny 
    pomysł! Tutaj jednak powinno to pokazywać pięciokart.





    \pagebreak
    \subsection*{Rozdanie 8}

    \begin{center}
        \hspace*{-12mm}%
        \ \vhand{4}{AKQ87}{K8543}{87} \\
        \vhand{AK875}{T5}{Q62}{T92}%
        \begin{minipage}{3cm}%
            \centering
            \vspace{-5mm}
            \nvul{N} \\[4mm]
            \nvul{W} \ \ \ \textbf{\large8} \ \ \ \nvul{E} \\[4mm]
            \nvul{S}
        \end{minipage}%
        \vhand{QJT6}{94}{AJT}{AJ54} \\
        \hspace*{-7mm}%
        \vhand{932}{J632}{97}{KQ63}
    \end{center}

    \subsubsection*{Proponowana licytacja}
    \begin{table}[h!]
        \centering
        \begin{tabular}{cccc}
            \nvul{W} & \nvul{N} & \nvul {E} & \nvul{S} \\
            -    & 1\hearts & \dbl & 2\hearts \\
            2\spades & 3\hearts & \pass & \pass \\
            3\spades & \pass & \pass & \pass \\
        \end{tabular}
    \end{table}

    \subsubsection*{Analiza}
    Pierwsza kluczowa decyzja w licytacji następuje w 
    drugim okrążeniu z ręką N. Partner pokazał fit, a nasze figury kierowe
    nie wezmą lew, jeśli przeciwnicy będą grali w piki.
    Ta karta aż prosi się o grę własną. Dlatego należy zawalczyć do 3\hearts.

    To samo powinien pomyśleć W, do którego licytacja zbiega.
    Ma niepokazanego piątego atuta okas AK, które przeciwnicy w 3\hearts\ 
    przebiją kierami. Dlatego dajemy 3\spades.


\end{document}