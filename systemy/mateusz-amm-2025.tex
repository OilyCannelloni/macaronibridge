\documentclass[12pt, a4paper]{article}
\usepackage{import}
\usepackage[left=50px,right=50px,top=100px,bottom=100px,paperwidth=8in,paperheight=50in]{geometry}


\import{../lib/}{bridge.sty}
\setmainlanguage{polish}

\title{\vspace{-2cm}AMM 2025}
\date{}
\author{Bartek Słupik \and Mateusz Francik}


\begin{document}
\maketitle

\section*{1\clubs}

\sequence{{1\clubs}}
\begin{options}[2]
	\item[1\diams\alrt] 0-6 \orr 7-10 bez 4M \orr 7-11 młode  
	\item[1\nt\alrt] \inv do 3\nt \vimp 
	\item[2\clubs\alrt] \gf na treflach, karach lub składzie równym \vimp
	\item[2\diams\alrt] MULTI 0-8 w zależnoci od założeń \vimp
	\item[2\hearts\alrt] Flannery
	\item[2\spades\alrt] Transfer na NT 
	\item[2\nt\alrt] Śmieciowy blok na treflach 0-5 nie dający dużej nadzieji na 3\nt do 18-20 \imp
	\item[3\clubs] Lepszy blok na treflach ok 6-9
\end{options}


\sequence{{1\clubs}{1\diams}}
\begin{options}[1]
	\item[1\hearts\alrt] Może być z dubla
	\item[2\diams\alrt] ACOL na treflach lub treflach i innym \imp
	\item[2\hearts] NAT ale limitowane do acola \nf
	\item[2\spades] jw
	\item[2\nt\alrt] silna ręka \clubs+\diams, ale gorsze niż acol
\end{options}


\sequence{{1\clubs}{2\clubs}}
\begin{options}[1]
	\item[2\diams\alrt] Skład zrównoważony, słabe lub silne. Może mieć starszą 4. \vimp
	\item[2\hearts] 5+\clubs 4\hearts
	\item[2\spades] 5+\clubs 4\spades
	\item[2\nt] nieużywane
	\item[3\diams] 5\clubs 4\diams
	\item[3\hearts] krótkość (tak jak po otw 1\nt)
	\item[3\spades] krótkość
\end{options}


\vspace*{1cm}
\section*{1\diams}

\sequence{{1\diams}}
\begin{options}[2]
	\item[2\diams] 10+ z 4-kartowym fitem. Forsuje do 3\diams.
	\item[3\clubs\alrt] Śmieciowy blok na karach 0-5 \imp
	\item[3\diams] Lepszy blok na karach 6-9
	\item[2\hearts\alrt] Flannery
	\item[2\spades\alrt] Transfer na NT 
\end{options}



\vspace*{1cm}
\section*{1\hearts}
\sequence{{1\hearts}}
\begin{options}[2]
	\item[2\spades\alrt] Mini Splinter 9-11PC \gf (2\nt\ \lsf) \imp
	\item[2\nt\alrt] Inwit z fitem
	\item[3\clubs\alrt] 9-11 z 4-fitem bez krótkości
	\item[3\diams\alrt] Mixed raise 6-9
	\item[3\spades] Splinter 12-15PC
	\item[3\nt\alrt] Spl \diams
	\item[4\diams] nieużywane
\end{options}


\vspace*{1cm}
\section*{1\spades}
\sequence{{1\spades}}
\begin{options}[2]
	\item[2\nt\alrt] Mini Splinter (3\clubs\ \lsf)
	\item[3\hearts\alrt] Inwit z fitem
	\item[3\nt] Spl \hearts
	\item[4\hearts] Do gry \imp
\end{options}



\vspace*{1cm}
\section*{Podniesienie koloru}
\sequence{{1\clubs}{1\hearts}{2\hearts}}
\begin{options}[2]
	\item[2\spades] \lsf
\end{options}

\sequence{{1\clubs}{1\hearts}{2\hearts}{2\spades}}
\begin{options}[2]
	\item[2\nt] brak krótkości \then 3\clubs = pytanie o dubla
	\item[3\clubs] krótkość niska
	\item[3\diams] krótkość wysoka
\end{options}

\sequence{{1\clubs}{1\hearts}{2\hearts}{2\spades}{2\ntx}{3\clubs}}
\begin{options}[2]
	\item[3\diams] brak dubla (4333)
	\item[3\hearts] dubel low
	\item[3\spades] dubel mid
	\item[3\nt] dubel high
	\item[4\clubs] 5422
\end{options}



\sequence{{1\diams}{1\hearts}{2\hearts}}
\begin{options}[2]
	\item[2\spades] \lsf
\end{options}

\sequence{{1\diams}{1\hearts}{2\hearts}{2\spades}}
\begin{options}[2]
	\item[2\nt] brak krótkości
	\item[3\clubs] krótkość niska
	\item[3\diams] krótkość wysoka
\end{options}


\sequence{{1\clubs}{1\hearts}{3\hearts}}
\begin{options}[2]
	\item[3\spades] \lsf
\end{options}



\vspace*{1cm}
\section*{1\nt}
\sequence{{1\ntx}}
\begin{options}[2]
	\item[2\spades] \inv lub \clubs. \then 3\hearts, 3\spades = \clubs\ + krótkość
	\item[2\nt] \diams lub \minor (wybierz lepszy) \then 3\diams = do gry, 3\hearts, 3\spades = \diams\ + krótkość
	\item[3\clubs] Puppet
	\item[3\diams] \inv NAT
	\item[3\hearts] krótkość
	\item[3\spades] krótkość
	\item[4\clubs] słabe na starych
\end{options}



\vspace*{1cm}
\section*{2\clubs}
\sequence{{2\clubs}}
\begin{options}[2]
	\item[2\diams] automat
\end{options}


\sequence{{2\clubs}{2\diams}}
\begin{options}[1]
	\item[2\hearts] \hearts lub 25+BAL \then 2\spades automat
	\item[2\nt] 23-24 BAL
	\item[3\clubs\alrt] Acol na \diams z starszą czwórką \then 3\diams ASK + ustalenie cue-bidem
	\item[3\diams\alrt] Acol na \diams bez bocznej czwórki \then 3\hearts, 3\spades = nat 5+
	\item[3\hearts] Samoustalenie
	\item[3\spades] Samoustalenie
	\item[3\nt] Młode 5+/5+
\end{options}


\sequence{{2\clubs}{2\diams}{2\hearts}{2\spades}}
\begin{options}[2]
	\item[2\nt] 25+ BAL
	\item[3\clubs] 5\hearts i \clubs
	\item[3\diams] 5\hearts i \diams
	\item[3\hearts] 6+\hearts
\end{options}


\vspace*{1cm}
\section*{2\diams Multi}
Siła od X do 8PC, gdzie w korzystnych X = 0 itd, w zielonych można taktycznie z piątki.

\sequence{{2\diams}}
\begin{options}[2]
	\item[2\nt] ASK \invp
	\item[3\clubs] GF na młodym
	\item[3\diams] \invp na ,,drugim`` starym \imp
\end{options}


\vspace*{1cm}
\section*{2\hearts\alrt\ 9-12 6+\hearts}
\sequence{{2\hearts}}
\begin{options}[2]
	\item[2\spades] 5+\spades\ \invp
	\item[2\nt] ASK \invp \then jeśli OTW pokaże dół to 3\hearts jest do gry
	\item[3\diams] Inwit z fitem
	\item[3\hearts] blok
\end{options}

\sequence{{2\hearts}{2\ntx}}
\begin{options}[1]
	\item[3\clubs] Dół z krótkością \then 3\diams\ \lsf\ niska/średnia/wysoka
	\item[3\diams] Góra z krótkością \then 3\hearts\ \lsf\ niska/średnia/wysoka
	\item[3\hearts] Dół bez krótkości
	\item[3\spades] 6\hearts 4\spades
	\item[3\nt] Góra bez krótkości 
\end{options}


\vspace*{1cm}
\section*{2\spades\alrt\ 9-12 6+\spades}
\sequence{{2\spades}}
\begin{options}[2]
	\item[2\nt] ASK \invp \then jeśli OTW pokaże dół to 3\spades jest do gry
	\item[3\clubs\alrt] 5+\hearts\ \invp \vimp
	\item[3\hearts] Inwit z fitem
	\item[3\spades] blok
\end{options}

\sequence{{2\spades}{2\ntx}}
\begin{options}[1]
	\item[3\clubs] Dół z krótkością \then 3\diams\ \lsf\ niska/średnia/wysoka
	\item[3\diams] Góra z krótkością \then 3\hearts\ \lsf\ niska/średnia/wysoka
	\item[3\hearts] 6\spades 4\hearts
	\item[3\spades] Dół bez krótkości
	\item[3\nt] Góra bez krótkości
\end{options}

\sequence{{2\spades}{3\clubs}}
\begin{options}[1]
	\item[3\diams\alrt] Dubel \hearts
	\item[3\hearts] Fit i śmieciowa karta
	\item[3\spades] 0-1 \hearts i śmieć, \nf
	\item[3\nt] 0-1 \hearts i przyjęcie inwitu
\end{options}



\section*{Bloki}

\begin{itemize}
	\item 3\clubs\diams W zielonych może być z 6 waleta, w czerwonych koniecznie 7 As lub Król
	\item 3\hearts\spades Może być z 6 z kartą za bardzo układową na multi lub taktycznie 
	\item 3\nt\ Chcę otworzyć 4\hearts lub 4\spades, ale karta jest nieco za silna. Około równowartość 10-12PC zależnie od założeń. 4\clubs ASK transfer jak na multi
\end{itemize}


\section*{Szlemiki}
\begin{itemize}
	\item Na pytanie o asy odpowiadamy \textbf{bez króli} - tylko 4 odpowiedzi.
	\item Na pytanie o damę odpowiadamy \textbf{bez króli} - nie/tak.
	\item Kolorowe Króle - na pytanie o króle +1 = król \clubs lub 2 pozostałe, +2 = król \diams lub 2 pozostałe, wrzucenie 6M = brak króli
	\br
	\item Na kolorze młodszym 4\nt jest zawsze do gry, chyba że w jakiejś dzikiej dwustronnej to może być Last Train. \textbf{Nigdy nie jest o asy.}
	\item Pytaniem o asy na \clubs jest 5\diams, a na \diams - 5\hearts.
	\item W licytacjach splinterowych, jeśli został tylko 1 cue-bid to jest on Last Trainem
	\br
	\item Niepoważne 3\spades i 3\nt - na kolorach starszych w 2/1 pokazujemy układ rewersowo bez nadwyżek.
	\item Mając 2 Asy i Damę nie można dać zjazdu
	\item Odzywka Serious forsuje cuebid od partnera
\end{itemize}






\section*{Przeciwnik wchodzi}

\compsequence{{1\clubs}{\dbl}} \\ System ON (2\nt śmieciowy blok itp)

\compsequence{{1\diams}{\dbl}} \\ System ON tylko 2\nt = dobra ręka z fitem \diams, a 2\diams jest słabe

\compsequence{{1\hearts}{\dbl}}
\begin{compoptions}[3]
	\item[1\nt] Konstruktywne z fitem (jak 2\hearts w jednostronnej) \imp
	\item[2\clubs...] \nf
	\item[2\nt] \invp z fitem \imp
\end{compoptions}

\compsequence{{1\spades}{\dbl}} \\ Jak wyżej


\compsequence{{1\clubs}{1\hearts}}
\begin{compoptions}[3]
	\item[\dbl] Bez pików
	\item[2\hearts] \invp bez trzymania lub duża karta z fitem \clubs
\end{compoptions}

\compsequence{{1\diams}{1\hearts}}
\begin{compoptions}[3]
	\item[\dbl] Bez pików
	\item[2\hearts] \invp bez trzymania lub duża karta z fitem \diams
\end{compoptions}

\compsequence{{1\hearts}{1\spades}}
\begin{compoptions}[3]
	\item[2\spades] \gf z fitem
	\item[2\nt] \inv z fitem
\end{compoptions}


\compsequence{{1\ntx}{2\clubs}}
\begin{compoptions}[3]
	\item[\dbl] 8+PC karna
	\item[2\diams] do gry
	\item[2\hearts] do gry
	\item[2\spades\alrt] MŁODE \vimp
	\item[2\nt] lebensohl
	\item[3\clubs] GF
\end{compoptions}


\compsequence{{1\ntx}{2\diams}}
\begin{compoptions}[3]
	\item[\dbl] 8+PC karna
	\item[2\hearts] do gry
	\item[2\spades] do gry
	\item[2\nt] lebensohl
	\item[3\clubs] GF
\end{compoptions}

\compsequence{{1\ntx}{2\hearts}}
\begin{compoptions}[3]
	\item[\dbl] 3+PC negatywna NIE \invp
	\item[2\spades] do gry
	\item[2\nt] lebensohl
	\item[3\hearts] krótkość \hearts
\end{compoptions}


\section*{Przeciwnik otwiera}

Na 1\nt po polsku

\subsection*{Obrona na Multi}

\compsequence{{2\diams\alrt}}
\begin{compoptions}[2]
	\item[\dbl] 14-16 BAL
	\item[2\nt] 17-19 BAL \then stayman transfery
\end{compoptions}

\compsequence{{2\diams\alrt}{\dbl}{\rdbl}}
\begin{compoptions}[4]
	\item[2\hearts] do gry
	\item[2\spades] do gry
	\item[2\nt] lebensohl (słabe na \clubs/\diams lub inwit na starym)
	\item[3\clubs] Stayman
	\item[3\diams] transfer \gf
	\item[3\hearts] transfer \gf
\end{compoptions}



\compsequence{{2\diams\alrt}{P}{2\hearts\alrt}}
\begin{compoptions}[4]
	\item[\dbl] takeout do \hearts
	\item[2\nt] 15-17 BAL \then stayman transfery
\end{compoptions}

\compsequence{{2\diams\alrt}{P}{2\spades+}}
\begin{compoptions}[4]
	\item[\dbl] takeout do tego co RHO licytuje
\end{compoptions}



\subsection*{Nasze wejście kolorem}

\compsequence{{1\diams}{1\spades}{P}}
\begin{compoptions}[4]
	\item[2\clubs\alrt] Drury \imp
	\item[2\diams\alrt] Dobra ręka bez fitu
	\item[2\nt] Naturalne (13-14)
\end{compoptions}


\compsequence{{1\diams}{1\spades}{[coś]}}
\begin{compoptions}[4]
	\item[2\clubs] Naturalne \imp
	\item[2\diams\alrt] (kolor otwarcia) Dobra ręka z fitem \imp
	\item[2\nt] Świetna ręka z 4-karotwym fitem (bo mamy \dbl lub \rdbl) \vimp
\end{compoptions}






\end{document}