\documentclass[12pt, a4paper]{article}
\usepackage{import}
\usepackage[paperwidth=21cm,paperheight=500cm,margin=1in]{geometry}


\import{../lib/}{bridge.sty}
\setmainlanguage{polish}

\title{\vspace{-2cm}Lista ważnych sekwencji w systemie ,,GŁĄB''}
\date{}
\author{Bartek Słupik \and Krystyna Gasińska}


\begin{document}
\maketitle






\section*{Otwarcia}
\vspace*{0.8cm}
\begin{options}
    \item[1\clubs] Strefowy trefl, może zawierać 5\diams w składzie zrównoważonym
    \item[1\diams] 5+\diams, 11-22, skład niezrównoważony
    \br
    \item[2\clubs] Acol
    \item[2\diams] Multi, 5/6\major 0-8 w zależności od założeń \\
                    Na 3 ręce: Naturalny blok 5+\diams \\
                    Na 4 ręce: 6+\diams, 8-13
    \item[2\hearts] 6+\hearts 9-12 \\
                    Na 3 ręce: Naturalny blok 5+\hearts
    \item[2\spades] 6+\spades 9-12 \\
                    Na 3 ręce: Naturalny blok 5+\spades
\end{options}


\section*{1\clubs}
\sequence{{1\clubs}}
\begin{options}[2]
    \item[1\diams] 0-6 dowolny (oprócz innych odp), \\
    7-10 BAL, może mieć 4\major jeśli bardzo \nt ręka \\
    7+ co najmniej 5-4 \minor
    \item[1\nt] \inv do 3\nt
    \item[2\clubs] \gf na 5+\clubs, 5+\diams lub składzie równym, nie daje możliwości uzgodnienia fitu 4-4\major
    \item[2\diams] Multi, 0-8
    \item[2\hearts] 5\spades, 4+\hearts, 4-9
    \item[2\spades] Blok na \clubs \\
                    \inv na \clubs \\
                    Układowy \gf na \clubs z krótkością   
    \item[2\nt] Blok na \minor co najmniej 5/4, 0-8
    \item[3\clubs] Mixed raise, (5)6+\clubs
    \item[3\diams] \inv, 6+\diams   
\end{options}

\subsection*{1\diams = Pseudo Negat}
\sequence{{1\clubs}{1\diams}}
\begin{options}[1]
    \item[1\hearts] 2+\hearts
    \item[1\spades] 4\spades
    \item[2\diams] Acol na 5+\clubs i bocznej czwórce lub długich \clubs, \gf
    \item[2\hearts] 19-22 5\clubs 4\hearts, \nf
    \item[2\nt] 19-22 5\clubs, 4\diams  
\end{options}

\sequence{{1\clubs}{1\diams}{1\hearts}}
\begin{options}[2]
    \item[1\spades] Transfer na \nt \\
                      9-11 z 5\clubs lub 5\diams (wyniesienie 1\nt w 2\minor)
    \item[1\nt] 4\hearts \vimp
    \item[2\minor] 0-8 NAT
    \item[2\major] \inv na młodych 5-4 z krótkością  (2\nt bez)
\end{options}

\sequence{{1\clubs}{1\diams}{1\spades}}
\begin{options}[2]
    \item[2\hearts] \minor i krótkość \hearts (nat dajemy multi)
    \item[2\spades] NAT
    \item[2\nt] \minor bez krótkości \hearts
\end{options}

\sequence{{1\clubs}{1\diams}{2\diams}}
\begin{options}[2]
    \item[2\hearts] ASK (odp 2\spades NAT, wyżej TRSF)
\end{options}

\subsection*{1\ntx\ = \inv BAL}
\sequence{{1\clubs}{1N}}
\begin{options}[1]
    \item[2\clubs] do gry
    \item[2\diams] Stayman
    \item[2\major] krótkość licytowana 
\end{options}

\subsection*{2\clubs = \gf na \clubs, \diams lub równym}
\sequence{{1\clubs}{2\clubs}}
\begin{options}[1]
    \item[2\diams] Skład zrównoważony
    \item[2\major] 5\clubs, 4\major (schemat 2\nt do \diams)
    \item[2\nt] 5\clubs, 4\diams (3m ustala i ASK LSF n/l/h)
    \item[3\clubs] 6+\clubs (3\diams ASK LSF n/l/m/h), nie ustala! ODP może mieć \diams
    \item[3\diams, 3\major] 4441 z licytowaną krótkością
    \item[3\nt] 4441\clubs  
\end{options}

\sequence{{1\clubs}{2\clubs}{2\diams}}
\begin{options}[2]
    \item[2\hearts] ASK
\end{options}

\sequence{{1\clubs}{2\clubs}{2\diams}{2\hearts}}
\begin{options}[1]
    \item[2\spades] Co najwyżej 4/3 w młodych (2\nt ASK - odp 3\minor NAT)
    \item[2\nt] Dokładnie 4/4 w młodych
    \item[3\minor] 5+\minor (5332)  
\end{options}


\subsection*{2\hearts Flannery}
\sequence{{1\clubs}{2\hearts}}
\begin{options}[1]
    \item[2\nt] ASK \gf
    \item[3\clubs] do gry
    \item[3\diams] \inv na punkty
\end{options}

\sequence{{1\clubs}{2\hearts}{2N}}
\begin{options}[2]
    \item[3\clubs] 5-5+ (3\diams ASK LSF l/h/rl/rh)
    \item[3\diams] brak krótkości
    \item[3\hearts] krótkość \clubs
    \item[3\spades] krótkość \diams  
\end{options}

\subsection*{2\spades BlokoInwitoSplinterka}
\sequence{{1\clubs}{2\spades}}
\begin{options}[1]
    \item[2\nt] góra (zajmuję NT z BAL ręki)
    \item[3\clubs] dół
    \item[3X] krótkość, F \then 4\clubs
\end{options}


\subsection*{Rebid 2\clubs}
\sequence{{1\clubs}{1\spades}{2\clubs}}
\begin{options}[2]
    \item[2\diams] \gf
    \item[2\hearts] \inv na 5/4+
    \item[2\spades] 6+\spades 9-10
    \item[2\nt] \inv\ \nf   
\end{options}



\vspace{3cm}
\section*{1\diams}

\sequence{{1\diams}}
\begin{options}
  \item[1\nt] 4-10 bez 4\major
  \item[2\clubs] GF \clubs, \diams lub równy
  \item[2\diams] 10-12 bez 4\major, może mieć tylko 2\diams \vimp
  \item[2\major] NAT blok 7-9
  \item[2\nt] BlokoInwitoSplinterka, blok na \diams, inwit z fitem lub \gf z krótkością
  \item[3\clubs] \inv na \clubs
  \item[3\diams] Mixed raise
  \item[3\major] Splinter słabszy niż przez 2\nt     
\end{options}

\subsection*{2\ntx\ BlokoInwitoSplinterka}
\sequence{{1\diams}{2N}}
\begin{options}
  \item[3\clubs] Góra
  \item[3\diams] Dół 
\end{options}

\subsection*{Gazilli 1\ntx}
\sequence{{1\diams}{1\spades}{1N\alrt}}
\begin{options}[1]
    \item[2\clubs] 8+
    \item[2\diams] do gry
    \item[2\hearts] 5/5 \major\ \fonce (2\spades po 1\hearts = słabe z jakimś singlem karo)   
    \item[2\spades] 5\spades, 5\clubs\ \nf (2\hearts po 1\hearts = 5\hearts, 5\clubs)
\end{options}

\sequence{{1\diams}{1\spades}{1N\alrt}{2\clubs}}
\begin{options}[1]
    \item[2\diams] 5+\diams, 4\hearts słabe
    \item[2\hearts] 3\spades, \gf 
    \item[2\spades] 4\spades, \gf  
    \item[2\nt] 4\clubs (3\minor ustala i \lsf\ n/l/h)
    \item[3\clubs] 6+\diams (3\diams\ \lsf\ n/l/h!! - z 6/4\clubs dajemy 2\nt od razu, nie ma krt \hearts) \imp 
    \item[3\diams] dużo dobrych \diams
    \item[3\hearts] 5/5 \minor, krótkość
    \item[3\spades] 5/5 \minor, krótkość
    \item[3\nt] 5\diams 332\spades    
\end{options}





\vspace{3cm}
\section*{1\hearts}


\sequence{{1\hearts}{1\spades}{2\clubs}{2\diams}}
\begin{options}
  \item[2\spades] 3\spades (2\nt = \lsf\ n/l/h)
  \item[2\nt] 4+\clubs (3\clubs = \lsf\ n/l/h)
  \item[3\clubs] 4+\diams
  \item[3\diams] 6\hearts (tylko brak krt lub krt \spades, bo nie ma 6-4! 3\hearts\ \lsf\ n/\spades)
  \item[3\hearts] dużo \hearts
  \item[3\spades] =4522 
  \item[3\nt] =5\hearts 332\spades 
\end{options}

\sequence{{1\hearts}{1\spades}{2N}}
\begin{options}
  \item[3\clubs] ASK (\clubs, \diams)
  \item[3\hearts] \nf 
\end{options}


\vspace{3cm}
\section*{1\spades}
\sequence{{1\spades}{1N}{2\clubs}{2\diams}}
\begin{options}
  \item[2\nt] 4+\clubs (3\clubs = \lsf\ n/l/h)
  \item[3\clubs] 4+\diams
  \item[3\diams] 3\hearts \vimp 
  \item[3\hearts] 6\spades (3\hearts\ \lsf\ n/l/m/h)
  \item[3\spades] dużo \spades
  \item[3\nt] =5\spades 332\hearts 
\end{options}
 


\section*{Transfery}


\subsection*{1\clubs}
Tak jak po otwarciu 1\nt

\subsection*{1\diams}
\compsequence{{1\diams}{2\clubs}}
\begin{compoptions}[3]
  \item[\dbl] neg 4/3
  \item[2\hearts, 2\spades] \fonce (wyższe niz kolor otw są \fonce)
  \item[2\nt] weak \clubs \orr \gf\ \clubs
  \item[3\clubs] \inv \diams 
  \item[3\diams, 3\major] preempt 
\end{compoptions}

\compsequence{{1\diams}{2\hearts}}
\begin{compoptions}[3]
  \item[\dbl] neg 4
  \item[2\spades] \fonce 
  \item[2\nt] weak \clubs \orr \gf\ \clubs
  \item[3\clubs] \inv \diams \orr \then 3\nt
  \item[3\diams, 3\spades] preempt 
\end{compoptions}

\compsequence{{1\diams}{2\spades}}
\begin{compoptions}[3]
  \item[\dbl] neg 4
  \item[2\nt] weak \clubs \orr \gf\ \clubs \orr \inv \diams
  \item[3\clubs] \invp\ 5+\hearts
  \item[3\diams] weak 
  \item[3\hearts] GF 6+ 
  \item[3\spades] krótkość 
\end{compoptions}


\compsequence{{1\diams}{3\clubs}}
\begin{compoptions}[3]
  \item[\dbl] neg 4/3+\major
  \item[3\diams] weak
  \item[3\hearts] 5+\spades\ \invp
  \item[3\spades] 5+\hearts\ \gf
  \item[4\minor] super/hyper (4\diams = krótkość \clubs!)    
\end{compoptions}  


\compsequence{{1\diams}{3\hearts}}
\begin{compoptions}[3]
  \item[\dbl] 5+\spades\ \invp
  \item[3\spades] 4\spades    
  \item[4\minor] super/hyper (4\diams = krótkość \hearts!)   
\end{compoptions}  


\compsequence{{1\diams}{3\spades}}
\begin{compoptions}[3]
  \item[\dbl] 4-5\hearts  
  \item[4\minor] super/hyper (4\diams = krótkość \spades!)   
\end{compoptions}  






\subsection*{1\major}


\compsequence{{1\hearts}{2\clubs}}
\begin{compoptions}[3]
  \item[\dbl] neg 4
  \item[2\diams] \nf 
  \item[2\spades] \fonce 
\end{compoptions}

\compsequence{{1\hearts}{2\spades}}
\begin{compoptions}[3]
  \item[\dbl] neg
  \item[2\nt] weak \clubs \orr \gf \clubs
\end{compoptions}



\compsequence{{1\hearts}{3\clubs}}
\begin{compoptions}[3]
  \item[\dbl] neg 4-5\spades (lub 6 śmieciowych)
  \item[3\diams] \invp\ \hearts. Wszystkie takie inwity+ tworzą PF ponad końcówką i 
  są limitowane przez 4\clubs Super.
  \item[3\hearts] weak
  \item[3\spades] \then \nt \orr \diams   
  \item[4\minor] super/hyper 
\end{compoptions}  



\compsequence{{1\hearts}{3\diams}}
\begin{compoptions}[3]
  \item[\dbl] neg 4-5\spades
  \item[3\hearts] weak 
  \item[3\spades] \then \nt \orr \clubs   
  \item[4\minor] super/hyper 
\end{compoptions}

  

\compsequence{{1\spades}{3\clubs}}
\begin{compoptions}[3]
  \item[\dbl] neg 4\hearts \orr \then \nt (po kontrze 3\spades) \orr \diams
  \item[3\diams] \invp\ \hearts
  \item[3\hearts] \invp\ \spades
  \item[3\spades] weak   
  \item[4\clubs] super/hyper 
\end{compoptions}

\compsequence{{1\spades}{3\diams}}
\begin{compoptions}[3]
  \item[\dbl] neg 4-5\hearts \orr \then \nt (po kontrze 3\spades) \orr \clubs
  \item[3\hearts] \invp\ \spades
  \item[3\spades] weak   
  \item[4\minor] super/hyper
\end{compoptions}

\compsequence{{1\spades}{3\hearts}}
\begin{compoptions}[3]
  \item[\dbl] neg
  \item[3\spades] weak
  \item[4\minor] super/hyper 
\end{compoptions}

\subsubsection*{Reagowanie na kontry}
\compsequence{{1\hearts}{3\diams}{\dbl}{P}}
\begin{compoptions}[1]
  \item[\pass] pamiętaj, bierzemy zapis zamiast ryzykować rozjazdu
  \item[3\hearts] 2-\spades, raczej brak trzymania, nic o extra kierach
  \item[3\spades] 3+\spades
  \item[3\nt] gramy
  \item[4\clubs] natural
  \item[4\diams] 4\spades i spora nadwyżka
  \item[4\hearts] dużo \hearts, ale minimum      
  \item[4\spades] 4\spades
\end{compoptions}

\compsequence{{1\spades}{3\diams}{\dbl}{P}}
\begin{compoptions}[1]
  \item[\pass] pamiętaj, bierzemy zapis zamiast ryzykować rozjazdu
  \item[3\hearts] 3+\hearts
  \item[3\spades] 2-\hearts, dupa wołowa
  \item[3\nt] gramy
  \item[4\clubs] natural
  \item[4\diams] 4\spades i spora nadwyżka
  \item[4\hearts] propozycja z niezłej 4   
  \item[4\spades] dużo \spades
\end{compoptions}



\end{document}
