\documentclass[12pt, a4paper]{article}
\usepackage{import}


\makeatletter
\ExplSyntaxOn
\clist_new:N\l_gtext_First_clist
\clist_new:N\l_gtext_Last_clist
\int_new:N\l_gtext_MaxIndex_int
\int_new:N\l_gtext_Ratio_int
\newcommand{\gr@dient}[8]{
  \int_set:Nn\l_gtext_MaxIndex_int{\int_eval:n{\str_count:n{#1}}}
  \int_step_inline:nnn{1}{\l_gtext_MaxIndex_int}{
      \exp_args:Ne\str_if_eq:nnTF{\str_item:Nn{#1}{##1}}{~}{}{
        \int_set:Nn\l_gtext_Ratio_int{\int_eval:n{\l_gtext_Ratio_int+1}}
      }
        \color_select:nn{#8}{
          \int_eval:n{(\int_use:N\l_gtext_Ratio_int*#5+(\l_gtext_MaxIndex_int-##1)*#2)/\l_gtext_MaxIndex_int},
          \int_eval:n{(\int_use:N\l_gtext_Ratio_int*#6+(\l_gtext_MaxIndex_int-##1)*#3)/\l_gtext_MaxIndex_int},
          \int_eval:n{(\int_use:N\l_gtext_Ratio_int*#7+(\l_gtext_MaxIndex_int-##1)*#4)/\l_gtext_MaxIndex_int}
      }\str_item:Nn{#1}{##1}
  }
}

\NewDocumentCommand\gradient{mmmm}{{
  \clist_set:Nn\l_gtext_First_clist {#3}
  \clist_set:Nn\l_gtext_Last_clist {#4}
  \gr@dient{#2}
  {\clist_item:Nn\l_gtext_First_clist{1}}
  {\clist_item:Nn\l_gtext_First_clist{2}}
  {\clist_item:Nn\l_gtext_First_clist{3}}
  {\clist_item:Nn\l_gtext_Last_clist{1}}
  {\clist_item:Nn\l_gtext_Last_clist{2}}
  {\clist_item:Nn\l_gtext_Last_clist{3}}
  {#1}
}}
\ExplSyntaxOff

\NewDocumentCommand\gradientRGB{mmm}{
  \gradient{RGB}{#1}{#2}{#3}
}
\NewDocumentCommand\gradientHSB{mmm}{
  \gradient{HSB}{#1}{#2}{#3}
}
\makeatother

\newcommand{\rainbow}[1]{\!\!\textbf{\gradientHSB{#1}{0,240,200}{240,240,200}}}


\import{../lib/}{bridge.sty}
\setmainlanguage{polish}

\title{\vspace{-2cm}System Naturalny ,,PIETRASZ''}
\date{}
\author{Bartek Słupik \and Marta Pietraszek}


\begin{document}
\maketitle

\section*{Założenia}

\begin{itemize}
    \item Otwarcia 1\clubs, 2\clubs, 2\diams oraz odpowiedzi 1\clubs - 1\diams  i 1\clubs - 2\diams są sztuczne.
    \item Pozostała licytacja jest całkowicie \textbf{naturalna} zgodnie z poprawną techniką naturalnego bilansowania.
    Kolory z piątki, potem z czwórki.
    \item Licytacja szybka, do celu pokazuje słabą kartę. Np 1\hearts -- 4\hearts zalicytujemy z kartą słabe 12pc z 3-kartowym fitem.
    \item Licytacja wolniejsza pokazuje znaczące nadwyżki, np 1\hearts -- 2\clubs -- 2\diams -- 4\hearts to jakieś marne 14pc.
\end{itemize}

\section*{Otwarcia}
\vspace*{0.8cm}
\begin{options}
    \item[1\clubs] Skład równy (4441, 5\xdiams332) 12-14 lub 18-20 \orr NAT \clubs
    \item[1\diams] \xdiams, skład nierówny 
    \item[1\hearts] \hearts
    \item[1\spades] \spades
    \item[1\nt] 15-17 skład w miarę równy
    \item[2\clubs] Acol
    \item[2\diams] \rainbow{WILKOSZ} = Dowolne 5-5 oprócz 5-5 na młodych, 0-9 PC
\end{options}

\pagebreak

\section*{Odpowiedzi}
\sequence{{1\clubs}}
\begin{options}[2]
    \item[1\diams\alrt] Nie mam starszej piątki ale coś w karcie jest \vimp 
    \item[1\hearts] 5+\hearts
    \item[1\spades] 5+\spades
    \item[1\nt] Gramy
    \item[2\clubs] 5+\clubs\ \gf
    \item[2\diams] \rainbow{WILKOSZ} 5-5 oprócz młodych
    \item[2\hearts] 6+\hearts\ blok
    \item[2\spades] 6+\spades\ blok
    \item[2\nt] inwit     
\end{options}

\sequence{{1\diams}}
\begin{options}[2]
    \item[1\hearts] (4)5+\hearts
    \item[1\spades] (4)5+\spades
    \item[1\nt] Brak starszej piątki, brak kar
    \item[2\clubs] 5+\clubs\ \gf
    \item[2\diams] jakiś syf, może być bez kar, ale gramy
    \item[2\hearts] 6+\hearts blok
    \item[2\spades] 6+\spades blok
    \item[2\nt] inwit     
\end{options}

\sequence{{1\hearts}}
\begin{options}[2]
    \item[1\spades] (4)5+\spades
    \item[1\nt] Gramy
    \item[2\clubs] \gf SZTUCZNY
    \item[2\diams] \gf 
    \item[2\hearts] \hearts
    \item[2\nt] inwit
    \item[3\hearts] inwit
    \item[3\nt] Gramy
    \item[4\hearts] Gramy        
\end{options}

\pagebreak
\section*{Rebidy}
\sequence{{1\clubs}{1\hearts}}
\begin{options}[1]
    \item[2\nt] 18-20 \nf
    \item[3\hearts] Dużo pkt, 3+\hearts (z trójki, partner ma 5)
\end{options}

\sequence{{1\diams}{1\hearts}}
\begin{options}[1]
    \item[2\nt] 18-20 F do 3\diams
    \item[3\hearts] Dużo pkt, 3+\hearts (z trójki, zakładasz że partner ma 5)
\end{options}

\section*{Wejścia}

\compsequence{{(1\clubs)}}
\begin{compoptions}[2]
    \item[2\diams] \rainbow{WILKOSZ}
\end{compoptions}

\section*{Strefa szlemowa}
\begin{itemize}
    \item \textbf{Nie gramy sigma pytaniem o asy, bo to sztuczna konwencja}
    \item 5\major = inwit do szlemika
\end{itemize}



\end{document}