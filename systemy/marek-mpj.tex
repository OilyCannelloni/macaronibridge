\documentclass[12pt, a4paper]{article}
\usepackage{import}

\import{../lib/}{bridge.sty}
\setmainlanguage{polish}

\title{\vspace{-2cm}MPJ 2024}
\date{}
\author{Bartek Słupik \and Marek Dzierżawa}


\begin{document}
\maketitle

\section{Negat}
\begin{itemize}
    \item W negacie upychamy ręce 7-10 bez starszej czwórki. Gram tak na co dzień i to działa świetnie.
    \item Trochę bardziej złożony relay 1\hearts daje lepszą kontrolę nad bilansem, kosztem niezagrania 1\hearts na
    ficie 4-3 lub 4-4.
    \item Negat zawiera GF na karach, ale nie zawiera GFa na równym.
\end{itemize}

\vspace{0.5cm}
\subsubsection*{Odpowiedzi na 1\clubs}
\sequence{{1\clubs}}
\begin{options}[2]
    \item[1\diams] 0-6 ANY \orr 7-10 BAL bez 4M \orr \gf\ \diams 
    \item[1\nt] 11-12 \inv do 3\nt \imp 
    \item[2\clubs] \gf\ 5+\clubs (2\diams = BAL może mieć 4M) 
    \item[2\diams] Multi (0-8) w zależności od założeń \imp  
    \item[2\hearts] Flannery \textbf{4-9} 
    \item[2\spades] Transfer \nt
    \item[2\nt] \gf 15+ BAL, może mieć 4M \vimp
    \item[3\clubs] 7-9, 6+\clubs (10-11 przez 2\spades i wyniesienie 2\nt)
    \item[3\diams] 10-11, 6+\diams
    \item[3\nt] 13-15 BAL 
\end{options}

\sequence{{1\clubs}{1\ntx}}
\begin{options}[1]
    \item[2\clubs] Do gry
\end{options}

\sequence{{1\clubs}{2\ntx}}
\begin{options}[1]
    \item[3\clubs] Stayman
    \item[3\diams] 6+\clubs bez krótkości
    \item[3\hearts] 5+\clubs, krótkość
    \item[3\spades] 5+\clubs, krótkość   
\end{options}

\subsubsection*{Negat dalej}
\sequence{{1\clubs}{1\diams}}
\begin{options}[1]
    \item[1\hearts] dowolna ręka w miarę BAL do 14PC \vimp
    \item[1\spades] 15+ 5+\clubs \vimp
    \item[1\nt] 18-20 BAL
    \item[2\clubs] 11-14 6\clubs  
    \item[2\diams] acol \clubs | acol \clubs\diams \imp
\end{options}

\sequence{{1\clubs}{1\diams}{1\hearts}}
\begin{options}[2]
    \item[\pass] 5+\hearts
\end{options}

\sequence{{1\clubs}{1\diams}{1\spades}}
\begin{options}[2]
    \item[1\nt] Brak fitu \clubs, 0-7PC
    \item[2\clubs] 3+\clubs, 0-7PC
    \item[2\diams] Do gry, 0-7PC
    \item[Inne] NAT, pamiętając, że jest tu silny wariant GFa na \diams
\end{options}



\pagebreak
\section{Sekwencje z młodymi}
\begin{itemize}
    \item 1\diams --- 2\diams forsuje do 3\diams
    \item 1\diams --- 1\major --- 2\diams jest zawsze 6+\diams (co powoduje, że 1\diams - 1\spades - 1\nt może
    mieć singla \spades)
    \item Z ręką typu 1-3-4-5 lub 3-1-3-6 można zalicytować 1\clubs --- 1\hearts --- 2\hearts z trójki.
    \item Tak samo 1\diams --- 1\major --- 2\major może być z 3 jeśli ma krótkość
\end{itemize}

\subsection*{Rebid młodego}
\sequence{{1\clubs}{1\hearts}{2\clubs}}
\begin{options}[2]
    \item[2\diams] \gf
    \item[2\hearts] 9-11, 6+\hearts\ \inv (bo multi!)
    \item[2\spades] ASK stop
    \item[2\nt] \inv\ \nf (z 6\clubs poprawiamy!)
    \item[3\clubs] \inv\ \nf
\end{options}

\sequence{{1\clubs}{1\spades}{2\clubs}}
\begin{options}[2]
    \item[2\diams] \gf
    \item[2\hearts] 10-11, 4\hearts 5\spades \nf
    \item[2\spades] 9-11, 6+\spades\ \inv (bo multi!)
\end{options}

\sequence{{1\diams}{1\hearts}{2\diams}}
\begin{options}[2]
    \item[2\hearts] 9-11, 6+\hearts\ \inv
    \item[2\spades] ASK stop
    \item[2\nt] \invp, forsuje do 3\diams! (3\clubs góra, 3\diams dół)  \imp
\end{options}

\sequence{{1\diams}{1\spades}{2\diams}}
\begin{options}[2]
    \item[2\hearts] 10-11, 4\hearts 5\spades\ \nf
    \item[2\spades] 9-11, 6+\spades\ \inv
    \item[2\nt] \invp, forsuje do 3\diams! (3\clubs góra, 3\diams dół)  \imp
\end{options}


\subsection*{Karo coś 2 trefl}
\sequence{{1\diams}{1\hearts}{2\clubs}}
\begin{options}[2]
    \item[2\diams] do 9-10 PC, z 16+ OTW powinien licytować dalej
    \item[2\hearts] 9-11 6+\hearts\ \inv \imp
    \item[2\spades] ASK stop
    \item[2\nt] \gf
    \item[3\clubs] \inv
    \item[3\diams] \inv      
\end{options}

\sequence{{1\diams}{1\spades}{2\clubs}}
\begin{options}[2]
    \item[2\diams] do 9-10 PC, z 16+ OTW powinien licytować dalej
    \item[2\hearts] 10-11 4\hearts 5\spades\ \inv
    \item[2\spades] 9-11 6+\spades \inv \imp
\end{options}

\subsection*{Otwarcie 1\diams}
\sequence{{1\diams}}
\begin{options}[2]
    \item[1\nt] 6-10
    \item[2\clubs] \gf\ 5+\clubs (4M) lub 3\diams 
    \item[2\diams] Inverted, forsuje do 3\diams 
    \item[2\hearts] Flannery 4-9
    \item[2\spades] Transfer \nt, forsuje do 3\clubs \imp
    \item[2\nt] \inv   
    \item[3\clubs] blok na \diams 
    \item[3\diams] mixed 
\end{options}


\sequence{{1\diams}{2\clubs}}
\begin{options}[1]
    \item[2\diams] brak 4\major
    \item[2\major] NAT bez nadwyżek
\end{options}

\sequence{{1\diams}{2\clubs}{2\diams}}
\begin{options}[2]
    \item[2\major] 5+\clubs 4\major
    \item[2\nt] 13-15 lub 18+ BAL (w domyśle ustala karo)
    \item[3\nt] 16-17 BAL
\end{options}


\sequence{{1\diams}{2\diams}}
\begin{options}[1]
    \item[2\hearts] Wartość
    \item[2\spades] Wartość
    \item[2\nt] NAT
    \item[3\clubs] NAT \imp  
    \item[3\diams] Słaby syf z długimi \diams bez przyszłości
    \item[3\major] SPL
\end{options}



\pagebreak
\section{Rebid 2\nt}
\subsection*{Po treflu}

\sequence{{1\clubs}{1\spades}{2\ntx}}
\begin{options}[2]
    \item[3\clubs] ASK o fit \spades lub o trefle
    \item[3\diams] ASK inwitujący grę w karo
    \item[3\hearts] 6+\hearts  
\end{options}

\sequence{{1\clubs}{1\spades}{2\ntx}{3\clubs}}
\begin{options}[1]
    \item[3\diams] 3-kartowy fit \spades \vimp
    \item[3\hearts] dubel \spades i 4+\clubs
    \item[3\spades] 4-kartowy fit \spades 
    \item[3\nt] nic z powyższych 
\end{options}

\sequence{{1\clubs}{1\spades}{2\ntx}{3\clubs}{3\diams}}
\begin{options}[2]
    \item[3\hearts] daj 4\clubs, jeśli masz 4 trefle \vimp
    \item[3\spades] ustalenie
\end{options}

\sequence{{1\clubs}{1\spades}{2\ntx}{3\diams}}
\begin{options}[1]
    \item[3\hearts] 3-kartowy fit \spades
    \item[3\spades] 4-kartowy fit \spades 
    \item[3\nt] nic z powyższych 
    \item[4\minor] cue na \diams (trzeba dać jak się ma 4\diams) 
\end{options}

\subsection*{Po karze}
\sequence{{1\diams}{1\spades}{2\ntx}{3\clubs}}
\begin{options}[1]
    \item[3\diams] 3-kartowy fit \spades
    \item[3\hearts] 6\diams
    \item[3\spades] 4-kartowy fit \spades 
\end{options}


\section{Otwarcie 2\clubs}

\sequence{{2\clubs}}
\begin{options}[2]
    \item[2\diams] automat
\end{options}

\sequence{{2\clubs}{2\diams}}
\begin{options}[1]
    \item[2\hearts] \hearts \orr \hearts\ + inny \orr 25+ \bal \qquad (2\spades -- automat)
    \item[2\spades] \spades (2\nt -- automat + transfery)
    \item[2\nt] 23-24 BAL
    \item[3\clubs] \diams \vimp
    \item[3\diams] \diams\ + 4\major \vimp
    \item[3\major] samoustalenie     
    \item[3\nt] młode 
\end{options}

\sequence{{2\clubs}{2\diams}{2\hearts}{2\spades}}
\begin{options}[1]
    \item[2\nt] 25+ BAL
    \item[3\clubs] transfer na 4+\diams (z dublem \hearts -- dajemy 3\hearts bo 6-4)
    \item[3\diams] transfer na 6+\hearts (zwykle 6331)
    \item[3\hearts] transfer na 4\spades   
    \item[3\spades] transfer na 4+\clubs 
\end{options}

\sequence{{2\clubs}{2\diams}{3\clubs}}
\begin{options}[2]
    \item[3\diams] ASK krótkość (powyżej 3NT ustalone \diams)
    \item[3\major] NAT, 5+ 
\end{options}

\sequence{{2\clubs}{2\diams}{3\diams}}
\begin{options}[2]
    \item[3\hearts] = 4\hearts (4\clubs ustala!)
    \item[3\spades] = 4\spades, brak 4\hearts 
\end{options}


\pagebreak
\section{Inne ustalenia}

\sequence{{1\clubs}{1\hearts}{2\hearts}}
\begin{options}[2]
    \item[2\spades] ASK 3/4 \gf
\end{options}

\sequence{{1\clubs}{1\hearts}{2\hearts}{2\spades}}
\begin{options}[1]
    \item[2\nt] 4\hearts (3\clubs = \lsf brak / niska / wysoka)
    \item[3\clubs] 3\hearts, krótkość niska (\diams)
    \item[3\diams] 3\hearts, krótkość wysoka (\spades)
\end{options}

\sequence{{1\clubs}{1\spades}{3\spades}}
\begin{options}[2]
    \item[3\nt] \lsf
\end{options}


\pagebreak
\section{Strefa szlemikowa}
\begin{itemize}
    \item 4\nt jest \textbf{do gry} jeśli nie mamy fitu, lub mamy \textbf{fit w młodym}
    \item 5\diams jest o asy na \clubs
    \item 5\hearts jest o asy na \diams
    \item 5\spades jest o asy na \hearts, jeśli 4\nt nie zostało użyte
\end{itemize}

\end{document}