\documentclass[12pt, a4paper]{article}
\usepackage{../lib/bridgetex2} 
\usepackage{polyglossia}
\setmainlanguage{polish}

\title{Jedno w stary}
\author{Bartek Słupik}

\begin{document}
    \maketitle
    \section{Założenia}
    1\major\ = 11+, 5+\major. Z 5/5 otwieramy 1\spades.

    \section{1\spades\ --- ?}
    \begin{itemize}
        \item 1\nt\ - \emph{Śmieciowe BezAtu} - 5-11 bez fitu na składzie dowolnym. Dalsza licytacja opisana w oddzielnym pliku.
        \br
        \item 2\clubs\ - \emph{Two Over One} - 13+, dalsza licytacja w oddzielnym pliku.
        \item 2\diams\ - 5+\diams, \gf
        \item 2\hearts\ - 5+\hearts, \gf
        \br
        \item 2\spades\ - 7-10 z fitem. Konstruktywne.
        \item 2\nt\ - Inwit siłowy (11-12) \textbf{z fitem} zwykle 3-kartowym.
        \br
        \item 3\clubs\ - Układowy inwit \textbf{z fitem} \spades\ oraz 5+\clubs
        \item 3\diams\ - Układowy inwit \textbf{z fitem} \spades\ oraz 5+\diams
        \item 3\hearts\ - Ukladowy inwit \textbf{z fitem} \spades\ oraz 5+\hearts
        \item 3\spades\ - \emph{Mixed Raise} - 7-10 z 4-kartowym fitem
        \br
        \item 3\nt\ - do gry \emph{z fitem}, propozycja grania w NT mimo fitu
    \end{itemize}
    \section{Jak zapamiętać?}
    \begin{itemize}
        \item Obowiązuje Two Over One - odpowiedzi na poziomie 2 w inny kolor są \gf
        \item Wszystkie odzywki powyżej 2\nt\ włącznie są \textbf{z fitem} i znaczą \textbf{to, co widać}:
        \subitem NT oznacza mały fit
        \subitem Kolor oznacza kolor
        \subitem Podniesienie oznacza duży fit
    \end{itemize}
\end{document}

