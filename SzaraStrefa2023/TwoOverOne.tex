\documentclass[12pt, a4paper]{article}
\usepackage{../lib/bridgetex2} 
\usepackage{polyglossia}
\setmainlanguage{polish}

\title{Two Over One}
\author{Bartek Słupik}

\begin{document}
    \maketitle
    \section{O licytacji sforsowanej do końcówki}
    Niektórym odzywkom przypisuje się własność \textbf{forsowania do końcówki} (\gf),
    gdyż znacznie upraszcza to system. Licytacja przebiega wtedy naturalnie i według kilku prostych zasad:
    \begin{itemize}
        \item Nie wolno spasować przed końcówką.
        \item Licytacja \textbf{szybka}, w szczególności wrzucenie końcówki od razu, oznacza \textbf{minimalną, martwą kartę},
        która nie daje nadziei na znalezienie szlemika.
        \item Licytacja \textbf{powolna} zwykle wskazuje \textbf{nadwyżki}, gdyż dążymy do
        wymiany informacji, w celu znalezienia potencjalnego szlemika.
        \item Skład licytujemy naturalnie, pokazując np. kolory z czwórki
        \item Jeśli przeciwnik zdecyduje się wejść do licytacji, nie dajemy mu grać bez kontry lub gramy sami.
    \end{itemize}

    \section{Założenia}
    Odpowiedź na otwarcie 1 w kolor kolorem na poziomie 2 jest \gf\ i pokazuje:
    \begin{itemize}
        \item \textbf{Dowolny skład}, możliwy też fit, jeśli było to 2\clubs
        \item \textbf{5+} w kolorze, jeśli odpowiedź była \textbf{bez przeskoku}, np. 1\hearts\ --- 2\diams
        \item \textbf{6+} w kolorze, jeśli odpowiedź była \textbf{z przeskokiem}, np. 1\diams\ --- 2\spades
    \end{itemize}
    
    \section{Przykłady}
    \subsection{Rebidy}
    \begin{itemize}
        \item 1\spades\ - 2\clubs\ --- 2\diams\ = licytowana czwórka
        \item 1\spades\ - 2\clubs\ --- 2\spades\ = wydłużenie (6+\spades)
        \item 1\spades\ - 2\clubs\ --- 2\nt\ = brak czwórki i wydłużenia (5332)
        \item 1\hearts\ - 2\clubs\ --- 2\spades\ = licytowana czwórka, \textbf{Nie obiecuje nadwyżek!}
        \item 1\hearts\ - 2\diams\ --- 3\clubs\ = licytowana czwórka, \textbf{Nie obiecuje nadwyżek!}
        \item 1\spades\ - 2\diams\ --- 3\diams\ = pokazanie fitu, ale nie trzeba tego robić, 
        jeśli mamy np 6-tego pika to lepiej pokazać go najpierw.
    \end{itemize}

    \subsection{Objaśnianie 2\clubs}
    \begin{itemize}
        \item 1\spades\ - 2\clubs\ --- 2\diams\ - 2\hearts\ = licytowana czwórka, 5+\clubs
        \item 1\spades\ - 2\clubs\ --- 2\diams\ - 2\spades\ = \textbf{sfitowanie} pików, silna karta
        \item 1\spades\ - 2\clubs\ --- 2\diams\ - 2\nt\ = śmietnik bez trefli ani fitu
        \item 1\spades\ - 2\clubs\ --- 2\diams\ - 3\clubs\ = 6+\clubs
    \end{itemize}

    \pagebreak
    \section{Jak zbilansować się przed szlemikiem?}
    Jeśli sfitowaliśmy kolor starszy w licytacji \gf, musimy mieć możliwość \textbf{sprawdzenia szans na szlemika}.
    Nie pokazujemy siły wcześniej, więc musimy wymyślić coś innego.

    \subsection{Powaga sytuacji}
    Dzielimy ręce na \textbf{cztery kategorie} pod względem siły:
    \begin{itemize}
        \item Ręce słabe, niedające szans na szlemika - zwykle u otwierającego będzie to 11-13 HCP
        oraz ręka niemająca nadwyżek układowych ani asów/dobrych atutów. Zamykamy licytację poprzez 4\major.
        \item Ręce lekko zachęcające do zagrania szlemika, 
        tzw. \textbf{niepoważne} - około 14-16 HCP, 
        biorąc dużą poprawkę na układ i ładność. Licytujemy \emph{niepoważne} 3\nt
        \item Ręce mocno inwitujące szlemika, tzw. \textbf{poważne} które potrzebują niewiele od partnera.
        Licytujemy \emph{cue-bida}.
        \item Ręce przesądzające szlemika.
    \end{itemize}

    Wniosek:
    \begin{formal}
        \itshape
        Jeśli spełnione są trzy warunki:
        \begin{itemize}
            \item Licytacja jest \gf
            \item Mamy ustalony fit w kolorze \textbf{starszym}
            \item Jesteśmy poniżej lub na wysokości 3 w nasz kolor
        \end{itemize}
        wtedy:
        \begin{itemize}
            \item 3\nt\ jest lekkim inwitem do szlemika
            \item Odzywki w kolory to cue-bidy, mocno inwitują szlemika
        \end{itemize}
    \end{formal}

    \pagebreak
    \section{Jak nie grać 3\ntx\ bez zatrzymań?}

    Mając fit w kolorze starszym używaliśmy 3\nt\ jako odzywki bilansującej,
    gdyż na pewno nie chcemy grać takiego kontraktu. W kolorze młodszym jest inaczej - 
    3\nt\ zwykle jest \textbf{jedyną dobrą końcówką}. Zatem wprowadzamy ustalenie:
    \begin{formal}
        \itshape
        Jeśli wiemy, że nie mamy fitu w kolorze starszym, wszystkie odzywki 
        poniżej 3\nt\ mają na celu ustalenie, czy jest to dobry kontrakt.
    \end{formal}

    3\nt\ wolno nam zalicytować wyłącznie, jeśli wiemy o trzymaniu w każdym kolorze.
    Np. w licytacji
    {\webidding{
        1\hearts & 2\clubs \\
        3\clubs & ?
    }}
    kolejne odzywki miałyby następujące znaczenie:
    \begin{itemize}
        \item 3\nt\ = do gry
        \item 3\diams\ = Trzymam \diams\, ale nie licytuję 3\nt --- nie trzymam \spades!
        \item 3\spades\ = Trzymam \spades\, ale nie trzymam \diams!
    \end{itemize}
    \begin{formal}
        \itshape
        Jeśli licytacja zmierza ku 3\nt, licytowanie kolorów oznacza trzymanie w tym kolorze
        oraz problem w innym - bo z trzymaniem we wszystkim mówimy 3\nt.
    \end{formal}

    \subsection{Szlemiki w młode}
    Jeśli mamy ustalony fit w kolorze młodszym, \textbf{przekroczenie} poziomu 3\nt\ obiecuje
    duże nadwyżki. Dopiero wtedy zaczynamy zastanawiać się nad szlemikiem - nigdy poniżej 3\nt!
\end{document}

