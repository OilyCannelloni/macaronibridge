\documentclass[12pt, a4paper]{article}
\usepackage{import}

\import{../lib/}{bridge.sty}
\setmainlanguage{polish}

\title{\vspace{-2cm}Taka strefa, że ćmy latają}
\author{}
\date{}

\begin{document}
\maketitle
\section{Założenia}
\begin{itemize}
    \item 1\clubs\ = 2+\clubs, otwarcia na poziomie 1 limitowane do Acola
    \item System posiada niewielkie wady, które jednak go znacznie upraszczają i uodporniają na pomyłki
    \item Gramy schematem pytania o krótkość Low Shortage First  - oznaczanym \lsf\ (dopóki ktoś mnie nie przekona że odwrotnie jest lepiej xd)
    \item \textbf{Niszczymy Multi} \imp
    \item Gramy lebensohlem w wielu dziwnych miejscach - \leb
    \item System został przystosowany do grania nim po piwie
\end{itemize}


\pagebreak
\section{Two Over One}
\subsection*{Założenia}
\begin{itemize}
    \item Gramy Niepoważnym 3\spades\ i 3\nt, zatem pokazujemy skład bez informacji o sile
    \item 1\clubs\ - 2\clubs\, 1\clubs\ - 2\diams\ jest \gf
    \item W kolorach młodszych każda odzywka poniżej 3\nt\ dotyczy 3\nt, a nie szlemika
    \item Jeśli licytowane były 2 kolory, licytujemy posiadane zatrzymanie
    \item Jeśli licytujemy 3 kolory, czwarty kolor wskazuje na problem z trzymaniem
\end{itemize}

\subsection*{Ustalenia}
\subsubsection*{Niepoważne}
\begin{itemize}
    \item Mając 2 asy z damą nie wolno dać fast arrival
    \item Mając cuebid należy go pokazać jeśli partner pokazał poważną rękę
    \item Ostatni cuebid to Last Train (żeby się wyratować z powyższego)
\end{itemize}

\subsubsection*{Rebid koloru}
    Rebid koloru pokazuje 6+, np w sekwencji 1\hearts\ - 2\clubs\ --- 2\hearts.
    Po takiej sekwencji:
    \begin{itemize}
        \item 2\spades\ = (2)3+\hearts\ \lsf\ brak/niska/średnia/wysoka
        \item 2\nt\ = 5+\clubs, 4\spades\ lub 4144
        \item 3\clubs\ = 6+\clubs
        \item 3\diams\ = 5+\clubs, 4\diams
        \item 3\hearts\ = dokładnie 2\hearts\
    \end{itemize}
    Dwukartowy fit można generalnie ukrywać na rzecz pokazania reszty składu. Wtedy odzywka niemożliwa powyżej
    3\nt\ ustala silnie kolor.
    Na pikach analogicznie - 2\nt\ jest \lsf.

\subsubsection*{Rebid 2\ntx}
    Pokazuje dokładnie 5332. Można tak też dać z 5422, które bardzo nie chce pokazywać drugiej czwórki.
    Dalsza licytacja NAT

\subsubsection*{Rebid 3\clubs, 3\diams}
    Pokazuje 5-4, raczej w składzie niezrównoważonym z niezłą czwórką.

\subsubsection*{Rebid splinterem}
    Rebid w nowy kolor z przeskokiem to splinter.
    \begin{itemize}
        \item Do 2\clubs\ jest z piątki: 1\spades\ - 2\clubs\ --- 3\diams\ = splinter 5+\clubs
        \item Do 2\diams, 2\hearts\ z czwórki.
    \end{itemize}

\subsubsection*{Rebid w kolor z przeskokiem}
    Np. 1\spades\ - 2\diams\ --- 3\spades. Samoustalenie koloru. Niepoważne ON.

\subsubsection*{GF przez 1\spades}
    Odpowiedź 1\spades, a potem sforsowanie do koncówki pokazuje 5+\spades. \\
    Np. 1\diams\ - 1\spades\ --- 2\diams\ - 3\clubs\ = 5+\spades. \br

    Jeśli chcemy sfitować \hearts\ do \gf\ pokazując najpierw \spades\ - skaczemy w 3\hearts:\\
    1\hearts\ - 1\spades\ --- 2\clubs\ - 3\hearts. Ignorujemy Gazilli, jednak mając na uwadze że 2\clubs\ mogło być sztuczne

\pagebreak
\subsection*{Otwarcie w kolor młodszy}
\subsubsection*{1\clubs\ --- 2\clubs}
\begin{itemize}
    \item 2\diams\ = 12-14 lub 18-19 BAL, może mieć starszą czwórkę \imp
    \item 2\hearts\ = 5+\clubs, 4+\hearts\ niekoniecznie nadwyżkowe
    \item 2\spades\ = 5+\clubs, 4+\spades 
    \item 2\nt\ = 5+\clubs, 4+\diams \vimp
\end{itemize}

\subsubsection*{1\diams\ --- 2\clubs}
\begin{itemize}
    \item 2\diams\ = \textbf{6+}\diams\ może mieć starszą czwórkę \imp
    \item 2\hearts\ = (4)5\diams, 4+\hearts\ niekoniecznie nadwyżkowe
    \item 2\spades\ = (4)5\diams, 4+\spades 
    \item 2\nt\ = 12-14 lub 18-19 5332
\end{itemize}



\pagebreak
\section{Otwarcie 1\clubs}
\subsection*{1\diams\ Negat}
\subsubsection*{Rebid odpowiadającego}
\begin{itemize}
    \item 1\hearts, 1\spades, 1\nt, 2\clubs = słabe
    \item 2\diams\ = 7-9(10) 5+\diams
    \item 2\major\ (drugi stary) = 7-11 na młodych
    \item 2\nt\ = 11-12, 5\diams
    \item 3\diams\ = 10-12, 6+\diams
\end{itemize}

\subsection*{2\hearts\ Flannery 5+\spades\ 4+\hearts}
\begin{itemize}
    \item 2\nt\ = \lsf\ brak/niska/wysoka \exq
    \item 3\clubs\ = do gry
    \item 3\diams\ = czwarty kolor \gf
\end{itemize}

\subsection*{2\spades\ Transfer i 2\ntx\ Inwit}
\begin{itemize}
    \item 2\nt\ = 11-12 BAL
    \item 3\clubs\ = jak normalny rebid 2\clubs\ (11-15)
    \item 3\diams\ = jak normalny rebid 3\clubs\ (15-17) \imp
    \item 3\hearts, 3\spades\ = krótkość (coś jak 1\nt\ - 3\hearts)
    \item 3\nt\ = 13-14 BAL
    \item 4\nt\ = 18-20 BAL
\end{itemize}

\subsection*{3\clubs\ Mixed Raise}
6+\clubs, 7-9PC

\subsection*{GF na \clubs}
1\clubs\ -- 1\hearts\ --- 3\diams\ oraz 1\clubs\ -- 1\spades\ --- 3\diams\ to \gf\ z 6+\clubs\ (podwójny rewers +1)


\pagebreak
\section{Otwarcie 1\diams}
\subsection{2\diams\ Inverted}
\begin{itemize}
    \item 10+PC, 4+\diams. Forsuje do 3\diams. Ten, kto przekracza 3\diams, przyjmuje inwit.
    Licytujemy posiadane zatrzymania, lub \nt\ z bilansu mając wszystkie.
    \item Rebid 3\hearts, 3\spades\ = splinter
\end{itemize}

\subsection*{2\hearts\ Flannery 5+\spades\ 4+\hearts}
\begin{itemize}
    \item 2\nt\ = \lsf\ brak/niska/wysoka \exq
    \item 3\clubs\ = czwarty kolor \gf
    \item 3\diams\ = do gry
\end{itemize}

\subsection*{2\spades\ Transfer i 2\ntx\ Inwit}
\begin{itemize}
    \item 2\nt\ = 11-12 BAL
    \item 3\clubs\ = NAT \nf
    \item 3\diams\ = jak normalny rebid 2\diams, \nf! Z 15-17 na karach dajemy krótkość
    \item 3\hearts, 3\spades\ = krótkość (coś jak 1\nt\ - 3\hearts\ ale może być z \diams)
    \item 3\nt\ = 13-14 BAL
    \item 4\nt\ = 18-20 BAL
\end{itemize}

\subsection*{3\clubs\ Inwit NAT}
6+\clubs, 7-9

\subsection*{3\diams\ Mixed Raise}
4+\diams, 7-9

\subsection*{GF na \diams}
1\diams\ -- 1\spades\ --- 2\nt, a potem po checkbacku 3\clubs\ dajemy 3\diams.


\pagebreak
\section{Podniesienie koloru po One Over One}
Przykłady są dla sekwencji 1\clubs\ - 1\hearts, ale wszystkie działają analogicznie
\subsection*{Podniesienie 2\hearts}
\begin{itemize}
    \item 2\spades\ = \lsf\ brak/niska/wysoka \inv+
    \item 2\nt, 3\clubs, 3\diams\ = inwity wartościowe, 2\nt\ = \spades
    \item 3\hearts\ = ogólny inwit
    \item 3\spades, 4\clubs, 4\diams\ = splinter - tu nie gramy niewygodnym bo OTW jest ograniczony!
    \item 3\nt\ = propozycyjne
\end{itemize}

\subsection*{Inwit 3\hearts}
\begin{itemize}
    \item 3\spades\ = \lsf\ brak/niska/wysoka
    \item 3\nt\ = krótkość \spades\ - to nie jest propozycyjne, bo OTW jest niezrównoważony i na \spades\ nie jest dostępne, bo to LSF
\end{itemize}

\subsection*{Bezkrótkościowe 3\ntx}
Jak nazwa wskazuje, 1\clubs\ - 1\hearts\ -- 3\nt\ = fit \hearts\ i brak krótkości. Leciutko propozycyjne. \vimp

\subsection*{Splintery 3\spades, 4\clubs, 4\diams}
Tu nie gramy niewygodnym, bo bezkrótkościowy \gf\ z fitem jest ważniejszy, szczególnie żeby odciążyć rebid 2\nt\ z rąk z fitem!
Jest to owszem lekko teoretycznie nieoptymalne, ale come on, staram się napisać prosty system na grę przy piwie czyli 3 ligę.



\pagebreak
\section{Rebid na poziomie 1}
\subsection*{Pozycja podwójnego magistra}
\begin{itemize}
    \item 2\clubs\ = inwit lub transfer na \diams\
    \item 2\nt\ = transfer na \clubs\
    \item 3\clubs, 3\diams\ = 5\minor\ -- 5\major, \gf, zależy co licytowaliśmy wcześniej
    \item 1\clubs\ - 1\spades\ --- 1\nt\ - 3\spades\ = \gf\ 6+\spades, 3N = niepoważne. Na \hearts\ tak samo.
    \item 3\nt\ = do gry
    \item 2\clubs\ - 2\diams\ --- 3\nt\ = propozycyjne ze starszą piątką (5332) \imp
\end{itemize}



\pagebreak
\section{Rebid 2\ntx\ po otwarciu w kolor młodszy}
Charakterystyka:
\begin{itemize}
    \item \textbf{Nie może} być z fitem do starszej czwórki partnera (chyba że obrzydliwa karta 4333)
    \item Może być z 6+m na składzie nierównym w sekwencji 1\diams\ - 1\hearts\ --- 2\nt \imp
    \item W pozostałych przypadkach jest na równym lub 2=2=5=4 - z rękami które nie nadawały się na rebid 3\clubs.
    \item Olewamy szukanie szlemika w karo po otwarciu 1\clubs, bo to zbyt rzadkie (skoro nie daliśmy 2\nt!)
    \item \textbf{What we bypass, we deny!}
\end{itemize}

\subsection*{1\clubs\ -- 1\hearts\ --- 2\ntx}
\subsubsection*{3\clubs\ = 4+\clubs\ i/lub pytanie}
\begin{itemize}
    \item 3\diams\ = potencjalny fit \clubs
    \item 3\hearts\ = 3\hearts, bez \clubs
    \item 3\spades\ = 4\spades, bez \hearts\ i \clubs
    \item 3\nt\ = 3253, bo to zostaje.
\end{itemize}

\subsection*{1\diams\ -- 1\hearts\ --- 2\ntx}
\subsubsection*{3\clubs\ = Pytanie lub długie \clubs}
\begin{itemize}
    \item 3\diams\ = \textbf{6+\diams!}
    \item 3\hearts\ = 3\hearts 
    \item 3\spades\ = 4\spades\ w składzie 4=2=5=2
    \item 3\nt\ = 3253, bo to zostaje.
\end{itemize}



\pagebreak
\section{Rewersy}
Wbrew głupiemu popularnemu przekonaniu rewersy są \fonce a nie \gf.
\subsection*{Odpowiedzi na rewers}
\subsubsection*{Slowdown Bid}
Zalicytowanie po rewersie odzywki niższej z czwartego koloru lub 2\nt\ wskazuje bardzo słabą rękę
(jakieś 6-8) i zwalnia z forsingu do końcówki. Następnie:
\begin{itemize}
    \item Odzywki poniżej 3 w kolor otwarcia są słabe i do gry (nie pokazuje 6+) - chyba że odpowiadający 
    to na coś poprawi, wtedy to też jest do gry 
    \item Jeśli Slowdown'em jest 2\nt, otwierający musi wrócić na swój kolor ze słabą ręką
\end{itemize}

\subsubsection*{Inne odzywki}
Poniżej Slowdown Bid - forsują. Ten, kto przekracza slowdown bid nie licytując go, ustala \gf

\subsection*{Jump Shifty}
Są to rewersy z przeskokiem. Pokazują:
\begin{itemize}
    \item 5-4 (1\minor\ - 1\hearts\ --- 2\spades\ oraz 1\diams\ - 1x --- 3\clubs)
    \item 5-5 (1\major\ - 1x --- 3\minor)
    \item 6-5 (1\hearts\ - 1\nt\ --- 2\spades) \imp
\end{itemize} 
i są ściśle \gf!



\pagebreak
\section{Otwarcie 1\hearts}
\begin{itemize}
    \item 1\nt\ może zawierać 4-7 z fitem
    \item 2\hearts\ = konstruktywne 8-10
    \item 2\spades\ = inwit z krótkością, 2\nt\ = \lsf
    \item 2\nt\ = inwit zwykły 3+\hearts
    \item 3\clubs, 3\diams\ = inwit 6+\minor
    \item 3\hearts\ = Mixed Raise 7-9 4+\hearts, 3\spades\ = \lsf
    \item 3\spades, 3\nt, 4\clubs\ = splinter
    \item 4\diams\ = "chcę grać końca, ale za słabe na gf"
\end{itemize}

\section{Otwarcie 1\spades}
\begin{itemize}
    \item 1\nt\ może zawierać 4-7 z fitem, czasami też czterokartowym
    \item 2\spades\ = konstruktywne 8-10
    \item 2\nt\ = inwit z krótkością, 3\clubs\ = \lsf
    \item 3\clubs, 3\diams\ = inwit 6+\minor
    \item 3\hearts\ = inwit zwykły 3+\spades
    \item 3\spades\ = Mixed Raise 7-9 4+\spades, 3\nt\ = \lsf
    \item 3\nt, 4\clubs, 4\diams\ = splinter
    \item 4\hearts\ = "chcę grać końca, ale coś tam mam"
\end{itemize}



\pagebreak
\section{Otwarcie 1\major\ na 3 i 4 ręce}
\subsection*{Charakterystyka}
\begin{itemize}
    \item Styl: A i tak mnie nie spałujecie piwo piwo
    \item Z czwórki: jeśli ma to sens
\end{itemize}

\subsection*{2\clubs\ Drury}
Używamy agresywnie, dodając sobie za krótkości. Od ok. 9PC, ale zdarza się z 6PC!
\begin{itemize}
    \item 2\diams\ = 12+, mamy szansę na końcówkę
    \item 2\major\ = Otworzyłem z czwórki albo równowartości czwórki
\end{itemize}

\subsection*{Inne odzywki} 
\begin{itemize}
    \item 3\hearts, 3\spades\ = blok
    \item 3\nt\ = mini-splinter (z singla, 4\clubs\ = \lsf)
    \item 4\clubs, 4\diams, 4\hearts = splinter z \textbf{renonsu}
\end{itemize}

\subsubsection*{Dziwne sekwencje}
\begin{itemize}
    \item 1\spades\ -- 2\clubs\ --- 2\diams\ -- 2\nt\ = \lsf\ brak/niska/średnia/wysoka \imp
    \item 1\hearts\ -- 2\clubs\ --- 2\diams\ -- 2\spades\ = \lsf
\end{itemize}



\pagebreak
\section{Gazilli}
\subsection*{1\hearts\ --- 1\spades}
\subsubsection*{1\hearts\ --- 1\spades \\ 2\clubs\ --- ?}
\begin{itemize}
    \item 2\diams\ = 8+ skład dowolny
    \item 3\hearts\ = \gf\ z fitem \hearts. 3\spades\ = \ns
    \item 2\nt\ = słabe na młodych
    \item 3\spades\ = 6+\spades, \inv
    \item 4\clubs, 4\diams\ = splintery na \hearts
\end{itemize}
\subsubsection*{1\hearts\ --- 1\spades \\ 2\clubs\ --- 2\diams \\ ?}
\begin{itemize}
    \item 2\spades\ = dokładnie 3\spades, 16+. Dalej: 2\nt\ = ask o skład, 3\spades\ = ustalenie, 3\nt\ = \ns
    \item 2\nt\ = 2533
    \item 3\spades\ = 4\spades, \gf, raczej bez krótkości
\end{itemize}
\subsubsection*{1\hearts\ --- 1\spades \\ 2\clubs\ --- 2\hearts \\ ?}
\begin{itemize}
    \item 2\spades\ = 5\hearts3\spades, forsuje do 3\major
    \item 2\nt\ = 18-20, 3\minor\ = \nf
\end{itemize}

\subsection*{Inne \vimp}
\begin{itemize}
    \item 1\hearts\ - 1\nt\ --- 2\spades\ = 6\hearts5\spades
    \item 1\major\ - 1\nt\ --- 2\nt\ = 6\major5\minor, 3\clubs\ = ASK -> 3\diams\ = \clubs, 3\hearts\ = \diams
    \item 1\major\ - 1\nt\ --- 3\major\ = \gf\ samoustalenie
\end{itemize}



\pagebreak
\section{Otwarcie 1\ntx}
\subsection*{1\ntx\ --- ?}
\begin{itemize}
    \item 3\clubs\ = krótkość \clubs, 5+\diams, 4\major\ \gf
    \subitem 3\diams\ = fit \diams, lekka próba do przodu (3\major\ NAT)
    \subitem 3\major\ = z czwórki
    \subitem 4\clubs\ = potężne wyłączenie \clubs\ z fitem \diams
    \subitem 4\diams\ = gramy w \clubs\ ale bez szału
    \item 3\diams\ = krótkść \diams, 5+\clubs, 4\major\ \gf
    \subitem 3\major\ = z czwórki
    \subitem 4\clubs\ = gramy w trefle ale bez szału
    \subitem 4\diams\ = potężne wyłączenie \diams\ z fitem \clubs
    \item 3\hearts\ = krótkość \hearts, 5+/4+ \minor
    \item 3\spades\ = krótkość \spades, 5+/4+ \minor
\end{itemize}

\subsection{1\ntx\ --- 2\clubs \\ 2\diams\ --- ?}
\begin{itemize}
    \item 2\hearts\ = 4+\hearts/4+\spades\ do pasa
    \item 2\spades\ = 5+\spades\ do pasa
    \item 3\clubs\ = pytanie o skład na młodych
    \subitem 3\diams\ = 4+/4+\minor (3\hearts\ ustala \clubs, 3\spades\ ustala \diams) \imp
    \subitem 3\hearts\ = 5+\clubs
    \subitem 3\spades\ = 5+\diams
    \subitem 3\nt\ = 4333 lub \textls{paskudna} karta z 4/4\minor
    \item 3\diams\ = pytanie o starsze trójki (4\clubs\ = obie -> 4\diams\hearts\ = trsf)
    \item 3\hearts\ = krótkość
    \item 3\spades\ = krótkość
\end{itemize}

\subsection*{Inne}
\begin{itemize}
    \item Inwit z 5/4 w \major\ = transfer, potem drugi stary
    \item GF z 5/4 \major\ = stayman, 3\diams
    \item Zalicytowanie drugiego starego po Staymanie i odp w stary ustala silnie
\end{itemize}



\pagebreak
\section{Otwarcie 2\ntx}
\subsection*{2\ntx\ --- 3\clubs \\ 3\diams\ --- ?}
\begin{itemize}
    \item 3\hearts\ = 4\spades\ (po odpowiedzi 3\nt\ \textbf{pełny} Minor Puppet o czwórki i trójki!)
    \item 3\spades\ = 4\hearts
    \item 4\clubs\ = \emph{Minor Puppet}
    \item 4\diams\ = 4\hearts\ oraz 4\spades\ (tu nie ma MP o młodsze trójki - nie mieści się!)
\end{itemize}

\subsection*{2\ntx\ --- 3\clubs \\ 3\spades\ --- ?}
Po odp. 3\hearts\ tak samo.
\begin{itemize}
    \item 4\clubs\ = \emph{Minor Puppet}
    \item 4\diams\ = \emph{Minor Puppet} o trójki
    \item 4\hearts\ = silne ustalenie \spades
\end{itemize}

\subsection*{2\ntx\ --- 3\diams \\ 3\hearts\ --- ?}
Po transferze na \spades\ tak samo
\begin{itemize}
    \item 3\spades\ = 4\spades5\hearts
    \item 4\clubs\ = \emph{Minor Puppet}
    \item 4\diams\ = \emph{Minor Puppet} o trójki
    \item 4\hearts\ = inwit do szlemika z 6+\hearts
\end{itemize}

\subsection*{4\clubs\ Minor Puppet o czwórki}
Forsuje do szlemika, jeśli jest fit i asy
\begin{itemize}
    \item 4\diams\ = posiadam młodszą czwórkę, ale nie piątkę
    \subitem 4\hearts\ = 4\clubs
    \subitem 4\spades\ = 4\diams
    \item 4\hearts\ = 5+\clubs
    \item 4\spades\ = 5+\diams, dalej po wszystkich tych odzywkach z fitem odpowiadamy schodkami jak na asy, oprócz 4\nt\ do gry
\end{itemize}

\subsection*{4\diams\ Minor Puppet o trójki}
\begin{itemize}
    \item 4\hearts\ = 3+\clubs, 3+\diams\ (4\spades\ ustala \clubs, 4\nt\ do gry, 5\clubs\ ustala \diams)
    \item 4\spades\ = 3+\clubs, 2\diams\ (wszystko oprócz 4\nt\ ustala \clubs)
    \item 4\nt\ = 2\clubs, 3+\diams\ (wszystko ustala \diams)
\end{itemize}


\subsection*{Inne}
\begin{itemize}
    \item 2\nt\ - 3\spades\ = młody 6+ lub młode 5/4, wymusza automat 3\nt:
    \subitem 4\clubs, 4\diams\ = 6+ NAT
    \subitem 4\hearts\ = krótkość \hearts, 5+/4+
    \subitem 4\spades\ = krótkość \spades, 5+/4+
    \item 2\nt\ - 4\clubs\ = 5/5 stare (4\diams\ = dobre z fitem \hearts, 4\nt+ = odp. na asy z fitem \spades)
\end{itemize}



\pagebreak
\section{Otwarcie 2\clubs\ Acol}
\subsection*{Odpowiedzi}
\begin{itemize}
    \item 2\diams\ = pozytywne
    \item 2\hearts\ = negatywne
    \item 2\spades, 2\nt\, 3\clubs, 3\diams\ = kolor w sile co najmniej \hearts KQJTxx 
\end{itemize}

\subsection*{2\clubs\ --- 2\hearts\ --- ?}
\begin{itemize}
    \item \pass\ = dopuszczalne
    \item 2\spades\ = forsuje raz, więc z maxem 2\hearts\ z fitem 4\spades!
    \item 2\nt, 3\clubs, 3\diams\ = \nf
    \item 3\hearts\ = \gf
\end{itemize}

\subsection*{2\clubs\ -- 2\diams\ --- 2\hearts\ -- ?}
\begin{itemize}
    \item 2\spades\ = pokaż układ transferem (od 2\nt)
    \item 2\nt, 3\clubs, 3\diams\ = raczej singiel \hearts, własny długi kolor z 3 honorami
\end{itemize}

\subsection*{2\clubs\ -- 2\diams\ --- 2\spades\ -- ?}
\begin{itemize}
    \item 2\nt\ = pokaż układ transferem (od 3\clubs)
    \item 3\clubs, 3\diams, 3\hearts\ = raczej singiel \spades, własny długi kolor z 3 honorami
\end{itemize}



\pagebreak
\section{Bloki}
\subsection*{Na poziomie 2}
Naturalne, jakość zależna od założeń.
\begin{itemize}
    \item Po otwarciu 2\diams, 2\nt\ = Ogust:
    \subitem 3\clubs\ = minimum, słaby kolor
    \subitem 3\diams\ = minimum, dobry kolor (3 honory lub AK)
    \subitem 3\hearts\ = maximum, słaby kolor
    \subitem 3\spades\ = maximum, dobry kolor
    \subitem 3\nt\ = AKQxxx
    \item Po 2\hearts: 2\spades\ = \lsf, 2\nt\ = 5+\spades\ \fonce
    \item Po 2\spades: 2\nt\ = \lsf
\end{itemize}

\subsection*{Na poziomie 3}
\begin{itemize}
    \item W młode - przed pasem partnera całkiem solidne, w czerwonych bardzo solidne
    \item W stare - bardzo agresywne również przed pasem partnera
\end{itemize}

\subsection*{4\clubs\ Preempt Keycard \vimp} 
Działa bezpośrednio po każdym bloku.
\begin{itemize}
    \item 4\diams\ = 0 asów
    \item 4\hearts\ = 1 bez Q
    \item 4\spades\ = 1 z Q
    \item 4\nt\ = 2 bez Q
    \item 5\clubs\ = 2 z Q
\end{itemize}


\pagebreak
\section{Wejście kolorem}
\subsection*{Wejście na poziomie 1}
\subsubsection*{(1\clubs) -- 1\hearts\ -- (\passx) -- ?}
\begin{itemize}
    \item 2\clubs\ = z fitem niezależnie od znaczenia 1\clubs
    \item 2\nt\ = NAT 13-14 \imp
    \item 3\clubs\ = Mixed raise 7-9 z 4-kartowym fitem
    \item 3\hearts\ = Blokujące
\end{itemize}

\subsubsection*{(1\clubs) -- 1\hearts\ -- (coś) -- ?}
\begin{itemize}
    \item coś = 1 w kolor, wtedy 2\clubs\ jest z fitem, a 2 w kolor odpowiedzi jest NAT
    \item coś = 2 w kolor, wtedy 3\clubs\ jest NAT
    \item \dbl\ i \rdbl\ = 9+PC i dokładnie dubelek \hearts
    \item 2\nt\ = 10+PC z \textbf{4-kartowym fitem} \imp
    \item 3\hearts\ = Blokujące
    \item 3\clubs\ (k. otw. z przeskokiem) = Mixed raise 7-9 z 4-kartowym fitem
\end{itemize}


\pagebreak
\section{Wejście dwukolorówką}
\subsection*{Wejście 2\diams\ stare}
\begin{itemize}
    \item 2\nt\ = pytanie o krótkość (3\clubs\ = \clubs, 3\diams\ = \diams, 3\hearts\ = void \clubs, 3\spades\ = void \diams)
    \item 3\clubs\ = INV+ z \hearts
    \item 3\diams\ = INV+ z \spades
    \item 3\major\ = Mixed Raise z dwoma grającymi wartościami
\end{itemize}

\subsection*{Wejście 2\ntx\ młode}
\begin{itemize}
    \item 3\hearts\ = INV+ z \clubs\
    \item 3\spades\ = INV+ z \diams\
\end{itemize}

\subsection*{Wejście 2\hearts\ = \spades\ + \minor}
\begin{itemize}
    \item 2\nt\ = ask INV+
    \subitem 3\hearts\ = dobre z \clubs
    \subitem 3\spades\ = dobre z \diams
    \item 3\diams\ = INV+ z fitem \spades \imp
    \item 3\spades\ = Mixed Raise z dwoma grającymi wartościami
\end{itemize}

\subsection*{Wejście 2\spades\ = \hearts\ + \minor}
\begin{itemize}
    \item 2\nt\ = ask INV+
    \subitem 3\hearts\ = dobre z \clubs 
    \subitem 3\spades\ = dobre z \diams
    \item 3\diams\ = INV+ z fitem \hearts \imp
\end{itemize}  

 

\pagebreak
\section{Przeciwnik wchodzi dwukolorówką}
\subsection*{Wejście dwukolorówką specyficzną}
Stosujemy zasadę "Lower for lower". W naszym kolorze INV+, w nielicytowanym \gf.
\\[.6em]

\begin{minipage}{0.5\linewidth}
    \subsubsection*{1\hearts\ -- (2\ntx) -- ?}
    \begin{itemize}  
        \item 3\clubs\ = INV+ z \hearts
        \item 3\diams\ = \gf\ z 5+\spades
        \item 3\hearts, 3\spades\ = do gry
    \end{itemize} 
\end{minipage}
\begin{minipage}{0.3\linewidth}
    \subsubsection*{1\spades\ -- (2\ntx) -- ?}
        \begin{itemize} 
            \item 3\clubs\ = \gf\ z 5+\hearts
            \item 3\diams\ = INV+ z \spades
            \item 3\hearts, 3\spades\ = do gry
        \end{itemize}
\end{minipage}

\subsection*{Wejście dwukolorówką niespecyficzną}
Stosujemy zasadę forsowania z 2\nt\ jak po wejściu kolorem
\\[.6em]

\begin{minipage}{0.5\linewidth}
    \subsubsection*{1\hearts\ -- (2\hearts) -- ?}
    \begin{itemize}  
        \item 2\nt\ = \inv\ z fitem
        \item 3\clubs, 3\diams\ = NAT słabe (wrr!)
        \item 3\spades\ = \gf\ z fitem
    \end{itemize} 
\end{minipage}%
\begin{minipage}{0.5\linewidth}
    \subsubsection*{1\spades\ -- (2\spades) -- ?}
        \begin{itemize} 
            \item 2\nt\ = \gf\ z fitem
            \item 3\clubs, 3\diams\ = NAT słabe (wrr!)
            \item 3\hearts\ = \inv\ z fitem
        \end{itemize}
\end{minipage}

\pagebreak
\section{Przeciwnik wchodzi kontrą}
Na razie:
\begin{itemize}
    \item \rdbl\ 10+ (z fitem tylko \gf)
    \item podniesienie = słabe
    \item 1\nt\ = 8-10 z fitem
    \item 2\nt\ = inwit z fitem
    \item kolory \nf
\end{itemize}


\pagebreak
\section{Przeciwnik otwiera 2\diams\ Multi lub Wilkosz}
Ku chwale Erica Kokisha, JEBAĆ MULTI PAŁA PUNKTY 
\subsection*{Wejścia na multi}
\subsubsection*{(2\diams*) --- ?}
\begin{itemize}
    \item \dbl\ = 14-16 BAL lub objaśniak \vimp
    \item 2\nt\ = 17-19 BAL
    \br
    \item Z kontrą wywoławczą do jednego koloru pasujemy i czekamy co zrobią.
    Wznowienie zbiegającego 2\hearts\ jest pełnobilansowe!
\end{itemize}

\subsection*{Partner kontruje na punkty \vimp} 
\subsubsection*{(2\diams*) --- \dbl\ --- (\rdbl**)}
** Cokolwiek co grają P/C. Trzeba uważać bo po \pass\ nasza odzywka może byc wymuszona.
\begin{itemize}
    \item 2\hearts, 2\spades\ = do gry
    \item 2\nt\ = \leb, relay 3\clubs
    \subitem \pass, 3\diams\ = do gry
    \subitem 3\hearts, 3\spades\ = \inv
    \item 3\clubs\ = Puppet Stayman \imp
    \item 3\diams, 3\hearts\ = transfer (jak po otwarciu 2\nt)
\end{itemize}

\subsubsection*{(2\diams*) --- \dbl\ --- (3\hearts)}
\begin{itemize}
    \item \dbl\ = negatywna z \textbf{pikami} (do naturalnych kierów)
    \item 3\spades\ = \gf
\end{itemize}

"Jeśli odpowiadający pokazał naturalnie jakiś kolor (np. 2\hearts/\spades\ albo \pass\ = 5+\diams),
 to transfer na ten kolor bądź późniejsze zalicytowanie tego koloru jest pytaniem o trzymanie."


 \pagebreak
\section{Przeciwnik otwiera dwukolorówką lub blokiem}

\subsection*{(2\spades) --- \dbl\ --- (\passx) --- ?}
\begin{itemize}
    \item 3\clubs, 3\diams, 3\hearts\ = \inv
    \item 3\spades\ = \gf\ bez 4\hearts
    \item 3\nt\ = do gry \br
    \item 2\nt\ \rightarrow\ 3\clubs, 3\diams, 3\hearts\ = do pasa
    \item 2\nt\ \rightarrow\ 3\spades\ = \gf\ z 4\hearts
    \item 2\nt\ \rightarrow\ 3\nt\ = \gf\ \textbf{z 4\hearts} z trzymaniem (choice of games)
\end{itemize}

\subsection*{(2\hearts) --- \dbl\ --- (\passx) --- ?}
\begin{itemize}
    \item 3\clubs, 3\diams, 3\spades\ = \inv
    \item 3\hearts\ = \gf\ bez 4\spades
    \item 3\nt\ = do gry \br
    \item 2\nt\ \rightarrow\ 3\clubs, 3\diams\ = do pasa
    \item 2\nt\ \rightarrow\ 3\hearts\ = \gf\ z 4\spades
    \item 2\nt\ \rightarrow\ 3\nt\ = \gf\ \textbf{z 4\spades} z trzymaniem (choice of games)
\end{itemize}

\subsection*{Inne}
\begin{itemize}
    \item (2\hearts) 3\hearts\ = Michaels NIE \gf
    \item (2\hearts) 4\clubs/4\diams\ = ULTRA dwukolorówka z licytowanym i \spades,
    pierwszy schodek ustala \textbf{stary}, drugi ustala \textbf{młody}
\end{itemize}


\pagebreak
\section{Strefa szlemowa}
\subsection*{Asy}
Odpowiedzi: bez bocznych króli. Pytanie o dame: \textbf{nie/tak} bez bocznych króli.
\subsubsection*{4\ntx\ --- (5\minor\major) --- ?}
Jeśli \textbf{możemy} zalicytować nasz kolor na poziomie 5:
\begin{itemize}
    \item \pass\ = 1/4 lub karny - \textbf{forsuje do kontry!}
    \item \dbl\ = 0/3
    \item +1 = 2 (jeśli to naszego koloru to nie pokazujemy Q)
\end{itemize}
Jeśli \textbf{nie możemy} zalicytować nasz kolor na poziomie 5:
\begin{itemize}
    \item \pass\ = 1/3 lub karny - \textbf{forsuje do kontry!}
    \item \dbl\ = 0/2/4
    \item +1 = renons w kolorze przeciwnika
\end{itemize}

\subsection*{Króle}
\begin{itemize}
    \item +1 = najniższy lub 2 pozostałe
    \item +2 = środkowy lub 2 pozostałe
    \item +3 = najwyższy lub 2 pozostałe
    \item zjazd = 0 lub brak miejsca
    \item z trzema dokładamy 7
\end{itemize}

\subsection*{Exclusion}
\begin{itemize}
    \item 0/1/2
\end{itemize}

\end{document}