\documentclass[12pt, a4paper]{article}
\usepackage{../lib/bridgetex2}
\usepackage{polyglossia}
\usepackage{enumitem}
\setmainlanguage{polish}

\newcommand{\lsf}{\color{WildStrawberry}\textbf{ASK LSF}\color{black}}
\newcommand{\leb}{\color{CadetBlue}\textbf{LEB}\color{black}}
\newcommand{\relay}{\color{BurntOrange}\textbf{Relay}\color{black}}

\title{\vspace{-2cm}Taka strefa, że ćmy latają}
\author{}
\date{}

\begin{document}
\maketitle
\section{Założenia}
\begin{itemize}
    \item 1\clubs\ = 2+\clubs, otwarcia na poziomie 1 limitowane do Acola
    \item System posiada niewielkie wady, które jednak go znacznie upraszczają i uodporniają na pomyłki
    \item Gramy schematem pytania o krótkość Low Shortage First  - oznaczanym \lsf\ (dopóki ktoś mnie nie przekona że odwrotnie jest lepiej xd)
    \item \textbf{Niszczymy Multi}
    \item Gramy lebensohlem w wielu dziwnych miejscach - \leb
    \item System został przystosowany do grania nim po piwie
\end{itemize}


\pagebreak
\section{Two Over One}
\subsection*{Założenia}
\begin{itemize}
    \item Gramy Niepoważnym 3\spades\ i 3\nt, zatem pokazujemy skład bez informacji o sile
    \item 1\clubs\ - 2\clubs\, 1\clubs\ - 2\diams\ jest \gf
    \item W kolorach młodszych każda odzywka poniżej 3\nt\ dotyczy 3\nt, a nie szlemika
    \item Jeśli licytowane były 2 kolory, licytujemy posiadane zatrzymanie
    \item Jeśli licytujemy 3 kolory, czwarty kolor wskazuje na problem z trzymaniem
\end{itemize}

\subsection*{Ustalenia}
\subsubsection{Niepoważne}
\begin{itemize}
    \item Mając 2 asy z damą nie wolno dać fast arrival
    \item Mając cuebid należy go pokazać jeśli partner pokazał poważną rękę
    \item Ostatni cuebid to Last Train (żeby się wyratować z powyższego)
\end{itemize}

\subsubsection*{Rebid koloru}
    Rebid koloru pokazuje 6+, np w sekwencji 1\hearts\ - 2\clubs\ --- 2\hearts.
    Po takiej sekwencji:
    \begin{itemize}
        \item 2\spades\ = (2)3+\hearts\ \lsf\ brak/niska/średnia/wysoka
        \item 2\nt\ = 5+\clubs, 4\spades lub 4144
        \item 3\clubs\ = 6+\clubs
        \item 3\diams\ = 5+\clubs, 4\diams
        \item 3\hearts\ = dokładnie 2\hearts\ (niewytestowane czy działa)
    \end{itemize}
    Dwukartowy fit można generalnie ukrywać na rzecz pokazania reszty składu. Wtedy odzywka niemożliwa powyżej
    3\nt\ ustala silnie kolor.
    Na pikach analogicznie - 2\nt\ jest \lsf.

\subsubsection*{Rebid 2\ntx}
    Pokazuje dokładnie 5332. Można tak też dać z 5422, które bardzo nie chce pokazywać drugiej czwórki.
    Dalsza licytacja NAT

\subsubsection*{Rebid 3\clubs, 3\diams}
    Pokazuje 5-4, raczej w składzie niezrównoważonym z niezłą czwórką.

\subsubsection*{Rebid splinterem}
    Rebid w nowy kolor z przeskokiem to splinter.
    \begin{itemize}
        \item Do 2\clubs\ jest z piątki: 1\spades\ - 2\clubs\ --- 3\diams\ = splinter 5+\clubs
        \item Do 2\diams, 2\hearts\ z czwórki.
    \end{itemize}

\subsubsection*{Rebid w kolor z przeskokiem}
    Np. 1\spades\ - 2\diams\ --- 3\spades. Samoustalenie koloru. Niepoważne ON.

\subsubsection*{GF przez 1\spades}
    Odpowiedź 1\spades, a potem sforsowanie do koncówki pokazuje 5+\spades. \\
    Np. 1\diams\ - 1\spades\ --- 2\diams\ - 3\clubs\ = 5+\spades. \br

    Jeśli chcemy sfitować \hearts\ do \gf\ pokazując najpierw \spades\ - skaczemy w 3\hearts:\\
    1\hearts\ - 1\spades\ --- 2\clubs\ - 3\hearts. Ignorujemy Gazilli, jednak mając na uwadze że 2\clubs\ mogło być sztuczne

\pagebreak
\section{Otwarcie 1\clubs}
\subsection*{1\diams\ Negat}
\subsubsection*{Rebid odpowiadającego}
\begin{itemize}
    \item 1\spades, 1\nt, 2\clubs = słabe
    \item 2\diams\ = 7-9(10) 5+\diams
    \item 2\spades, 2\nt = 11-12, 5\diams, niezrównoważone. Zależy z której ręki chcemy \nt.
    \item 3\diams\ = 10-12, 6+\diams
\end{itemize}

\subsection*{2\hearts\ Flannery 5+\spades\ 4+\hearts}
\begin{itemize}
    \item 2\nt\ = \lsf\ brak/niska/wysoka
    \item 3\clubs\ = do gry
    \item 3\diams\ = czwarty kolor \gf
\end{itemize}

\subsection*{2\spades\ Transfer na NT}
\begin{itemize}
    \item 2\nt\ = 11-12 BAL
    \item 3\clubs\ = jak normalny rebid 2\clubs\ (11-15)
    \item 3\diams\ = jak normalny rebid 3\clubs\ (15-17)
    \item 3\hearts, 3\spades\ = krótkość (coś jak 1\nt\ - 3\hearts)
    \item 3\nt\ = 13-14 BAL
    \item 4\nt\ = 18-20 BAL
\end{itemize}

\subsection*{3\clubs\ Mixed Raise}
6+\clubs, 7-9PC

\subsection*{GF na \clubs}
1\clubs\ -- 1\hearts\ --- 3\diams\ oraz 1\clubs\ -- 1\spades\ --- 3\hearts\ to \gf\ z 6+\clubs


\pagebreak
\section{Otwarcie 1\diams}
\subsection{2\diams\ Inverted}
\begin{itemize}
    \item 10+PC, 4+\diams. Forsuje do 3\diams. Ten, kto przekracza 3\diams, przyjmuje inwit.
    Licytujemy posiadane zatrzymania, lub \nt\ z bilansu mając wszystkie.
    \item Rebid 3\hearts, 3\spades\ = splinter
\end{itemize}

\subsection*{2\hearts\ Flannery 5+\spades\ 4+\hearts}
\begin{itemize}
    \item 2\nt\ = \lsf\ brak/niska/wysoka
    \item 3\clubs\ = czwarty kolor \gf
    \item 3\diams\ = do gry
\end{itemize}

\subsection*{2\spades\ Transfer na NT}
\begin{itemize}
    \item 2\nt\ = 11-12 BAL
    \item 3\clubs\ = jak normalny rebid 2\diams\ (11-15)
    \item 3\diams\ = jak normalny rebid 3\diams\ (15-17)
    \item 3\hearts, 3\spades\ = krótkość (coś jak 1\nt\ - 3\hearts)
    \item 3\nt\ = 13-14 BAL
    \item 4\nt\ = 18-20 BAL
\end{itemize}

\subsection*{3\clubs\ Inwit NAT}
6+\clubs, 7-9

\subsection*{3\diams\ Mixed Raise}
4+\diams, 7-9

\subsection*{GF na \diams}
1\diams\ -- 1\spades\ --- 3\hearts\ = \gf\, 6+\diams. \\
Po odpowiedzi 1\hearts\ dajemy 2\nt\ - więcej niżej


\pagebreak
\section{Podniesienie koloru po One Over One}
Przykłady są dla sekwencji 1\clubs\ - 1\hearts, ale wszystkie działają analogicznie
\subsection*{Podniesienie 2\hearts}
\begin{itemize}
    \item 2\spades\ = \lsf\ brak/niska/wysoka \gf\ 
    \item 2\nt, 3\clubs, 3\diams\ = inwity wartościowe, 2\nt\ = \spades
    \item 3\hearts\ = ogólny inwit
    \item 3\spades, 4\clubs, 4\diams\ = splinter - tu nie gramy niewygodnym bo OTW jest ograniczony!
    \item 3\nt\ = propozycyjne
\end{itemize}

\subsection*{Inwit 3\hearts}
\begin{itemize}
    \item 3\spades\ = \lsf\ brak/niska/wysoka
    \item 3\nt\ = cue \spades\ - to nie jest propozycyjne, bo OTW jest niezrównoważony i na \spades\ nie jest dostępne, bo to LSF
\end{itemize}

\subsection*{Bezkrótkościowe 3\ntx}
Jak nazwa wskazuje, fit \hearts\ i brak krótkości. Leciutko propozycyjne

\subsection*{Splintery 3\spades, 4\clubs, 4\diams}
Tu nie gramy niewygodnym, bo bezkrótkościowy \gf\ z fitem jest ważniejszy, szczególnie żeby odciążyć rebid 2\nt\ z rąk z fitem!
Jest to owszem lekko teoretycznie nieoptymalne, ale come on, staram się napisać prosty system na grę przy piwie czyli 3 ligę.



\pagebreak
\section{Rebid na poziomie 1}
\subsection*{Pozycja podwójnego magistra}
\begin{itemize}
    \item 2\clubs\ = inwit lub transfer na \diams\
    \item 2\nt\ = transfer na \clubs\
    \item 3\clubs, 3\diams\ = 5\minor-5\major, \gf, zależy co licytowaliśmy wcześniej
    \item 1\clubs\ - 1\spades\ --- 1\nt\ - 3\spades\ = \gf\ 6+\spades, 3N = niepoważne. Na \hearts\ tak samo.
    \item 3\nt\ = do gry
    \item 2\clubs\ - 2\diams\ --- 3\nt\ = propozycyjne ze starszą piątką
\end{itemize}



\pagebreak
\section{Rebid 2\ntx\ po otwarciu w kolor młodszy}
Charakterystyka:
\begin{itemize}
    \item Nie może być z fitem do starszej czwórki partnera
    \item Może być z 6+m na składzie nierównym w sekwencji 1\diams\ - 1\hearts\ --- 2\nt
    \item W pozostałych przypadkach jest na równym lub 2=2=5=4 - z rękami które nie nadawały się na rebid 3\clubs.
    \item Olewamy szukanie szlemika w karo po otwarciu 1\clubs, bo to zbyt rzadkie (skoro nie daliśmy 2\diams!)
    \item \textbf{What we bypass, we deny!}
    \item Skok w 4\clubs\ ustala kolor otwarcia i pyta o asy
    \item Skok w 4\diams\ ustala kolor odopwiadającego i pyta o asy
\end{itemize}

\subsection*{1\clubs\ -- 1\hearts\ --- 2\ntx}
\subsubsection*{3\clubs\ = 4+\clubs\ i/lub pytanie}
\begin{itemize}
    \item 3\diams\ = potencjalny fit \clubs
    \item 3\hearts\ = 3\hearts, bez \clubs
    \item 3\spades\ = 4\spades, bez \hearts\ i \clubs
    \item 3\nt\ = 3253, bo to zostaje.
\end{itemize}

\subsection*{1\diams\ -- 1\hearts\ --- 2\ntx}
\subsubsection*{3\clubs\ = Pytanie lub długie \clubs}
\begin{itemize}
    \item 3\diams\ = \textbf{6+\diams!}
    \item 3\hearts\ = 3\hearts 
    \item 3\spades\ = 4\spades\ w składzie 4=2=5=2
    \item 3\nt\ = 3253, bo to zostaje.
\end{itemize}



\pagebreak
\section{Rewersy}
Wbrew głupiemu popularnemu przekonaniu rewersy są \fonce a nie \gf.
\subsection*{Odpowiedzi na rewers}
\subsubsection*{Slowdown Bid}
Zalicytowanie po rewersie odzywki niższej z czwartego koloru lub 2\nt\ wskazuje bardzo słabą rękę
(jakieś 6-8) i zwalnia z forsingu do końcówki. Następnie:
\begin{itemize}
    \item Odzywki poniżej 3 w kolor otwarcia są słabe i do gry (nie pokazuje 6+) - chyba że odpowiadający 
    to na coś poprawi, wtedy to też jest do gry 
    \item Jeśli Slowdown'em jest 2\nt, otwierający musi wrócić na swój kolor ze słabą ręką
\end{itemize}

\subsubsection*{Inne odzywki}
Są \gf\ nawet jeśli są poniżej slowdown bid.

\subsection*{Jump Shifty}
Są to rewersy z przeskokiem. Pokazują 5-4 i są ściśle \gf!



\pagebreak
\section{Otwarcie 1\hearts}
\begin{itemize}
    \item 1\nt\ może zawierać 4-7 z fitem
    \item 2\hearts\ = konstruktywne 8-10
    \item 2\spades\ = inwit z krótkością, 2\nt\ = \lsf
    \item 2\nt\ = inwit zwykły 3+\hearts
    \item 3\clubs, 3\diams\ = inwit 6+\minor
    \item 3\hearts\ = Mixed Raise 7-9 4+\hearts
    \item 3\spades, 3\nt, 4\clubs\ = splinter
    \item 4\diams\ = "chcę grać końca, ale za słabe na gf"
\end{itemize}

\section{Otwarcie 1\spades}
\begin{itemize}
    \item 1\nt\ może zawierać 4-7 z fitem, czasami też czterokartowym
    \item 2\spades\ = konstruktywne 8-10
    \item 2\nt\ = inwit z krótkością, 3\clubs\ = \lsf
    \item 3\clubs, 3\diams\ = inwit 6+\minor
    \item 3\hearts\ = inwit zwykły 3+\spades
    \item 3\spades\ = Mixed Raise 7-9 4+\spades
    \item 3\nt, 4\clubs, 4\diams\ = splinter
    \item 4\hearts\ = "chcę grać końca, ale coś tam mam"
\end{itemize}


\pagebreak
\section{Otwarcie 1\major\ na 3 i 4 ręce}
\subsection*{Charakterystyka}
\begin{itemize}
    \item Styl: A i tak mnie nie spałujecie piwo piwo
    \item Z czwórki: jeśli ma to sens
\end{itemize}

\subsection*{2\clubs\ Drury}
Używamy agresywnie, dodając sobie za krótkości. Od ok. 9PC, ale zdarza się z 6PC!
\begin{itemize}
    \item 2\diams\ = 12+, mamy szansę na końcówkę
    \item 2\major\ = Otworzyłem z czwórki albo równowartości czwórki
\end{itemize}

\subsection*{Inne odzywki} 
\begin{itemize}
    \item 3\hearts, 3\spades\ = blok
    \item 3\nt\ = mini-splinter (z singla, 4\clubs\ = \lsf)
    \item 4\clubs, 4\diams, 4\hearts = splinter z \textbf{renonsu}
\end{itemize}

\subsubsection*{Last Train 2\hearts}
1\spades\ -- 2\clubs\ --- 2\diams\ -- 2\hearts\ = jakiś tam inwit, żeby siędało zatrzymać w 2\spades (nic o kierach)


\section{Gazilli}
\section{Otwarcie 1\ntx}
\section{Otwarcie 2\ntx}
\section{Otwarcie 2\clubs\ Acol}
\section{Bloki}
\section{Wejście kolorem}
\section{Wejście dwukolorówką}
\section{Przeciwnik wchodzi kolorem}
\section{Przeciwnik wchodzi dwukolorówką}
\section{Przeciwnik wchodzi kontrą}
\section{Przeciwnik otwiera 2\diams\ Multi}
\section{Przeciwnik otwiera dwukolorówką lub blokiem}
\section{Strefa szlemowa}
\section{Pas forsujący i kontry karne}

\end{document}