\documentclass[12pt, a4paper]{article}
\usepackage{../lib/bridgetex2} 
\usepackage{polyglossia}
\setmainlanguage{polish}

\title{Acol}
\author{Bartek Słupik}

\begin{document}
    \maketitle
    \section{Założenia}
    \begin{itemize}
        \item 22+ BAL lub dowolny \gf
        \item Licytacja 2\clubs\ --- 2\hearts\ --- 2\nt\ --- \pass\ jest dopuszczalna - to jedyna sekwencja zwalniająca
        z \gf, nie mamy bilansu na końcówkę.
        \item System bazuje głównie na naturalnej licytacji otwierającego. Odpowiadający licytuje 
        naturalnie, jeśli coś ma ciekawego, jak nie to zwykle następna odzywka do góry jest oczekująca.
    \end{itemize}

    \section{2\clubs\ --- ?}

    \begin{itemize}
        \item 2\diams\ - Oczekujące, co najmniej Król lub 4+\hcp
        \item 2\hearts\ - Negat, 0-3\hcp, bez Króla
        \item 2\spades\ - Własny kolor 5+\spades, 2 z top 3 honorów, 8+ \hcp
        \item 2\nt\ - Własny kolor 5+\hearts, 2 z top 3 honorów, 8+ \hcp
        \item 2\clubs\ - Własny kolor 6+\clubs, 2 z top 3 honorów, 8+ \hcp
        \item 2\diams\ - Własny kolor 6+\diams, 2 z top 3 honorów, 8+ \hcp
    \end{itemize}

    \section{2\clubs\ --- 2\hearts \\ ?}
    \begin{itemize}
        \item 2\spades\ - 5+\spades, \fonce
        \item 2\nt\ - 22-23 BAL, \nf. Jest to \textbf{jedyna} odzywka, która może zostać spasowana przez odpowiadającego
        \item Inne naturalnie - \fonce. Każda następna odzywka otwierającego też jest \fonce.
    \end{itemize}

    \pagebreak
    \section{2\clubs\ --- 2\diams}
    \begin{itemize}
        \item 2\hearts\ - Kokish Relay, wymusza automat 2\spades 
        \item 2\spades\ - Naturalne, 5+\spades 
        \subitem 2\nt\ - Oczekujące
        \subitem 3\clubs, 3\diams, 3\hearts\ - 5+ naturalne z niezłej jakości kolorem
        \subitem 3\spades\ - silne z fitem
        \subitem 4\clubs, 4\diams, 4\hearts\ - splintery
        \item 2\nt\ - 22-23 BAL, gramy systemowo
        \item 3\clubs\ - 6+\clubs 
        \subitem 3\diams\ - oczekujące
        \subitem 3\hearts, 3\spades\ - Naturalne z piątki
        \item 3\diams\ - 6+\diams
        \subitem 3\hearts, 3\spades\ - Naturalne z piątki lub dobrej czwórki
        \item 3\hearts, 3\spades, 4\clubs, 4\diams\ - samoustalenia
        \item 3\nt - 26-27 BAL
    \end{itemize}

    \subsection{2\clubs\ --- 2\diams \\ 2\hearts\ --- 2\spades \\ ?}
    \begin{itemize}
        \item 2\nt\ - 24-25 BAL
        \item 3\clubs\ - 6+\hearts, ręka jednokolorowa. Jest to zamienione z 3\hearts, żeby dało się sfitować przez 3\hearts
        \item 3\diams\ - 5+\hearts, 4+\diams
        \item 3\hearts\ - 5+\hearts, 4+\clubs 
        \subitem 4\diams\ fituje \hearts\ silnie.
        \item 3\spades\ - 5+\hearts, 4\spades
        \subitem 4\clubs\ fituje \hearts\ silnie. 
        \subitem 4\diams\ fituje \spades\ silnie. 
    \end{itemize}

    \pagebreak
    \section{Po wejściu przeciwnika}
    \begin{itemize}
        \item \pass - Nic nie mówiący, ręka bez kształtu na kontrę wywoławczą
        i bez 5-kartu do pokazania. Otwierający powinien wznowić kontrą z krótkością w kolorze wejścia
        \item \dbl - wywoławcza
        \item Kolor - 6+\hcp, nie musi być super jakość - chodzi o walkę jeśli podniosą blok 
        \item 2\nt
    \end{itemize}

\end{document}

