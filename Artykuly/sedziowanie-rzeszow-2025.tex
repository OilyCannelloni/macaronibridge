\documentclass[12pt, a4paper]{article}
\usepackage{import}
\usepackage[paperwidth=21cm,paperheight=40cm,margin=1in]{geometry}

\import{../lib/}{bridge.sty}
\title{Podkarpacki Mityng Brydżowy 2025}

\author{Bartek Słupik}

\newcounter{board}
\newcommand\nextboard{\stepcounter{board}\theboard}

\begin{document}
\maketitle

\subsection*{Zbyt szybka przebitka}
S gra 4\hearts. Z ręki \xspades J, którego zamierza przebić w stole dziewiątką.
Jednak niespodziewanie kolor nie dzieli się, a W wbija się \xhearts 10. Rozgrywający prosi o dołożenie dziewiątki, od razu orientując 
się w sytuacji.

Czy możemy pozwolić mu na wycofanie karty?

\textbf{Nie.} Przepis 45C4(b) mówi jasno, że możliwość wycofania karty jest tylko w przypadku przejęzyczenia,
ale nie po utracie koncentracji, która miała miejsce przy stole.

\subsection*{Panie sędzio założenia!}
Dostałem wezwanie, by zweryfikować, czy w rozdaniu 26 na pewno 
obie strony są po partii. Zaiste tak jest. Pamiętając stringa
\textbf{,,oneb-nebo-ebon-bone''} i indeksując go numerem rozdania modulo 16,
otrzymujemy założenia (o = nikt, b = both po partii).

\subsection*{Sędziowski Clash Squeeze}
Grane jest 3\hearts\dbl, ale atutów już nie ma. Rozgrywający S ma
\xdiams Qx i wychodzi blotką. W tym momencie E poza kolejnością dokłada do lewy Asa z \xdiams Ax.
W tymczasem miał \xdiams Kx. Co robimy?

\textbf{Przepis 57A.} Rozgrywający może nakazać W dołożenie najstarszej 
lub najmłodszej karty w kolorze lewy. Oczywiście As będzie zagrany.


\subsection*{Skradziono mi odzywkę}

\handdiagramv[\nextboard]{\vhand{AJ5}{AQ9872}{-}{J543}}
				{\vhand{-}{K53}{96543}{KT972}}
                {\vhand{9432}{6}{KQJT72}{86}}
                {\vhand{KQT876}{JT4}{A8}{AQ}}{}

\allbidding{
    \nvul{W} & \nvul{N} & \nvul{E} & \nvul{S} \\
     \pass & 1\spades & 2\hearts & \dbl\alrt \\
     \pass & 4\spades & \dbl & \pass \\
     5\diams & \dbl & 5\hearts & \dbl
}

Kontra - po stronie NE brak alertu, po stronie SW - silny bergen.
S twierdzi, że zalicytował 5\diams, bo założył, że partner ma renons
pik. W KK EW mają zapis 1\spades - 2\hearts = bergen (jednostronna),
uparcie twierdzą, że ustalali kontrę ,,skradziono mi odzywkę'', ale tego w systemie nie ma.
Co robimy?

\textbf{Zmiana wyniku na 4\spades\dbl\ W -4.}
Sędzia zakłada, że ma do czynienia z błędnym wyjaśnieniem, a nie błędną zapowiedzią,
jeśli nie ma dowodów, że jest inaczej. Na gębę w takie absurdalne ustalenia nie wierzymy.
Wykonujemy panel z ręką S w pozycji zbiegającego 4\spades \dbl.



\subsection*{Transferowy blok}

\handdiagramv[\nextboard]{\vhand{A}{A6}{Q652}{AQJ854}}
				{\vhand{QJ2}{QJ975}{3}{KT97}}
                {\vhand{65}{KT84}{AKJ974}{3}}
                {\vhand{KT98743}{32}{T8}{62}}{}

\allbidding{
    \nvul{W} & \nvul{N} & \nvul{E} & \nvul{S} \\
        - & - & - & 1\diams \\
        3\hearts\alrt & 4\hearts & \pass & \pass \\
        \pass
}

3\hearts - W do S: Transferowy blok na piki, E do N: brak alertu
Wynik: bez jednej. EW grają ze sobą pierwszy raz, nie mają systemu.
Ustalili, że ,,grają transferowymi blokami'', co oboje potwierdzają przekonująco.
Co robimy?

\textbf{Zmiana wyniku na +3/-3 IMP.}
Nie ma dowodu, że takie ustalenie odnosi się również do wejść blokiem.
Tak jak w poprzednim problemie, odzywka 3\hearts jest tak nieschematyczna,
że nie można bezdyskusyjnie zaufać, że obowiązuje również na wejściu.
Zatem przyjmujemy, że prawidłowe tłumaczenie wejścia 3\hearts to naturalne.


\subsection*{Zaginiony pas}

\allbidding{
    \nvul{W} & \nvul{N} & \nvul{E} & \nvul{S} \\
    - & 1\diams & \pass & 2\clubs \\
    \pass & 3\diams & \pass\alrt & 4\nt \\
    \pass & 4\hearts & \pass\alrt 
}

Gramy na zasłonach. Zapowiedź niewystarczająca przyjechała na 
stronę SW. W przegródce deski należącej do E znajdowały się wtedy dwa pasy.
Co robimy?

\textbf{Zbieramy fakty.}
Przepytujemy graczy, żeby dowiedzieć się, czy brakujący pas
(pewnie puknięcie) nastąpił teraz, czy w poprzednim okrążeniu.
Okazuje się, że teraz pas był fizycznie położony, a brakło go po 3\diams.

\textbf{E zaakceptował zapowiedź niewystarczającą.}
Na zasłonach, dopóki nieprawidłowość nie zostanie przesunięta na 
drugą stronę zasłony, musi być poprawiona bez dalszych konsekwencji.

Gdy jednak dostanie się na drugą stronę, obowiązują przepisy MPB.


\pagebreak
\subsection*{Duży brak transferu}

\handdiagramv[\nextboard]{\vhand{QT}{KQ5}{AT76}{A832}}
				{\vhand{753}{8742}{J54}{K64}}
                {\vhand{42}{AJT963}{9}{Q975}}
                {\vhand{AKJ986}{-}{KQ832}{JT}}{}

\allbidding{
    \nvul{W} & \vul{N} & \nvul{E} & \vul{S} \\
    & 1\nt & \pass & 4\hearts\alrt \\
    4\spades\alrt & \pass & 4\nt & \dbl \\
    5\diams & \pass & \pass & \dbl \\
    \pass & \pass & 5\hearts & \dbl \\
    5\spades & \pass & 6\diams & \dbl
}

4\hearts: S do W: Naturalne, N do E - transfer.
4\spades: E do N: dwukolorówka.

Największym problemem w tym rozdaniu jest znalezienie odpowiedzi na
pytanie: jak brzmi poprawne wyjaśnienie? W tej sekwencji 
nie możemy przyjąć ,,braku ustaleń''. Gracze mają swoje wersje głęboko we krwi
i zawsze po jednej stronie padnie jakieś konkretne wyjaśnienie.

Podjąłem decyzję o przeanalizowaniu dwóch równoległych problemów,
przyjmując za poprawne obie wersje, a następnie wzięciu wyniku bardziej korzystnego
dla niewykraczających.

W pierwszym przypadku, dla 4\hearts = NAT, E spasuje na 4\spades.
Ze względu na format BAM i absurdalny wynik na drugim stole nie panelowałęm,
ale należałoby sprawdzić, co licytuje S na zbiegające 4\spades.

W drugim przypadku (4\hearts = trsf) W nie wejdzie pikami,
N da 4\spades, a S wyniesie w 5\hearts, które przegra.

Werdykt (jeśli byłby potrzebny): \textbf{Wynik ważony, 
4\spades\dbl +1 i 5\hearts -1} w zależności od odpowiedzi panelowanych.

\pagebreak
\subsection*{Marks próbuje ugrać maksa u sędziego}

\handdiagramv[\nextboard]{\vhand{QJT5}{QJ9}{A85}{K32}}
				{\vhand{4}{T72}{964}{QT9876}}
                {\vhand{6}{653}{KJT732}{AJ4}}
                {\vhand{AK98732}{AK84}{Q}{5}}{NSEW}

\allbidding{
    \vul{W} & \vul{N} & \vul{E} & \vul{S} \\
    1\clubs & \dbl & \pass & 1\diams \\
    3\spades & \pass & \pass & 4\diams \\
    4\hearts & \pass\alrt & \pass & \dbl   
}

Pas nastąpił po około 3-minutowym namyśle, potwierdzonym po drugiej stronie zasłony.
Przepanelowałem rękę S w końcowej pozycji. 
\begin{itemize}
    \item 4 graczy kontruje nie rozważając alternatyw
    \item 1 gracz kontruje, rozważając pasa
    \item 1 gracz pasuje, rozważając kontrę
\end{itemize}

Namysł jednoznacznie sugeruje chęć skontrowania.

Trzeba sprawdzić, czy pas jest logiczną alternatywą. Rzeczywiście, 
niektóry go wybierają, ale nie można powiedzieć, że znaczna część graczy go rozważa.
Zdecydowałem się \textbf{utrzymać wynik.}


\subsection*{Czy otwieramy kiery?}

\handdiagramv[\nextboard]{\vhand{AJT974}{93}{62}{A76}}
				{\vhand{Q832}{K6}{AQT94}{J2}}
                {\vhand{K65}{AJ2}{K5}{KQT54}}
                {\vhand{-}{QT8754}{J873}{983}}{NSEW}

\allbidding{
    \vul{W} & \vul{N} & \vul{E} & \vul{S} \\
    1\diams & 1\nt & 2\hearts\alrt & \pass \\
    2\spades\alrt & \pass & 3\diams & 3\nt 
}

2\hearts: W do S - naturalne, E do N - transfer na piki.
Na drugim stole w piki wzięto 11 lew. 

Poprawnym wyjaśnieniem było ,,naturalne''. NS, po otrzymaniu prawidłowego
wyjaśnienia, znajdą piki i zagrają 4\spades. Ale E zorientuje się w sytuacji 
i przypomni sobie system (do 1\diams grają NAT, a do innych otwarć - transferami).
Zawistuje w \xhearts K, EW wezmą 3 lewy.
\textbf{Wynik utrzymany.}




\pagebreak
\subsection*{Książkowy namysł}

\handdiagramv[\nextboard]{\vhand{K754}{KQ96}{T62}{J9}}
				{\vhand{Q862}{T}{Q53}{KT752}}
                {\vhand{T}{AJ542}{97}{Q8643}}
                {\vhand{AJ93}{873}{AKJ84}{A}}{EW}

\allbidding{
    \vul{W} & \nvul{N} & \vul{E} & \nvul{S} \\
     & \pass & \pass & 2\hearts\alrt \\
    \dbl & 3\hearts & 3\spades & \pass \\
    4\spades & \pass\alrt & \pass & 5\hearts \\
    \dbl
}

2\hearts - dwukolorówka \\
\pass - po długim namyśle 

Aby zmienić wynik należy wykazać, czy pas jest logiczną alternatywą
dla S. Wystarczyło przepanlelować dwie osoby. Dla obydwu pas 
był oczywisty. To wystarcza, by stwierdzić, że ,,niektórzy go
wybierają''.


\subsection*{Ciekawa fałszywka}

Dziadek zagrywa kiera, do którego E dokłada karo. Puszczamy w ręce
i W bije waletem. W tym momencie E zauważa, że został mu jeszcze \xhearts K.

Jako, że strona wykraczająca nie zagrała w kolejnej lewie, 
fałszywka musi zostać poprawiona. Każda karta wycofana przez obrońców
zostaje przygwożdżona. Zatem w tej lewie \xhearts J został pobity \xhearts K,
a następnie E musiał zagrać przygwożdżone karo.

\subsection*{O kartę za późno}

S gra \xspades A, do którego E nie dokłada. Następnie \xspades K, którego W przebija \xhearts 9.
Teraz E orientuje się, że ma jeszcze do pika.

Jako, że strona EW zagrała w kolejnej leewie po fałszywym renonsie, 
nie można już go poprawić. Gra toczy się dalej, a E dokłada pika, któ©eko powinien był
dołożyć wcześniej.

Wykraczający nie wziął lewy z fałszywką, ale strona wykraczająca wzięła
przebitkę w kolejnej lewie. Zatem należy się jedna lewa rekompensaty.


\subsection*{Pasywny wist w siódemkę karo}

\handdiagramv[\nextboard]{\vhand{752}{J9}{KJT532}{98}}
				{\vhand{Q6}{A864}{AQ}{AKQJ2}}
                {\vhand{AKT9}{KQ32}{974}{T4}}
                {\vhand{J843}{T65}{87}{7653}}{NS}

\allbidding{
    \nvul{W} & \vul{N} & \nvul{E} & \vul{S} \\
    1\clubs & \pass & 1\spades\alrt & \dbl\alrt \\
    \pass & 3\spades & \pass & 3\nt
}

1\spades - dowolna ręka bez starszych czwórek, \textbf{od 0 pc}.

\dbl - E do N: brak alertu i pytania, W do S - koniecznie zawsze 5 pików.

S zawistował w \xdiams 7. Argumentuje, że przy tłumaczeniu 
,,piki lub objaśniak'' zawistuje agresywnie \xspades A, a następnie w kiera.

Pozostaje zatem przeprowadzić panel wistowy. S to zawodnik światowej klasy,
także ciężko było znaleźć osoby na podobnym poziomie.
Odpowiedzi prezentują się tak:

\begin{itemize}
    \item Zawsze w \xhearts Q
    \item Zawsze w \xdiams 7
    \item Zawsze w \xspades A (dwukrotnie),
    \item Przy wyjaśnieniu ,,5 pików zawsze'' w \xhearts 2, 
    a z dopuszczonym silnym wariantem w \xspades A.
\end{itemize}

Należy jednak zauważyć, że wist w karo jest pasywny. Jedyny panelowany,
który zmienił zdanie pod wpływem wyjaśnienia już wcześniej 
wyciągał agresywnego kiera. 

Zatem nie znaleźlismy żadnego poparcia dla argumentacji S o zmianie 
wistu karowego na kierowy. \textbf{Wynik utrzymany.} Teoretycznie, 
,,wątpliwości należy rozstrzygać na korzyść strony niewykraczającej''. 
Zatem wystarczyłby choć jeden przypadek zmiany decyzji z argumentacją identyczną
do wistującego, by zmienić wynik.


\subsection*{Nietrafiony claim}

\handdiagramv[\nextboard]{}
				{\vhand{AK}{K}{95}{-}}
                {\vhand{9}{-}{JT87}{-}}
                {}{NS}

Rozgrywamy 5\diams\dbl. Zostały dwa atuty, ale nie wiemy jak się dzielą.
Mamy lekkie podejrzenia, że E ma oba, bo to on skontrował.

Deklaracja brzmiałą: ,,Gram pika, reszta moja''. Jednak E po wzięciu
gra ponownie pika i teraz S musi zgadnąć, czym przebić. Oczywiście od razu broni się,
że to oczywiste, że małym, bo E skontrował, więc ma więcej kar.

Przepisy o deklaracjach jednak nie mają litości. Przebicie dużym na podział 1-1
to normalne zagranie. Gracz mógłby w cwany sposób sprawdzać podział - jeśli są 1-1 z
dziewiątką z prawej, EW natychmiast się poddadzą. W innym przypadku zagramy na podział 0-2,
bo jeden z 1-1 już wykluczyliśmy. 


\subsection*{Stare i Multi, czyli klasyka}

\handdiagramv[\nextboard]{\vhand{T832}{AJ642}{Q62}{8}}
				{\vhand{Q5}{Q7}{KJT9}{AKQJ4}}
                {\vhand{AKJ74}{T9853}{8}{T5}}
                {\vhand{96}{K}{A7543}{97632}}{}

\allbidding{
    \nvul{W} & \vul{N} & \nvul{E} & \vul{S} \\
     & & 1\clubs & 2\diams\alrt \\
    \dbl\alrt & 2\hearts & \dbl & \pass \\
    3\clubs & \pass & 3\hearts & \pass \\
    3\nt
}

2\diams: S do W: stare, N do E: multi. Gracze nie mieli systemu.
Prawidłowe wyjaśnienie to zatem brak ustaleń.
E, przepytywany, stwierdził, że do wyjaśnienia ,,stare'' nie zalicytuje 
pytania o trzymanie 3\hearts, bo ,,będzie się bał o obydwa kolory''.
Do wyjaśnienia ,,brak ustaleń'' nie był w stanie udzielić odpowiedzi.

Rozdanie było ciężkie w analizie, bo gracz, który z 5-4 w starych 
licytuje 2\hearts do multi, może być bardzo nieprzewidywalny w dalszej licytacji.

3\nt-2 płaciło 35\% dla EW mimo chodzącego 4\major na drugiej linii.
5\clubs-1 płaciło 65\%.

Przez brak możliwości dokładnej analizy potencjalnych przebiegów rozdania
bez wykroczenia, zdecydowaliśmy sie rozdać \textbf{60-40}.




\end{document}










































