\documentclass[12pt, a4paper]{article}
\usepackage{import}

\import{../../lib/}{bridge.sty}
\setmainlanguage{polish}
\title{\vspace{-2cm}Minor Puppet Stayman}
\author{Bartek Słupik\\\small{troche poprawione przez}\\Krystyna Gasińska}

\begin{document}
    \maketitle
    \section{Motywacja}
    Dawno dawno temu pewnemu sławnemu brydżyście przyszło następujące rozdanie:
    \begin{hand}[h!]
        \centering
        \vhand{AK92}{97}{AKJ7}{AQx} \we \vhand{T}{AT84}{Q862}{K742}
        \caption{Smutne, nudne 3 bułki}
    \end{hand}

    Na większości stołów licytacja przebiegła wedle schematów oczywistych dla
    wszystkich grających silnym 2\nt\ i Puppet Staymanem:
    \begin{table}[h]
        \centering
        \begin{tabular}{cc}
            2\nt & 3\clubs \\
            3\diams & 3\spades \\
            3\nt & 
        \end{tabular}
    \end{table}

    Mamy spore nadwyżki bilansowe, ale nadal nie jest to inwit do szlemika. Pas na 3\nt\ jest smutny,
    ale konieczny, gdyz 4\clubs\ i 4\diams\ obiecywałoby pięciokart. No to gramy i bierzemy +1. Czy dało się lepiej?

    Na chwilę załóżmy, że żyjemy w utopijnym świecie, w którym partner rozumie nasze pozasystemowe odzywki dokładnie tak,
    jak chcemy. Naturalnie chcielibyśmy poszukać fitu w kolorze młodszym, gdyż krótkość pik pozwoli nam bez
    większych problemów zagrać szlemika. Niech zatem 4\clubs\ będzie pytaniem o skład.

    Partner powinien móc rozróżnić młodsze czwórki i piątki, więc zalicytuje 4\diams\ jako "Mam młodszą czwórkę, ale nie mam piątki".
    Odpowiadamy 4\hearts\ jako "Mam czwókę trefli", partner 4\nt\ "Nie mamy fitu", na co my 6\diams!
    Tak na prawdę już przed rozgrywką możemy zapisać sobie 12 IMP.

    \section{System}
    Sekwencje Minor Puppet Stayman rozpoczynają się zwykle od wyniesienia kontraktu 3\nt przez rękę niezlimitowaną.
    Można później pokusić się o ustalenie rónież innych sekwencji, w których konwencja ta działa.

    \subsection{4\clubs\ --- ?}
    \begin{itemize}
        \item 4\diams\ - Mam młodszą czwórkę, lecz nie mam piątki
        \item 4\hearts\ - 5+\clubs
        \item 4\spades\ - 5+\diams 
        \item 4\nt\ - Brak 4+ \minor\
    \end{itemize}

    \subsubsection{4\clubs\ --- 4\diams \\ ?}
    \begin{itemize}
        \item 4\hearts\ - 4\clubs
        \item 4\spades\ - 4\diams
        \item 4\nt\ - \soff
    \end{itemize}

    \subsubsection{4\clubs\ --- 4\diams \\ 4\hearts\ --- ?}
    \begin{itemize}
        \item 4\spades\ - Fit \clubs\ lepsza ręka
        \item 4\nt\ - \soff
        \item 5\clubs\ - gorsza ręka
    \end{itemize}
    Licytacja po odpowiedzi 4\spades\ analogiczna.

    \subsection{4\clubs\ --- 4\hearts}
    \begin{itemize}
        \item 4\spades\ - Fit \clubs\ lepsza ręka
        \item 4\nt\ - \soff
        \item 5\clubs\ - gorsza ręka
    \end{itemize}
    Licytacja po odpowiedzi 4\spades\ analogiczna.

    \section{Rozszerzenie - pseudotransfery}
    Na początek mała zmiana zwykłej Strefy - odpowiedź 3\spades\ na 2\nt\ wymusza automat 3\nt\ i może oznaczać:
    \begin{itemize}
        \item 6+\minor
        \item 5+/4+ \minor
    \end{itemize}
    Wtedy zyskujemy kolejne dwie odzywki wynoszące 3\nt:
    ...3\nt\ - 4\diams\ (transfer na \clubs) i ...3\nt\ - 4\hearts\ (tr. na \diams)
    \subsection{2\ntx\ --- 3\spades \\ 3\ntx\ --- ?}
    \begin{itemize}
        \item 4\clubs\ - 6+\diams, \textbf{kolory na odwrót} żeby grać z dobrej strony
        \item 4\diams\ - 6+\clubs 
        \item 4\hearts\ - 5+\clubs, 4\diams
        \item 4\spades\ - 4\clubs, 5+\diams 
        \item 4\nt\ - 5+\clubs, 5+\diams
    \end{itemize}
    \subsection{...3\ntx\ --- 4\diams}
    \begin{itemize}
        \item 4\major\ fit, cue
        \item 4\nt\ - \soff
        \item 5\clubs\ fit, gorsza ręka
    \end{itemize}
    Licytacja po 4\hearts\ analogiczna

    \section{Problemy licytacyjne}

    \begin{hand}[h]
        \centering
        \vhand{AKJ7}{K76}{Q3}{AQJ2}
        \webidding{
            \vul{W} & \vul{E} \\
            2\nt & 3\clubs \\
            3\diams & 3\spades \\ 
            3\nt & 4\clubs \\
            4\diams & 4\hearts \\ 
            5\hearts & 6\clubs
        }
        \vhand{9}{AQ43}{K972}{K984}
        \caption{6\nt\ na impasie i podziale, 6\clubs\ idzie}
    \end{hand}

    \begin{hand}[h]
        \centering
        \vhand{KJ53}{A7}{QJ43}{AKQ}
        \webidding{
            \vul{W} & \vul{E} \\
            2\nt & 3\clubs \\
            3\diams & 4\clubs \\ 
            4\diams & 4\hearts \\
            4\nt & 6\diams \\ 
        }
        \vhand{4}{K62}{AKT98}{JT73}
        \caption{6\nt\ na impasie, 6\diams\ idzie}
    \end{hand}

    \begin{hand}[h]
        \centering
        \vhand{AKQ}{A932}{QJ43}{A5}
        \webidding{
            \vul{W} & \vul{E} \\
            2\nt & 3\spades \\
            3\nt & 4\spades \\ 
            6\diams
        }
        \vhand{832}{K6}{AKT98}{JT72}
        \caption{6\nt\ bez szans, 6\diams\ idzie}
    \end{hand}

\end{document}