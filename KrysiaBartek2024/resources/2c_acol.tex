\documentclass[12pt, a4paper]{article}
\usepackage{import}

\import{../../lib/}{bridge.sty}
\setmainlanguage{polish}

\author{Krystyna Gasińska\\\small{troche tekstu ukradzione od}\\Bartek Słpuik}
\date{}
\title{2\clubs\ Acol bez kontroli}

\begin{document}
\maketitle
\section{Granie bez kontroli}
\begin{itemize}
    \item System jest prosty i przejrzysty, nie wymaga szczegółowych ustaleń
    \item Miejsce na pokazanie składu, brak rozjazdów bez fitu na poziomie 5
    \item Silne ręce odpowiadającego i tak da się rozlicytować
\end{itemize}

\section{Otwieranie acola}
Ręka musi spełniać jeden z dwóch warunków:
\begin{itemize}
    \item 23+ PC
    \item 9 lew w ręce lub $\leq 4$ przegrywające
\end{itemize}

\section{Odpowiedzi na acola}
\begin{itemize}
    \item 2\diams\ - dowolna sensowna ręka
    \item 2\hearts, 2\spades, 2\nt, 3\clubs, 3\diams\ = naturalne z piątki,
    co najmniej (3)4 honory (z AKQJT) w kolorze (lub coś co gra jak 4 honory)
\end{itemize}


\pagebreak
\section{2\clubs\ - 2\diams\ --- ?}
\begin{itemize}
    \item 2\hearts - Kokish
    \item 2\spades\ - naturalne, ręka raczej niezrównoważona
    \item 2\nt\ - \bal\ 23-24 PC, może być ze 5\major\ i 4\minor lub 6\anysuit{x}
    \item 3\clubs\ - (5)6+\clubs
    \item 3\diams\ - 6+\diams, raczej tylko długie kara, bo jest problem z pokazywaniem dwukolorówek. 
    Z rękami dwukolorowymi na karach - 1\diams.
    \item 3\hearts, 3\spades, 4\clubs, 4\diams\ - samoustalenia
\end{itemize}

\section{Licytacja relayowa odpowiadającego}
\raggedright
Odpowiadający w swojej drugiej odzywce może zalicytować odzywkę o 1 wyższą, by zapytać się OTW o skład.
Przykładowo: \\[1em]
\webidding{
    2\clubs & 2\diams \\
    2\hearts & \conventional{2\spades}
}
\webidding{
    2\clubs & 2\diams \\
    2\spades & \conventional{2\ntx}
} \\[1em]
\raggedright
Te odzywki nie są naturalne i tylko pytają o skład.

\begin{formal}
    W sekwencji 2\clubs\ - 2\diams\ --- 3\diams\ relay \textbf{nie występuje}. 3\hearts\ i 3\spades\ są naturalne,
    z piątki. Dlatego należy uważać z acolem na karach i ręce z 6\diams\ i 4\major\ otwierać 1\diams.
\end{formal}

Kiedy nie dawać relay'a? Kolor 5+ z czterama z pięciu honorów mogliśmy pokazać w pierwszym kółku.
W drugim kółku - pokazujemy kolor naturalnie, jeśli ma \textbf{dokładnie 3 honory}. No i oczywiście
\diams QJTxx nie jest warte pokazywania - dajmy OTW się odlicytować!

\pagebreak
\section{Smaczki dla zaawansowanych}
\raggedright
Pojawia sie problem w sekwencjach typu: \\[1em]

\webidding{
    2\clubs\ & 2\diams \\
    2\spades\ & 2\nt \\
    3\spades\ & ?
} \\[1em] \raggedright

Jak tu silnie ustalić piki? Rozwiązanie - transfery po pytającym relay'u!

\subsection*{2\clubs\ --- 2\diams\ \\ 2\spades\ --- 2\nt\ \\ ?}
\begin{itemize}
    \item 3\clubs\ = 5\spades, 4+\diams
    \item 3\diams\ = 5\spades, 4\hearts
    \item 3\hearts\ = 6+\spades
    \item 3\spades\ = 5\spades, 4\clubs
\end{itemize}

Takie transfery przyjmujemy \textbf{tylko z fitem!} Bez fitu licytujemy naturalnie. Np:

\webidding{
    2\clubs\ & 2\diams \\
    2\spades\ & 2\nt \\
    \conventional{3\hearts} & 3\spades
} \\[1em] \raggedright

3\diams\ pokazało szóstego pika, więc 3\spades\ ustala kiery silnie! Bez fitu dajemy 3\nt.

\raggedright
Jak to się łączy z Kokishem: \\[1em]
\subsection*{2\clubs\ --- 2\diams\ \\ 2\hearts\ --- 2\spades\ \\ ?}
\begin{itemize}
    \item 2\nt\ = \bal\ 24+
    \item 3\clubs = 5\hearts4\diams
    \item 3\diams = 6+\hearts
    \item 3\hearts = 5\hearts4\spades
    \item 3\spades = 5\hearts4\clubs
\end{itemize}

\end{document}