\documentclass[12pt, a4paper]{article}
\usepackage{import}

\import{../../lib/}{bridge.sty}

\begin{document}

\handdiagramh{\hhand{J73}{AJ5}{AK73}{874}}{13}{}{}{\hhand{AKT964}{K82}{JT}{J5}}{12}{}{}{WENS}

\begin{center}
    \begin{tabular}{cccc}
        \vul{W} & \vul{N} & \vul {E} & \vul{S} \\
        -- & -- & \pass & 1\spades \\
        2\clubs & 2\nt & \pass & 3\spades \\
        \pass & 4\spades & \multicolumn{2}{l}{all pass}
    \end{tabular}
\end{center}

W leads A, K and Q\clubs. E follows with T\clubs (discourage), 6\clubs and 7\hearts. You ruff the
third club and play A\spades, both opponents follow. What do you do next? Can you secure the contract?
\begin{center}
    ***
\end{center}

Normally we would play K\spades (and not finesse). 
However if W started with 6 clubs it is now more probable that trumps are 1-3.
From the bidding we know that E has at most 3-4 points, so it is possible he has Q\spades.
Should we, therefore, finesse? You will fail if both finessing queen fails and the second
finesse that
we decide to take (\diams or \hearts) also fails (and if E does not have Q\spades, he may have
Q\diams or Q\hearts).
Can you do better? Yes, we can play K\spades
and if Q\spades doesn't drop, play the third spade. E will take with queen and will be
endplayed. Playing either of the red suits (and that is all E can do now) finesses W's honor.
So this play will succeed with trumps 2-2 as well as with trumps 1-3. 
You can do nothing with trumps 3-1.

Conclusion. Establishing the most probable card distribution is not a goal in itself, it is
just a step to finding the best line of play.

\handdiagramh{\hhand{J73}{AJ5}{AK73}{874}}{13} 
        {\hhand{52}{742}{Q86542}{T6}}{2}
        {\hhand{AKT964}{K82}{JT}{J5}}{12}
        {\hhand{Q8}{QT6}{9}{AKQ932}}{13}
        {WENS}

\end{document}