\documentclass[12pt, a4paper]{article}
\usepackage{import}

\import{../../lib/}{bridge.sty}

\title{Sława 2024}
\author{Krysia \& Bartek}
\begin{document}
\maketitle

\section*{Kongresowy 27.06 15:00}

8. 

\handdiagramv{\vhand{98}{T42}{KT75}{J873}}
        {\vhand{A72}{J53}{AQJ43}{T6}}
        {\vhand{QT53}{AQ}{982}{AQ54}}
        {\vhand{KJ64}{K9876}{6}{K92}}
        {}

\begin{table}[h!]
    \centering
    \begin{tabular}{cccc}
        \nvul{W} & \nvul{N} & \nvul {E} (ja) & \nvul{S} \\
        \pass & \pass & 1\diams & \dbl \\
        \rdbl & 1\hearts & \pass & \pass \\
        \dbl & \pass & \pass & 1\spades \\
        \dbl & 2\clubs & \pass & \pass \\
        \dbl & all \pass & & \\
    \end{tabular}
\end{table}

Po tej obfitej w kontry licytacji przeciwnicy znaleźli się w 2\clubs\dbl.
Wist 6\clubs przejęty królem, wyjście w singla karo do J\diams. Następnie należało wyjść
damą (Lavinthal), żeby partner przebił i wyszedł pikiem do asa. Nie możemy jednak wyjść teraz do kolejnej przebitki,
bo partner będzie wpuszczony. Należy najpierw zebrać asa karo, następnie wyjść ostatnim karem do przebitki.
Rozgrywający może co prawda przebić wysoko, ale nie wyrzuci już przegrywającego kiera na króla karo.
Jedyne co mu zostaje to zebrać ostatnie 2 atuty waletem i wykonać nieudany impas kiera, po 
którym partner zbierze jeszcze króla pik. Zbierzemy:
K\clubs, J\diams, przebitkę karo, A\spades, A\diams, K\hearts i K\spades obkładając bez 2.

Na naszej linii chodzi co prawda 4\hearts, ale z zaledwie 6 par na sali, które znalazły się w tym kontrakcie, 
jedynie 2 zdołały wygrać. Za 2\clubs\xdbl-2 można było zgarnąć >90\%.

\vspace{0.2cm}

5.

\handdiagramv{\vhand{KQ87}{AT92}{A3}{KJ5}}
        {\vhand{--}{QJ7643}{J72}{A643}}
        {\vhand{J432}{5}{T96}{Q9872}}
        {\vhand{AT965}{K8}{KQ854}{T}}
        {NS}

\begin{table}[h!]
    \centering
    \begin{tabular}{cccc}
        \nvul{W} & \vul{N} & \nvul {E} (ja) & \vul{S} \\
        - & \alrts{1\clubs} & \pass & 1\diams \\
        \alrts{1\spades} & 1\nt & \pass & \pass \\
        2\diams & \dbl & \pass & 3\clubs \\
        all \pass & & & \\
    \end{tabular}
\end{table}

1\clubs = 16+. Z ręką \nvul{E} powinnam zablokować 2\hearts, tym bardziej,
że na pewno nie mamy bilansu na końcówkę. Po pasie i jednoznacznym negacie przeciwnika na \vul{S},
partner wszedł 1\spades \small{CRASH}em, pokazując piki i kara (lub kiery i trefle).

\newpage

9.

\handdiagramv{\vhand{AKQJT7}{6}{K842}{K3}}
        {\vhand{4}{AK8432}{QJ6}{QJ7}}
        {\vhand{985}{75}{T97}{T9842}}
        {\vhand{632}{QJT9}{A53}{A65}}
        {EW}

\begin{table}[h!]
    \centering
    \begin{tabular}{cccc}
        \vul{W} & \nvul{N} & \vul {E} (ja) & \nvul{S} \\
        - & 1\spades & 2\hearts & \pass \\
        2\spades & 3\spades & 4\hearts & \pass \\
        \pass & 4\spades & 5\hearts & \pass \\
        6\hearts & all \pass & & \\
    \end{tabular}
\end{table}

Dużo się działo w tej licytacji. 2\spades na \vul{W} nie musi pokazywać fitu,
ale zaalertowałam jako \invp z fitem. Po 3\spades przeciwnika pas jest
\textbf{forsujący}, bo przeciwnicy chcą grać powyżej tresholdu inwitu (3\hearts).
Wyniosłam w 4\hearts, co było raczej poprawne, jeśli zakładamy fit u partnera (jeśli nie,
powinnam spasować, wtedy wyniósłby partner). Po 4\spades przeciwnika należało już spasować,
zagralibyśmy 4\spades\xdbl-2 za blisko 50\%. Po 5\hearts partner spodziewał się ręki typu
\xspades x \xhearts AKxxxxx \xdiams KQxx \xclubs x, z którą szlemik jest górny.

Przeciwnik z \nvul{S} był natomiast bliski wyniesienia w 6\spades\ :)

\newpage

\section*{Kongresowy 28.06 15:00}

4.

\handdiagramv{\vhand{AQ97}{QJ76}{A753}{6}}
        {\vhand{K864}{KT5}{2}{AJT82}}
        {\vhand{JT32}{2}{Q984}{Q753}}
        {\vhand{5}{A9843}{KJT6}{K94}}
        {NSEW}

\begin{table}[h!]
    \centering
    \begin{tabular}{cccc}
        \vul{W} (ja) & \vul{N} & \vul {E} & \vul{S} \\
        1\hearts & \pass & 2\clubs & \pass \\
        2\diams & \pass & 4\hearts & \pass \\
        \pass & \dbl & all \pass & \\
    \end{tabular}
\end{table}

Wist 6\clubs do damy i mojego króla. Czwarte QJ\hearts jest logicznym założeniem po kontrze \vul{N}.
Zagrać należy więc kiera i waleta zabić królem, następnie ściągnąć A\hearts i wyjść 9\hearts. \vul{N} weźmie damą
i w cokolwiek odejdzie, możemy zebrać atuty i dobre trefle.

Więcej problemu mamy, jeśli \vul{S} nie położy damy w pierwszej lewie. Wtedy jedyną linią jest zagranie kiera do damy i króla a następnie
karo do T\diams, z nadzieją, że Q\diams jest u \vul{S}.

Większość graczy wypuszczała nieskontrowane 4\hearts grając kiera do króla i oddając tym samym na J\hearts i Q\hearts.

\newpage

5.

\handdiagramv{\vhand{5}{KQJT76}{74}{T984}}
        {\vhand{AQJ973}{A}{KQJT2}{3}}
        {\vhand{82}{843}{A93}{KJ752}}
        {\vhand{KT64}{952}{865}{AQ6}}
        {NS}

\begin{table}[h!]
    \centering
    \begin{tabular}{cccc}
        \nvul{W} (ja) & \vul{N} & \nvul {E} & \vul{S} \\
        - & 2\diams & 4\diams & \pass \\
        5\diams & \multicolumn{3}{l}{kurtyna...} \\
    \end{tabular}
\end{table}

4\diams = leaping \diams+\major, o czym przydałoby się pamiętać...

\newpage

23.

\handdiagramv{\vhand{AQ72}{A74}{642}{AQT}}
        {\vhand{J64}{JT5}{QT8753}{7}}
        {\vhand{KT983}{Q}{A9}{KJ985}}
        {\vhand{5}{K98632}{KJ}{6432}}
        {NSEW}

\begin{table}[h!]
    \centering
    \begin{tabular}{cccc}
        \vul{W} & \vul{N} & \vul {E} & \vul{S} (ja) \\
        - & - & - & 1\spades \\
        \pass & 2\clubs & \pass & \alrts{3\hearts} \\
        \pass & 4\nt & \pass & 5\hearts \\
        \pass & 5\nt & \pass & 6\clubs \\
        \pass & 7\spades & all \pass & \\
    \end{tabular}
\end{table}

3\hearts to 5+\clubs i singiel \hearts. 6\clubs pokazuje K\clubs (lub dwa pozostałe króle).
Z ręki \vul{N} widać zatem, że kara będzie można wyrzucić na trefla (po czym przebić karo pozostałe w ręce \vul{S}), 
więc kontrakt może położyć jedynie podział atutów 4-0 lub zły podział trefli -- jeśli nie posiadam waleta...

Przeciwnik zawistował 6\hearts odmiennie a ja przepuściłam do damy...\\
Teraz nawet zły podział atutów nie kładzie kontraktu, niestety -- dzieliły się.

\newpage

25.

\handdiagramv{\vhand{AK72}{K73}{963}{864}}
        {\vhand{T}{QJ986}{Q}{KJ9532}}
        {\vhand{94}{A52}{AKJ842}{A7}}
        {\vhand{QJ8653}{T4}{T75}{QT}}
        {EW}

\begin{table}[h!]
    \centering
    \begin{tabular}{cccc}
        \vul{W} & \nvul{N} & \vul {E} & \nvul{S} (ja) \\
        - & \pass & 2\diams & \dbl \\
        2\hearts & \dbl & \pass & \pass \\
        2\spades & \dbl & 3\clubs & \pass \\
        \pass & \dbl & all \pass & \\
    \end{tabular}
\end{table}

2\diams = Wilkosz. 

Wist A\diams, K\diams przebite. Kier zagrany do K\hearts, karo przebite.
Kolejny kier wzięty asem. Następnie należało wyjść pikiem do K\spades partnera, który
zagra K\spades -- przebite. Rozgrywający może wziąć na damę kier. 

\handdiagramv{\hhand{72}{--}{--}{864}}
        {\hhand{}{J9}{--}{KJ9}}
        {\hhand{--}{--}{J84}{A7}}
        {\hhand{QJ8}{--}{--}{QT}}
        {EW}

Następnie, niezależnie
czy rozgrywający postanowi ściągnąć atuty czy ponownie wyjść w kiera, zostanie skrócony.
Ważne, żeby próbę ściągnięcia atutów zabić asem dopiero za drugim razem -- po czym
wyjść w karo pozbawiając rozgrywającego ostatniego atuta.

% K1064
% 952
% 865
% AQ6
% {KT64}{952}{865}{AQ6}

% AQJ973
% A
% KQJ102
% 3 	
% {AQJ973}{A}{KQJ102}{3}

% 82
% 843
% A93
% KJ752
% {82}{843}{A93}{KJ752}

% AK72
% K73
% 963
% 864 	

% QJ8653
% 104
% 1075
% Q10 	
	
% 10
% QJ986
% Q
% KJ9532 	
	
% 94
% A52
% AKJ842
% A7 

% {AK72}{K73}{963}{864}
% {QJ8653}{104}{1075}{Q10}
% {10}{QJ986}{Q}{KJ9532}
% {94}{A52}{AKJ842}{A7}

\end{document}