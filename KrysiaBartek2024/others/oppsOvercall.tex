\documentclass[12pt, a4paper]{article}
\usepackage{import}

\import{../../lib/}{bridge.sty}

\title{Opps' overcall}
\author{Krystyna Gasińska}
\begin{document}
\maketitle

\section{1\clubs -- (\anysuit{\textbullet}) -- ?}
\begin{itemize}
    \item Po wejściu 1\nt kolory na poziomie 2 są \nf.
    \item Po 1\clubs -- 1\diams/1\nt -- ?\\
    2\clubs = stare.
    \item Transfer na KP jest \invp, ask stopper.
    \item Po wejściu kolorowym kolory na poziomie 1 są transferami, po wejściu \dbl są \nat.
    \item 1\nt po wejściu jest zawsze \nat.
    \item 2\nt i kolory na poziomie 3 są \inv.
    \item Po wejściu 1\nt (lub \dbl) z punktami dajemy \dbl (\rdbl), więc nie ma sensu,
    żeby 2\nt było \nat. Są to młode.\\
    \vspace{0.1cm}
    \hrule
    \item Po wejściach blokiem (do 3\diams włącznie):
    \begin{itemize}
        \item \dbl jest wywoławcza
        \item kolory na poziomie 2 = \nf
        \item kolory na poziomie 3 (razem z 2\nt) = transfer, \invp
        \item transfer na KP to ask stopper, poza
        1\clubs -- (2\diams) -- ?\\
        3\clubs = stare (\gf),\\
        3\spades = ask stopper.
    \end{itemize}
\end{itemize}





\end{document}