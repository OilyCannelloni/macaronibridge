\documentclass[12pt, a4paper]{report}
\usepackage{import}

\import{../../lib/}{bridge.sty}
\setmainlanguage{english}

\usepackage[colorlinks=true, linkcolor=blue]{hyperref}
\usepackage[most]{tcolorbox}
\usepackage{tikz}
\usepackage{changepage}

\newcommand{\qqq}{{\color{white}{\textbf{\colorbox{red}{\large{?}}}}}}
\newcommand{\iii}{{\color{white}{\textbf{\colorbox{green}{\large{!}}}}}}

\title{\spades\clubs Strefa \xdiams\xhearts}
\author{Krysia \& Oliwia}
\begin{document}
\maketitle

\section*{\colorbox{blue!30}{Ogólne ustalenia}}
\addcontentsline{toc}{section}{Ogólne ustalenia}
\begin{itemize}
    \item Strefa
    \item Acol Bartka (brak kontroli + transfery + Kokish)
    \item 2 over 1 bez pokazywania siły
    \item niepoważne 3\nt
    \item proste Gazilli
    \item LSF
    \item Michaels (pełen przedział siły)
    \item \nat\ bloki (2\diams też) -- albo patrz: \textbf{propozycja} na końcu \qqq
    \item \textbf{brak pytania o asy 4\nt na młodych} 4\nt = \nat\ \iii
    \item W dwustronnej podnienia z trójki + kontra bilansowa (nie fit) \iii
    \item Exclusion 03/14
    \item 1\diams - 2\diams = inverted (\invp, po nim trzymania, nie nat) \qqq
    \item rewersy nie-\gf\ \qqq
    \item na pytanie o asy odpowiadamy bez króli
    \item kolorowe króle (ustalony kolor = brak króli)
    \item DONT (obrona po 1\nt\dbl) \qqq
    \item Transfer Lebensohl po wejściach na 1\nt
    \item obrona Jassema (na 1\nt)
    \item obrona na Multi Kokisha (pała punkty) \qqq
    \item CRASH po silnych otwarciach \qqq
    \item transfer na karo po (1\major) 1\nt i po (2\major) 2\nt\
    \item transfery po 1\major (\dbl)
    \item do (naszego) wejścia na poziomie 1 wszystko \fonce\ z czwórki \qqq
    \item do (naszego) wejścia na poziomie 2 wszystko \nf\ (z piątki...?) \qqq
    \item Lavithal, wist odmienny, zrzutki wiadomo
\end{itemize} 

\newpage

\section*{\colorbox{blue!30}{Po otwarciu 1\major }}
\addcontentsline{toc}{section}{Po otwarciu 1\major }

-- TODO --

\section*{\colorbox{blue!30}{1\major\ (\dbl)}}
\addcontentsline{toc}{section}{1\major x}

-- TODO --

\section*{\colorbox{blue!30}{Sekwencje z LSF}}
\addcontentsline{toc}{section}{Sekwencje z LSF}

-- TODO --

\section*{\colorbox{blue!30}{Rewersy \qqq}}
\addcontentsline{toc}{section}{Rewersy}

-- TODO --

\section*{\colorbox{blue!30}{Podniesienia z trójki \iii}}
\addcontentsline{toc}{section}{Podniesienia z trójki}

-- TODO --

\section*{\colorbox{blue!30}{Obrona na Multi Kokisha \qqq}}
\addcontentsline{toc}{section}{Obrona na Multi Kokisha}

-- TODO --

\section*{\colorbox{blue!30}{Licytacja szlemikowa na \minor \iii}}
\addcontentsline{toc}{section}{Licytacja szlemikowa na \minor}

-- TODO --

\section*{\colorbox{blue!30}{Michaels -- nasz lub przeciwnika}}
\addcontentsline{toc}{section}{Michaels}

-- TODO: coś tam trzeba ustalić kiedyś --

\section*{\colorbox{blue!30}{Propozycja: szwedzkie otwarcia 2\major \qqq}}
\addcontentsline{toc}{section}{Propozycja: szwedzkie otwarcia 2\major}

Z Bartkiem gram teraz otwarcia 2\major w sile 9-12 i to jest imo super.\\
Mamy wtedy 2\diams jako Multi czyli jak wszyscy.\\
Co to daje? Mamy co zrobić z rękami za silnymi na blok, za słabymi na otwarcie,
np.
\chhand{AKJxxx}{Qxx}{xx}{xx}
albo
\chhand{x}{AQJxxx}{Kx}{xxx}
Fajnie by było z nimi blokować, ale fajnie by też było nie grać 2\major zamiast końcówki.\\
Destruktywne bloki $\leq$ 8 nadal są (w Multi).

Po tym jest trochę systemu, ale w sumie głównie \nat, jedyna przeszkoda to
znajdowanie kierów po 2\spades (bo 2\major jest z dowolną ręką 9-12, może mieć 4OM).

Dodatkowy argument: Kacper tak gra.

Oczywiście są też wady, czasem zagramy bezszansowną końcówkę, gdy cała sala zagra 2\hearts (patrz system poniżej),
ale rzadko znajdziemy się w złej końcówce.

\textbf{I jeszcze jedna propozycja}, w sumie niezależna od powyższej:\\
Ażeby nasze wejścia 2\major na 1\minor były \textbf{silne} (tzn. też jakieś 9-12).\\
To jest podobno w mecie i mi się mega podoba: po pierwsze jak z silniejszą ręką i 6\major
wchodzimy 1\major, i przeciwnicy coś tam jeszcze mówią, to nie odprzedaliśmy ani siły
ani 6-kartu, np:

\noindent
(1\clubs) -- 1\hearts -- (1\spades) -- P\\
(2\spades) -- ?

\noindent
Trzymamy: \chhand{xx}{AQJxxx}{xx}{KQx}

Po drugie wchodząc słabe 2\major specjalnie nie utrudnimy licytacji, jak przeciwnicy mają bilans
to i tak zagrają końcówkę, a po trzecie słabe 2\major łatwo spałować.

Tzn. ,,specjalnie nie utrudnimy'' to głównie w odniesieniu do lepszych przeciwników,
bo na takich, którzy się w dowolnej dwustronnej rozjeżdżają, to może nie.

Po tym nie ma specjalnych ustaleń, wszystko tak jak normalnie po wejściu na poziomie 2.

\noindent
(1\clubs) -- 1\hearts -- (P) -- 1\nt\\
(P) -- ?

Tu 2\hearts jest słabą wersją (8-), 3\hearts z nadwyżką (13+).

\vspace{0.5cm}

Szwedzkie otwarcia:

\subsubsection*{\alrts{2\hearts} -- ?}
\begin{itemize}
    \item 2\spades = 5+\spades\ \fonce
    \item 2\nt = ask \invp\ (bez fitu lub dowolny \gf)
    \item 3\clubs = \gf na młodym (3\diams = 2+ w obu \minor, 3\hearts = 1-\clubs, 3\spades = 1-\diams)
    \item 3\diams = \invp\ z fitem (przyjęcie: 3\nt bez krótkości, inne: cue z krótkości)
    \item 3\hearts = do gry (podniesienie ,,bloku'')
    \item 3\spades/4\minor = splinter
    \item 3\nt = do gry
\end{itemize}

\subsubsection*{\alrts{2\hearts} -- 2\spades \\ ?}
\begin{itemize}
    \item 2\nt = krótkość \spades, lepsza ręka (\fton 3\hearts)
    \item 3\minor = 2\spades, z wartości
    \item 3\hearts = bez fitu, gorsza ręka
    \item 3\spades = z fitem \nf
    \item 3\nt = z fitem bez krótkości, pozostałe cue z krótkości
\end{itemize}

\subsubsection*{\alrts{2\hearts} -- 2\nt \\ ?}
\begin{itemize}
    \item 3\clubs = boczna czwórka (3\diams = ask: \spades/\clubs/diams)
    \item 3\diams = krótkość \minor (3\hearts = do gry, 3\spades = ask)
    \item 3\hearts = słaba ręka bez krótkości
    \item 3\spades = krótkość \spades (nawet jak nie mamy na przyjęcie inwitu!)
    \item 3\nt = lepsza ręka bez krótkości
    \item 4\minor = 7\hearts + krótkość
\end{itemize}

Czasem przekraczamy treshold inwitu nie mając na przyjęcie (np 3\spades = krótkość).
Efekt jest taki, że częściej gramy końcówki :)

\subsubsection*{\alrts{2\spades} -- ?}
\begin{itemize}
    \item 2\nt = ask \invp\ (bez fitu lub dowolny \gf)
    \item 3\clubs = 5+\hearts\ \fonce\ (to jedno bardzo sztuczne miejsce) \imp
    \item 3\diams = \gf\ na młodym (podobnie jak po 2\hearts)
    \item 3\hearts = \invp\ z fitem (przyjęcie: 3\nt bez krótkości, inne: cue z krótkości)
    \item 3\spades = do gry (podniesienie ,,bloku'')
    \item 3\nt = do gry
    \item 4\anysuit{x} = splinter
\end{itemize}

\subsubsection*{\alrts{2\spades} -- 3\clubs \\ ?}
\begin{itemize}
    \item 3\diams = 2\hearts, lepsza ręka
    \item 3\hearts = fit, gorsza ręka
    \item 3\spades = gorsza ręka bez fitu
    \item 3\nt = krótkość \hearts, lepsza ręka
    \item 4\minor = fit + krótkość
\end{itemize}

Po 2\spades -- 2\nt tak jak po 2\hearts.

Ofc tu jest dużo zrobione źle, fit w drugim starym można znajdować znacznie lepiej,
\nt można nie zajmować z nierównej (otwierającej) ręki itd, ale
to usztucznia system (a nie wiem czy Ci się wgl podoba pomysł).
To jest wersja jest do zapamiętania w 3 minuty.

Fun fact: na lidze strzeliłam blefa otwarciem 2\spades trzymając coś w stylu:
\chhand{KJxxxx}{Qxxxx}{xxx}{--}
na trzeciej w zielonych. Bardzo się zmartwiłam, gdy na stole wyjechał podwójny fit.
Na szczęście nam żadna końcówka nie szła, a przeciwnikom szło 3\nt (które grała zała sala) :)

\end{document}