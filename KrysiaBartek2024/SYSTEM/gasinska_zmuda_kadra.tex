\documentclass[12pt, a4paper]{report}
\usepackage{import}

\import{../../lib/}{bridge.sty}
\setmainlanguage{english}

\usepackage[colorlinks=true, linkcolor=blue]{hyperref}
\usepackage[most]{tcolorbox}
\usepackage{tikz}
\usepackage{changepage}

\title{\spades\clubs Strefa \xdiams\xhearts}
\author{Krysia Gasińska \& Oliwia Żmuda}
\begin{document}
\maketitle

\renewcommand{\contentsname}{Spis treści}
\tableofcontents

\chapter*{\colorbox{Plum!30}{Ogólne ustalenia}}
\addcontentsline{toc}{chapter}{Ogólne ustalenia} {

    \vspace{-0.4cm}

    \section*{\colorbox{blue!30}{Licytacja jednostronna}}
    \addcontentsline{toc}{section}{Licytacja jednostronna}
    Otwarcia:
    \begin{itemize}
        \item Strefa (1\clubs = 2+)
        \item 1\diams w słabej sile obiecuje niezbalansowaną rękę\\
        otwarcie 1\clubs zawiera 12-14 \bal z 5\diams
        \item 1\nt = 15-17 \bal
        \item Z 18-20 \bal z 5\diams otwieramy 1\diams
        \item 2\nt = 21-22 \bal
        \item 2\clubs = Acol (\gf lub 23-24 \bal)
        \item 2\diams = Multi
        \item 2\major = 6+\major, 9-12 (szwedzkie)
        \item 3\nt = kontruktywny blok na kolorze starszym
    \end{itemize}

    Dalsza licytacja:
    \begin{itemize}
        \item półpreferencja kolorów starszych
        \item Gazilli po 1\major i po 1\diams (!)
        \item rewersy \textbf{nie są} \gf
        \item schemat \lsf\ (brak krótkości pierwszy)
        \item nie sprzedajemy siły po 2/1 (tylko skład)
        \item niepoważne 3\nt
        \item Blackwood 4\nt tylko na kolorze starszym!\\
        na młodym 4\nt = \nat
        \item Exclusion 03/14
        \item \textit{Kolorowe Króle} (pierwsza odzywka  = brak króli)
        \item na pytanie o asy \textbf{nie} odpowiadamy z królami
        \item pytanie o asy na młodym (szlemowe): 5\minor+1
        \item po acolu \textbf{nie} gramy kontrolami (2\diams = automat)\\
        drugi kolor otwierającego pokazujemy transferem
        \item silne przyjęcia po 1\nt i po 2\nt
        \item oryginalny system uzgadniania koloru młodszego po Puppet Staymanie\\
        patrz: Minor Puppet Stayman
    \end{itemize}

    \section*{\colorbox{blue!30}{Licytacja dwustronna}}
    \addcontentsline{toc}{section}{Licytacja dwustronna}
    \begin{itemize}
        \item Drury i KP (kolor przeciwnika) po wejściu
        \item po naszym wejściu na poziomie 1 nic nie forsuje chyba
        że z przeskokiem (lub KP lub 2\clubs = drury)
        \item po naszym wejściu na poziomie 2 nowy kolor jest \nf
        \item po wejściu przeciwnika nowy kolor na poziomie 2 = \nf,\\
        na poziomie 3 = \gf
        \item niższa z odzywek 2\nt/KP = \invp, wyższa = mixed raise
        \item podniesienia z trójki! (tylko w dwustronnej)
        \item transfery po 1\major (\dbl)
        \item DONT po 1\nt (\dbl)
        \item Lebensohl + Rubensohl po wejściu na 1\nt
        \item Jassem
        \item pełen przedział Michaelsa (nie mini-maxi)
        \item obrona Kokisha na 2\diams = Multi (Wilkosz): \dbl = 14-16 \bal
    \end{itemize}

    \section*{\colorbox{blue!30}{Sygnalizacja}}
    \addcontentsline{toc}{section}{Sygnalizacja}
    \begin{itemize}
        \item wist odmienny
        \item do wistu marka/demarka
        \item Lavinthal (dużo)
        \item ilościówki niezwykle rzadko (w ewidentnych sytuacjach)
        \item w środkowej fazie rozdania wychodzimy zazwyczaj odmiennie,
        czasem Lavinthalem (czasem losowo)
    \end{itemize}

}

\part*{\colorbox{RoyalPurple!30}{Licytacja jednostronna}}
\addcontentsline{toc}{chapter}{Licytacja jednostronna} {

    \chapter*{\colorbox{Plum!30}{Otwarcia na poziomie 1}}
    \addcontentsline{toc}{chapter}{Otwarcia na poziomie 1} {

        \section*{\colorbox{blue!30}{1\clubs}}
        \addcontentsline{toc}{section}{1\clubs} {  
            \subsubsection*{1\clubs\ -- ?}
            \begin{itemize}
                \item 1\diams = negat/młode (bez ręki 16+!)
                \item 2\clubs = \gf\ \nat\ lub \bal (może mieć 4\major)
                \item 2\diams = \gf \nat (półpreferencja)
                \item 2\hearts = Flannery (2\nt = \lsf)
                \item 2\spades = trsf na \nt\ \invp
                \item 3\clubs = blok
                \item 3\anysuit{x} = splinter
            \end{itemize}

            \subsubsection*{1\clubs\ -- \alrts{2\clubs} \\ ?}
            \begin{itemize}
                \item 2\diams\ = \bal (może mieć 4\major) !
                \item 2\major = 5\clubs4\major
                \item 2\nt\ = 5\clubs4\diams !
                \item 3\clubs\ = \clubs
            \end{itemize}

            \subsubsection*{1\clubs\ -- 2\clubs \\
                            2\diams -- ?}
            \begin{itemize}
                \item 2\nt = 12-14 lub 18-20 \bal
                \item 3\nt = 15-17 \bal
                \item inne odzywki pokazują \gf na treflach
            \end{itemize}
        }

        \section*{\colorbox{blue!30}{1\diams}}
        \addcontentsline{toc}{section}{1\diams} { 
            \subsubsection*{1\diams -- ?}
            \begin{itemize}
                \item 2\diams = inverted raise (\invp\ 4+\diams)\\
                    po nim sprzedajemy trzymania! Forsuje do 3\diams
                \item 2\hearts = Flannery
                \item 2\spades = trsf na \nt
                \item 3\diams = blok
                \item 3\anysuit{x} = splinter
            \end{itemize}

            \subsubsection*{1\diams -- 1\major \\
                            ?}
            \begin{itemize}
                \item 1\spades forsuje
                \item 1\nt = Gazilli! dowolne 16+
            \end{itemize}

            \subsubsection*{1\diams -- 1\major \\
                            1\nt -- ?}
            \begin{itemize}
                \item 2\clubs = dowolne 8+ \gf
                \item pozostałe odzywki słabe \nf
            \end{itemize}

            \subsubsection*{1\diams -- 1\hearts \\
                            1\nt -- 2\clubs\\
                            ?}
            \begin{itemize}
                \item 2\diams = 3\hearts
                \item 2\hearts = ustala
                \item 2\nt = 18-20 \bal z 2\hearts
            \end{itemize}

            \subsubsection*{1\diams -- 1\spades \\
                            1\nt -- 2\clubs\\
                            ?}
            \begin{itemize}
                \item 2\diams = 4\hearts
                \item 2\hearts = 3\spades
                \item 2\spades = 4\spades
                \item 2\nt = 18-20 \bal z 2\spades i 2-3\hearts
            \end{itemize}
        }

        \section*{\colorbox{blue!30}{1\major}}
        \addcontentsline{toc}{section}{1\major} { 
            \subsubsection*{1\major -- ?}
            \begin{itemize}
                \item 1\nt = półforsujące
                \item 2\clubs = dowolny \gf
                \item 2\major = 8-10
                \item 2\nt = \invp
                \item 3\clubs = mixed raise
                \item 3\diams = mini splinter
                \item 3\major = blok
                \item 3\nt = niewygodny splinter
            \end{itemize}
            
            \subsubsection*{1\hearts -- ?}
            \begin{itemize}
                \item 2\spades = blok
            \end{itemize}
            
            \subsubsection*{1\spades -- ?}
            \begin{itemize}
                \item 3\hearts = \nat\ \inv\ \nf
                \item 4\hearts = do gry !!!
            \end{itemize}
        }

        \section*{\colorbox{blue!30}{1\nt}}
        \addcontentsline{toc}{section}{1\ntx} {
            \subsubsection*{1\nt -- ?}
        \begin{itemize}
            \item 2\spades = trsf na \clubs lub \inv
            \item 2\nt = trsf na \diams
            \item 3\clubs = Puppet Stayman
            \item 4\clubs = 5\hearts 5\spades
        \end{itemize}

        \vspace{0.3cm}
        Po transferach: Supperaccepty.\\
        Po transferze na kolor starszy, drugi kolor starszy ustala go silnie.\\
        Po 3\clubs również można zastosować Minor Puppet Stayman (patrz: 2\nt).

        \vspace{0.3cm}
        Smolen:\\
        Z (54) \major licytujemy:
        \begin{itemize}
            \item w słabej sile: Stayman, a po 2\diams swoją piątkę 2\major
            \item w sile inwitu: transfer na piątkę, potem drugi stary
            \item silne: Stayman, po 2\diams swoją \textbf{czwórkę} 3\major
        \end{itemize}
        }

        \section*{\colorbox{blue!30}{Podwójny magister}}
        \addcontentsline{toc}{section}{Podwójny magister} {  
            Również w licytacji dwustronnej! Gdy ma to sens.

            Licytacja:
            \vspace{-0.3cm}
            \subsubsection*{1\clubs -- 1\spades\\
                            1\nt -- 2\hearts}
            pokazuje rękę silniejszą niż na Flannery (10-11pc), \then 2\nt = \lsf

            \vspace{0.4cm}
            Po rebidzie 1\spades na 2\clubs nie trzeba odpowiadać 2\diams!
            Zwłaszcza po otwarciu 1\diams, z silną ręką.. Otwierający może
            chcieć spasować na 2\diams.
        }

        \section*{\colorbox{blue!30}{Rewersy}}
        \addcontentsline{toc}{section}{Rewersy} {
            Rebid 1\spades jest forsujący.
            Powtórzenie swojego koloru po rewersie jest slow-downem,
            pokazuje minimum i zwalnia z \gf. Nie obiecuje więcej kart w
            swoim kolorze. Np:
            \subsubsection*{1\clubs -- 1\spades\\
                            2\diams -- 2\spades\\
                            ?}
            2\spades = minimum (7-8pc), 4+\spades\ !
            \begin{itemize}
                \item 2\nt = \gf
                \item 3\minor = \nf
            \end{itemize}

        }
    }

    \chapter*{\colorbox{Plum!30}{Acol 2\clubs}}
    \addcontentsline{toc}{chapter}{Acol 2\clubs} {
        \subsubsection*{2\clubs -- ?}
        \begin{itemize}
            \item 
        \end{itemize}
    }

    \chapter*{\colorbox{Plum!30}{Multi 2\diams}}
    \addcontentsline{toc}{chapter}{Multi 2\diams} {
        \subsubsection*{2\diams -- ?}
        \begin{itemize}
            \item 
        \end{itemize}
    }

    \chapter*{\colorbox{Plum!30}{Otwarcia 2\major}}
    \addcontentsline{toc}{chapter}{Otwarcia 2\major} {

        \section*{\colorbox{blue!30}{2\hearts}}
        \addcontentsline{toc}{section}{2\hearts} { 
            \subsubsection*{2\hearts -- ?}
            \begin{itemize}
                \item 
            \end{itemize}
        }

        \section*{\colorbox{blue!30}{2\spades}}
        \addcontentsline{toc}{section}{2\spades} { 
            \subsubsection*{2\spades -- ?}
            \begin{itemize}
                \item 
            \end{itemize}
        }
    }

    \chapter*{\colorbox{Plum!30}{2{\Large{NT}} i Minor Puppet Stayman}}
    \addcontentsline{toc}{chapter}{2\ntx\ i Minor Puppet Stayman} {
        2\nt = 21-22 \bal, może zawierać 5\major / 6\minor
        \subsubsection*{2\nt -- ?}
        \begin{itemize}
            \item 3\clubs = Puppet Stayman
            \item 3\spades = trsf na 3\nt\ !!!
            \item 3\nt = 5\spades 4\hearts\ !!!
            \item 4\clubs = 5\hearts 5\spades (\then 4\diams = lepsza ręka)
        \end{itemize}
    }

    \vspace{0.5cm}

    W każdej sekwencji, w której nie został ustalony kolor starszy,\\
    (np. 2\nt -- 3\hearts -- 3\spades -- ?)\\
    odzywki 4\minor to \textbf{Minor Puppet Stayman}.

    4\clubs = pytanie o młodsze 5-tki i 4-ki,\\
    4\diams = pytanie o młodsze czwórki.

    \vspace{0.1cm}

    Przykłady:    
    \subsubsection*{2\nt -- 3\clubs\\
                    3\diams -- 3\hearts\\
                    3\nt -- 4\diams\\
                    4\hearts -- 4\nt\\
                    5\clubs -- 5\hearts\\
                    5\spades -- 7\diams}

    Otwierający pokazał: starszą czwórkę (3\diams), brak 4\spades (3\nt),
    obie starsze trójki (4\hearts), lepszą rękę (5\clubs) -- mógł dać 5\diams,
    1/3 asy (5\spades).

    Odpowiadający: zapytał o kolory starsze (3\clubs), pokazał 4\spades,
    zapytał o młodsze trójki (4\diams), ustalił karo (4\nt),
    zapytał o asy (5\hearts).

    \subsubsection*{2\nt -- 3\hearts\\
                    3\spades -- 4\clubs\\
                    4\diams -- 4\hearts\\
                    4\nt -- \pass
                    }

    Otwierający pokazał: 2\spades (brak fitu), brak młodszej piątki,
    ale młodszą czwórkę (4\diams), brak fitu -- 4 trefli (4\nt).
    Wiemy, więc, że ma 2 piki, 4 kara, 2-3 trefli, 4-5 kierów.

    Odpowiadający: pokazał 5 pików, zapytał o młode (4\clubs),
    zapytał o fit do (4) trefli (4\hearts).

    \vspace{0.5cm}

    \textbf{Wyjątkiem}, w którym 4\diams \textbf{nie jest} pytaniem o trójki, jest sekwencja:
    \subsubsection*{2\nt -- 3\clubs\\
                    3\diams -- 4\diams\\
                    ...}
    4\diams pokazuje obie starsze czwórki (ale 4\clubs jest normalnie pytaniem o młode).



    \chapter*{\colorbox{Plum!30}{3\ntx}}
    \addcontentsline{toc}{chapter}{3\ntx} {
        Otwarcie 3\nt jest konstruktywnym blokiem na kolorze starszym 
        (może być w sile otwarcia -- \textit{konstruktywność} zależy od pozycji i założeń)
        \subsubsection*{3\nt -- ?}
        \begin{itemize}
            \item 4\clubs = pokaż kolor transferem
            \item 4\diams = pokaż kolor
        \end{itemize}
    }
}

\end{document}