\documentclass[12pt, a4paper]{report}
\usepackage{import}

\import{../../lib/}{bridge.sty}
\setmainlanguage{english}

\usepackage[colorlinks=true, linkcolor=blue]{hyperref}
\usepackage[most]{tcolorbox}
\usepackage{tikz}
\usepackage{changepage}

\title{\spades\clubs Strefa Żaczek 2025 \xdiams\xhearts}
\author{--}
\begin{document}
\maketitle

%\tableofcontents

%\section{Otwarcia}

Ogólne ustalenia:
\begin{itemize}
	\item Drury
	\item Michaels
	\item Gazilli (proste - do dogadania w poniedziałek)
	\item Podwójny Magister
	\item Puppet Stayman po 2\nt
	\item nowy kolor na poziomie 3 - \gf, na poziomie 2 - \nf
	\item do wejścia na poziomie 2 nowy kolor \textbf{forsuje}, np:\\ (1\spades) - 2\clubs - (\pass) - 2\hearts
	\item do wejścia na poziomie 2 nowy kolor \textbf{nie} forsuje, np:\\ (1\clubs) - 1\hearts - (\pass) - 1\spades
	\item jeśli do dyspozycji jest kolor przeciwnika, to 2\clubs jest \nat (nie Drury)
	\item pytanie o asy 14/03 (odpowiedź 2+Q bez króli)
	\item ,,Kolorowe Króle" (brak, najmłodszy lub 2 pozostałe, ...)
	\item kontra fit (nie lubię jej ale tak chyba będzie najłatwiej)
	\item po wejściu przeciwnika kontrą: rekontra siłowa, reszta tak jak bez kontry
	\item wejścia blokiem 2\hearts/2\spades w sile 8-12!!! (trzeba alertować)
	\item wejście 2\diams\ na 1\clubs = stare
\end{itemize}

%\sequence{{?}}
\vspace{2cm}

\begin{options}[1]
	\item[1\diams] 5\diams w dowolnej sile (18-19\bal z 5\diams również, potem 2\nt)
\item[] lub 4441 - ale np z 4\xspades1\xhearts4\xdiams4\clubs wolę otwierać karo
\item[] (mam wygodny rebid)
    \item[2\clubs] 22-23 \bal\ lub Acol
    \item[2\diams/\hearts/\spades] blok \nat (2\nt\ pytanie)
    \item[2\nt] 20-21 \bal
\end{options}

\sequence{{1\hearts}{?}}
\begin{options}[2]
    \item[1\nt] półforsujące (potem Gazilli)
    \item[2\clubs/2\diams] \gf
    \item[2\hearts] konstruktywne 8-10 + fit
    \item[2\spades] blok \nat
    \item[2\nt] \inv\ + fit
    \item[3\clubs] \inv\ \nat
    \item[3\diams] \inv\ \nat
    \item[3\hearts] blok
\end{options}

\sequence{{1\spades}{?}}
\begin{options}[2]
    \item[1\nt] półforsujące (potem Gazilli)
    \item[2\clubs/2\diams/2\hearts] \gf
    \item[2\spades]konstruktywne 8-10 + fit
    \item[2\nt] \inv\ + fit
    \item[3\clubs] \inv\ \nat
    \item[3\diams] \inv\ \nat
    \item[3\hearts] \inv\ \nat\ \nf
    \item[3\spades] blok
\end{options}

\sequence{{1\nt}{?}}
\begin{options}[2] 
	\item[2\spades] transfer na \clubs lub inwit
	\item[2\nt] transfer na młode lub kara (odpowiadamy lepszym młodszym)
	\item[3\major] młode z singlem \major
	\item[4\diams/\hearts] Texas (również po wejściu przeciwnika)
\end{options}
Po wejściu przeciwnika na 1\nt:\\
kolory na poziomie 2 do gry, na poziomie 3 \gf\\
2\nt Lebensohl

\sequence{{2\clubs}{?}}
\begin{options}[2] 
	\item[2\diams] Automat! (nie kontrole)
\end{options}

\sequence{{2\clubs}{2\diams}{?}}
\begin{options}[1] 
	\item[2\hearts] Kokish (kiery lub 24+ \bal
	\item[2\spades] \nat
	\item[2\nt] 22-23 \bal\ (dalej jak po otw 2\nt)
\end{options}

\newpage

Gazilli

\subsubsection*{1\hearts\ -- 1\spades\ \\ ?}
\begin{itemize}
    \item 2\clubs\ = 5\hearts4\clubs\ 11-15 lub 16+ HCP \fonce
    \item 2\nt = 6\hearts4\minor, (15)16+
\end{itemize}

\subsubsection*{1\hearts\ -- 1\ntx\ \\ ?}
\begin{itemize}
    \item 2\clubs\ = 5\hearts\clubs\ 11-15 lub 16+ HCP \fonce
    \item 2\diams\ = 5\hearts4\diams\ 11-15
    \item 2\hearts\ = 11-15
    \item 2\spades\ = 6\hearts5\spades\ \gf
    \item 2\ntx\ = 6\hearts4\minor\ (15)16+
    \item 3\clubs\ = 5\hearts5\clubs\ \invp
    \item 3\diams\ = 5\hearts5\diams\ \invp
    \item 3\hearts\ = autouzgodnienie
\end{itemize}

\subsubsection*{1\spades\ -- 1\ntx\ \\ ?}
\begin{itemize}
    \item \pass\ = 5332 12-14
    \item 2\clubs\ = 5\spades\clubs\ 11-15 lub 16+ HCP \fonce
    \item 2\diams\ = 5\spades4\diams\ 11-15
    \item 2\hearts\ = 5\spades4\hearts\ 11-15
    \item 2\spades\ = 11-15
    \item 2\ntx\ = 6\spades4\minor\ (15)16+
    \item 3\clubs\ = 5\spades5\clubs\ \invp
    \item 3\diams\ = 5\spades5\diams\ \invp
    \item 3\hearts\ = 5\spades5\hearts\ \invp
    \item 3\spades\ = autouzgodnienie
\end{itemize}

\subsubsection*{1\hearts\ -- 1\spades\ \\ 2\clubs\ -- ?}
\begin{itemize}
    \item \diams\ = 8+
    \item \hearts\ = 2\hearts\ 5-7
    \item \spades\ = dobre 5\spades\ 5-7
    \item 2\ntx\ = 1-\hearts\ 5-7
    \item 3\clubs\ = 6+\clubs\ 5-7
    \item 3\diams\ = 6+\diams\ 5-7
    \item 3\hearts = \hearts\ fit, \gf
\end{itemize}

\subsubsection*{1\hearts\ -- 1\ntx\ \\ 2\clubs\ -- ?}
\begin{itemize}
    \item 2\diams\ = 8+
    \item 2\hearts\ = 2-3\hearts\ 5-7
    \item 2\spades\ = 55\minor\ 5-7
    \item 2\nt\ = 1-\hearts\ 5-7
    \item 3\clubs\ = 6+\clubs\ 5-7
    \item 3\diams\ = 6+\diams\ 5-7
\end{itemize}

\subsubsection*{1\spades\ -- 1\ntx\ \\ 2\clubs\ -- ?}
\begin{itemize}
    \item 2\diams\ = 8+
    \item 2\hearts\ = 5\hearts\ 5-7
    \item 2\spades\ = 2-3\spades\ 5-7
    \item 2\ntx\ = 1-\spades\ 5-7
    \item 3\clubs\ = 6+\clubs\ 5-7
    \item 3\diams\ = 6+\diams\ 5-7
\end{itemize}

\subsubsection*{1\hearts\ -- 1\spades\ \\ 2\clubs\ -- 2\diams \\ ?}
\begin{itemize}
    \item 2\hearts\ = 5\hearts4\clubs\ 11-15
    \item 2\spades\ = 5\hearts, =3\spades\ 16+
    \item 2\ntx\ = 5332 18-20
    \item 3\clubs\ = 5\hearts4\clubs\ 16+
    \item 3\diams\ = 5\hearts4\diams\ 16+
    \item 3\hearts\ = 6\hearts\ 16+
    \item 3\spades\ = 5\hearts4\spades\ \gf
\end{itemize}

\subsubsection*{1\hearts\ -- 1\ntx\ \\ 2\clubs\ -- 2\diams \\ ?}
\begin{itemize}
    \item 2\hearts\ = 5\hearts4\clubs\ 11-15
    \item 2\spades\ = 5\hearts4\spades\ 16+
    \item 2\ntx\ = 5332 18-20
    \item 3\clubs\ = 5\hearts4\clubs\ 16+
    \item 3\diams\ = 5\hearts4\diams\ 16+
    \item 3\hearts\ = 6\hearts\ 16+
\end{itemize}

\subsubsection*{1\spades\ -- 1\ntx\ \\ 2\clubs\ -- 2\diams \\ ?}
\begin{itemize}
    \item 2\hearts\ = 5\spades4\hearts\ 16+
    \item 2\spades\ = 5\spades4\clubs\ 11-15
    \item 2\ntx\ = 5332 18-20
    \item 3\clubs\ = 5\spades4\clubs\ 16+
    \item 3\diams\ = 5\spades4\diams\ 16+
    \item 3\spades\ = 6\spades\ 16+
\end{itemize}

\subsubsection*{1\major -- 1\spades/1\nt\\
                2\nt -- ?}
\begin{itemize}
    \item 3\clubs = \pass/popraw
    \item 3\diams = pytanie, \gf
    \item 3\major = ustala \major
    \item 3\twosuit{\spades}{\hearts} = \nat
\end{itemize}

\subsubsection*{1\major -- 1\spades/1\nt\\
                2\nt -- 3\diams\\
                ?}
\begin{itemize}
    \item 3\hearts = \clubs (3\spades = pytanie o siłę, 3\nt = słabsze)
    \item 3\spades = \diams dobra ręka
    \item 3\nt = \diams
\end{itemize}

\end{document}











