\documentclass[12pt, a4paper]{article}
\usepackage{import}

\import{../../../lib/}{bridge.sty}
\setmainlanguage{english}

\usepackage[colorlinks=true, linkcolor=blue]{hyperref}
\usepackage[most]{tcolorbox}
\usepackage{tikz}
\usepackage{changepage}

\title{Strefa dla Marcina i Pietrasza na MPJ 2024}
\author{Krysia}
\begin{document}
\maketitle

Alertujemy!!! Otwarcia 2\clubs, 2\diams, 2\nt, wszystko po acolu,
wszystko po 2\nt, Ogusta (i odpowiedzi), inwit z fitem (2\nt), magistra,
[po 1\nt : 2\spades i 2\nt], \gf 2\clubs.

Nie alertujemy nic powyżej 3\nt.

\subsubsection*{1\clubs\ -- ?}
\begin{itemize}
    \item 1\diams = negat, 0-6pc
    \item 1\major = 4+\major
    \item 1\nt = 7-10pc, bez 4\major
    \item 2\clubs = \gf z \clubs
    \item 2\diams = \gf z \diams
    \item 2\hearts = \gf z 6\hearts
    \item 2\spades = \gf z 6\spades
    \item 2\nt = 11-12pc, bez 4\major
    \item 3\nt = do gry
    \item 4\nt = 19-21pc (inwit do 6\nt)
\end{itemize}

\subsubsection*{1\clubs\ -- 1\diams\\
                ?}
\begin{itemize}
    \item 1\major = 3+\major (z trójką nie pasujemy na to)
    \item 1\nt = 18-20pc, potem system jak po otwarciu 1\nt
    \item 2\clubs = 5+\clubs
    \item 2\diams/2\hearts/2\spades = bardzo silna ręka (19+), rewers (5\clubs4\anysuit{x}), ale nie forsuje
    \item 3\clubs = bardzo silna ręka z \clubs (ale nie forsuje)
\end{itemize}

\subsubsection*{1\diams\ -- ?}
\begin{itemize}
    \item 1\major = 4+\major
    \item 1\nt = 7-10pc, bez 4\major
    \item 2\clubs = \gf z \clubs
    \item 2\diams = \invp, 4+\diams (potem pokazujemy zatrzymania, forsuje do 3\diams)
    \item 2\hearts = \gf z 6\hearts
    \item 2\spades = \gf z 6\spades
    \item 2\nt = 11-12pc, bez 4\major
    \item 3\diams = dużo \diams, blok
    \item 3\nt = do gry
    \item 4\nt = 19-21pc (inwit do 6\nt)
\end{itemize}

\subsubsection*{1\major\ -- ?}
\begin{itemize}
    \item \pass = 0-6pc
    \item 1\nt = 7-11pc, bez fitu
    \item \alrts{2\clubs} = \gf (niekoniecznie z treflami, dlatego alert)
    \item 2\diams = \gf z \diams
    \item 2\major = 7-10pc, fit
    \item \alrts{2\nt} = inwit z fitem
    \item 3\major = mixed raise (4+\major, 6-9pc)
    \item 3\nt = do gry
    \item 4\minor = splinter
\end{itemize}

\subsubsection*{1\hearts\ -- ?}
\begin{itemize}
    \item 1\spades = 4+\spades, bez fitu
    \item 2\hearts = 7-10pc, fit
    \item 2\spades = \gf z 6\spades
    \item 3\spades = splinter
\end{itemize}

\subsubsection*{1\spades\ -- ?}
\begin{itemize}
    \item 2\hearts = \gf z \hearts
    \item 2\spades = 7-10pc, fit
    \item 4\hearts = splinter
\end{itemize}

\subsubsection*{1\nt\ -- ?}
\begin{itemize}
    \item 2\clubs = Stayman
    \item 2\diams = transfer na \hearts
    \item 2\hearts = transfer na \spades
    \item 2\spades = transfer na \clubs lub inwit
    \item 2\nt = transfer na \diams
    \item 3\nt = do gry
    \item 4\diams = Texas
    \item 4\hearts = Texas
    \item 4\nt = inwit do 6\nt
\end{itemize}

\subsubsection*{\alrts{2\clubs} -- ?}
Po acolu jesteśmy sforsowani do końcówki!
\begin{itemize}
    \item \alrts{2\diams} = automatyczna odzwyka!!!
\end{itemize}

\subsubsection*{\alrts{2\clubs} -- \alrts{2\diams}\\
                ?}
\begin{itemize}
    \item 2\hearts = 5+\hearts
    \item 2\spades = 5+\spades
    \item 2\nt = \bal -- potem system jak po otwarciu 2\nt
    \item 3\clubs = 5+\clubs
    \item 3\diams = 5+\diams
\end{itemize}

\subsubsection*{\alrts{2\clubs} -- \alrts{2\diams}\\
                2\hearts -- ?}
\begin{itemize}
    \item \alrts{2\spades} = sztuczne pytanie o skład! bez fitu
\end{itemize}

\vspace{0.3cm}

\alrts{2\diams}, 2\hearts, 2\spades = BLOKI z 6+\\
\alrts{2\nt} = Ogust (pytanie o siłę)\\
nowy kolor forsuje na kółko (\fonce)

\subsubsection*{2\anysuit{x} -- \alrts{2\nt}\\
                ?}
\begin{itemize}
    \item \alrts{3\clubs} = mało siły, słaby kolor
    \item \alrts{3\diams} = mało siły, dobry kolor
    \item \alrts{3\hearts} = dużo siły, słaby kolor
    \item \alrts{3\spades} = dużo siły, dobry kolor
    \item 3\nt = w sumie wyjęło mi się 2\anysuit{x} zamiast 1\anysuit{x}
\end{itemize}

\subsubsection*{\alrts{2\nt} -- ?}
\alrts{2\nt} = 21-22pc, \bal, może być starsza piątka, młodsza szóstka, singlowa figura\\
\begin{itemize}
    \item \alrts{3\clubs} = Puppet Stayman
    \item \alrts{3\diams} = transfer na \hearts
    \item \alrts{3\hearts} = transfer na \spades
    \item 3\nt = do gry
    \item 4\clubs = \gf, ustala trefle, bezpośrednie 4\nt jest propozycją gry
    \item 4\diams = \gf, ustala karo, bezpośrednie 4\nt jest propozycją gry
\end{itemize}

przyjęcie transferu = BRAK FITU\\
z fitem dajemy cuebid lub 3\nt (propozycyjne z 3-kartowym fitem)

\subsubsection*{\alrts{2\nt} -- \alrts{3\clubs}\\
                ?}
\begin{itemize}
    \item \alrts{3\diams} = 4\major, brak 5\major
    \item \alrts{3\hearts} = 5\hearts (3\spades fituje kiery! szlemikowe)
    \item \alrts{3\spades} = 5\spades (4\hearts fituje piki! szlemikowe)
    \item 3\nt = brak 4\major
\end{itemize}

\subsubsection*{\alrts{2\nt} -- \alrts{3\clubs}\\
                3\diams -- ?}
\begin{itemize}
    \item \alrts{3\hearts} = 4\spades (!!!)
    \item \alrts{3\spades} = 4\hearts (!!!)
    \item 4\diams = obie starsze czwórki
\end{itemize}

odpowiedzi na pytanie o asy:\\
\begin{itemize}
    \item 4\clubs = 1/4 asy (król atu jest asem)
    \item 4\diams = 0/3 asy
    \item 4\hearts = 2 asy bez damy atu
    \item 4\spades = 2 asy z damą atu
\end{itemize}

kolejna odzywka po odpowiedzi to pytanie o damę 
(jeśli jeszcze nie pokazaliśmy damy lub jej braku), odpowiadamy:\\
pierwsza odzywka = brak damy\\
druga odzywka = dama

kolejna odzywka jest pytaniem o króle, odpowiadamy:\\
pierwsza odzywka = 0,\\
druga odzywka = 1 itd.

\newpage

Rewersy\\
pokazują 5-4 i \gf, 18+, na przykład:

1\hearts -- 1\nt\\
2\spades

1\clubs -- 1\hearts\\
2\diams

\vspace{0.3cm}

Magister\\
\textbf{po 3 odzywkach na poziomie 1 bez negatu}\\ 2\clubs jest inwitem (2\diams automat),
2\diams jest \gf.\\
Uwaga na sekwencje w stylu:\\
1\diams -- 1\hearts\\
1\spades -- 2\diams\\
2\diams jest \gf, nie do gry!\\
Możemy też dawać 2\clubs z zamiarem spasowania na 2\diams.\\
Obie odzywki {\color{red}alertujemy} (jako podwójny magister).

\vspace{0.3cm}

Licytacja dwustronna\\
Po naszym wejściu na poziomie 1 nic nie forsuje, chyba że z przeskokiem,\\
2\clubs jest naturalne, kolor przeciwnika jest z fitem (silniejsza ręka), np:\\
(1\diams) -- 1\hearts -- (P) -- \alrts{2\diams}, albo:\\
(1\clubs) -- 1\hearts -- (P) -- \alrts{2\clubs}, ale:\\
(1\diams) -- 1\hearts -- (P) -- 2\clubs jest naturalne.\\
Pamiętamy o istnieniu objaśniaka.\\
Po naszym wejściu (nie-blokiem) na poziomie 2 nowy kolor jest z piątki i forsuje,
kolor przeciwnika jest \invp, pyta o trzymanie, niekoniecznie jest z fitem.\\
Po (naszej) kontrze licytujemy z bilansu -- nic poza kolorem przeciwnika nie forsuje.\\
Kontra na trefla jest wywoławcza, negatywna (po niej 1\diams naturalne).\\
Po wejściu przeciwnika na 1\ntx\ :\\
- po wejściu 2\clubs : \dbl = Stayman, reszta tak jak bez wejścia\\
- po innych wejściach odzywki na poziomie 2 są \nf, na poziomie 3: \gf.\\
Po wejściu przeciwnika kontrą licytujemy tak jakby jej nie było.\\
W ogólności, po wejściach przeciwnika, nowy kolor na poziomie 2 nie forsuje, na poziomie 3 jest \gf.\\
Gramy \textbf{Drury}! (Ale tylko w jednostronnej).

\newpage

Licytacja:\\
1\clubs -- (1\diams) -- 1\hearts\\
pokazaliśmy 4 (nie 5) kierów.

Po naszym bloku kontry są karne, np:\\
2\spades -- (3\hearts) -- \dbl

Wejścia na 1\nt :\\
Wchodzimy tylko z układem! Z punktami (nawet 20pc pasujemy).\\
Natomiast żeby wejść nie trzeba mieć 12pc, tylko solidne kolory i układ.\\
\dbl = 4\major + 5\minor \then 2\clubs = spasuj/popraw, 2\diams = pokaż starszy\\
2\clubs = \major (54) \then 2\diams = pokaż dłuższy, 2\major = wybór samodzielny.\\
2\diams = multi (6+\major) \then 2\hearts = spasuj popraw\\
2\major = 5\major i 4\minor \then 2\nt = pokaż młodszy\\
2\nt = młode (55) \then 3\minor = wybór\\
3\major = więcej starszego, bardziej blokujące

\vspace{0.3cm}

Do wistu dajemy małe jak chcemy kontynuacji w ten kolor, duże jak chcemy zmiany na inny,
środkowe jak nie mamy preferencji. (marka/demarka).

Potem zrzucamy Lavinthal, czyli preferencja koloru (w który kolor
chcemy, żeby partner wyszedł lub w którym mamy figury).

\vspace{0.3cm}

Jak przeciwnicy pytają o system/zrzutki:
\begin{itemize}
    \item gramy strefą
    \item staramy się wistować odmiennie (pierwszy wist, w środkowej fazie gry nie mamy ustaleń, ew. też odmiennie)
    \item do wistu zrzucamy markę/demarkę, staramy się zrzucać Lavinthala
    \item nie zrzucamy ilościówek
    \item nie gramy potwierdzeniem wistu
\end{itemize}

\vspace{0.4cm}

\center
Good luck!

\end{document}