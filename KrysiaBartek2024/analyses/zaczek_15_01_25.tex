
\documentclass[12pt, a4paper]{article}
\usepackage{import}

\import{../../lib/}{bridge.sty}

\title{Żaczek 15.01.2025}
\author{Krysia Gasińska \& Bartek Słupik}

\begin{document}
\maketitle


\pagebreak
\section*{Rozdanie 1}
\handdiagramv{\vhand{93}{Q765}{AJ82}{A98}}
{\vhand{K}{KJT93}{Q764}{KJ7}}
{\vhand{AQ7}{A82}{T953}{QT4}}
{\vhand{JT86542}{4}{K}{6532}}
{}

\begin{table}[h!]
    \centering
    \begin{tabular}{cccc}
        \nvul{W} & \nvul{N} & \nvul{E} & \nvul{S}\\
		  -  & \pass & 1\hearts & \pass \\
          \pass & 1\nt & 2\diams & \pass \\
          2\spades & all \pass & & \\
    \end{tabular}
\end{table}

[KG] Większość sali grała 2 lub 3\nt na naszej linii. 
2\spades jest -1, przez
nieuwagę nie obłożyliśmy, ale za -1 też byłby słaby zapis.
Czy 2\nt na wznówce jest do gry...?

\pagebreak
\section*{Rozdanie 2}
\handdiagramv{\vhand{AJ85}{AJ6}{752}{932}}
{\vhand{T72}{Q95}{43}{KQT74}}
{\vhand{Q943}{K8732}{AQ}{65}}
{\vhand{K6}{T4}{KJT986}{AJ8}}
{NS}

\begin{table}[h!]
    \centering
    \begin{tabular}{cccc}
        \nvul{W} & \vul{N} & \nvul{E} & \vul{S}\\
		  -  &  -  & \pass & 1\hearts \\
          2\diams & 2\hearts & all \pass & \\
    \end{tabular}
\end{table}

[KG] Niezbyt udany wist J\diams dał mi darmową lewę.
Zebrałam atuty oddając lewę na Q\hearts, odwrót w \diams.
Musiałam zdecydować czy zagrać na drugiego K\spades czy
drugą T\spades. Uznałam że trzeci K\spades u \nvul{W}
jest bardzo prawdopodobny, jako że pokazał już
(tylko) 2 blotki kier. Niestety król był drugi.

\pagebreak
\section*{Rozdanie 3}
\handdiagramv{\vhand{AJ874}{QJ85}{J2}{J4}}
{\vhand{K2}{A742}{AK4}{KT92}}
{\vhand{T}{K963}{QT95}{Q863}}
{\vhand{Q9653}{T}{8763}{A75}}
{EW}

\begin{table}[h!]
    \centering
    \begin{tabular}{cccc}
        \vul{W} & \nvul{N} & \vul{E} & \nvul{S}\\
		  -  &  -  &  -  & \pass \\
          \pass & \alrts{2\diams} & 2\nt & \pass \\
          3\hearts & \pass & 3\spades & all \pass \\
    \end{tabular}
\end{table}

[KG] Zaalertowałam 2\diams jako Multi, bo wydawało mi się,
że gramy Wilkoszem na 3-ciej ręce tylko w czerwonych.
Przeciwnicy znaleźli się nieprzyjemnym kontrakcie z
podziałem 5-1. Decyzją sędziego dostaliśmy wynik ważony.

\pagebreak
\section*{Rozdanie 4}
\handdiagramv{\vhand{KT64}{J7}{KT542}{Q2}}
{\vhand{A93}{KQ5}{76}{J9743}}
{\vhand{75}{AT94}{A983}{K86}}
{\vhand{QJ82}{8632}{QJ}{AT5}}
{NSEW}

\begin{table}[h!]
    \centering
    \begin{tabular}{cccc}
        \vul{W} & \vul{N} & \vul{E} & \vul{S}\\
		\pass & \pass & \pass & 1\clubs \\
        \pass & 1\hearts & \pass & 1\nt \\
        all \pass & & & \\
    \end{tabular}
\end{table}

[KG] Przeciwnicy grający szaloną \textit{dubeltówką}
znaleźli się w 1\nt. Rozgrywający pokazał
4-5 kierów i 0-2 piki. Dziadek: 4-5 pików. Mając najwyraźniej wenę, zawistowałam
2\spades, a partner przepuścił, dzięki czemu rozgrywający
wziął pierwszą lewę na 7\spades. +2, 0\%.

\pagebreak
\section*{Rozdanie 5}
\handdiagramv{\vhand{T865}{T85}{T4}{A764}}
{\vhand{KJ}{A32}{KQ962}{QT8}}
{\vhand{A42}{K976}{AJ7}{K93}}
{\vhand{Q973}{QJ4}{853}{J52}}
{NS}

\begin{table}[h!]
    \centering
    \begin{tabular}{cccc}
        \nvul{W} & \vul{N} & \nvul{E} & \vul{S}\\
		  -  & & & \\

    \end{tabular}
\end{table}

\pagebreak
\section*{Rozdanie 6}
\handdiagramv{\vhand{KQJ8}{AK92}{T85}{53}}
{\vhand{4}{QJT65}{Q943}{KT6}}
{\vhand{T9632}{3}{AJ62}{972}}
{\vhand{A75}{874}{K7}{AQJ84}}
{EW}

\begin{table}[h!]
    \centering
    \begin{tabular}{cccc}
        \vul{W} & \nvul{N} & \vul{E} & \nvul{S}\\
		  -  &  -  & \pass & \pass \\
          1\nt & \pass & 2\diams & \pass \\
          2\hearts & \pass & 2\nt & \pass \\
          3\hearts & all \pass & & \\
    \end{tabular}
\end{table}

[KG] Bardzo kuszący był pas na 2\nt, na szczęście się powstrzymałam.
Wist K\spades.
Przeciwnicy mogliby mi utrudnić życie wychodząc w piki i skracając
stół, ale \nvul{N} zdecydował się atakować trefle.

\pagebreak
\section*{Rozdanie 7}
\handdiagramv{\vhand{Q}{7543}{9862}{7642}}
{\vhand{K54}{KQ9}{QJ5}{AK93}}
{\vhand{JT9632}{T82}{KT43}{}}
{\vhand{A87}{AJ6}{A7}{QJT85}}
{NSEW}

\begin{table}[h!]
    \centering
    \begin{tabular}{cccc}
        \vul{W} (K) & \vul{N} & \vul{E} (B) & \vul{S}\\
		  -  &  -  &  -  & 2\spades \\
          2\nt & \pass & 6\nt & all \pass \\
    \end{tabular}
\end{table}

[KG] Po długim namyśle partner zdecydował się na 
wrzucenie 6\nt -- po rozdaniu ustaliliśmy jak 
sprawdzić młode po (2\major) 2\nt. Wist pikowy przejęłam
królem, zaimpasowałam \diams, który ku mojemu wielkiemu zaskoceniu
stał (chociaż faktycznie otwierający coś tam na ten blok
powinien niby mieć). W dodatku pan Henryk z jakiegoś
powodu nie zagrał króla, dając mi, niestety
niewykorzystaną, możliwość postawienia przymusu:

\handdiagramv{}
{\hhand{x}{-}{J}{-}}
{\hhand{x}{-}{K}{-}}
{\hhand{Ax}{-}{-}{-}}
{NSEW}

Co ciekawe, za 12 lew dostaliśmy ~30\%. Nikt nie wstawiał K\diams...?

\pagebreak
\section*{Rozdanie 8}
\handdiagramv{\vhand{QT9}{AT3}{J8}{AQ954}}
{\vhand{A8}{Q954}{KQ975}{T8}}
{\vhand{K7542}{K7}{T2}{KJ32}}
{\vhand{J63}{J862}{A643}{76}}
{}

\begin{table}[h!]
    \centering
    \begin{tabular}{cccc}
        \nvul{W} & \nvul{N} & \nvul{E} & \nvul{S}\\
		\pass & 1\nt & \dbl & 2\clubs \\
        \pass & 2\diams & \dbl & 2\spades \\
        all \pass & & & \\
    \end{tabular}
\end{table}

[KG] Podczas licytacji (i wistu -- w atut) żyłam jeszcze poprzednim rozdaniem...

\pagebreak
\section*{Rozdanie 9}
\handdiagramv{\vhand{984}{}{J653}{AQT932}}
{\vhand{J5}{QJ8754}{KT7}{87}}
{\vhand{AKQT76}{AKT}{Q9}{J5}}
{\vhand{32}{9632}{A842}{K64}}
{EW}

\begin{table}[h!]
    \centering
    \begin{tabular}{cccc}
        \vul{W} & \nvul{N} & \vul{E} & \nvul{S}\\
		  -  & 3\clubs & \pass  & 3\nt \\
            all \pass & & & \\
    \end{tabular}
\end{table}

[KG] Przeciwnicy jako jedyni na sali znaleźli się w 3\nt,
a ja nie trafiłam wistu (6\hearts -- a trafiłam w długość partnera!) 
pozwalając im
wziąć wszystkie lewy... To nie był wygrany stolik.

\pagebreak
\section*{Rozdanie 10}
\handdiagramv{\vhand{973}{A}{K85432}{874}}
{\vhand{AJ52}{T7}{AT9}{J652}}
{\vhand{KQ86}{QJ6}{Q76}{T93}}
{\vhand{T4}{K985432}{J}{AKQ}}
{NSEW}

\begin{table}[h!]
    \centering
    \begin{tabular}{cccc}
        \vul{W} & \vul{N} & \vul{E} & \vul{S}\\
		  -  &  -  & & \\

    \end{tabular}
\end{table}

\pagebreak
\section*{Rozdanie 11}
\handdiagramv{\vhand{K9763}{84}{AKJ2}{63}}
{\vhand{QJ52}{A97}{976}{A72}}
{\vhand{A8}{QT2}{QT83}{J954}}
{\vhand{T4}{KJ653}{54}{KQT8}}
{}

\begin{table}[h!]
    \centering
    \begin{tabular}{cccc}
        \nvul{W} & \nvul{N} & \nvul{E} & \nvul{S}\\
		  -  &  -  &  -  & \\

    \end{tabular}
\end{table}

\pagebreak
\section*{Rozdanie 12}
\handdiagramv{\vhand{JT9843}{K32}{T6}{86}}
{\vhand{}{A764}{AQ542}{JT53}}
{\vhand{652}{T985}{K83}{AQ2}}
{\vhand{AKQ7}{QJ}{J97}{K974}}
{NS}

\begin{table}[h!]
    \centering
    \begin{tabular}{cccc}
        \nvul{W} & \vul{N} & \nvul{E} & \vul{S}\\
		\\

    \end{tabular}
\end{table}

\pagebreak
\section*{Rozdanie 13}
\handdiagramv{\vhand{872}{8765}{93}{AQ76}}
{\vhand{QJT63}{AJ}{KT}{KJ92}}
{\vhand{A5}{QT3}{AJ76542}{T}}
{\vhand{K94}{K942}{Q8}{8543}}
{NSEW}

\begin{table}[h!]
    \centering
    \begin{tabular}{cccc}
        \vul{W} (K) & \vul{N} & \vul{E} (B) & \vul{S}\\
		  -  & \pass & 1\nt & 2\diams \\
          \dbl & 2\hearts & \pass & 3\diams \\
          \pass & \pass & 3\spades & all \pass \\
    \end{tabular}
\end{table}

[KG] Przeciwnicy -- kompletny rozjazd. \vul{S} długo myślał
nad ostatnim  pasem, więc i ja miałam czas do namysłu.
Mam fit, możemy mieć bilans. Ale. Znając Bartka, z 
dobrymi 17 punktami z 5 pikami, nigdy nie otworzy 1\nt.
Ja mam K9 -- może mieć gorszą piątkę, lub 15-16pc.
Mam drugą babcię w kolorze przeciwnika, którego
Bartek nie chciał spałować, więc nie wygląda jakby miała wziąć.
Mam króla kier, który też może nie znaleźć dopełnienia,
bo 2\hearts też nie zostało spałowane. W dodatku A\hearts
może równie dobrze być u \vul{N} (żaden z nich nie pokazał
kierów w licytacji). Bartek ma zapewne punkty w treflach,
ale niekoniecznie długość. 3\spades było idealnym kontraktem,
co się działo w rozgrywce -- nie wiem, ale wzięliśmy +1.

\pagebreak
\section*{Rozdanie 14}
\handdiagramv{\vhand{A}{JT85}{873}{AQ653}}
{\vhand{Q6}{AKQ973}{A}{9872}}
{\vhand{T732}{2}{KJT942}{K4}}
{\vhand{KJ9854}{64}{Q65}{JT}}
{}

\begin{table}[h!]
    \centering
    \begin{tabular}{cccc}
        \nvul{W} & \nvul{N} & \nvul{E} & \nvul{S}\\
		  -  &  -  & 1\hearts & \pass \\
          2\spades & \pass & 4\spades & all \pass \\
    \end{tabular}
\end{table}

[KG] Mimo dość silnej ręki zdecydowałam się na blok, więc ucieszyłam się z dokładu.
Szybki wist 5\hearts. Wzięłam asem, zagrałam Q\spades,
wzięte asem, 8\hearts. Zabiłam, nawet jak położę 9\hearts to
\nvul{S} może obłożyć wychodząc teraz karem i zabierając mi dojście do stołu.
To był słaby przeciwnik i może by na to nie wpadł, ale nie dałam mu szansy,
zabiłam figurą kończąc -2. 
Abstrahując od faktu, że kontrakt jest zawsze -1, czy mogłam znać rozkład kar?
Wist, w końcu w kolor dziadka, był bardzo pewny, właściwie spodziewałam się singla.
Niby przeciwnik nie powinien mieć JT(x) -- zawistowałby waletem. 
Ale skoro
zarówno wist jak i kontynuacja były tak tak szybkie,
a \nvul{S} dołożył dwójkę, która powinna być ilościówką,
a nie mogła być z dubla (\nvul{N} miałby H85), może
singiel u \nvul{S} był do wymyślenia.


\pagebreak
\section*{Rozdanie 15}
\handdiagramv{\vhand{QJT94}{J852}{84}{A8}}
{\vhand{K62}{T64}{QT5}{KJ42}}
{\vhand{A8}{AK973}{AKJ2}{T3}}
{\vhand{753}{Q}{9763}{Q9765}}
{NS}

\begin{table}[h!]
    \centering
    \begin{tabular}{cccc}
        \nvul{W} & \vul{N} & \nvul{E} & \vul{S}\\
		  -  &  -  &  -  & \\

    \end{tabular}
\end{table}

[Dlaczego oni grali w piki i czemu wzięli 11...?]

\pagebreak
\section*{Rozdanie 16}
\handdiagramv{\vhand{K984}{K64}{A96}{KQ8}}
{\vhand{AT63}{A}{QJT72}{J93}}
{\vhand{}{QJT9852}{K84}{A74}}
{\vhand{QJ752}{73}{53}{T652}}
{EW}

\begin{table}[h!]
    \centering
    \begin{tabular}{cccc}
        \vul{W} & \nvul{N} & \vul{E} & \nvul{S}\\
		\\

    \end{tabular}
\end{table}

\pagebreak
\section*{Rozdanie 17}
\handdiagramv{\vhand{AJT63}{JT982}{74}{Q}}
{\vhand{KQ842}{5}{A95}{KJ82}}
{\vhand{97}{AQ64}{QJ863}{64}}
{\vhand{5}{K73}{KT2}{AT9753}}
{}

\begin{table}[h!]
    \centering
    \begin{tabular}{cccc}
        \nvul{W} & \nvul{N} & \nvul{E} & \nvul{S}\\
		  -  & & & \\

    \end{tabular}
\end{table}

\pagebreak
\section*{Rozdanie 18}
\handdiagramv{\vhand{KJ862}{}{QT52}{KT97}}
{\vhand{T94}{98}{A864}{Q532}}
{\vhand{3}{AKJT5432}{9}{AJ8}}
{\vhand{AQ75}{Q76}{KJ73}{64}}
{NS}

\begin{table}[h!]
    \centering
    \begin{tabular}{cccc}
        \nvul{W} & \vul{N} & \nvul{E} & \vul{S}\\
		  -  &  -  & & \\

    \end{tabular}
\end{table}

\pagebreak
\section*{Rozdanie 19}
\handdiagramv{\vhand{J9842}{A4}{A}{AJ832}}
{\vhand{65}{JT98}{J5432}{K6}}
{\vhand{T73}{65}{Q976}{Q954}}
{\vhand{AKQ}{KQ732}{KT8}{T7}}
{EW}

\begin{table}[h!]
    \centering
    \begin{tabular}{cccc}
        \vul{W} & \nvul{N} & \vul{E} & \nvul{S}\\
		  -  &  -  &  -  & \\

    \end{tabular}
\end{table}

\pagebreak
\section*{Rozdanie 20}
\handdiagramv{\vhand{KQ85}{Q85}{AQ962}{9}}
{\vhand{AJ9}{AJ92}{KJ5}{832}}
{\vhand{T32}{K643}{T}{AK754}}
{\vhand{764}{T7}{8743}{QJT6}}
{NSEW}

\begin{table}[h!]
    \centering
    \begin{tabular}{cccc}
        \vul{W} & \vul{N} & \vul{E} & \vul{S}\\
		\pass & 1\diams & \dbl & \rdbl \\
        \pass & \pass & 1\hearts & all \pass \\
    \end{tabular}
\end{table}

[KG] Bez 3, pierwsze (i ostatnie) 100\% tego wieczoru.

\pagebreak
\section*{Rozdanie 21}
\handdiagramv{\vhand{T976}{KQT9654}{2}{4}}
{\vhand{A}{}{J9864}{JT98652}}
{\vhand{KJ3}{A3}{AKT75}{K73}}
{\vhand{Q8542}{J872}{Q3}{AQ}}
{NS}

\begin{table}[h!]
    \centering
    \begin{tabular}{cccc}
        \nvul{W} & \vul{N} & \nvul{E} & \vul{S}\\
		  -  & 4\hearts & 4\nt & \dbl \\
          5\clubs & \pass & \pass & \dbl \\
          all \pass & & & \\
    \end{tabular}
\end{table}

[KG] No i co tu robić z tyloma punktami...? Jak
potem uzgodniliśmy, lepszy byłby pas na 4\nt i
pała (atakująco-obronna?) na 5\clubs.

\pagebreak
\section*{Rozdanie 22}
\handdiagramv{\vhand{43}{754}{QJT85}{AJT}}
{\vhand{AKQ5}{KT8}{K64}{K53}}
{\vhand{J98762}{QJ92}{A3}{6}}
{\vhand{T}{A63}{972}{Q98742}}
{EW}

\begin{table}[h!]
    \centering
    \begin{tabular}{cccc}
        \vul{W} & \nvul{N} & \vul{E} & \nvul{S}\\
		  -  &  -  & 1\clubs & 1\spades \\
          3\clubs & \pass & 3\nt & all \pass \\
    \end{tabular}
\end{table}

[KG] Maraton wypuszczania.
Wist pikiem na 3\nt po takiej licytacji był niezbyt udany,
Q\hearts byłoby o wiele lepsze.
Bartek wziął (potem) na
J\clubs i odwrócił Q\diams, drugie 
\diams wzięłam asem i zapomniawszy już o porażce z rozdania
4 wyszłam małym kierem. Dając rozgrywającemu drugą już
w tym rozdaniu darmową lewę. Swoje -- 2 kara i 2 trefle.

A w tych założeniach chyba lepsze byłoby 2\spades (9-12).

\pagebreak
\section*{Rozdanie 23}
\handdiagramv{\vhand{3}{Q82}{985432}{AK5}}
{\vhand{KJ9}{KJ4}{AT76}{Q72}}
{\vhand{AQ864}{T973}{Q}{T98}}
{\vhand{T752}{A65}{KJ}{J643}}
{NSEW}

\begin{table}[h!]
    \centering
    \begin{tabular}{cccc}
        \vul{W} & \vul{N} & \vul{E} & \vul{S}\\
		  -  &  -  &  -  & \\

    \end{tabular}
\end{table}

\pagebreak
\section*{Rozdanie 24}
\handdiagramv{\vhand{K8}{Q97}{KT42}{QT94}}
{\vhand{QJT96}{J}{953}{K632}}
{\vhand{4}{AKT5432}{AJ8}{75}}
{\vhand{A7532}{86}{Q76}{AJ8}}
{}

\begin{table}[h!]
    \centering
    \begin{tabular}{cccc}
        \nvul{W} & \nvul{N} & \nvul{E} & \nvul{S}\\
		\\

    \end{tabular}
\end{table}

\pagebreak
\section*{Rozdanie 25}
\handdiagramv{\vhand{6}{QJ8642}{Q87}{974}}
{\vhand{AQ973}{T73}{J52}{82}}
{\vhand{KJT84}{K5}{T94}{K63}}
{\vhand{52}{A9}{AK63}{AQJT5}}
{EW}

\begin{table}[h!]
    \centering
    \begin{tabular}{cccc}
        \vul{W} & \nvul{N} & \vul{E} & \nvul{S}\\
		  -  & & & \\

    \end{tabular}
\end{table}

\pagebreak
\section*{Rozdanie 26}
\handdiagramv{\vhand{AQJ7}{T952}{J6}{J32}}
{\vhand{T96}{KJ764}{873}{94}}
{\vhand{K42}{A}{KQ92}{KT875}}
{\vhand{853}{Q83}{AT54}{AQ6}}
{NSEW}

\begin{table}[h!]
    \centering
    \begin{tabular}{cccc}
        \vul{W} & \vul{N} & \vul{E} & \vul{S}\\
		  -  &  -  & & \\

    \end{tabular}
\end{table}

\pagebreak
\section*{Rozdanie 27}
\handdiagramv{\vhand{}{J9875}{AK}{KQJ532}}
{\vhand{KQ85}{AT6}{854}{A76}}
{\vhand{JT74}{KQ43}{QJ92}{9}}
{\vhand{A9632}{2}{T763}{T84}}
{}

\begin{table}[h!]
    \centering
    \begin{tabular}{cccc}
        \nvul{W} & \nvul{N} (B) & \nvul{E} & \nvul{S} (K)\\
		  -  &  -  &  -  & \\
          ?? & & & \\
          ? & 4\spades & \pass & 4\nt \\
          \pass & 5\clubs & \pass & 5\hearts \\
          all \pass & & & \\
    \end{tabular}
\end{table}

[yyy co to były za dymy w licytacji???]

[KG] Pół sali grało 4 lub 5 \hearts z pałą, nam
nie udało się wystarczająco sprowokować przeciwników...

\end{document}        
