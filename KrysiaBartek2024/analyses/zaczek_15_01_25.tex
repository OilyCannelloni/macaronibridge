
\documentclass[12pt, a4paper]{article}
\usepackage{import}

\import{../../lib/}{bridge.sty}

\title{Żaczek 15.01.2025}
\author{Krysia Gasińska \& Bartek Słupik}

\begin{document}
\maketitle


\pagebreak
\section*{Rozdanie 1}
\handdiagramv{\vhand{93}{Q765}{AJ82}{A98}}
{\vhand{K}{KJT93}{Q764}{KJ7}}
{\vhand{AQ7}{A82}{T953}{QT4}}
{\vhand{JT86542}{4}{K}{6532}}
{}

\begin{table}[h!]
    \centering
    \begin{tabular}{cccc}
        \nvul{W} & \nvul{N} & \nvul{E} & \nvul{S}\\
		  -  & \pass & 1\hearts & \pass \\
          \pass & 1\nt & 2\diams & \pass \\
          2\spades & all \pass & & \\
    \end{tabular}
\end{table}

[KG] Większość sali grała 2 lub 3\nt na naszej linii. 
2\spades jest -1, przez
nieuwagę nie obłożyliśmy, ale za -1 też byłby słaby zapis.
Czy 2\nt na wznówce jest do gry...?

[B] Wiemy, że mamy 23 pkt na linii, 
a dajemy przeciwnikowi grać 2\spades w zielonych. 
Przecież nawet 100 może być słabym zapisem! 
Trzeba cos zrobić, pała jest OK ale 2\nt 
lepsze, ze względu na bogatą w wysokie blotki 
kartę. Z niższymi blotkami \dbl bardziej, 
bo 2\nt może się wywalać na sprzedanych złych 
podziałach.
W rozgrywce przeciwnik przebił \xdiams3 
czwórką kier, więc nie wsadziłem ósemki. 
Ale przynajmniej wist w trefla wyszedł.


\pagebreak
\section*{Rozdanie 2}
\handdiagramv{\vhand{AJ85}{AJ6}{752}{932}}
{\vhand{T72}{Q95}{43}{KQT74}}
{\vhand{Q943}{K8732}{AQ}{65}}
{\vhand{K6}{T4}{KJT986}{AJ8}}
{NS}

\begin{table}[h!]
    \centering
    \begin{tabular}{cccc}
        \nvul{W} & \vul{N} & \nvul{E} & \vul{S}\\
		  -  &  -  & \pass & 1\hearts \\
          2\diams & 2\hearts & all \pass & \\
    \end{tabular}
\end{table}

[KG] Niezbyt udany wist J\diams dał mi darmową lewę.
Zebrałam atuty oddając lewę na Q\hearts, odwrót w \diams.
Musiałam zdecydować czy zagrać na drugiego K\spades czy
drugą T\spades. Uznałam że trzeci K\spades u \nvul{W}
jest bardzo prawdopodobny, jako że pokazał już
(tylko) 2 blotki kier. Niestety król był drugi.

[B] \nvul{E} spasował na otwarciu w korzystnych, 
co podnosi szanse na 3 trefle u \nvul{W}. 
Co dołożył \nvul{E} w pierwszej lewie? 
Jesli \xdiams 3 to koniec rozdania.

[KG] A no faktycznie, \nvul{E} spasował to nie ma 6
trefli i dołożył ilościówkę/markę to ma 2 kara, było
do wymyślenia.

\pagebreak
\section*{Rozdanie 3}
\handdiagramv{\vhand{AJ874}{QJ85}{J2}{J4}}
{\vhand{K2}{A742}{AK4}{KT92}}
{\vhand{T}{K963}{QT95}{Q863}}
{\vhand{Q9653}{T}{8763}{A75}}
{EW}

\begin{table}[h!]
    \centering
    \begin{tabular}{cccc}
        \vul{W} & \nvul{N} & \vul{E} & \nvul{S}\\
		  -  &  -  &  -  & \pass \\
          \pass & \alrts{2\diams} & 2\nt & \pass \\
          3\hearts & \pass & 3\spades & all \pass \\
    \end{tabular}
\end{table}

[KG] Zaalertowałam 2\diams jako Multi, bo wydawało mi się,
że gramy Wilkoszem na 3-ciej ręce tylko w czerwonych.
Przeciwnicy znaleźli się nieprzyjemnym kontrakcie z
podziałem 5-1. Decyzją sędziego dostaliśmy wynik ważony. 

\pagebreak
\section*{Rozdanie 4}
\handdiagramv{\vhand{KT64}{J7}{KT542}{Q2}}
{\vhand{A93}{KQ5}{76}{J9743}}
{\vhand{75}{AT94}{A983}{K86}}
{\vhand{QJ82}{8632}{QJ}{AT5}}
{NSEW}

\begin{table}[h!]
    \centering
    \begin{tabular}{cccc}
        \vul{W} & \vul{N} & \vul{E} & \vul{S}\\
		\pass & \pass & \pass & 1\clubs \\
        \pass & 1\hearts & \pass & 1\nt \\
        all \pass & & & \\
    \end{tabular}
\end{table}

[KG] Przeciwnicy grający szaloną \textit{dubeltówką}
znaleźli się w 1\nt. Rozgrywający pokazał
4-5 kierów i 0-2 piki. Dziadek: 4-5 pików. Mając najwyraźniej wenę, zawistowałam
2\spades, a partner przepuścił, dzięki czemu rozgrywający
wziął pierwszą lewę na 7\spades. +2, 0\%.

[B] Partner do zmiany.

[KG] Nie no wist jakiś z dupy, chyba legitnie nie ma
układu na który on jest lepszy od damy. A jak mam
\xspades Q872 to kładąc 9 możesz wypuścić.

\pagebreak
\section*{Rozdanie 5}
\handdiagramv{\vhand{T865}{T85}{T4}{A764}}
{\vhand{KJ}{A32}{KQ962}{QT8}}
{\vhand{A42}{K976}{AJ7}{K93}}
{\vhand{Q973}{QJ4}{853}{J52}}
{NS}

\begin{table}[h!]
    \centering
    \begin{tabular}{cccc}
        \nvul{W} & \vul{N} & \nvul{E} & \vul{S}\\
		  -  & \pass  & 1\nt & \pass \\
		  \pass & \pass 	
    \end{tabular}
\end{table}

Wist \xhearts 6 wzięty w stole (1). Ze względu na brak komunikacji i tempa gra przez karo nie wygląda obiecująco. Zagrałem pika licząc na jaki komunikacyjny wpust S. Ten zabił i odszedł w pika (2).
Teraz \xdiams K. S zabił i wyszedł w trefla do Asa (wyrzuciłem Q), N zagrał kiera, którego przepuciłem i wziąłem kolejnego (3). Teraz trefl, S wskoczył Królem, sciągnął kiera i wyszedł w trefla do J (4).
Ściągnąłem \xspades Q (5) i niestety nie było przymusu (6). Noga warta chyba jakie 75\%.

\pagebreak
\section*{Rozdanie 6}
\handdiagramv{\vhand{KQJ8}{AK92}{T85}{53}}
{\vhand{4}{QJT65}{Q943}{KT6}}
{\vhand{T9632}{3}{AJ62}{972}}
{\vhand{A75}{874}{K7}{AQJ84}}
{EW}

\begin{table}[h!]
    \centering
    \begin{tabular}{cccc}
        \vul{W} & \nvul{N} & \vul{E} & \nvul{S}\\
		  -  &  -  & \pass & \pass \\
          1\nt & \pass & 2\diams & \pass \\
          2\hearts & \pass & 2\nt & \pass \\
          3\hearts & all \pass & & \\
    \end{tabular}
\end{table}

[KG] Bardzo kuszący był pas na 2\nt, na szczęście się powstrzymałam.
Wist K\spades.
Przeciwnicy mogliby mi utrudnić życie wychodząc w piki i skracając
stół, ale \nvul{N} zdecydował się atakować trefle.

\pagebreak
\section*{Rozdanie 7}
\handdiagramv{\vhand{Q}{7543}{9862}{7642}}
{\vhand{K54}{KQ9}{QJ5}{AK93}}
{\vhand{JT9632}{T82}{KT43}{}}
{\vhand{A87}{AJ6}{A7}{QJT85}}
{NSEW}

\begin{table}[h!]
    \centering
    \begin{tabular}{cccc}
        \vul{W} (K) & \vul{N} & \vul{E} (B) & \vul{S}\\
		  -  &  -  &  -  & 2\spades \\
          2\nt & \pass & 6\nt & all \pass \\
    \end{tabular}
\end{table}

[KG] Po długim namyśle partner zdecydował się na 
wrzucenie 6\nt -- po rozdaniu ustaliliśmy jak 
sprawdzić młode po (2\major) 2\nt. Wist pikowy przejęłam
królem, zaimpasowałam K\diams, który ku mojemu wielkiemu zaskoczeniu
stał (chociaż faktycznie otwierający coś tam na ten blok
powinien niby mieć). W dodatku pan Henryk z jakiegoś
powodu nie zagrał króla, dając mi, niestety
niewykorzystaną, możliwość postawienia przymusu:

\handdiagramv{}
{\hhand{x}{-}{J}{-}}
{\hhand{x}{-}{K}{-}}
{\hhand{Ax}{-}{-}{-}}
{NSEW}

Co ciekawe, za 12 lew dostaliśmy ~30\%. Nikt nie wstawiał K\diams...?

\pagebreak
\section*{Rozdanie 8}
\handdiagramv{\vhand{QT9}{AT3}{J8}{AQ954}}
{\vhand{A8}{Q954}{KQ975}{T8}}
{\vhand{K7542}{K7}{T2}{KJ32}}
{\vhand{J63}{J862}{A643}{76}}
{}

\begin{table}[h!]
    \centering
    \begin{tabular}{cccc}
        \nvul{W} & \nvul{N} & \nvul{E} & \nvul{S}\\
		\pass & 1\nt\alrt & \pass & 2\clubs \\
        \pass & 2\diams & \dbl & 2\spades \\
        all \pass & & & \\
    \end{tabular}
\end{table}

[KG] Podczas licytacji (i wistu -- w atut) żyłam jeszcze poprzednim rozdaniem...

\pagebreak
\section*{Rozdanie 9}
\handdiagramv{\vhand{984}{}{J653}{AQT932}}
{\vhand{J5}{QJ8754}{KT7}{87}}
{\vhand{AKQT76}{AKT}{Q9}{J5}}
{\vhand{32}{9632}{A842}{K64}}
{EW}

\begin{table}[h!]
    \centering
    \begin{tabular}{cccc}
        \vul{W} & \nvul{N} & \vul{E} & \nvul{S}\\
		  -  & 3\clubs & \pass  & 3\nt \\
            all \pass & & & \\
    \end{tabular}
\end{table}

[KG] Przeciwnicy jako jedyni na sali znaleźli się w 3\nt,
a ja nie trafiłam wistu (6\hearts -- a trafiłam w długość partnera!) 
pozwalając im
wziąć wszystkie lewy... To nie był wygrany stolik.

\pagebreak
\section*{Rozdanie 10}
\handdiagramv{\vhand{973}{A}{K85432}{874}}
{\vhand{AJ52}{T7}{AT9}{J652}}
{\vhand{KQ86}{QJ6}{Q76}{T93}}
{\vhand{T4}{K985432}{J}{AKQ}}
{NSEW}

\begin{table}[h!]
    \centering
    \begin{tabular}{cccc}
        \vul{W} & \vul{N} & \vul{E} & \vul{S}\\
		  -  &  -  & \pass & \pass \\
		  1\hearts & 3\diams & \dbl & \pass \\
		  4\hearts & 

    \end{tabular}
\end{table}

[B] Cwaniak zagrał kiera do 10 i kolejne 20\%

\pagebreak
\section*{Rozdanie 11}
\handdiagramv{\vhand{K9763}{84}{AKJ2}{63}}
{\vhand{QJ52}{A97}{976}{A72}}
{\vhand{A8}{QT2}{QT83}{J954}}
{\vhand{T4}{KJ653}{54}{KQT8}}
{}

\begin{table}[h!]
    \centering
    \begin{tabular}{cccc}
        \nvul{W} & \nvul{N} & \nvul{E} & \nvul{S}\\
		  -  &  -  &  -  & \pass \\
		  \pass & 1\spades& \pass& 1\nt \\
		  \pass & 2\diams & \pass & 3\diams \\

    \end{tabular}
\end{table}

[B] Jakies 100\% totalnie z dupy, 3\diams (imo poprawne, widać że mają fit jakis) zablokowało przeciwnika.


\pagebreak
\section*{Rozdanie 12}
\handdiagramv{\vhand{JT9843}{K32}{T6}{86}}
{\vhand{}{A764}{AQ542}{JT53}}
{\vhand{652}{T985}{K83}{AQ2}}
{\vhand{AKQ7}{QJ}{J97}{K974}}
{NS}

\begin{table}[h!]
    \centering
    \begin{tabular}{cccc}
        \nvul{W} & \vul{N} & \nvul{E} & \vul{S}\\
	1\nt & \pass & 2\clubs & \pass	\\
	2\spades & \pass & 3\nt 

    \end{tabular}
\end{table}

Wist \xspades J. 11 lew, ale nie wiem co się działo.


\pagebreak
\section*{Rozdanie 13}
\handdiagramv{\vhand{872}{8765}{93}{AQ76}}
{\vhand{QJT63}{AJ}{KT}{KJ92}}
{\vhand{A5}{QT3}{AJ76542}{T}}
{\vhand{K94}{K942}{Q8}{8543}}
{NSEW}

\begin{table}[h!]
    \centering
    \begin{tabular}{cccc}
        \vul{W} (K) & \vul{N} & \vul{E} (B) & \vul{S}\\
		  -  & \pass & 1\nt & 2\diams \\
          \dbl & 2\hearts & \pass & 3\diams \\
          \pass & \pass & 3\spades & all \pass \\
    \end{tabular}
\end{table}

[KG] Przeciwnicy -- kompletny rozjazd. \vul{S} długo myślał
nad ostatnim  pasem, więc i ja miałam czas do namysłu.
Mam fit, możemy mieć bilans. Ale. Znając Bartka, z 
dobrymi 17 punktami z 5 pikami, nigdy nie otworzy 1\nt. \textbf{zniewaga}
Ja mam K9 -- może mieć gorszą piątkę, lub 15-16pc.
Mam drugą babcię w kolorze przeciwnika, którego
Bartek nie chciał spałować, więc nie wygląda jakby miała wziąć.
Mam króla kier, który też może nie znaleźć dopełnienia,
bo 2\hearts też nie zostało spałowane. W dodatku A\hearts
może równie dobrze być u \vul{N} (żaden z nich nie pokazał
kierów w licytacji). Bartek ma zapewne punkty w treflach,
ale niekoniecznie długość. 3\spades było idealnym kontraktem,
co się działo w rozgrywce -- nie wiem, ale wzięliśmy +1.

[B] Dostałem wist w \xhearts 3 XD

\pagebreak
\section*{Rozdanie 14}
\handdiagramv{\vhand{A}{JT85}{873}{AQ653}}
{\vhand{Q6}{AKQ973}{A}{9872}}
{\vhand{T732}{2}{KJT942}{K4}}
{\vhand{KJ9854}{64}{Q65}{JT}}
{}

\begin{table}[h!]
    \centering
    \begin{tabular}{cccc}
        \nvul{W} & \nvul{N} & \nvul{E} & \nvul{S}\\
		  -  &  -  & 1\hearts & \pass \\
          2\spades & \pass & 4\spades & all \pass \\
    \end{tabular}
\end{table}

[KG] Mimo dość silnej ręki zdecydowałam się na blok, więc ucieszyłam się z dokładu.
Szybki wist 5\hearts. Wzięłam asem, zagrałam Q\spades,
wzięte asem, 8\hearts. Zabiłam, nawet jak położę 9\hearts to
\nvul{S} może obłożyć wychodząc teraz karem i zabierając mi dojście do stołu.
To był słaby przeciwnik i może by na to nie wpadł, ale nie dałam mu szansy,
zabiłam figurą kończąc -2. 
Abstrahując od faktu, że kontrakt jest zawsze -1, czy mogłam znać rozkład kar?
Wist, w końcu w kolor dziadka, był bardzo pewny, właściwie spodziewałam się singla.
Niby przeciwnik nie powinien mieć JT(x) -- zawistowałby waletem. 
Ale skoro
zarówno wist jak i kontynuacja były tak tak szybkie,
a \nvul{S} dołożył dwójkę, która powinna być ilościówką,
a nie mogła być z dubla (\nvul{N} miałby H85), może
singiel u \nvul{S} był do wymyślenia.

[B] Nie masz wcale dużo na ten blok. 
W tej odzywce nie chodzi o to, żeby 
koniecznie wyłączyć przeciwnika z 
licytacji jeli masz 2PC. Bardziej 
o to, żeby znajdować takie 
\textbf{dobre końcówki} jak tu - 
wcale nie musisz mieć dubelka kier 
a po rebidzie 3\hearts jest słabo. 
Niestety w tym przypadku bezsystemie 
przynosiło 70\%.


\pagebreak
\section*{Rozdanie 15}
\handdiagramv{\vhand{QJT94}{J852}{84}{A8}}
{\vhand{K62}{T64}{QT5}{KJ42}}
{\vhand{A8}{AK973}{AKJ2}{T3}}
{\vhand{753}{Q}{9763}{Q9765}}
{NS}

\begin{table}[h!]
    \centering
    \begin{tabular}{cccc}
        \nvul{W} & \vul{N} & \nvul{E} & \vul{S}\\
		  -  &  -  &  -  & 1\clubs \\
		  \pass & 1\spades & \pass & 2\diams \\
		  \pass & 2\nt\alrt & \pass & 4\spades

    \end{tabular}
\end{table}

[KG] Dlaczego oni grali w piki i czemu wzięli 11...?

[B] Bo to nadal był ten smieszny pan na \vul{S}.

[KG] Aaa no tak. Wyłożył dziadka mówiąc: \textit{brak fitu to nie choroba}.
Jak Rafcio na kadrze da znowu pytanie o ulubioną maksymę
brydżową to mam nowego kandydata. I jeszcze po rozdaniu 
się nas pytał jakiego wgl kontraktu tu by mógł szukać jak nie 4\spades XD

\pagebreak
\section*{Rozdanie 16}
\handdiagramv{\vhand{K984}{K64}{A96}{KQ8}}
{\vhand{AT63}{A}{QJT72}{J93}}
{\vhand{}{QJT9852}{K84}{A74}}
{\vhand{QJ752}{73}{53}{T652}}
{EW}

\begin{table}[h!]
    \centering
    \begin{tabular}{cccc}
        \vul{W} & \nvul{N} & \vul{E} & \nvul{S}\\
	  \pass & 1\nt & \dbl & 4\diams \\
	  \pass & 4\hearts & \pass & \pass \\
	  4\spades & \dbl & \pass & 5\hearts \\
      all \pass & & & \\
    \end{tabular}
\end{table}

[B] W obronie partner do wymiany. W licytacji 
natomiast jesli już mamy dawać 4\spades 
to lepiej od razu, bo wtedy by se zajęli 
ze złej ręki, co może robić. Sama odzywka 
jest imo graniczna, dobre rozczytanie, że 
na pewno nie obłożymy (krótkoć pik lub 9 
kierów u oppsa) i że 8 lew jest na pewno. 
Raczej na impy taką próbę można podjąć, 
bo 500 vs 450 to nie tak dużo. Na impy 
\dbl też będzie trochę mocniejsza.

[KG] Trochę mnie powstrzymało, że zapytałam
tego cwela w kapeluszu czy 4\diams to Texas, a on
dał mi do zrozumienia, że nie wie xd No a potem
tak jak mówisz, widać, że nie obłożymy.

\pagebreak
\section*{Rozdanie 17}
\handdiagramv{\vhand{AJT63}{JT982}{74}{Q}}
{\vhand{KQ842}{5}{A95}{KJ82}}
{\vhand{97}{AQ64}{QJ863}{64}}
{\vhand{5}{K73}{KT2}{AT9753}}
{}

\begin{table}[h!]
    \centering
    \begin{tabular}{cccc}
        \nvul{W} & \nvul{N} & \nvul{E} & \nvul{S}\\
		  -  & 2\hearts & 2\spades & 4\hearts \\
    \end{tabular}
\end{table}
[B] Wist \xspades K. Rozgrywający zabił asem i oddał \hearts K. Ściągnęlimy młode figury, -2.

\pagebreak
\section*{Rozdanie 18}
\handdiagramv{\vhand{KJ862}{}{QT52}{KT97}}
{\vhand{T94}{98}{A864}{Q532}}
{\vhand{3}{AKJT5432}{9}{AJ8}}
{\vhand{AQ75}{Q76}{KJ73}{64}}
{NS}

\begin{table}[h!]
    \centering
    \begin{tabular}{cccc}
        \nvul{W} & \vul{N} & \nvul{E} & \vul{S}\\
		  -  &  -  & \pass & 4\hearts \\

    \end{tabular}
\end{table}
[B] Nothing hand. Pik, kier, karo, trefl na wiscie trafiony.

\pagebreak
\section*{Rozdanie 19}
\handdiagramv{\vhand{J9842}{A4}{A}{AJ832}}
{\vhand{65}{JT98}{J5432}{K6}}
{\vhand{T73}{65}{Q976}{Q954}}
{\vhand{AKQ}{KQ732}{KT8}{T7}}
{EW}

\begin{table}[h!]
    \centering
    \begin{tabular}{cccc}
        \vul{W} & \nvul{N} & \vul{E} & \nvul{S}\\
		  -  &  -  &  -  & \pass \\
		  1\hearts & 2\hearts & \dbl & 2\nt \\
		  4\hearts & \pass & \pass & 4\spades \\
		  \dbl & 5\clubs & \pass &\pass \\
		  \dbl

    \end{tabular}
\end{table}

[B] \dbl bez alertu, bez zawahania, ``Nie, nie wygrasz 2\hearts''. Z licytacją \nvul{S} się zgadzam. Wyniosłem, bo po takiej sekwencji wygląda, jakby piki się nie dzieliły i chcę uniknąć jakiego skrótowego disastera.

[KG] O wow, nawet nie ogarnęłam, że on miał 5pc XD\\
No i nie spodziewałam się po takiej licytacji, że partner jest tak silny.
I ma 3 asy. 4\hearts idzie, więc obrona niby opłacalna,
ale sala grała 4\spades\dbl-2.

\pagebreak
\section*{Rozdanie 20}
\handdiagramv{\vhand{KQ85}{Q85}{AQ962}{9}}
{\vhand{AJ9}{AJ92}{KJ5}{832}}
{\vhand{T32}{K643}{T}{AK754}}
{\vhand{764}{T7}{8743}{QJT6}}
{NSEW}

\begin{table}[h!]
    \centering
    \begin{tabular}{cccc}
        \vul{W} & \vul{N} & \vul{E} & \vul{S}\\
		\pass & 1\diams & \dbl & \rdbl \\
        \pass & \pass & 1\hearts & all \pass \\
    \end{tabular}
\end{table}

[KG] Bez 3, pierwsze (i ostatnie) 100\% tego wieczoru.

\pagebreak
\section*{Rozdanie 21}
\handdiagramv{\vhand{T976}{KQT9654}{2}{4}}
{\vhand{A}{}{J9864}{JT98652}}
{\vhand{KJ3}{A3}{AKT75}{K73}}
{\vhand{Q8542}{J872}{Q3}{AQ}}
{NS}

\begin{table}[h!]
    \centering
    \begin{tabular}{cccc}
        \nvul{W} & \vul{N} & \nvul{E} & \vul{S}\\
		  -  & 4\hearts & 4\nt & \dbl \\
          5\clubs & \pass & \pass & \dbl \\
          all \pass & & & \\
    \end{tabular}
\end{table}

[KG] No i co tu robić z tyloma punktami...? Jak
potem uzgodniliśmy, lepszy byłby pas na 4\nt i
pała (atakująco-obronna?) na 5\clubs.

[BS] Nie no chyba jest git. Teraz widzę, 
że karta raczej nie wygląda na 11 lew w kiery, 
szczególnie że te są pewnie 4-0. 
Trzeba zatłuc 5m. W mógł 
zrobić pięknego false-carda damą karo w 
pierwszej lewie, wtedy przebitka nie do wymyślenia.

[KG] Czy ja wiem? Po co miałbyś wistować w dubla w kolorze dziadka, 
mając max. singla atu. False-card automatyczny, więc
o niczym nie świadczy. Nie sądzę, żebyś nie miał innego wistu
(jak widać tu miałeś neutralne \xhearts K albo pika),
zwłaszcza, że chyba atutowy brzmi dobrze po takiej licytacji.
Już nie mówiąc o szybkości wistu, ale powiedzmy, że
na to bym nie popatrzyła :)

\pagebreak
\section*{Rozdanie 22}
\handdiagramv{\vhand{43}{754}{QJT85}{AJT}}
{\vhand{AKQ5}{KT8}{K64}{K53}}
{\vhand{J98762}{QJ92}{A3}{6}}
{\vhand{T}{A63}{972}{Q98742}}
{EW}

\begin{table}[h!]
    \centering
    \begin{tabular}{cccc}
        \vul{W} & \nvul{N} & \vul{E} & \nvul{S}\\
		  -  &  -  & 1\clubs & 1\spades \\
          3\clubs & \pass & 3\nt & all \pass \\
    \end{tabular}
\end{table}

[KG] Maraton wypuszczania.
Wist pikiem na 3\nt po takiej licytacji był niezbyt udany,
Q\hearts byłoby o wiele lepsze.
Bartek wziął (potem) na
J\clubs i odwrócił Q\diams, drugie 
\diams wzięłam asem i zapomniawszy już o porażce z rozdania
4 wyszłam małym kierem. Dając rozgrywającemu drugą już
w tym rozdaniu darmową lewę. Swoje -- 2 kara i 2 trefle.

A w tych założeniach chyba lepsze byłoby 2\spades (9-12).

[B] Tak, 2\spades lepsze.

\pagebreak
\section*{Rozdanie 23}
\handdiagramv{\vhand{3}{Q82}{985432}{AK5}}
{\vhand{KJ9}{KJ4}{AT76}{Q72}}
{\vhand{AQ864}{T973}{Q}{T98}}
{\vhand{T752}{A65}{KJ}{J643}}
{NSEW}

\begin{table}[h!]
    \centering
    \begin{tabular}{cccc}
        \vul{W} & \vul{N} & \vul{E} & \vul{S}\\
		  -  &  -  &  -  &  \pass \\
		  \pass & 3\diams 

    \end{tabular}
\end{table}

[B] Brzydka i silna ta karta, ale na 2\spades oppsów muszę się odezwać. Otwieramy w każde 6 pików, więc przegrywam tylko jak trafię u partnerki 5 pików, singla i dużo pkt. Trafiłem za 37\%. Imo w pytę.


\pagebreak
\section*{Rozdanie 24}
\handdiagramv{\vhand{K8}{Q97}{KT42}{QT94}}
{\vhand{QJT96}{J}{953}{K632}}
{\vhand{4}{AKT5432}{AJ8}{75}}
{\vhand{A7532}{86}{Q76}{AJ8}}
{}

\begin{table}[h!]
    \centering
    \begin{tabular}{cccc}
        \nvul{W} & \nvul{N} & \nvul{E} & \nvul{S}\\
		1\spades & \pass & 4\spades & all \pass \\
    \end{tabular}
\end{table}

[B] Wejście jest obowiązkowe. W najgorszym wypadku 
300, ale przeważnie do chodzącej końcówki. 
Fit dostajesz na 90\% bo często wejdę 2\nt 
z singlem kier. 

[KG] No tak. Bałam się, bo w WJ mogą mieć dowolną siłę
na takie coś. Ale w sumie często to jednak jest blok,
a nie bilansowe z 3 kartowym fitem. Zwłaszcza na ostatnim
stole na barometrze.

\pagebreak
\section*{Rozdanie 25}
\handdiagramv{\vhand{6}{QJ8642}{Q87}{974}}
{\vhand{AQ973}{T73}{J52}{82}}
{\vhand{KJT84}{K5}{T94}{K63}}
{\vhand{52}{A9}{AK63}{AQJT5}}
{EW}

\begin{table}[h!]
    \centering
    \begin{tabular}{cccc}
        \vul{W} & \nvul{N} & \vul{E} & \nvul{S}\\
		  -  & 2\diams & \pass & 2\hearts  \\
		  3\clubs

    \end{tabular}
\end{table}

[B] Nic ciekawego.

\pagebreak
\section*{Rozdanie 26}
\handdiagramv{\vhand{AQJ7}{T952}{J6}{J32}}
{\vhand{T96}{KJ764}{873}{94}}
{\vhand{K42}{A}{KQ92}{KT875}}
{\vhand{853}{Q83}{AT54}{AQ6}}
{NSEW}

\begin{table}[h!]
    \centering
    \begin{tabular}{cccc}
        \vul{W} & \vul{N} & \vul{E} & \vul{S}\\
		  -  &  -  & \pass & 1\clubs \\
		  \dbl & \pass & 1\hearts & \dbl \\
		  \pass & 4\spades

    \end{tabular}
\end{table}

[B] Trochę może świr, bo \vul{S} na kontrę nie musi mieć nie wiadomo jakich nadwyżek. Ale to wygląda na jedyny grywalny kontrakt. Nikt na sali się nie odważył, grano jakies 3\nt z obory wyjęte.
Może mogłem pomysleć, żeby ustawić kontrującego na 
wiście, tylko jak? I tak przegram, 
ale obrona będzie musiała coś pograć.

\pagebreak
\section*{Rozdanie 27}
\handdiagramv{\vhand{}{J9875}{AK}{KQJ532}}
{\vhand{KQ85}{AT6}{854}{A76}}
{\vhand{JT74}{KQ43}{QJ92}{9}}
{\vhand{A9632}{2}{T763}{T84}}
{}

\begin{table}[h!]
    \centering
    \begin{tabular}{cccc}
        \nvul{W} & \nvul{N} (B) & \nvul{E} & \nvul{S} (K)\\
		  -  &  -  &  -  & \pass \\
          \pass & 1\clubs & \dbl & \rdbl \\
          1\spades & 2\hearts & 2\spades & 3\hearts \\
          3\spades & 4\spades & \pass & 4\nt \\
          \pass & 5\clubs & \pass & 5\hearts \\
          all \pass & & & \\
    \end{tabular}
\end{table}

[KG] Pół sali grało 4 lub 5\hearts z pałą, nam
nie udało się wystarczająco sprowokować przeciwników...

[BS] \rdbl fuj.

[KG] E tam fuj.

\pagebreak
\section*{Rozdanie z turnieju czwartkowego 16.01.2025}
\handdiagramh{\hhand{AQxx}{Axxxx}{J}{Axx}}{}
{}{}
{\hhand{KJxxxx}{-}{AKxxx}{xx}}{}
{}{}
{}

\begin{table}[h!]
    \centering
    \begin{tabular}{cccc}
        \nvul{W} & \nvul{N} (B) & \nvul{E} & \nvul{S} (K)\\
		  - & - & 2\spades & \pass \\
          2\nt & \pass & 3\clubs & \pass \\
          3\diams & \pass & 4\diams & \dbl \\
          4\hearts & \pass & 5\hearts & \pass \\
          5\nt & \pass & 6\hearts & \dbl \\
          7\spades & \pass & \pass & \dbl \\
          \rdbl & all \pass & & \\
    \end{tabular}
\end{table}

\end{document}        
