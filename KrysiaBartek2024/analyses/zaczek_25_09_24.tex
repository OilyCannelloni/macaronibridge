\documentclass[12pt, a4paper]{article}
\usepackage{import}

\import{../../../../lib/}{bridge.sty}

\title{Żaczek 25.09.24}
\author{Krysia \& Bartek}
\begin{document}
\maketitle

\vspace{-0.5cm}
\section*{Rozdanie 1}
\vspace{-0.5cm}
\handdiagramv
{\vhand{876}{A2}{A752}{J742}}
{\vhand{AQ43}{KQJT4}{Q}{K83}}
{\vhand{KJ5}{987653}{3}{Q65}}
{\vhand{T92}{--}{KJT9864}{AT9}}{}

\vspace{-0.7cm}

\begin{table}[h!]
    \centering
    \begin{tabular}{cccc}
        \nvul{W} (K) & \nvul{N} & \nvul {E} (B) & \nvul{S} \\
        -- & \pass & 1\hearts & \pass \\
        \alrts{1\nt} & \pass & \alrts{2\clubs} & \pass \\
        \alrts{2\diams} & \pass & 2\spades & \pass \\
        \alrts{2\ntx} & \pass & \alrts{3\ntx} & all \pass \\
    \end{tabular}
\end{table}

Zamiast odzywki 2\nt lepsze byłoby 3\diams. 2\nt było pytaniem o skład,
3\nt pokazało 54 i krótkość karo. Wist 7\spades do damy i króla,
walet pik zabity asem. Wyszłam w karo obawiając się jedynie odwrotu w
trefla, który zabrałby mi dojście do lew kierowych i na koniec oddałabym trefla (=).
Odwrót nastąpił w pika i reszta lew była moja (+2), a obrońcy nie wzięli nawet na asa kier.

\hfill -K

\pagebreak
\section*{Rozdanie 2}
\handdiagramv
{\vhand{632}{A765}{Q9}{AK32}}
{\vhand{KQT8}{J83}{754}{864}}
{\vhand{J94}{QT42}{K86}{QT5}}
{\vhand{A75}{K9}{AJT32}{J97}}
{NS}

Po oczywistej licytacji znalazłam się (\nvul{W}) w kontrakcie 1\nt.
Obrońcy ściągnęli 4 trefle i wyszli w kiera, ze stołu małe, od \vul{S} dama, z ręki król.
\vul{S} na czwartego trefla wyrzucił pika, co jeszcze mocniej zasugerowało waleta po stronie \vul{N},
zatem zaimpasowałam go. Odwrót \vul{S} w karo musiałam przejąć asem i pozostało mi jedynie zebrać pika (-2).

\hfill -K

A nie, jednak jestem debilem, trzeba było te dwa piki z góry zebrać i potem zaimpasować waleta, wtedy wezmę -1.
Albo grać na A\hearts u \vul{N} i figury karo rozdzielone, żeby wziąć swoje, ale jak to się nie uda to mogę skończyć bez więcej.

\hfill -K

\pagebreak
\section*{Rozdanie 3}
\handdiagramv
{\vhand{Q543}{AK5}{Q9}{KJ73}}
{\vhand{AT72}{--}{T8542}{Q852}}
{\vhand{J96}{JT97432}{J}{A4}}
{\vhand{K8}{Q86}{AK763}{T96}}
{WE}

\begin{table}[h!]
    \centering
    \begin{tabular}{cccc}
        \vul{W} (K) & \nvul{N} & \vul {E} (B) & \nvul{S} \\
        -- & -- & -- & \alrts{2\diams} \\
        \pass & \alrts{3\diams} & \pass & 4\hearts \\
        all \pass & & & \\
    \end{tabular}
\end{table}

Wist A\diams, K\diams przebity, od partnera 4\diams (marka) i T\diams (Lavinthal).
Rozgrywający zebrał atuty i z nieznanego powodu wyszedł ze stołu
Q\spades. Bartek wskoczył asem (inaczej wypuszczamy) i również
wyszedł w pika. \nvul{S} nie zgadł i podłożył waleta. 
Pozostało mu jedynie zaimpasować trefla,
a jako że impas nie stał, skończył bez 2.

\hfill -K

\pagebreak
\section*{Rozdanie 4}
\handdiagramv{\vhand{K7}{QJ75}{75}{KJ542}}
{\vhand{T93}{T2}{AKJ42}{AQ9}}
{\vhand{AQ654}{AK963}{Q6}{T}}
{\vhand{J82}{84}{T983}{8763}}
{NSEW}

\begin{table}[h!]
    \centering
    \begin{tabular}{cccc}
        \vul{W} (K) & \vul{N} & \vul{E} (B) & \vul{S}\\
        \pass & \pass & 1\nt & 2\spades \\
        all \pass & & & \\
    \end{tabular}
\end{table}

Otwarcie 1\nt zablokowało przeciwników, którzy bez ustaleń nie znaleźli
końcówki kierowej. Zawistowałam w pika, rozgrywający ściągnął atuty i zaimpasował trefla,
Bartek ściągnął A\diams, do którego Krysia zrzuciła \xdiams 3. Bartek (debil) nie ściągnął \xdiams K bo uznał trójkę za markę.
Nie ściągnęliśmy przez to K\diams, +3.

\hfill -KB

No i słusznie uznał 3 za markę, powinnam dać wysokie,
ale odruchowo dawałam ilościówki do wszystkiego z powodu deficytu punktów.

\hfill -K

Nie ufa się w zrzutki, kiedy widać że on ma dubla a Ty resztę lol. Jestem pajacem i tyle.

\hfill -B

Wtf, mam pozycję na markę i tyle XD

\hfill -K

\pagebreak
\section*{Rozdanie 5}
\handdiagramv
{\vhand{AK9832}{KQ96}{5}{JT}}
{\vhand{}{AT4}{AQ832}{A8543}}
{\vhand{JT64}{J82}{T64}{K72}}
{\vhand{Q75}{753}{KJ97}{Q96}}
{NS}

\begin{table}[h!]
    \centering
    \begin{tabular}{cccc}
        \nvul{W} (K) & \vul{N} & \nvul{E} (B) & \vul{S}\\
        -- & 1\spades & \alrts{2\nt} & 3\spades \\
        4\diams & 4\spades & 5\diams & all \pass \\
    \end{tabular}
\end{table}

Po wiście K\spades moją najlepszą szansą (chyba?) jest zagranie
na trefle 2-3 i K\clubs u \vul{S}, mimo otwarcia na \vul{N}. Swoje.

\hfill -K

\pagebreak
\section*{Rozdanie 6}
\handdiagramv{\vhand{KQT}{Q872}{KT}{A985}}
{\vhand{J8}{AK654}{97643}{T}}
{\vhand{9753}{JT93}{Q2}{KQ6}}
{\vhand{A642}{}{AJ85}{J7432}}
{EW}

\begin{table}[h!]
    \centering
    \begin{tabular}{cccc}
        \vul{W} (K) & \nvul{N} & \vul{E} (B) & \nvul{S}\\
         & & \pass & \pass \\
        1\diams & \dbl & 1\hearts & 1\spades \\
        \pass & \pass & 2\spades & \pass \\
        3\clubs & \pass & 3\diams \\  
        all \pass
    \end{tabular}
\end{table}

Uznałem, że ręka \vul{E} jest za słaba na dorzucenie końcówki - możemy łatwo oddawać 3 topy.
Wist \xspades K przepuszczony, następnie \xspades T (??). Krysia wyrzuciła trefla a następnie wzięła 10 lew.
Należało jednak zauważyć, że kiery są prawie stuprocentowo 4-4, przebić 2 z nich a następnie zagrać na kara 2-2 biorąc 11.

\hfill -B

Ja tego do końca nie widzę, tzn. jak są to super, a jak nie są? Zostanę z x\diams xx\clubs x\spades
w ręce i xx\diams x\hearts w stole, nie wiem czy \nvul{N} ma xxx\clubs czy x\diams xx\clubs, jak teraz
źle zagram to wezmę swoje, a tak to zawsze +1.

\hfill -K

\pagebreak
\section*{Rozdanie 7}
\handdiagramv{\vhand{KT9652}{T5}{986}{AK}}
{\vhand{QJ74}{}{KJT5}{QJ754}}
{\vhand{A3}{A8432}{AQ4}{962}}
{\vhand{8}{KQJ976}{732}{T83}}
{NSEW}

\begin{table}[h!]
    \centering
    \begin{tabular}{cccc}
        \vul{W} & \vul{N} (B) & \vul{E} & \vul{S} (K)\\
        -- & -- & -- & 1\hearts \\
        \pass & 1\spades & 2\clubs & \pass \\
        \pass & 2\spades & all \pass & \\
    \end{tabular}
\end{table}

Wznowiłem ,,tylko'' 2\spades oczekując maksymalnie dubla w stole oraz niedzielących się kolorów 
(wejście z QJ -- pewnie ma układ)\\
W rozgrywce chyba nic się nie dzieje. Wist J\diams.\\

\hfill -B

\pagebreak
\section*{Rozdanie 8}
\handdiagramv{\vhand{A82}{AKJT5}{K83}{K9}}
{\vhand{53}{82}{A75}{AT8643}}
{\vhand{KQJT764}{76}{QJ}{J2}}
{\vhand{9}{Q943}{T9642}{Q75}}
{}

\begin{table}[h!]
    \centering
    \begin{tabular}{cccc}
        \nvul{W} & \nvul{N} (B) & \nvul{E} & \nvul{S} (K)\\
        \pass & 1\hearts & \pass & 1\spades \\
        \pass & 2\clubs & \dbl & 2\diams \\
        \pass & 2\nt & \pass & 3\spades \\
        \pass & 4\clubs & \dbl & 4\spades \\
        all \pass & & & \\
    \end{tabular}
\end{table}

Ja z ręką S dałbym 4\spades zamiast 2\diams ale to chyba kwestia stylu.
Pojawił się temat kontr wistowych na cue-bidy: przyjęliśmy ustalenie:
\begin{itemize}
    \item \rdbl = as lub renons
    \item \pass = brak kontroli (\rdbl partnera = as)
    \item inny cue = dobrze położony król lub singiel
\end{itemize}
Czy to dobre nie wiem ale chyba tak się gra.

\hfill -B

Wist Q\clubs wypuszczał, ale zagrałam kiery z góry licząc że dama spadnie.

\hfill -K

\pagebreak
\section*{Rozdanie 9}
\handdiagramv{\vhand{AJ}{KQJ872}{J4}{JT3}}
{\vhand{9}{T943}{QT73}{A962}}
{\vhand{K7654}{6}{AK62}{K75}}
{\vhand{QT832}{A5}{985}{Q84}}
{EW}

\begin{table}[h!]
    \centering
    \begin{tabular}{cccc}
        \vul{W} & \nvul{N} (B) & \vul{E} & \nvul{S} (K)\\
        -- & 1\hearts & \pass & 1\spades \\
        \pass & 2\hearts & \pass & 3\nt \\
        all \pass & & & \\
    \end{tabular}
\end{table}

Zdecydowałem się na pokazanie 14+ mimo tych waletów, zostałem może i słusznie opierdolony.
Jednak jestem na pierwszej w korzystnych i otwarcie 2\hearts może być lekko
naciągane, aby kryć bardzo szeroki przedział multi. Na + jest jednak kolor KQJ, dojście oraz wygląda na to że przeciwnik może łatwo
dać lewę na wiście.\\
Wist 8\clubs.

\hfill -B

Dla potomnych: Bartek pokazał 14+, bo gramy otwarcie 2\major = 10-13.\\
W rozgrywce nic się nie działo, po wiście 8\clubs wziętym K\clubs oddałam 3 trefle i A\hearts.

\hfill -K

\pagebreak
\section*{Rozdanie 10}
\handdiagramv{\vhand{652}{A843}{K975}{A5}}
{\vhand{AQ8}{T652}{A863}{82}}
{\vhand{JT9}{J7}{QT42}{KT64}}
{\vhand{K743}{KQ9}{J}{QJ973}}
{NSEW}

\begin{table}[h!]
    \centering
    \begin{tabular}{cccc}
        \vul{W} & \vul{N} (B) & \vul{E} & \vul{S} (K) \\
        -- & -- & \pass & \pass \\
        2\clubs & \dbl & \pass & 2\diams \\
        all \pass & & & \\
    \end{tabular}
\end{table}

Graliśmy sobie na pierwszym stole i zadziałał u mnie instynkt unikania gry w obronie. Zapolowałem lekko świrową kontrą.

Krysia musiała wynieść śmieci. Wist \xclubs Q zabity asem (1). \xdiams 5 do T i J. W ściągnął 3 piki i zagrał \xhearts K do A (2).
Ze stołu blotka karo do \xdiams A i kontynuacja kiera do K i kiera przebitego w ręce (3). 
Ściągnięty \xclubs K (4). Teraz trefl przebity bez sensu siódemką, co oddawało lewę.

Czy kolor karowy gra się do 10? Krysia policz bo ja nie umiem ale chyba rzeczywiście tak.

\hfill -B

Przepałowałam, wyszło mi że do 10.

\hfill -K

\pagebreak
\section*{Rozdanie 11}
\handdiagramv{\vhand{Q643}{T}{A5432}{865}}
{\vhand{AJT85}{AK93}{97}{J4}}
{\vhand{9}{J8742}{JT86}{AQ9}}
{\vhand{K72}{Q65}{KQ}{KT732}}
{}

\begin{table}[h!]
    \centering
    \begin{tabular}{cccc}
        \nvul{W} & \nvul{N} (B) & \nvul{E} & \nvul{S} (K)\\
        1\clubs & \pass & 1\spades & \pass \\
        1\nt & \pass & 2\clubs & \pass \\
        2\spades & \pass & 4\spades & all \pass \\
    \end{tabular}
\end{table}

Wist \xdiams J do Asa, odwrót \xhearts T. Rozgrywający zakręca \xspades J do Q, trefl do Asa, przebitka, claim. 

\hfill -B

\pagebreak
\section*{Rozdanie 12}
\handdiagramv{\vhand{A82}{Q72}{AKJT}{KJ5}}
{\vhand{T7}{K83}{Q932}{9762}}
{\vhand{KJ964}{A4}{654}{A84}}
{\vhand{Q53}{JT965}{87}{QT3}}
{NS}

\begin{table}[h!]
    \centering
    \begin{tabular}{cccc}
        \nvul{W} & \vul{N} (B) & \nvul{E} & \vul{S} (K)\\
        -- & 1\diams & \pass & 1\spades \\
        \pass & 2\nt & \pass & 3\clubs \\
        \pass & 3\hearts & \pass & 3\spades \\
        \pass & 3\nt & \pass & 4\clubs \\
        \pass & 4\diams & \pass & 4\hearts \\
        \pass & 4\spades & all \pass & \\
    \end{tabular}
\end{table}

Otwarłem karo z czwórki bo tak się umówiliśmy. Natomiast raczej nie powinienem tak robić z 18-20, tylko z 12-14.
Dałem 3NT które chciałem do gry, ale niestety było Non Serious. Na szczęście nie sprowokowałem partnerki zbyt mocno.

\hfill -B

\pagebreak
\section*{Rozdanie 13}
\handdiagramv{\vhand{T7654}{653}{J73}{82}}
{\vhand{KQ}{T742}{KT8}{AJT4}}
{\vhand{82}{AQ9}{A965}{K763}}
{\vhand{AJ93}{KJ8}{Q42}{Q95}}
{NSEW}

\begin{table}[h!]
    \centering
    \begin{tabular}{cccc}
        \vul{W} & \vul{N} (B) & \vul{E} & \vul{S} (K) \\
        -- & \pass & 1\nt & \pass \\
        3\nt & all \pass & & \\
    \end{tabular}
\end{table}

Wist \xhearts 9 nie przemawia do mnie kompletnie. Ale co ja będę gadał, jak 9 wzięła lewę i to jeszcze z dołożoną ilościówką.
To skończyło rozdanie. Ja bym wyjął karo.

\hfill -B

A taką miałam wenę. Bardzo nie chciałam wypuścić wistem spod figury, a u Bartka
mogłam liczyć na maksymalnie waleta. Natomiast faktycznie mogłam w pika.

\hfill -K

Pik może mi podgrywać damę, której rozgrywający nie trafi. Karo żeby wypuszczało to się muszą grube dymy dziać.

\hfill -B

\pagebreak
\section*{Rozdanie 14}
\handdiagramv{\vhand{52}{AQ}{T9643}{JT84}}
{\vhand{AT43}{J87643}{Q}{32}}
{\vhand{QJ87}{T92}{A82}{Q65}}
{\vhand{K96}{K5}{KJ75}{AK97}}
{}

\begin{table}[h!]
    \centering
    \begin{tabular}{cccc}
        \nvul{W} & \nvul{N} (B) & \nvul{E} & \nvul{S} (K)\\
        -- & -- & 2\diams & \pass \\
        2\nt & \pass & 3\diams & \pass \\
        4\hearts \\
    \end{tabular}
\end{table}

Niestety odwaliłem tu pajacerkę turnieju. Wist \xdiams 9 do A i trefl do Asa.
Rozgrywający usunął 2 piki na kara, przeszedł pikiem na stół i zagrał kiera do K i A.
Tu trochę automatycznie wyszedłem w pika. \nvul{W} zabił królem i zagrał kiera.

No i nastąpił kasztan. Pomyślałem o promocji, bo partnerka dołożyła \xhearts 9.
Jednak nie wyszedłem w karo, bojąc się, że to wyjście pod podwójny renons, a \nvul{W} wyrzuci trefla.

Jest to niepoprawne myślenie. Jeśli \nvul{W} nie ma \xclubs K, wyrzuciłby przecież w trzeciej lewie trefla ze stołu, a nie pika.
Zatem wyjście w karo na pewno nie wypuszcza, a daje możliwość pięknego obłożenia końcówki jako jedyni na sali.

\hfill -B

\textbf{{\color{red}I}{\color{orange}l}{\color{LimeGreen}o}{\color{cyan}ś}{\color{blue}c}{\color{purple}i}{\color{red}ó}{\color{orange}w}{\color{LimeGreen}k}{\color{cyan}ę}}
w karach powinnam zrzucić a nie kasztan. Aczkolwiek nie wiem czy to nie jakiś Lavinthal na K\spades, którego spokojnie mogę mieć.

\hfill -K

\pagebreak
\section*{Rozdanie 15}
\handdiagramv{\vhand{AJ96542}{K5}{2}{A87}}
{\vhand{7}{AQ9642}{QJ65}{65}}
{\vhand{QT8}{J8}{K94}{KQJ92}}
{\vhand{K3}{T73}{AT873}{T43}}
{NS}

\vspace{-0.5cm}
\begin{table}[h!]
    \centering
    \begin{tabular}{cccc}
        \nvul{W} & \vul{N} (B) & \nvul{E} & \vul{S} (K) \\
        -- & -- & -- & 1\clubs \\
        \pass & 1\spades & 2\hearts & \pass \\
        \pass & 4\spades & all \pass & \\
    \end{tabular}
\end{table}

Wist Q\diams. Zagrałem jak przedszkolak i wsadziłem Króla natychmiast gryząc się w język 
(choć to nic nie zmienia ale daje szansę na błąd). Oddałem 3 lewy i przyszło trafić pika.
Biorąc pod uwagę, że jeśli \nvul{W} ma Króla, to \nvul{E} wszedł na 9PC, a \nvul{W} nie podniósł z fitem z niezłym 7PC, zagrałem z góry.

Jest to niepoprawne, gdyż mam więcej przesłanek. Gdyby \nvul{W} miał 4 kiery, raczej nie spasowałby na 2\hearts. Zatem \nvul{E} ma ich 6.
\nvul{W} nie ma także \xdiams ATxxxx, bo często wszedłby 3\diams. \nvul{E} nie ma ponadto singla trefl (brak wistu).

W ten sposób \nvul{E} ma zwykle 1-6-4-2, a karta którą ma zdecydowanie wystarcza na wejście. Zatem przesłanka punktowa jest niewystarczająca.
\nvul{E} nadal może mieć \hhand{K}{AQxxxx}{QJx}{xxx}, ale jest to mniej prawdopodobne.

Czy \nvul{W} ze swoją kartą nie powinienbył podnieść w 3\hearts? Raczej nie. 
Król pik jest źle położony oraz nie mamy figury w kierach i partner może wypuścić na wiście.

Co ciekawe za 4\spades= było 20\% bo sala wistowała w kiera.

\hfill -B

\pagebreak
\section*{Rozdanie 16}
\handdiagramv{\vhand{AJ32}{T7}{QT8}{8754}}
{\vhand{6}{K854}{AJ53}{AJT2}}
{\vhand{K95}{A96}{K97642}{3}}
{\vhand{QT874}{QJ32}{}{KQ96}}
{EW}

\begin{table}[h!]
    \centering
    \begin{tabular}{cccc}
        \vul{W} & \nvul{N} (B) & \vul{E} & \nvul{S} (K) \\
        1\spades & \pass & 2\clubs & \pass \\
        2\hearts & \pass & 4\hearts & all \pass \\
    \end{tabular}
\end{table}

Wist 7\clubs. Rozgrywający myśli 3 minuty, po czym bije w stole i gra pika do \xspades T i J. 
Z braku lepszego pomysłu kontynuuję trefla do przebitki. \nvul{S} gra \xhearts A i blotkę, którą \vul{W} bije. Następnie ściąga trefle, przebija pika,
gra karo. \nvul{S} bez sensu wskakuje Królem wypuszczając kontrakt. Jak rozgrywający to przegrał - odeszło w zapomnienie.

\hfill -B

Nooo booo mógł mieć singlową Q\diams...

\hfill -K

\pagebreak
\section*{Rozdanie 17}
\handdiagramv{\vhand{A876}{3}{KQ65}{K853}}
{\vhand{43}{K8642}{T92}{JT7}}
{\vhand{KT95}{QT95}{AJ7}{64}}
{\vhand{QJ2}{AJ7}{843}{AQ92}}
{}

\begin{table}[h!]
    \centering
    \begin{tabular}{cccc}
        \nvul{W} & \nvul{N} (B) & \nvul{E} & \nvul{S} (K) \\
        -- & 1\diams & \pass & 1\hearts \\
        \pass & 1\spades & \pass & 2\spades \\
        all \pass & & & \\
    \end{tabular}
\end{table}

Wist 10\clubs. Jakaś nuda. +2.

\hfill -B

\pagebreak
\section*{Rozdanie 18}
\handdiagramv{\vhand{96532}{A874}{A5}{95}}
{\vhand{J7}{T3}{K8732}{AKQ8}}
{\vhand{KQT8}{J96}{Q964}{J7}}
{\vhand{A4}{KQ52}{JT}{T6432}}
{NS}

\begin{table}[h!]
    \centering
    \begin{tabular}{cccc}
        \nvul{W} & \vul{N} (B) & \nvul{E} & \vul{S} (K)\\
        -- & -- & \alrts{1\nt} & \pass \\
        2\clubs & \pass & 2\diams & \pass \\
        3\nt & all \pass & & \\
    \end{tabular}
\end{table}

Wist K\spades, który bierze \spades 4, 6 (unblock ilościówką!) i 7. Dalej kontynuacja \spades 8.

Czy dało się to policzyć? Otóż tak!
\begin{itemize}
    \item Systemowo zrzutka jest ilościowa
    \item Brakuje nam J, 9, 5, 3, 2 - czyli 6 jest wysoka!
    \item Jeśli \vul{N} zrzuca z trzech pików, to E musi mieć 4, w tym jednego pod stołem (dał 2\diams).
\end{itemize}

Wniosek - \vul{N} ma 5 pików i trzeba wyjść damą. Na szczęście bez konsekwencji.

\hfill -B

\pagebreak
\section*{Rozdanie 19}
\handdiagramv{\vhand{7}{JT654}{K874}{J73}}
{\vhand{A863}{K7}{QT9}{KT84}}
{\vhand{Q94}{A832}{AJ}{A952}}
{\vhand{KJT52}{Q9}{6532}{Q6}}
{EW}

\begin{table}[h!]
    \centering
    \begin{tabular}{cccc}
        \vul{W} (B) & \nvul{N} & \vul{E} (K) & \nvul{S} \\
        -- & -- & -- & 1\nt \\
        \pass & 2\diams & \pass & 2\hearts \\
        all \pass & & & \\
    \end{tabular}
\end{table}

Wist 5\diams do T i J. \xhearts A, kier do K. \xclubs 8 do Q.

No i wtopa. Wyszedłem nieco automatycznie w pika, bo stwierdziłem, że \vul{E} na pewno nie wychodzi ósemką mając KT i jest ryzyko, 
że S usunie pika z dziadka. Jednak niezależnie od tego, należało zabezpieczyć się grając \xspades K! A następnie trefla, który
już nie mógł wypuszczać lewy. 

\hfill -B

A ja wyszłam 8\clubs żeby nie sugerować rozgrywającemu tego króla, ale jeśli miałby AQ\clubs
nie szkodziłoby mu przepuścić, więc mogłam wyjść poprawnie.

\hfill -K

\pagebreak
\section*{Rozdanie 20}
\handdiagramv{\vhand{A3}{AQT95}{KT9}{Q86}}
{\vhand{T52}{8742}{65}{AJ54}}
{\vhand{QJ76}{K}{Q842}{T973}}
{\vhand{K984}{J63}{AJ73}{K2}}
{NSEW}

\begin{table}[h!]
    \centering
    \begin{tabular}{cccc}
        \vul{W} (B) & \vul{N} & \vul{E} (K) & \vul{S}\\
        1\clubs & \dbl & \pass & 1\spades \\
        \pass & 2\hearts & \pass & 2\nt \\
        \pass & 3\nt & all \pass & \\
    \end{tabular}
\end{table}

Wist 6\hearts, pasywny, od partnerki \xhearts 2. Natychmiast biorę to za lavinthala (ilościówka nie ma sensu).
Rozgrywający gra karo z ręki.

Widzę, że trefle jadą. Policzyłem lewy jeśli puszczę: 2 piki, 5 kierów i karo = 8. Niewiedzieć czemu jednak 
perspektywa natychmiastowego efektownego wyjścia \xclubs K była zbyt kusząca.

Należało puścić, dając S szansę na błędne zagranie do Króla, które przegrywa.

\hfill -B

Czemu ilościówka nie ma sensu...? Akurat miałam i parzyście i trefla więc nie zastanawiałam się
co chcę zrzucić...

\hfill -K

\pagebreak
\section*{Rozdanie 21}
\handdiagramv{\vhand{KT732}{A98}{652}{AT}}
{\vhand{6}{KQJ74}{AQ7}{KJ95}}
{\vhand{AQ95}{T2}{J83}{Q843}}
{\vhand{J84}{653}{KT94}{762}}
{NS}

\begin{table}[h!]
    \centering
    \begin{tabular}{cccc}
        \nvul{W} (B) & \vul{N} & \nvul{E} (K) & \vul{S}\\
        -- & 1\spades & 2\hearts & 2\spades \\
        \pass & \pass & \dbl & \pass \\
        3\hearts & \pass & \pass & 3\spades \\
        all \pass & & & \\
    \end{tabular}
\end{table}

Tu puściłam widowiskowo. Wist K\hearts, od partnera 6. Rozgrywający ściągnął atuty, a ja
mając opór przed wyrzuceniem trefla wyrzuciłam wszystkie kiery.
Następnie zagrał małe karo, które wzięłam na 7\diams. Byłam pewna, że jestem wpuszczona.
Wyszłam treflem puszczając lewę. O K\diams mogłam wiedzieć z 6\hearts Bartka, która była
Lavinthalem, ale nawet o tym nie pomyślałam. Swoje.

Aczkolwiek Bartek mógł położyć T\diams zamiast puszczać, wtedy nie mam okazji do pomyłki.

\hfill -K

\pagebreak
\section*{Rozdanie 22}
\handdiagramv{\vhand{873}{953}{A543}{AK7}}
{\vhand{9}{K762}{KT862}{JT4}}
{\vhand{KJT64}{AQ8}{Q97}{92}}
{\vhand{AQ52}{JT4}{J}{Q8653}}
{EW}

\begin{table}[h!]
    \centering
    \begin{tabular}{cccc}
        \vul{W} (B) & \nvul{N} & \vul{E} (K) & \nvul{S}\\
        -- & -- & \pass & 1\spades \\
        \pass & \alrts{1\nt} & all \pass & \\
    \end{tabular}
\end{table}

1\nt = inwit z fitem, 4333\\
Wist 6\diams. Brałem piki i wychodziłem w trefla. Zdziwko jak \nvul{N} pokazał \xclubs AK. Płaskie jak cep.

\hfill -B

\pagebreak
\section*{Rozdanie 23}
\handdiagramv{\vhand{T83}{KJ762}{854}{97}}
{\vhand{AK652}{T5}{T62}{A86}}
{\vhand{QJ974}{93}{AKQ93}{Q}}
{\vhand{}{AQ84}{J7}{KJT5432}}
{NSEW}

\begin{table}[h!]
    \centering
    \begin{tabular}{cccc}
        \vul{W} (B) & \vul{N} & \vul{E} (K) & \vul{S}\\
        -- & -- & -- & 1\spades \\
        2\clubs & 2\hearts & 2\nt & 4\spades \\
        4\nt & \pass & 5\clubs & \dbl \\
        all \pass & & & \\
    \end{tabular}
\end{table}

Po mojej stronie stołu było śmiesznie, nie mogłam sobie wyobrazić rąk przeciwników.
2\nt było jakieś naturalne, inwitujące, w sumie ucieszyłabym się grając 2 (albo 3) \nt.
Na 4\nt też prawie spasowałam, ale ja bym to musiała grać. Myślałam, że 4\nt to młode,
ale i tak uważam, że się dogadaliśmy.\\

\hfill -K

Wist 8\spades

Skok \vul{S} w końcówkę wywołał niemałą konsternację przy stole. Musi mieć jakiś potężny układ, może 6-5 z renonsem trefl?
Dużo wskazuje, że Krysia mogłą sobie ubzdurać, że 2NT to cośtam z czerwonymi. Bałem się dać 5\clubs, a 4NT jest dobrym 
zabezpieczeniem. 

\hfill -B

\pagebreak
\section*{Rozdanie 24}
\handdiagramv{\vhand{QJ}{J3}{AQ432}{JT98}}
{\vhand{KT96}{AK75}{KJ8}{K2}}
{\vhand{83}{Q8642}{6}{Q7654}}
{\vhand{A7542}{T9}{T975}{A3}}
{}

\begin{table}[h!]
    \centering
    \begin{tabular}{cccc}
        \nvul{W} (B) & \nvul{N} & \nvul{E} (K) & \nvul{S}\\
        \pass & 1\diams & 1\nt & \pass \\
        2\hearts & \pass & 3\spades & \pass \\
        4\spades & all \pass & & \\
    \end{tabular}
\end{table}

Wist 6\diams, oddane A\diams i przebitka.

\hfill -K

\pagebreak
\section*{Rozdanie 25}
\handdiagramv{\vhand{AQ852}{A7}{T52}{754}}
{\vhand{T9}{KT643}{K43}{K92}}
{\vhand{J4}{95}{AQJ986}{AJT}}
{\vhand{K763}{QJ82}{7}{Q863}}
{EW}

\begin{table}[h!]
    \centering
    \begin{tabular}{cccc}
        \vul{W} (B) & \nvul{N} & \vul{E} (K) & \nvul{S}\\
        -- & \pass & \pass & 3\diams \\
        all \pass & & & \\
    \end{tabular}
\end{table}

Wist 2\spades\\
Nie no, gość miał 13, dyskusyjne to otwarcie.

\hfill -K

++ myślałem że miał 12.

\hfill -B

\pagebreak
\section*{Rozdanie 26}
\handdiagramv{\vhand{K6}{QT654}{A95}{QJ6}}
{\vhand{AJ73}{J9872}{2}{T94}}
{\vhand{QT852}{A3}{KJ843}{A}}
{\vhand{94}{K}{QT76}{K87532}}
{NSEW}

\begin{table}[h!]
    \centering
    \begin{tabular}{cccc}
        \vul{W} (B) & \vul{N} & \vul{E} (K) & \vul{S}\\
        -- & -- & \pass & 1\spades \\
        \pass & 2\hearts & \pass & 3\diams \\
        \pass & 3\nt & all \pass & \\
    \end{tabular}
\end{table}

Wist 9\clubs. Zagrany pik do K i A, następnie trefl. Zabiłem Królem (mam wszystkie dojścia), zagrałem trefla.

Trochę gdybanie. Ale może lepiej przepuścić? Skąd rozgrywający ma widzieć, kto ma trefle? 
Zabicie bardzo sugeruje posiadanie \xdiams QT. Zresztą, \vul{N} odrzucając impas karo wygrywa!

\hfill -B

Wpisane na złą linię :(

\hfill -K

\pagebreak
\section*{Rozdanie 27}
\handdiagramv{\vhand{Q743}{2}{AQ9532}{AT}}
{\vhand{62}{873}{J64}{98653}}
{\vhand{AT5}{KJ954}{T}{J742}}
{\vhand{KJ98}{AQT6}{K87}{KQ}}
{}

\begin{table}[h!]
    \centering
    \begin{tabular}{cccc}
        \nvul{W} (B) & \nvul{N} & \nvul{E} (K) & \nvul{S}\\
        -- & -- & -- & \alrts{2\diams} \\
        2\nt & all \pass & & \\
    \end{tabular}
\end{table}

Wist 5\diams wzięty waletem (1). Skoro mam mniej pików, to na pewno ma je \nvul{S}, prawda? 
Zagrałem kiera, w trakcie czego \nvul{N} wyciągnął kartę z ręki, więc położyłem \xhearts T, która wzięła (2).
No dobra, teraz to już na 100\% ma piki.
Zgrywam \xclubs KQ (3), \nvul{N} bije drugiego i gra pika do Asa, \nvul{N} odgrywa pika którego biję Królem (4).

O zgrozo, \nvul{N} dokłada! Dopiero teraz przypomniałem sobie otwarcie 3\diams Giorgi'a i ogarnąłem, że on umie w bloki.
Rozliczam rozdanie raz jeszcze i gram \xspades 9, licząc na wpust karowy.
\nvul{N} bije i gra pika do mojej blotki (5).

To umożliwia mi drugą próbę wpustu. Oddaję \xhearts 6 do \xhearts 9. S musi teraz dać mi dostęp do trefli lub lewę kierową, co czyni. (6, 7).
Noga. Ale z jaką historią!

\hfill -B


\end{document}