\documentclass[12pt, a4paper]{article}
\usepackage{import}

\import{../../lib/}{bridge.sty}

\title{GP Tarnów 13.10.2024\\część I}
\author{Krysia (\& Oliwia)}
\begin{document}
\maketitle

\section*{Rozdanie 1} % 3ntS Qs -2
\handdiagramv{\vhand{976}{KJ92}{K4}{AK93}}
{\vhand{T543}{A8543}{6}{875}}
{\vhand{K}{QT}{Q98732}{QT62}}
{\vhand{AQJ82}{76}{AJT5}{J4}}
{}

[KG] Ja na \nvul{W}. Jak tam się dolicytowali do 3\nt -- nie wiem,
ale raczej mnie w tej licytacji nie było. Wist \xspades Q.
Mogę wyciągnąć \xspades 8? Albo  w ogóle \xspades A? Szkoda, że nie pamiętam licytacji.


\pagebreak
\section*{Rozdanie 2} % 4sN Jh +1
\handdiagramv{\vhand{AT854}{92}{AK93}{K8}}
{\vhand{Q9}{JT53}{QJ4}{AJT2}}
{\vhand{K763}{AKQ8}{62}{974}}
{\vhand{J2}{764}{T875}{Q653}}
{NS}

\begin{table}[h!]
    \centering
    \begin{tabular}{cccc}
        \nvul{W} (K) & \vul{N} & \nvul{E} (O) & \vul{S}\\
        -- & -- & \pass & 1\clubs \\
        \pass & 1\spades & \pass & 2\spades \\
        \pass & 4\spades & all \pass & \\
    \end{tabular}
\end{table}

[KG] Wist \xhearts J. Jak rozgr. zaimpasuje \xhearts T to ma 13 lew.
Nie odważył się, ale czemu wziął tylko +1; na prawdę nie wiem,
może impasował pika (nie no nie ma jak impasować xd). 
Tu się nie da nie wziąć +2.

Dobra, zapytałam Oliwii, gość oddał 2 trefle XD Nieźle


\pagebreak
\section*{Rozdanie 3} % 4sE, 5d =
\handdiagramv{\vhand{A954}{96}{KJ86}{642}}
{\vhand{KQT76}{8}{T42}{AKT3}}
{\vhand{J2}{K754}{Q9753}{85}}
{\vhand{83}{AQJT32}{A}{QJ97}}
{EW}

\begin{table}[h!]
    \centering
    \begin{tabular}{cccc}
        \vul{W} (K) & \nvul{N} & \vul{E} (O) & \nvul{S}\\
        -- & -- & -- & \pass \\
        1\hearts & \pass & 1\spades & \pass \\
        2\hearts & \pass & 3\clubs & \pass \\
        3\diams & \pass & 3\spades & \pass \\
        4\spades & all \pass & & \\
    \end{tabular}
\end{table}

[KG] Bałam się że trafię na renons \hearts, więc dałam 4\spades, Oliwia wzięła = za 30\%. 
W 4\hearts mam górne 11. Daje się 4\hearts...?

\textit{Gość zapytał przed wistem czy wszystkie bidy naturalne, a Oliwia, że tak XD}

Ale w sumie mogłam 3\hearts zamiast 3\diams. Przedłużam kiery i widać, że się boję o trzymanie \diams.
A Oliwia z renonsem i bez trzymania da [cośtam], to zagramy w te kiery.

\pagebreak
\section*{Rozdanie 4} % 3ntS, Ad -2
\handdiagramv{\vhand{K9}{K}{T52}{KQJ6532}}
{\vhand{54}{J96}{QJ74}{T987}}
{\vhand{QJ8762}{AT87}{86}{A}}
{\vhand{AT3}{Q5432}{AK93}{4}}
{NSEW}

\begin{table}[h!]
    \centering
    \begin{tabular}{cccc}
        \vul{W} (K) & \vul{N} & \vul{E} (O) & \vul{S}\\
        1\hearts & 2\clubs & \pass & 2\spades \\
        \pass & 3\clubs & \pass & 3\nt \\
        all \pass & & & \\
    \end{tabular}
\end{table}

[KG] \vul{S} bez wahania wrzucił 3\nt więc uznałam że wist 
kierowy nie wygląda najlepiej. Zawistowałam \xdiams 
A i byłam 
z siebie dumna ale w sumie pół sali tak wistowało.
No a kierowy akurat nie wypuszcza.
Po ściągnięciu kar wzięłam \xspades A,
do którego gość wyrzucił króla. Wyszłam w kiera.
Zebrał piki, ale już nie miał powrotu na stół.
I tak powinien być -1. Chyba do jednego z kar wywalił pika.
No niezbyt to przemyślał.

\pagebreak
\section*{Rozdanie 5}
\handdiagramv{\vhand{AT3}{9}{AQJ953}{KT8}}
{\vhand{KQ864}{65}{T4}{Q763}}
{\vhand{95}{AKJ832}{K86}{A2}}
{\vhand{J72}{QT74}{72}{J954}}
{NS}

\begin{table}[h!]
    \centering
    \begin{tabular}{cccc}
        \nvul{W} (K) & \vul{N} & \nvul{E} (O) & \vul{S}\\
        -- & 1\diams & \pass & 1\hearts \\
        \pass & 2\diams & \pass & ? \\
        \pass & 3\nt & \pass & 4\diams \\
        \pass & 4\nt & all \pass & \\
    \end{tabular}
\end{table}

[KG] Nie jestem pewna licytacji, Oliwia chyba nie wchodziła -- ja bym weszła.
Na pewno \vul{S} wyniósł w 4\diams, a \vul{N} nie poczuł się wystarczająco
zachęcony. Wist \xspades 6 i wzięli tyle ile sie należało (+1).

Aczkolwiek wydaje mi się, że beztrosko pozbyłam się pików.
Kiera też pewnie jakiegoś wyrzuciłam. To może dobrze, że Oliwia
nie wchodziła -- gość mógłby, wiedząc, że nie mam już pików,
oddać mi bezpiecznie kiera i wziąć wszystkie pozostałe lewy.
Oliwka też raczej małego pika wyrzuciła, więc myślę, że i tak
mógł założyć, że wist nie poszedł z \xspades F64 i zaimpasować to serduszko.
W ogóle wydaje mi się, że wist był królem, a ja potem wyrzuciłam wałka na karo.
Ale może nie.

\vul{N} tłumaczył się, że nie chciał dać cue z singla.
Ale 4\spades w sumie mógł. Chyba, że po nim już 4\nt nie
zagrają i bał się o maxy.

\vspace{0.5cm}

Myślę, że z Bartkiem znaleźlibyśmy szlema:\\
1\diams -- 1\hearts\\
2\diams -- 2\spades\\
2\nt -- 3\hearts\\
3\nt -- 4\diams\\
4\spades -- 5\clubs\\
5\hearts -- 5\nt\\
7\diams

Gdzie: 2\spades = \gf, 5\hearts = asy.

\vspace{0.5cm}

Albo, jako że z ręki \vul{S} widać, że chcemy grać w
\diams, nie w \hearts :\\
1\diams -- 1\hearts\\
2\diams -- 2\spades\\
2\nt -- 3\diams\\
3\spades -- 4\clubs\\
4\spades -- ...

\pagebreak
\section*{Rozdanie 6}
\handdiagramv{\vhand{K92}{AT4}{A42}{T753}}
{\vhand{A874}{K9}{Q3}{AKQJ2}}
{\vhand{QJ53}{Q87653}{865}{}}
{\vhand{T6}{J2}{KJT97}{9864}}
{EW}

\begin{table}[h!]
    \centering
    \begin{tabular}{cccc}
        \vul{W} (O) & \nvul{N} & \vul{E} (K) & \nvul{S}\\
        -- & -- & 1\clubs & 2\diams \\
        \dbl & \rdbl & \pass & 2\hearts \\
        \pass & \pass & \dbl & all \pass \\
    \end{tabular}
\end{table}

[KG] Dobrze, że nie na impy...

Za to Oliwka dała pałę wistową, a ja się mocno zdziwiłam,
że chce pałować multi, ja mam 19, a w dziadku wyjeżdża 11...

Dobra, no to zakładając już tę pałę, co mam zrobić? Wrzucać 3\nt na koniec?
A bez pały na \vul{W}, jak pójdzie 2\hearts, pasować? dać 2\spades? Pała?

Pocieszenie jest takie, że nic nam nie idzie...

\pagebreak
\section*{Rozdanie 7}
\handdiagramv{\vhand{A873}{QJT87}{K}{QJ6}}
{\vhand{KT962}{K92}{3}{T753}}
{\vhand{Q5}{643}{QT864}{K84}}
{\vhand{J4}{A5}{AJ9752}{A92}}
{NSEW}

\begin{table}[h!]
    \centering
    \begin{tabular}{cccc}
        \vul{W} (K) & \vul{N} & \vul{E} (O) & \vul{S}\\
        -- & -- & -- & \pass \\
        1\nt & \alrts{2\clubs} & \pass & 2\hearts \\
        all \pass & & & \\
    \end{tabular}
\end{table}

[KG] Wist \xdiams A. RHO wolał kiery, więc wyobraziłam sobie 
\xspades K u partnerki. Wyszłam \xspades J, puszczone, Oliwia 
wzięła królem. Długi namysł, wyjście w pika (uf). 
Teraz gość wyszedł w karo, Oliwia przebiła, 
pik przebity. Oliwce mieści się jeszcze król. 
Powinnam wyjść w karo, na wypadek singlowego 
\xhearts K po drugiej stronie stołu, wtedy weźmiemy 2 
atutowe. Jakoś założyłam drugiego króla i
spokojnie wyszłam \xclubs A, \clubs. Rozgrywający wziął w 
stole i zagrał \xhearts Q, Oliwia co prawda miała drugiego 
królika, ale wskoczyła, dzięki czemu wzięliśmy jedną 
atutową (strata ok. 30\%). Nawet wiedząc że ma 
dwa atuty, powinnam wyjść karem, żeby nie dać jej 
szansy na taką pomyłkę.

\pagebreak
\section*{Rozdanie 8}
\handdiagramv{\vhand{Q973}{5}{QT92}{AK84}}
{\vhand{JT}{KT7643}{A87}{63}}
{\vhand{K862}{AJ8}{4}{QJT92}}
{\vhand{A54}{Q92}{KJ653}{75}}
{}

[KG] Licytacji nie pamiętam. 4\xspades \nvul{N}, wist \xclubs 3. 
Nie mam pojęcia co on
zrobił, żeby być -1.

\pagebreak
\section*{Rozdanie 9}
\handdiagramv{\vhand{J5}{JT76}{432}{8763}}
{\vhand{AT86}{K32}{AKQT8}{4}}
{\vhand{KQ}{Q84}{J65}{JT952}}
{\vhand{97432}{A95}{97}{AKQ}}
{EW}

\begin{table}[h!]
    \centering
    \begin{tabular}{cccc}
        \vul{W} (K) & \nvul{N} & \vul{E} (O) & \nvul{S}\\
        -- & \pass & 1\diams & \pass \\
        1\spades & \pass & 4\spades & all \pass \\
    \end{tabular}
\end{table}

[KG] Ajaj, pierwsze grane przez nas rozdanie to chyba moja największa
porażka na tym turnieju :')

\pagebreak
\section*{Rozdanie 10}
\handdiagramv{\vhand{KJ83}{A4}{KJ84}{983}}
{\vhand{42}{KJT83}{976}{Q75}}
{\vhand{A96}{Q76}{T532}{642}}
{\vhand{QT75}{952}{AQ}{AKJT}}
{NSEW}

\begin{table}[h!]
    \centering
    \begin{tabular}{cccc}
        \vul{W} (K) & \vul{N} & \vul{E} (O) & \vul{S}\\
        -- & -- & \pass & \pass \\
        1\nt & \pass & 2\diams & \pass \\
        2\hearts & all \pass & & \\
    \end{tabular}
\end{table}

[KG] Wist \xclubs 8. Oddałam 2 piki, \xdiams K, 2 atuty. Przeciwnicy
skomentowali, że nie dałam im szansy na pomyłkę (ze
ściągnięciem pików). Ale przecież nie zacznę od ściągania trefli,
więc nie wiem, jak to zrobić lepiej. Lepiej będzie jak zagram
do \xhearts K, zaimpasuje karo i jeśli nie ściągną teraz pików,
to już tego nie zrobią -- Zagram 4 trefle, jeśli \vul{N}
przebije czwartego, to wezmą jednego pika, jeśli 
\vul{S} przebije -- wezmą tylko jedną atutową.
Ale jak figury kier są na odwrót -- w ten sposób stracę lewę.


\pagebreak
\section*{Rozdanie 11}
\handdiagramv{\vhand{53}{A86}{AQ9876}{K9}}
{\vhand{QT876}{942}{J}{A543}}
{\vhand{A4}{QJT53}{T32}{JT2}}
{\vhand{KJ92}{K7}{K54}{Q876}}
{}

\begin{table}[h!]
    \centering
    \begin{tabular}{cccc}
        \nvul{W} (K) & \nvul{N} & \nvul{E} (O) & \nvul{S}\\
        -- & -- & -- & \pass \\
        1\clubs & 1\diams & 1\spades & 2\hearts \\
        2\spades & 4\hearts & 4\spades & all \pass \\
    \end{tabular}
\end{table}

[KG] W sumie dobra obrona, bo w kiery idzie 12, szkoda, 
że prawie nikt na sali nie zagrał. Ale przynajmniej
nie skontrowali. A jak na 4\hearts zawistuję w karo,
a \nvul{S} zabije asem, to już obkładamy.

\pagebreak
\section*{Rozdanie 12}
\handdiagramv{\vhand{AKJ87}{54}{AKT75}{T}}
{\vhand{T32}{98}{J862}{K982}}
{\vhand{95}{AQJT6}{9}{AQ754}}
{\vhand{Q64}{K732}{Q43}{J63}}
{NS}

3\nt\vul{S}\\
Wist 4\diams, poimpasował, wziął +3.

\pagebreak
\section*{Rozdanie 13}
\handdiagramv{\vhand{75}{KQT973}{K54}{Q4}}
{\vhand{4}{6}{QJ763}{AKT952}}
{\vhand{KJ32}{AJ852}{T982}{}}
{\vhand{AQT986}{4}{A}{J8763}}
{NSEW}

\begin{table}[h!]
    \centering
    \begin{tabular}{cccc}
        \vul{W} (K) & \vul{N} & \vul{E} (O) & \vul{S}\\
        -- & 2\diams & 3\clubs & 4\clubs \\
        \pass & 4\diams & \dbl & 4\hearts \\
        5\hearts & \dbl & \pass & \pass \\
        6\clubs & all \pass & & \\
    \end{tabular}
\end{table}

[KG] Bardzo chciałam pokazać moją potężną rękę.
Bałam się, że 4\nt kompletnie nie zostanie zrozumiane
i sobie je zagram (albo zagram 6\diams). 
Oliwia mogła też nie wiedzieć jaki jest kolor przeciwnika.
Dlatego ostatecznie spasowałam, co dało przeciwnikom szansę na
informację, jaki mają kolor. Na szczęście jakimś cudem
nie poszli w obronę. Oliwka tym razem dała \dbl\ na \diams
jako długość, ja zrozumiałam, że siła. Wg mnie mogła dać 4\nt.
Albo w ogóle od razu 4\nt. Dałam 5\hearts i na szczęście \vul{N}
skontrował, co dało mi dodatkowe kółko, żeby w końcu pokazać fit.
Miałam nadzieję, że pokazałam siłę i fit (chciałam
zainwitować szlema), ale nie zostałam zrozumiana.
Na dodatek wist poszedł ze złej ręki 
i ja rozgrywałam. Wist \xhearts K przejęty asem,
błyskawiczne wyjście małym pikiem (miał pół godziny na namysł
jak Oliwia kminiła co zrobić z wistem ze złej ręki).
Byłam 100\% pewna, że mogę zaimpasować. ale uznałam,
że dam sobie więcej szans niż moja niezawodna intuicja.
Zabiłam asem, zebrałam \xclubs A. Niestety nie 1-1.
Zebrałam drugiego \clubs planując przebijać w obu rękach,
a jeśli \xdiams K nie spadnie, zaimpasować go u \vul{S}.
Podczas przebijania już wiedziałam, że się uda,
\vul{N} pokazał 2 piki, 2 trefle, ma maksymalnie 3 kara -- król
spada albo jest u \vul{S}.

\vspace{0.5cm}

\textbf{14}-stego nie zagraliśmy.

\pagebreak
\section*{Rozdanie 15}
\handdiagramv{\vhand{975}{K85}{K863}{K63}}
{\vhand{AQ}{AQJ42}{54}{AT85}}
{\vhand{KJ63}{T}{AQJT7}{Q42}}
{\vhand{T842}{9763}{92}{J97}}
{NS}

\begin{table}[h!]
    \centering
    \begin{tabular}{cccc}
        \nvul{W} (K) & \vul{N} & \nvul{E} (O) & \vul{S}\\
        -- & -- & -- & 1\diams \\
        \pass & 1\nt & \dbl & \rdbl \\
        \pass & \pass & 2\hearts & \pass \\
        \pass & 3\diams & \pass & 3\nt \\
        all \pass & & & \\
    \end{tabular}
\end{table}

Coś im ewidentnie nie wyszło, bo bilansu to nie mają,
a gość sobie wrzucił 3\nt z singlem. Wist \xhearts Q,
dorzuciłam markę na 9. Wziął królem wyszedł w pika.
Oliwia wskoczyła i wyszła małym kierem, a ja wzięłam na 9.
Wzięłyśmy 4 kiery i 2 asy.

\pagebreak
\section*{Rozdanie 16}
\handdiagramv{\vhand{543}{J6}{KQT}{AKQJ9}}
{\vhand{QT86}{974}{9654}{63}}
{\vhand{AK97}{AKT852}{3}{T4}}
{\vhand{J2}{Q3}{AJ872}{8752}}
{EW}

\begin{table}[h!]
    \centering
    \begin{tabular}{cccc}
        \vul{W} (K) & \nvul{N} & \vul{E} (O) & \nvul{S}\\
        \pass & 1\clubs & \pass & 1\hearts \\
        \pass & 2\clubs & \pass & \alrts{2\diams} \\
        \pass & 3\diams & \pass & 4\clubs \\
        \pass & 4\diams & \pass & 4\nt \\
        \pass & 5\spades & \pass & 6\clubs \\
    \end{tabular}
\end{table}

No to sobie wymyślił kontrakt na maxy. 6\clubs na 7-atucie,
jako jedyni na sali. Też bym nie zawistowała w karo; wzięli +1.
Wist w atut. Ten kontrakt jest jakiś tragiczny xd

\pagebreak
\section*{Rozdanie 17}
\handdiagramv{\vhand{A972}{T965}{62}{AKQ}}
{\vhand{863}{Q7}{K93}{J9873}}
{\vhand{QJ4}{AK4}{AJ74}{542}}
{\vhand{KT5}{J832}{QT85}{T6}}
{}

Jakieś płaskie 3\nt. Oliwia zawistowała \xclubs 9 (?).

\pagebreak
\section*{Rozdanie 18}
\handdiagramv{\vhand{T9643}{A843}{T53}{4}}
{\vhand{K7}{KQ97}{94}{AKQ98}}
{\vhand{Q2}{JT652}{AKQ7}{J3}}
{\vhand{AJ85}{}{J862}{T7652}}
{NS}

\begin{table}[h!]
    \centering
    \begin{tabular}{cccc}
        \nvul{W} (K) & \vul{N} & \nvul{E} (O) & \vul{S}\\
        -- & -- & 1\clubs & 1\hearts \\
        1\spades & 3\hearts & 3\nt & all \pass \\
    \end{tabular}
\end{table}

Oliwia spodziewała się trochę więcej, ale na szczęście szło.
Wist \xdiams Q, \xhearts J.

\pagebreak
\section*{Rozdanie 19}
\handdiagramv{\vhand{972}{432}{T53}{J532}}
{\vhand{JT8}{976}{AQ84}{KQ7}}
{\vhand{AKQ3}{KT}{J62}{A864}}
{\vhand{654}{AQJ85}{K97}{T9}}
{EW}


\begin{table}[h!]
    \centering
    \begin{tabular}{cccc}
        \vul{W} (O) & \nvul{N} & \vul{E} (K) & \nvul{S}\\
        -- & -- & -- & 1\nt \\
        all \pass & & & \\
    \end{tabular}
\end{table}

Wist w pika, a potem to ja nie wiem co się stało,
ale w kiera wyszła dopiero Oliwia w 3 lewie przed końcem,
a w karo żadna z nas xd

\pagebreak
\section*{Rozdanie 20}
\handdiagramv{\vhand{5432}{AT8}{KQ65}{J3}}
{\vhand{AK6}{Q932}{A4}{Q754}}
{\vhand{QJ}{KJ76}{972}{K862}}
{\vhand{T987}{54}{JT83}{AT9}}
{NSEW}

\begin{table}[h!]
    \centering
    \begin{tabular}{cccc}
        \vul{W} (O) & \vul{N} & \vul{E} (K) & \vul{S}\\
        \pass & \pass & 1\nt & \pass \\
        \pass & \dbl & \pass & 2\hearts \\
        \dbl & all \pass & & \\
    \end{tabular}
\end{table}

Kontra wznówkowa i może powinnam unikać gry w obronie
po poprzednim rozdaniu, ale pas wyglądał lepiej.
Wist \xspades 9, w sumie nie wiem co się działo.

\pagebreak
\section*{Rozdanie 21}
\handdiagramv{\vhand{K7}{75}{QT85}{AJ854}}
{\vhand{J85432}{J82}{K76}{T}}
{\vhand{T6}{Q964}{J2}{Q9762}}
{\vhand{AQ9}{AKT3}{A943}{K3}}
{NS}

\begin{table}[h!]
    \centering
    \begin{tabular}{cccc}
        \nvul{W} & \vul{N} (O) & \nvul{E} & \vul{S} (K) \\
        -- & \pass & \pass & 1\nt \\
        \dbl & 3\nt & \pass & \pass \\
        \dbl & \pass & \pass & 4\clubs \\
        \dbl & all \pass \\
    \end{tabular}
\end{table}

No nie mogłam się powstrzymać mimo założeń xd\\
Przy naszym stoliku sędzia powinien być 2x,
najpierw \nvul{W} skomentował tłumaczenie partnera
(że pała = 54), że może mieć też objaśniak,
potem przed wistem powiedział, że blef to nie w tych założeniach.
Było -2, ale w sumie niewiele brakowało do -1.
Jakby zamienić \xdiams Q z \xdiams K. Ale trzeba przyznać,
że 10-kart wygrałam. Kusiło mnie też otwarcie 1\spades,
ale bałam się, że zagrają dobre na maxy 3\nt zamiast 4\spades.
Akurat 3\nt im nie szło. Może by się dobrze skończyło.

\pagebreak
\section*{Rozdanie 22}
\handdiagramv{\vhand{6542}{987}{AQ}{AQ86}}
{\vhand{}{Q653}{T8732}{5432}}
{\vhand{KQT983}{T4}{9654}{7}}
{\vhand{AJ7}{AKJ2}{KJ}{KJT9}}
{EW}

\begin{table}[h!]
    \centering
    \begin{tabular}{cccc}
        \vul{W} & \nvul{N} (O) & \vul{E} & \nvul{S} (K) \\
        -- & -- & \pass & 3\spades \\
        3\nt & 4\spades & \pass & \pass \\
        \dbl & all \pass & & \\
    \end{tabular}
\end{table}

Tym razem dobre założenia na blok, gorsze na wyniesienie...
No cóż.

\pagebreak
\section*{Rozdanie 23}
\handdiagramv{\vhand{A765}{JT54}{KT87}{K}}
{\vhand{K3}{A876}{J64}{QT42}}
{\vhand{9842}{K}{A532}{A753}}
{\vhand{QJT}{Q932}{Q9}{J986}}
{NSEW}

\begin{table}[h!]
    \centering
    \begin{tabular}{cccc}
        \vul{W} & \vul{N} (O) & \vul{E} & \vul{S} (K) \\
        -- & -- & -- & \pass \\
        \pass & 1\diams & \pass & 1\spades \\
        all \pass & & & \\
    \end{tabular}
\end{table}

Bardzo się ucieszyłam z otwarcia i już widziałam się w końcówce.
Zdziwiłam się na pas i brak wznówki (w sumie \vul{E} ma dobrą wznówkę?).
Wist \xclubs 9 (czemu oni wistują 9 spod waleta?), wzięty w stole.
Tak jak uczył mistrz Bartłomiej, zagrałam szybko wałka do singlowego króla,
który wziął (no w końcu coś poszło zgodnie z planem).
Następnie ściągnęłam 2 kara i \xclubs A, przebiłam 3 kiery i 2 trefle,
wszystko się jakimś cudem podzieliło. Na koniec jeszcze \xspades A; 11 lew.

\pagebreak
\section*{Rozdanie 24}
\handdiagramv{\vhand{QT9}{J9652}{K32}{96}}
{\vhand{3}{83}{J764}{AK8532}}
{\vhand{7542}{AKT}{AT9}{QT7}}
{\vhand{AKJ86}{Q74}{Q85}{J4}}
{}

\begin{table}[h!]
    \centering
    \begin{tabular}{cccc}
        \nvul{W} & \nvul{N} (O) & \nvul{E} & \nvul{S} (K) \\
        1\spades & \pass & \pass & \dbl \\
        \pass & 2\hearts & \pass & \pass \\
        2\spades & all \pass & & \\
    \end{tabular}
\end{table}

Ktoś tu bardzo chciał rozgrywać. Chociaż
pas na \nvul{E} chyba jeszcze dziwniejszy. 
Wziął -1, tyle się należało.
Udało nam się nie rozwiązać kar.

\pagebreak
\section*{Rozdanie 25}
\handdiagramv{\vhand{Q9}{K4}{J9863}{KJ42}}
{\vhand{AK532}{AT87}{AQ5}{6}}
{\vhand{74}{QJ3}{KT2}{AT753}}
{\vhand{JT86}{9652}{74}{Q98}}
{EW}

\begin{table}[h!]
    \centering
    \begin{tabular}{cccc}
        \vul{W} & \nvul{N} & \vul{E} & \nvul{S}\\
        -- & \pass & 1\spades & \pass \\
        \pass & 2\nt & \dbl & \rdbl \\
        \pass & 3\diams & \dbl & 4\clubs \\
        \pass & \pass & \dbl & all \pass \\
    \end{tabular}
\end{table}

Że Oliwka była po pasie to ogarnęłam po rozdaniu.
No cóż, Slowpoke. Po 3\diams zdecydowałam się
zaalertować 2\nt. No i kurczę! 3\diams było swoje.
No ale skąd mogłam wiedzieć. Nie spodziewałam się 5-4.
A 4\clubs\dbl-1 to minimax, ale nic nie trafiłam.
A właściwie mogłam. Widać było, że \vul{E} ma kara,
a trefle już dla zasady pałuje. 2 piki i kier na wiście,
w czwartej lewie \vul{E} wyszedł \xdiams A, \diams. Zabiłam,
bo uznałam, że \vul{W} jakaś babcia się mieści i bałam się przebitki.
Potem nie trafiłam też trefli: -2.

% \pagebreak
% \section*{Rozdanie 26}
% \handdiagramv{\vhand{KT75}{A954}{852}{T2}}
% {\vhand{AJ86}{QT2}{JT3}{J74}}
% {\vhand{42}{K87}{AK76}{AK98}}
% {\vhand{Q93}{J63}{Q94}{Q653}}
% {NSEW}

% \begin{table}[h!]
%     \centering
%     \begin{tabular}{cccc}
%         \vul{W} (K) & \vul{N} & \vul{E} (O) & \vul{S}\\

%     \end{tabular}
% \end{table}

% \pagebreak
% \section*{Rozdanie 27}
% \handdiagramv{\vhand{T83}{KJ7532}{T6}{QT}}
% {\vhand{K52}{86}{A9543}{K32}}
% {\vhand{AJ96}{AQ94}{82}{AJ8}}
% {\vhand{Q74}{T}{KQJ7}{97654}}
% {}

% \begin{table}[h!]
%     \centering
%     \begin{tabular}{cccc}
%         \nvul{W} & \nvul{N} & \nvul{E} & \nvul{S}\\

%     \end{tabular}
% \end{table}

% \pagebreak
% \section*{Rozdanie 28}
% \handdiagramv{\vhand{J543}{K97542}{}{K43}}
% {\vhand{Q6}{A}{T8653}{JT862}}
% {\vhand{T2}{QJ63}{AKQJ74}{7}}
% {\vhand{AK987}{T8}{92}{AQ95}}
% {NS}

% \begin{table}[h!]
%     \centering
%     \begin{tabular}{cccc}
%         \nvul{W} & \vul{N} & \nvul{E} & \vul{S}\\

%     \end{tabular}
% \end{table}

% \pagebreak
% \section*{Rozdanie 29}
% \handdiagramv{\vhand{T3}{A2}{AQJ953}{A73}}
% {\vhand{Q954}{T3}{86}{Q6542}}
% {\vhand{A62}{K874}{K42}{K98}}
% {\vhand{KJ87}{QJ965}{T7}{JT}}
% {NSEW}

% \begin{table}[h!]
%     \centering
%     \begin{tabular}{cccc}
%         \vul{W} (K) & \vul{N} & \vul{E} (O) & \vul{S}\\

%     \end{tabular}
% \end{table}

% \pagebreak
% \section*{Rozdanie 30}
% \handdiagramv{\vhand{A}{AQJ43}{KT9753}{Q}}
% {\vhand{KQJ964}{962}{J8}{J2}}
% {\vhand{T753}{T5}{A64}{A873}}
% {\vhand{82}{K87}{Q2}{KT9654}}
% {}

% \begin{table}[h!]
%     \centering
%     \begin{tabular}{cccc}
%         \nvul{W} & \nvul{N} & \nvul{E} & \nvul{S}\\

%     \end{tabular}
% \end{table}

% \pagebreak
% \section*{Rozdanie 31}
% \handdiagramv{\vhand{Q932}{AT62}{43}{KJ5}}
% {\vhand{AJ}{KJ853}{Q9}{Q843}}
% {\vhand{874}{Q74}{KT852}{A2}}
% {\vhand{KT65}{9}{AJ76}{T976}}
% {NS}

% \begin{table}[h!]
%     \centering
%     \begin{tabular}{cccc}
%         \nvul{W} & \vul{N} & \nvul{E} & \vul{S}\\

%     \end{tabular}
% \end{table}

% \pagebreak
% \section*{Rozdanie 32}
% \handdiagramv{\vhand{K75}{83}{T5}{KJ7643}}
% {\vhand{AQ93}{KT75}{AKJ7}{8}}
% {\vhand{T6}{QJ42}{6432}{T95}}
% {\vhand{J842}{A96}{Q98}{AQ2}}
% {EW}

% \begin{table}[h!]
%     \centering
%     \begin{tabular}{cccc}
%         \vul{W} & \nvul{N} & \vul{E} & \nvul{S}\\

%     \end{tabular}
% \end{table}

% \pagebreak
% \section*{Rozdanie 33}
% \handdiagramv{\vhand{QJT9}{T943}{AQ842}{}}
% {\vhand{752}{J5}{J9753}{AQ9}}
% {\vhand{AK84}{AQ76}{K}{JT74}}
% {\vhand{63}{K82}{T6}{K86532}}
% {}

% \begin{table}[h!]
%     \centering
%     \begin{tabular}{cccc}
%         \nvul{W} & \nvul{N} & \nvul{E} & \nvul{S}\\

%     \end{tabular}
% \end{table}

% \pagebreak
% \section*{Rozdanie 34}
% \handdiagramv{\vhand{86}{AQ76}{KT85}{Q86}}
% {\vhand{73}{KJ8}{Q64}{97432}}
% {\vhand{KQ94}{T952}{73}{KJ5}}
% {\vhand{AJT52}{43}{AJ92}{AT}}
% {NS}

% \begin{table}[h!]
%     \centering
%     \begin{tabular}{cccc}
%         \nvul{W} & \vul{N} & \nvul{E} & \vul{S}\\

%     \end{tabular}
% \end{table}

% \pagebreak
% \section*{Rozdanie 35}
% \handdiagramv{\vhand{954}{KQJT}{T4}{T654}}
% {\vhand{T6}{A85}{AK972}{A72}}
% {\vhand{K873}{7642}{85}{J93}}
% {\vhand{AQJ2}{93}{QJ63}{KQ8}}
% {EW}

% \begin{table}[h!]
%     \centering
%     \begin{tabular}{cccc}
%         \vul{W} & \nvul{N} & \vul{E} & \nvul{S}\\

%     \end{tabular}
% \end{table}

% \pagebreak
% \section*{Rozdanie 36}
% \handdiagramv{\vhand{95}{KJ982}{KJ6}{543}}
% {\vhand{J643}{3}{9543}{AQT8}}
% {\vhand{KQT872}{AT74}{2}{K6}}
% {\vhand{A}{Q65}{AQT87}{J972}}
% {NSEW}

% \begin{table}[h!]
%     \centering
%     \begin{tabular}{cccc}
%         \vul{W} (K) & \vul{N} & \vul{E} (O) & \vul{S}\\

%     \end{tabular}
% \end{table}

% \pagebreak
% \section*{Rozdanie 37}
% \handdiagramv{\vhand{QJT72}{86}{AT72}{T9}}
% {\vhand{K6543}{J2}{J93}{653}}
% {\vhand{A9}{AKQ5}{K6}{AKQ74}}
% {\vhand{8}{T9743}{Q854}{J82}}
% {NS}

% \begin{table}[h!]
%     \centering
%     \begin{tabular}{cccc}
%         \nvul{W} & \vul{N} & \nvul{E} & \vul{S}\\

%     \end{tabular}
% \end{table}

% \pagebreak
% \section*{Rozdanie 38}
% \handdiagramv{\vhand{KJ94}{T7432}{5}{AQJ}}
% {\vhand{T63}{AKJ}{A832}{T63}}
% {\vhand{752}{}{QJ64}{K97542}}
% {\vhand{AQ8}{Q9865}{KT97}{8}}
% {EW}

% \begin{table}[h!]
%     \centering
%     \begin{tabular}{cccc}
%         \vul{W} & \nvul{N} & \vul{E} & \nvul{S}\\

%     \end{tabular}
% \end{table}

% \pagebreak
% \section*{Rozdanie 39}
% \handdiagramv{\vhand{T543}{AKQ74}{K}{KQ4}}
% {\vhand{KQJ96}{T32}{AT8}{32}}
% {\vhand{872}{5}{QJ9543}{J76}}
% {\vhand{A}{J986}{762}{AT985}}
% {NSEW}

% \begin{table}[h!]
%     \centering
%     \begin{tabular}{cccc}
%         \vul{W} (K) & \vul{N} & \vul{E} (O) & \vul{S}\\

%     \end{tabular}
% \end{table}

% \pagebreak
% \section*{Rozdanie 40}
% \handdiagramv{\vhand{K83}{K965}{JT}{AJ83}}
% {\vhand{AQ762}{AQT}{A3}{976}}
% {\vhand{95}{J7432}{652}{K52}}
% {\vhand{JT4}{8}{KQ9874}{QT4}}
% {}

% \begin{table}[h!]
%     \centering
%     \begin{tabular}{cccc}
%         \nvul{W} & \nvul{N} & \nvul{E} & \nvul{S}\\

%     \end{tabular}
% \end{table}

% \pagebreak
% \section*{Rozdanie 41}
% \handdiagramv{\vhand{K83}{K965}{JT}{AJ83}}
% {\vhand{AQT4}{J2}{53}{AT632}}
% {\vhand{K5}{K87643}{Q}{KJ98}}
% {\vhand{832}{A9}{AKJ864}{Q5}}
% {EW}

% \begin{table}[h!]
%     \centering
%     \begin{tabular}{cccc}
%         \vul{W} & \nvul{N} & \vul{E} & \nvul{S}\\

%     \end{tabular}
% \end{table}

% \pagebreak
% \section*{Rozdanie 42}
% \handdiagramv{\vhand{Q6}{76}{9862}{AJT42}}
% {\vhand{KT83}{9832}{4}{9765}}
% {\vhand{AJ974}{JT4}{T73}{Q8}}
% {\vhand{52}{AKQ5}{AKQJ5}{K3}}
% {NSEW}

% \begin{table}[h!]
%     \centering
%     \begin{tabular}{cccc}
%         \vul{W} (K) & \vul{N} & \vul{E} (O) & \vul{S}\\

%     \end{tabular}
% \end{table}

% \pagebreak
% \section*{Rozdanie 43}
% \handdiagramv{\vhand{AK42}{62}{K865}{AT6}}
% {\vhand{Q97}{AKQT3}{AT97}{4}}
% {\vhand{T653}{J7}{QJ432}{82}}
% {\vhand{J8}{9854}{}{KQJ9753}}
% {}

% \begin{table}[h!]
%     \centering
%     \begin{tabular}{cccc}
%         \nvul{W} & \nvul{N} & \nvul{E} & \nvul{S}\\

%     \end{tabular}
% \end{table}

% \pagebreak
% \section*{Rozdanie 44}
% \handdiagramv{\vhand{K64}{K52}{T6}{JT984}}
% {\vhand{T982}{QJ873}{53}{73}}
% {\vhand{AQ753}{A9}{J72}{Q52}}
% {\vhand{J}{T64}{AKQ984}{AK6}}
% {NS}

% \begin{table}[h!]
%     \centering
%     \begin{tabular}{cccc}
%         \nvul{W} & \vul{N} & \nvul{E} & \vul{S}\\

%     \end{tabular}
% \end{table}

% \pagebreak
% \section*{Rozdanie 45}
% \handdiagramv{\vhand{8752}{764}{QJ}{AJT3}}
% {\vhand{QT93}{98532}{AK6}{5}}
% {\vhand{AJ}{AKJ}{9743}{K964}}
% {\vhand{K64}{QT}{T852}{Q872}}
% {NSEW}

% \begin{table}[h!]
%     \centering
%     \begin{tabular}{cccc}
%         \vul{W} (K) & \vul{N} & \vul{E} (O) & \vul{S}\\

%     \end{tabular}
% \end{table}

% \pagebreak
% \section*{Rozdanie 46}
% \handdiagramv{\vhand{8752}{764}{QJ}{AJT3}}
% {\vhand{QT93}{98532}{AK6}{5}}
% {\vhand{AJ}{AKJ}{9743}{K964}}
% {\vhand{K64}{QT}{T852}{Q872}}
% {}

% \begin{table}[h!]
%     \centering
%     \begin{tabular}{cccc}
%         \nvul{W} & \nvul{N} & \nvul{E} & \nvul{S}\\

%     \end{tabular}
% \end{table}

% \pagebreak
% \section*{Rozdanie 47}
% \handdiagramv{\vhand{8752}{764}{QJ}{AJT3}}
% {\vhand{QT93}{98532}{AK6}{5}}
% {\vhand{AJ}{AKJ}{9743}{K964}}
% {\vhand{K64}{QT}{T852}{Q872}}
% {NS}

% \begin{table}[h!]
%     \centering
%     \begin{tabular}{cccc}
%         \nvul{W} & \vul{N} & \nvul{E} & \vul{S}\\

%     \end{tabular}
% \end{table}

% \pagebreak
% \section*{Rozdanie 48}
% \handdiagramv{\vhand{J98}{AQ6}{AT97}{T82}}
% {\vhand{AQ42}{852}{632}{K53}}
% {\vhand{KT6}{J4}{J4}{AQJ964}}
% {\vhand{753}{KT973}{KQ85}{7}}
% {EW}

% \begin{table}[h!]
%     \centering
%     \begin{tabular}{cccc}
%         \vul{W} & \nvul{N} & \vul{E} & \nvul{S}\\

%     \end{tabular}
% \end{table}

% \pagebreak
% \section*{Rozdanie 49}
% \handdiagramv{\vhand{QJT5}{JT864}{5}{AQ3}}
% {\vhand{A983}{93}{Q96}{T942}}
% {\vhand{K2}{AK752}{KT742}{5}}
% {\vhand{764}{Q}{AJ83}{KJ876}}
% {}

% \begin{table}[h!]
%     \centering
%     \begin{tabular}{cccc}
%         \nvul{W} & \nvul{N} & \nvul{E} & \nvul{S}\\

%     \end{tabular}
% \end{table}

% \pagebreak
% \section*{Rozdanie 50}
% \handdiagramv{\vhand{AQ}{Q65}{T862}{AK42}}
% {\vhand{KT85}{AJ3}{Q}{QJ865}}
% {\vhand{9763}{942}{J953}{93}}
% {\vhand{J42}{KT87}{AK74}{T7}}
% {NS}

% \begin{table}[h!]
%     \centering
%     \begin{tabular}{cccc}
%         \nvul{W} & \vul{N} & \nvul{E} & \vul{S}\\

%     \end{tabular}
% \end{table}

\end{document}