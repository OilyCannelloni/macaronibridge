\documentclass[12pt, a4paper]{article}
\usepackage{import}

\import{../../lib/}{bridge.sty}

\title{Żaczek 30.09.24}
\author{Krysia \& Bartek}
\begin{document}
\maketitle

\section*{Rozdanie 1}
\handdiagramv{\vhand{T8542}{K3}{KJ3}{KJ7}}
{\vhand{Q7}{A2}{T9765}{Q963}}
{\vhand{93}{QJ985}{Q42}{852}}
{\vhand{AKJ6}{T764}{A8}{AT4}}
{}

\begin{table}[h!]
    \centering
    \begin{tabular}{cccc}
        \nvul{W} (K) & \nvul{N} & \nvul{E} (B) & \nvul{S}\\
        -- & \pass & \pass & \pass \\
        1\nt & all pass & & \\
    \end{tabular}
\end{table}

[KG] Bartek z 8pc zdecydował się nie inwitować,
3\nt po dość oczywistym wiście pikowym 
(u nas \xspades 4) niestety czapowe. Po innym też
idzie, chociaż nie tak bezproblemowo.

[KK] Pasowanie w 8pc na 1\nt partnera jest w pełni 
uzasadnioną decyzją, aczkolwiek tutaj mamy świetne 
8pc. Druga \xspades Q nie traci na wartości po otwarciu 
1\nt partnera, a do tego posiadamy T9765 jako 
źródło lew z dużą ilością potencjalnych dojść. 
\pass w tym przypadku nie jest poprawny, 
ale warto pasować w niektóre ręce z 8pc. 


\section*{Rozdanie 2}
\handdiagramv{\vhand{J6532}{5}{KJ6}{KQ53}}
{\vhand{874}{AK843}{Q82}{97}}
{\vhand{KT9}{JT97}{T743}{84}}
{\vhand{AQ}{Q62}{A95}{AJT62}}
{NS}

\begin{table}[h!]
    \centering
    \begin{tabular}{cccc}
        \nvul{W} (K) & \vul{N} & \nvul{E} (B) & \vul{S}\\
        -- & -- & \pass & \pass \\
        1\clubs & 1\spades & \dbl & \pass \\
        2\nt & \pass & \alrts{3\clubs} & \pass \\
        3\hearts & \pass & 4\hearts & all pass \\
    \end{tabular}
\end{table}

[KG] Tu niestety zlagowałam w licytacji, chyba ze względu
na wejście przeciwnika -- pokazałam 4 kiery.
Na szczęście po systemowym 3\diams (3 kiery) też
kończymy w 4\hearts. Wist \xclubs K pomógł, bez niego
pewnie starałabym się impasować trefle zamiast liczyć
na wejście 1\spades z piątego wałka. Ściągnęłam 2
kiery i zagrałam trefla. \vul{N} wyszedł w pika, po raz kolejny
ułatwiając mi zadanie. Oddałam \xhearts J, \xclubs Q i \xdiams K.

[KK] Tutaj na szczęście nie ma 17pc 
na 4432 bez potencjału na lewy. 
Ale nie mamy lew!!! Mamy potencjał lewowy, 
dobry kolor treflowy, ale nie bardzo dobry. 
Pojedyncze zatrzymania w czerwonych, 
niewsparte blotkami poza \xdiams 9. 
Po 1\nt też byście doszli do 4\hearts. 
Inna ręka? KQ♣ K♦. Idealne wartości, 
3\nt na impasie do inwitu. \xclubs K \xdiams K 
\xhearts J. 7pc, 
brak inwitu, ale końcówka jest na 
trafieniu trefla i impasie pik, a gramy na maxy. 
Najchętniej dałbym Ci bana na otwieranie 
\xclubs 1 w 17pc, ale czasami trzeba, więc 
tego nie zrobię. Otwarcie 1\nt zajmuje 
z dobrej ręki, blokuje wszelkie wejścia, 
np takie, które wskazują wist. Ogranicza 
przepływ informacji między przeciwnikami. 
Jest najpotężniejszym, najbardziej 
despotycznym i dominującym otwarciem w 
brydżu. Dużo częściej naciągamy 14pc 
do otwarcia 1\nt niż 17 do otwarcia 1\minor. 
Wyjątkami są ręce z kolorami starszymi, 
z którymi otwarciem 1\nt łatwo omijamy 
końcówki w kolory starsze, ale wasz system 
nawet nie walczy z tym problemem. 
Optymalnie jest otwieranie 1\nt z 5 
starszą w silę (14)15(16)pc, a 1\major w silę 16-17.

\pagebreak
\section*{Rozdanie 3}
\handdiagramv{\vhand{AQ65}{Q97}{K82}{J97}}
{\vhand{9842}{AKT54}{Q9}{64}}
{\vhand{3}{J632}{J654}{AQT5}}
{\vhand{KJT7}{8}{AT73}{K832}}
{EW}

\begin{table}[h!]
    \centering
    \begin{tabular}{cccc}
        \vul{W} (K) & \nvul{N} & \vul{E} (B) & \nvul{S}\\
        -- & -- & -- & \pass \\
        1\diams & \dbl & 1\hearts & \pass \\
        1\spades & \pass & 2\spades & all \pass \\
    \end{tabular}
\end{table}

[KG] Otwarłam z 11pc i po raz trzeci wylądowałam na rozgrywce.
Z pałą na \nvul{N} zupełnie się nie zgadzam.
Na 1\hearts niestety musiałam dać forsujące 1\spades 
(nie chciałam grać z punktami przed \nvul{N} 
z naciąganym otwarciem i
niedzielącymi się atutami i singlem w kolorze partnera).

Na szczęście po raz kolejny wszystko stało i
jeszcze wist wypuścił nadróbkę.

Wist \xspades 5 wzięty w stole, karo wzięte \xdiams K,
trefl do asa. Przebiłam karo i 2 trefle w stole, kiera w ręce,
na koniec oddając 2 piki.

[KK] Otwarcie nawet się trzyma, 
ale skład 4441 to najbardziej niewygodny, 
sussy skład jaki może być i ciężko się na nim 
licytuje, więc w 11pc można spasować, 
aczkolwiek masz bardzo ładną kartę.
Kontra na \nvul{N} ekstremalnie agresywna i 
nie widać co miałaby wnieść. Lokalizacja 
i jakość honorów w najlepszym wypadku 
średnia, a do tego skład 4333. 
Jedyne co ją broni to pierwszoręczny 
pas partnera, ale to i tak bezcelowe 
pokazywanie rozkładu siły przeciwnikom.

\pagebreak
\section*{Rozdanie 4}
\handdiagramv{\vhand{AKT5}{63}{Q73}{7543}}
{\vhand{Q64}{J98}{AJ6}{AJT6}}
{\vhand{8732}{QT54}{852}{K9}}
{\vhand{J9}{AK72}{KT94}{Q82}}
{NSEW}

\begin{table}[h!]
    \centering
    \begin{tabular}{cccc}
        \vul{W} & \vul{N} (B) & \vul{E} & \vul{S} (K) \\
        1\clubs & \pass & \alrts{2\spades} & \pass \\
        3\nt & all pass & & \\
    \end{tabular}
\end{table}

[KG] Wist \xspades A, dołożone \xspades 7 (demarka...? Może wsm
powinnam markę, jako że mam cztery piki?). Bartek zmienił atak na trefla,
wzięłam na króla i zagrałam w kiera. Rozgrywający
wziął na \xhearts 8 i na prawdę nie wiem dlaczego wziął tylko swoje...

[KK] O kurde, prawie że przykład z prezentacji xD \\
Przy zrzutce jakościowej nie myślimy o tym co mamy, 
tylko czy chcemy kontynuacji. Partner wistuje 
\xspades A, czyli ma 4, lub 3 piki. 
Niezależnie od tego ile, chcemy kontynuacji. 
\xspades Q i tak weźmie lewę. 
Nie musimy mieć figury, żeby markować. 
Skoro chcemy kontynuacji możemy dać markę. 
To znaczy musimy. Poza tym czemu 7, mamy 8. 
Nie ma zrzucania drugiej karty od góry jako 75\% 
demarki. Zrzucamy markę, albo demarkę. 
Jednoznacznie. Nie robimy partnerowi wody z mózgu.

\pagebreak
\section*{Rozdanie 5}
\handdiagramv{\vhand{K5}{A7}{987632}{J54}}
{\vhand{AQ82}{J4}{AKT}{KQT8}}
{\vhand{3}{K96532}{Q5}{9632}}
{\vhand{JT9764}{QT8}{J4}{A7}}
{NS}

\begin{table}[h!]
    \centering
    \begin{tabular}{cccc}
        \nvul{W} & \vul{N} (B) & \nvul{E} & \vul{S} (K) \\
        -- & \pass & 1\clubs & 2\hearts \\
        2\spades & \pass & 3\hearts & \pass \\
        3\nt & \pass & 4\spades & all pass \\
    \end{tabular}
\end{table}

[KG] Z 64 Mimo słabego koloru zdecydowałam się blokować 
(chociaż chyba nie mam na blok w tych założeniach?).
Wist w karo (z nadzieją na przebitkę?) wypuścił (+3).
Za zagraniem asa kier przemawia brak antykontry na 3\hearts,
ale jeśli faktycznie biorę przebitkę karową to po wiście asem nie będę
miała dojścia. Rozgrywający mógłby też mieć \xhearts K, w końcu pokazał trzymanie,
ale ja miałam do dyspozycji antykontrę. Chociaż może nie dałabym jej
z perspektywą 3\nt i szóstym QJT...?

Natomiast myślę, że z renonsem karo mogłabym dać na koniec pałę wistową,
licząc, że Bartek ma długość i się domyśli (chyba że pała jest jednoznacznie na trefla?).

[BS] Tak, pas na 3\hearts jest oczywisty ale weź tu pasa zauważ xdd\\
Pała jest wistowa na domyśl się i niekoniecznie 
musi być poprzedzona antykontrą. 
Może jeszcze 3 razy i będę widział takie pasy. 
W karo, bo źle policzyłem atuty i myślałem, że 
jak wezmę na \xspades K to dostaniesz klepę.

[KK] 8pc+25pc=33pc. Partnerka blokowała, 
brak antykontry=\xhearts K co najmniej. 
Zostaje jej \textbf{maksymalnie} 4pc.\\
1. nie wskazała chęci szukania przebitki, \\
2. Szansa na przebitkę jest bardzo mała.

Nie ma na blok, za mało lew. \xdiams Q w treflach 
by przeważyła.

\section*{Rozdanie 6}
\handdiagramv{\vhand{T74}{763}{76432}{65}}
{\vhand{9852}{AQ5}{Q95}{AKQ}}
{\vhand{AQ63}{KJ84}{J}{9843}}
{\vhand{KJ}{T92}{AKT8}{JT72}}
{EW}

\begin{table}[h!]
    \centering
    \begin{tabular}{cccc}
        \vul{W} & \nvul{N} (B) & \vul{E} & \nvul{S} (K)\\
        -- & -- & 1\nt & \alrts{2\clubs} \\
        3\nt & all pass & & \\
    \end{tabular}
\end{table}

[KG] Trochę mi się nudziło i w zielonych z przyzwoitymi punktami 
weszłam \\
2\clubs = \major z 44. Ja z ręką \vul{W} dałabym kontrę,
ale może założenia zachęciły przeciwnika do 3\nt.
Wist \xspades 3 wzięte królem. Rozgrywający ściągnął trefle i kara.
Wypuściło wyrzucenie na trefla \xspades T -- skończyłam
w przymusie wpustkowym, chyba mogłam jeszcze uratować sytuację singlując
króla kier (ale też mogło wypuścić?). Zostawienie \xspades T skończyłoby się tak samo,
bo żeby zerwać przymus musiałabym wyrzucić damę pik zostawiając \xspades A6.
Z ilościówek powinnam znać cały skład, ale skąd
mam wiedzieć o \xspades T...?

[KK] Prawidłowe wejście, na maxy. \\
Nie dajemy kontry, patrzymy na założenia.\\
Obrona zbyt high level. 
Pewnie można myśleć o marcę w pierwszej 
lewie i później pokazaniu składu, żeby wywalić \xspades Q.

\pagebreak
\section*{Rozdanie 7}
\handdiagramv{\vhand{T84}{J}{AKJ943}{J53}}
{\vhand{K532}{Q975}{7}{T742}}
{\vhand{AQ6}{AKT8}{852}{AKQ}}
{\vhand{J97}{6432}{QT6}{986}}
{NSEW}

\begin{table}[h!]
    \centering
    \begin{tabular}{cccc}
        \vul{W} & \vul{N} (B) & \vul{E} & \vul{S} (K) \\
        -- & -- & -- & \alrts{2\clubs} \\
        \pass & \alrts{2\diams} & \pass & 2\nt \\
        \pass & \alrts{3\spades} & \pass & 3\nt \\
        \pass & 4\diams & \pass & 4\hearts \\
        \pass & \alrts{5\hearts} & \pass & 5\nt \\
        \pass & 7\nt & all \pass & \\
    \end{tabular}
\end{table}

[KG] 7\nt bez \xdiams Q ale trzeba było gonić Pawła. Na 6\nt się
bierze nadróbkę, a 7 się przegrywa :( bezpiecznie na -1 się nie opłaca
jak wiadomo że cała sala ugra 6\nt. Grałam kara w strone stołu,
ale babka na \vul{W} dokładała beznamiętnie (tak, wiem, że Kacper by trafił).

[BS] \vul{E} do pierwszego karo dołożył siódemkę, 
jakby miał Q7 to by se ją zostawił, 
żeby umożliwić ci wzięcie piwa.

[KK] Nie rozumiem dlaczego na 6\nt się bierze 
nadróbkę, ale może wy znacie jakieś sekrety, 
których ja nie znam. \\
Jeszcze nie wiem z jakich przesłanek 
przystolikowych, ale bym trafił.

[KG] Już wyjaśnione, dobrze napisałam z tą nadróbką.

\pagebreak
\section*{Rozdanie 8}
\handdiagramv{\vhand{A9654}{542}{Q54}{K6}}
{\vhand{3}{QT63}{AKJT976}{5}}
{\vhand{QJT872}{AKJ}{}{QJT8}}
{\vhand{K}{987}{832}{A97432}}
{}

\begin{table}[h!]
    \centering
    \begin{tabular}{cccc}
        \nvul{W} & \nvul{N} (B) & \nvul{E} & \nvul{S} (K) \\
        3\clubs & \pass & \pass & 3\spades \\
        \pass & 4\spades & 5\diams & \pass \\
        \pass & 5\spades & all pass & \\
    \end{tabular}
\end{table}

[KG] W sumie jak pas jest forsujący, to ja powinnam wynieść.
Wist 3\diams, impas kier, +1. Nie znaleźli przebitki.

[KK] \pass jest forsujący, bo opsy robią 
sobie jaja, a to my mamy przewagę siły. 
Można 3\nt i wynieść w 4\diams. 
Albo po prostu 5\diams z bomby.  

\pagebreak
\section*{Rozdanie 9}
\handdiagramv{\vhand{J7}{KQJ}{AT765}{AJT}}
{\vhand{T94}{T754}{J93}{952}}
{\vhand{KQ83}{93}{K84}{8763}}
{\vhand{A652}{A862}{Q2}{KQ4}}
{EW}

\begin{table}[h!]
    \centering
    \begin{tabular}{cccc}
        \vul{W} & \nvul{N} (B) & \vul{E} & \nvul{S} (K) \\
        -- & 1\nt & all \pass & \\
    \end{tabular}
\end{table}

[KG] Tym razem ja nie inwitowałam z 8... +2, wist \xhearts 7.

[KK] Dobre 8, sztywniutki inwit. 
Chociaż bliżej pasa niż ręka Bartosha. 
Jakby zrobić z tego 4333 i może wykasować 
kilka 8 też bym spasował.

\pagebreak
\section*{Rozdanie 12}
\handdiagramv{\vhand{Q6}{953}{AQ92}{A864}}
{\vhand{J987}{AK762}{3}{QT7}}
{\vhand{AT5}{QJT8}{J84}{K32}}
{\vhand{K432}{4}{KT765}{J95}}
{NS}

\begin{table}[h!]
    \centering
    \begin{tabular}{cccc}
        \nvul{W} & \vul{N} (B) & \nvul{E} & \vul{S} (K) \\
        \pass & 1\clubs & 1\hearts & 1\nt \\
        \pass & \pass & \dbl & \rdbl \\
        2\diams & \dbl & 2\hearts & \pass \\
        2\spades & \pass$^*$ & \pass & \dbl \\
        all \pass & & & \\
    \end{tabular}
\end{table}

[KG] Pała na 1\nt jest co najmniej dziwna.\\
Został wezwany sędzia na namysł ($^*$), ale moja decyzja
była od niego niezależna. Bartek wiedział, że nie mam 4 pików,
a przeciwnicy mają fit i myślał nad wrzuceniem 2\nt. Wist \xclubs A, marka \xclubs 2,
Zmiana na 5\hearts, zabite asem. \xdiams do króla, wzięte asem.
Pod koniec Bartek wyszedł po podwójny renons w \clubs, co chyba nie miało znaczenia, 
ale było fajne i zmusiło rozgrywającego do nadbicia 
mojej \xspades 5 królem.\\
Finalnie -1, 2\nt okazało się być lepsze.

[BS] W rozgrywce nic się nie działo ale miałem straszną padakę 
i liczyłem wszystko 3 razy. Co do licytacji to widzę, że 
2\spades jest bez jednej, bardzo rzadko -2. 
Ale boje się że nie mamy lew na 2\nt. 
Znam skład wznawiajacego, 
więc na wznówkę musi mieć co najmniej niezłe kolory. 
Biorą pewnie 6 lew w kolorach starszych, 
a my nie mamy źródła lew, bo karo się nie dzieli, 
a trefli mamy za mało.

[KK] Nie rozumiem połowy waszych odzywek. 
Co to jest za sztuczka z 1\nt? Nie daję inwitu, 
bo może będą w 2\hearts to ich pierdolniemy. 
Żarcik, nie kontrujemy 2\hearts. 
\rdbl nie forsuje nas w sumie do niczego, 
bo ta sekwencja z 1\nt i \rdbl nie ma żadnego 
sensu, ani prawa bytu, więc uznałbym, 
że każda kontra jest po prostu karna, 
ale takiej licytacji to nie ma. 
Jakby poszła sensowniejsza sekwencja typu 
1\clubs -- 1\hearts -- 1\nt -- \dbl -- \rdbl uznałbym, 
że jesteśmy sforsowani do 2\spades, 
to może zbiec. Chyba nigdy w sekwencji z 
\rdbl na 1\nt nie jesteśmy sforsowani do 
2\nt. No, ale tu nic nie ma sensu, 
więc nawet nie wiem co mam o licytacji powiedzieć. 
Na oko nielicytowanie 1\nt w 11 zajebistych 
punktów z opozycją rozwiązałoby każdy problem. 
Ku zaskoczeniu licytowanie tego co się 
ma bywa najlepszą opcją.   

\pagebreak
\section*{Rozdanie 13}
\handdiagramv{\vhand{KT5}{JT4}{K832}{862}}
{\vhand{AQ74}{A87}{A54}{KJ3}}
{\vhand{93}{Q932}{JT97}{QT7}}
{\vhand{J862}{K65}{Q6}{A954}}
{NSEW}

\begin{table}[h!]
    \centering
    \begin{tabular}{cccc}
        \vul{W} & \vul{N} (B) & \vul{E} & \vul{S} (K) \\
        -- & \pass & 1\clubs & \pass \\
        1\spades & \pass & \alrts{2\diams} & \pass \\
        \alrts{2\hearts} & \pass & 4\spades & all pass \\
    \end{tabular}
\end{table}

[KG] Wist \xclubs, -1.

\pagebreak
\section*{Rozdanie 14}
\handdiagramv{\vhand{532}{J2}{A94}{KJ863}}
{\vhand{T984}{AK53}{T3}{A72}}
{\vhand{AK6}{T9}{KQ7652}{94}}
{\vhand{QJ7}{Q8764}{J8}{QT5}}
{}

\begin{table}[h!]
    \centering
    \begin{tabular}{cccc}
        \nvul{W} & \nvul{N} (B) & \nvul{E} & \nvul{S} (K) \\
        -- & -- & 1\clubs & 1\diams \\
        1\hearts & 2\diams & 2\hearts & 3\diams \\
        3\hearts & all pass & & \\
    \end{tabular}
\end{table}

[KG] \xdiams A, \xdiams, \xspades A, dorzucone \xspades 5,
i właściwie widać, że jest duże, chyba że Bartek miałby
\xspades Q5, a rozgrywająca dołożyła \xspades 7
zamiast \xspades 2 lub 3, żeby mnie zmylić.
Uznałam odruchowo, że \xspades 5 jest mała, ściągnęłam króla,
i rozgrywająca oddała jeszcze tylko \xclubs K, -1
zamiast -2. A 4\diams było swoje (na trafieniu trefla), 
może mogłam wrzucać...?

[KK] Sprzedałaś dokładnie to co miałaś przez 
3\diams, 4\diams jest statystycznie złe. Analizujemy blotki.

\pagebreak
\section*{Rozdanie 15}
\handdiagramv{\vhand{T984}{A82}{JT}{T642}}
{\vhand{Q63}{KJ7}{A74}{AJ87}}
{\vhand{K52}{Q653}{Q8532}{K}}
{\vhand{AJ7}{T94}{K96}{Q953}}
{NS}

\begin{table}[h!]
    \centering
    \begin{tabular}{cccc}
        \nvul{W} & \vul{N} (B) & \nvul{E} & \vul{S} (K) \\
        -- & -- & -- & 1\nt \\
        \pass & 3\nt \\ all pass & & \\
    \end{tabular}
\end{table}

[KG] Zdecydowałam się na wist kierowy, a nie karowy.
Wzięliśmy 3 kiery, karo i \xclubs K. Po karowym 
wzięlibyśmy 3 kara, 2 kiery i (prawie na pewno) \xclubs K.

[KK] Bardzo dobry wist, nie szukali starych.

\pagebreak
\section*{Rozdanie 16}
\handdiagramv{\vhand{JT5}{872}{9865}{A72}}
{\vhand{A4}{J95}{QJ7}{KJ965}}
{\vhand{KQ962}{AK64}{A3}{43}}
{\vhand{873}{QT3}{KT42}{QT8}}
{EW}

\begin{table}[h!]
    \centering
    \begin{tabular}{cccc}
        \vul{W} (K) & \nvul{N} & \vul{E} (B) & \nvul{S}\\
        \pass & \pass & \pass & 1\spades \\
        \pass & 2\spades & \pass & 4\spades \\
        all pass & & & \\
    \end{tabular}
\end{table}

[KG] Jeśli przeciwnicy mają w systemie drury, to wrzucenie
końcówki lekko świrowe. Tym razem udało mi się nie 
wypuścić wistem, chociaż pewnie logiczniej byłoby
wyciągnąć pika (ale niewiele tu brakuje żeby pik wypuszczał).
-1.

[BS] No ja bym wyjął pika w momencie kiedy \vul{S} wyjął 4\spades z boxa.

[KK] Wist w pika wydaje się być automatyczny, najbezpieczniejszy.

\pagebreak
\section*{Rozdanie 17}
\handdiagramv{\vhand{T9}{AK64}{2}{AKQ864}}
{\vhand{Q762}{J3}{J976}{JT9}}
{\vhand{KJ54}{T987}{8543}{7}}
{\vhand{A83}{Q52}{AKQT}{532}}
{}

\begin{table}[h!]
    \centering
    \begin{tabular}{cccc}
        \nvul{W} (K) & \nvul{N} & \nvul{E} (B) & \nvul{S}\\
        -- & 1\clubs & \pass & 1\diams \\
        \pass & 2\clubs & \pass & 2\hearts \\
        \pass & 4\hearts & all pass & \\
    \end{tabular}
\end{table}

[KG] Wist \xdiams A, dorzucone 7? Uznałam to za Lavinthala na pika
i bojąc się, że Rozgrywający łatwo wyimpasuje moją damę kier
i zbierze dzielące się trefle, wyszłam \xspades A,
oszczędzając rozgrywającemu
konieczności zgadywania pika, swoje. Do \xspades A dostałam \xspades 2,
więc tym bardziej liczyłam na króla u partnera.

[BS] Do \xdiams A wyrzuciłem środkowe, nw czy mam przesłanki, 
że akurat zostanie wzięte za wysokie. Chciałem kontynuacji 
kara, przypominając sobie podobne rozdanie sprzed paru miesięcy, 
gdzie rozgrywający miał renons trefl. Wtedy trzeba natychmiast 
skrócić stół i jeśli \nvul{S} musi oddać atuta dostaje drugi skrót i 
trefle idą spać do łóżeczka. \xspades 2 bo myślałem że masz króla ale idk.

[KK] Wszystko elegancko. Dobrze, że \xspades A, a nie blotka. 

\pagebreak
\section*{Rozdanie 18}
\handdiagramv{\vhand{AQ64}{54}{KQT65}{T3}}
{\vhand{T9}{KQJ}{A974}{KJ62}}
{\vhand{KJ752}{T9632}{J32}{}}
{\vhand{83}{A87}{8}{AQ98754}}
{NS}

\begin{table}[h!]
    \centering
    \begin{tabular}{cccc}
        \nvul{W} (K) & \vul{N} & \nvul{E} (B) & \vul{S}\\
        -- & -- & 1\clubs & \alrts{2\hearts} \\
        3\clubs & \pass & 3\diams & \pass \\
        3\nt & \pass & 5\clubs & all pass \\
    \end{tabular}
\end{table}

[KG] Powinnam dać 4\clubs zamiast 3\nt, ale na szczęście
partner uratował kontrakt bojąc się o piki. 5\clubs czapowe.

[BS] Typiarz wszedł 2\hearts = kiery i inny, stąd mój pomysł z pikami.

[KK] Nie musicie losować. Zakładam, że 
3\clubs było naturalne i 3\diams również? 
Jakieś z wartości? No nie wiem, tak czy 
inaczej można dać 3\hearts jako 3ci kolor, 
jeśli zostają 2 kolory to pokazujemy 
zatrzymania. Wtedy Bartek może dać 
3\spades; póltrzymanie, 
3\nt; trzymanie pik, 
4\clubs; brak trzymania i byście wszystko wiedzieli.
 Tak na dobrą sprawę 5\clubs 
 to decyzja wynikająca z braku zaufania.  

\pagebreak
\section*{Rozdanie 19}
\handdiagramv{\vhand{J8532}{KJ}{KJ9}{J92}}
{\vhand{KQ6}{AQ653}{AQ42}{7}}
{\vhand{T74}{942}{853}{QT64}}
{\vhand{A9}{T87}{T76}{AK853}}
{EW}

\begin{table}[h!]
    \centering
    \begin{tabular}{cccc}
        \vul{W} (K) & \nvul{N} & \vul{E} (B) & \nvul{S}\\
        \pass & \pass & 1\hearts & \pass \\
        2\clubs & \pass & 4\clubs & \pass \\
        4\hearts & all pass & \\
    \end{tabular}
\end{table}

[KG] No otworzyć to trzeba było. Na 4\clubs też mogłam 4\diams,
niby last train ale jakoś głupio mi było bez cue \diams
i z \xclubs AK do krótkości. Raczej i tak kończymy w 5\hearts.
A szlemik bardzo mierny. Wist \xclubs 4, +2.

[BS] Zrolowałem se lewę, bo zapomniałem, że istnieje 
coś takiego jak KJ sec i trzecie karo przebiłem dychą.

[KK] Z \vul{W} na prawdę poszedł \pass na otwarciu? 
Poza tym wszystko spoko, ale \pass? 
Raz w życiu widziałem bardziej zamulającego 
pasa w 11pc. Chociaż nie, ten jest imponujący xD

\pagebreak
\section*{Rozdanie 20}
\handdiagramv{\vhand{T976}{9}{KJ8765}{T2}}
{\vhand{843}{QJ842}{43}{J85}}
{\vhand{QJ}{A7653}{AQ}{AK43}}
{\vhand{AK52}{KT}{T92}{Q976}}
{NSEW}

[KG] Rozdanie na Pluteckich, dociągnęli do 5\diams bez bilansu,
idzie. Ale 3\nt lepsze, na podziale kar, a jeszcze wist 
w pika wypuszcza nadróbkę.

\pagebreak
\section*{Rozdanie 21}
\handdiagramv{\vhand{AJ4}{A85}{QJT7}{864}}
{\vhand{KQ8765}{KJ93}{}{T97}}
{\vhand{T93}{QT2}{K965}{KQJ}}
{\vhand{2}{764}{A8432}{A532}}
{NS}

\begin{table}[h!]
    \centering
    \begin{tabular}{cccc}
        \nvul{W} (K) & \vul{N} & \nvul{E} (B) & \vul{S}\\
        -- & \alrts{1\clubs} & 1\spades & \alrts{1\nt}\\
        \pass & 2\diams & 2\hearts & 3\diams \\
        all \pass & & & \\
    \end{tabular}
\end{table}

[KG] Myślałam nad pałą, mogło iść ale raczej widać, że ciężko.
Pluteccy pokazali 10-14 po obu stronach.\\
Wzięliśmy pika, przebitkę, dwa asy i kiera. Wist \xspades K.
Na asa trefl dostałam wysoką zrzutkę, żeby nie wychodzić w kiera,
wyszłam i tak, ale to chyba nic nie zmienia.

[BS] Rozwiązaliśmy mu kiery. Piki nie uciekają, przebitki nie ma.

[KK] Nie wiem czy bym nie wznowił kontrą na 
Pluteckich na maxy. Mamy petardę układową, 
a partner może tylko na to czekać. 
Po prostu głupio z takim układem odpuszczać. 
To byłby pewnie rodzaj ekstremalnie ostrej 
kontry atakująco-obronnej.

\pagebreak
\section*{Rozdanie 22}
\handdiagramv{\vhand{A5}{AJ87}{J842}{AT9}}
{\vhand{KQJ3}{92}{KT}{KQJ65}}
{\vhand{8}{KQT5}{AQ975}{743}}
{\vhand{T97642}{643}{63}{82}}
{EW}

\begin{table}[h!]
    \centering
    \begin{tabular}{cccc}
        \vul{W} (K) & \nvul{N} & \vul{E} (B) & \nvul{S}\\
        -- & -- & 1\clubs & 1\diams \\
        \pass & 1\hearts & 1\spades & 2\hearts \\
        3\spades &  \dbl & all \pass & \\
    \end{tabular}
\end{table}

[KG] Z 1\spades się zgadzam, nie mam chyba alternatywy
do tego 3\spades, na 2\spades od razu też nie mam.
Miałam idealną rękę na jednostronne multi, ale 
wejście popsuło plany. 3\spades jest -2,
więc not worth it. Wist \xhearts K.

[BS] Nie masz na jednostronne multi. Po co stawiać jak nie trzeba.

[KK] Jasny chuj, Twoje jajniki są większe 
niż niejedne jaja. Nie ma nawet rozmowy 
od jednostronnym multi, powinno być w 
przedziale 3-8, po partii jakieś 5-8. 
Możesz dać zawsze 2\spades, nie ma sensu stawiać.

\pagebreak
\section*{Rozdanie 23}
\handdiagramv{\vhand{9}{KQT9}{Q65}{AQT96}}
{\vhand{QT62}{A865}{KJ84}{8}}
{\vhand{}{J2}{AT93}{KJ75432}}
{\vhand{AKJ87543}{743}{72}{}}
{NSEW}

\begin{table}[h!]
    \centering
    \begin{tabular}{cccc}
        \vul{W} (K) & \vul{N} & \vul{E} (B) & \vul{S}\\
        -- & -- & -- & 3\clubs \\
        4\spades & 5\clubs & 5\diams & \pass \\
        5\spades & all \pass & & \\
    \end{tabular}
\end{table}

[KG] Bartek zrespektował mój brak współpracy. Kara nie trafiłam,
-2. Wist \xhearts K. Co ciekawe, prawie cała sala decydowała się na
(co najmniej) 6\clubs, które nie idzie, ale było
zdarzało się =. A jedna para nawet odważyła się skontrować 5\spades (XD).
Na 6\spades z ręką \vul{S} powinno się dać pałę mając jedną lewę (czyli tak jak tu),
a \vul{N} mając też jedną powinien spasować (\xhearts KQ chyba
wystarczy).

[KK] Spoko licytacja, wszystko git. 
Nie pasujemy w tak duże karty, nie 
dajemy też 4\spades, bo sprzedalibyśmy 
podacola, o tym będzie jeszcze na 
wykładzie. I to za 2 tygodnie. 

\pagebreak
\section*{Rozdanie 24}
\handdiagramv{\vhand{T943}{765}{J2}{A943}}
{\vhand{Q865}{A43}{T75}{K52}}
{\vhand{KJ2}{KJ9}{K3}{QJ876}}
{\vhand{A7}{QT82}{AQ9864}{T}}
{}

\begin{table}[h!]
    \centering
    \begin{tabular}{cccc}
        \nvul{W} (K) & \nvul{N} & \nvul{E} (B) & \nvul{S}\\
        1\diams & \pass & 1\spades & \pass \\
        2\diams & all \pass & & \\
    \end{tabular}
\end{table}

[KG] Wist \xclubs A, \clubs.

[BS] W drugiej wyszedł w trefla. 
Zabiłaś królem, impas karo i 
typ chyba wywalił waleta (chad) 
zmuszając cię do wpuszczenia się do ręki, +2.

\pagebreak
\section*{Rozdanie 25}
\handdiagramv{\vhand{K752}{K3}{AQ5}{QJ93}}
{\vhand{93}{872}{T6}{AK7642}}
{\vhand{Q84}{QJT94}{J9}{T85}}
{\vhand{AJT6}{A65}{K87432}{}}
{EW}

\begin{table}[h!]
    \centering
    \begin{tabular}{cccc}
        \vul{W} & \nvul{N} (B) & \vul{E} & \nvul{S} (K) \\
        -- & 1\nt & \pass & 2\diams \\
        \pass & 2\hearts & all \pass & \\
    \end{tabular}
\end{table}

[BS] Wist \xclubs K, +2.

Rozdanie z fajnym manewrem i festiwalem wypuszczania.
Wist \xclubs K po namyśle przebity (?). 
Natychmiastowe wyjście w karo, nie dałem się 
porobić i wziąłem na 9. 
Kier do asa i odblokowanie. 
Odwrót w karo wziąłem damą. 
Ściągnąłem 2 kiery.

Teraz uwaga bo się ustawiły Widły Mortona. 
Gram trefla do waleta. Jeśli \vul{E} zabije asem, 
daje mi komunikację do ręki 
żeby wyrzucić pika na trefla lub rozwiązuje mi kolor pikowy. 
Jeśli przepuści - już nie weźmie, 
bo wyrzucę trefla na karo.
\vul{E} zabił i wyszedł w pika, którego ze 
zdziwieniem wziąłem Królem (?) pokazując karty. +2.

[KK] Gratuluję 69 w drugiej lewie. 
No powiem, że całkiem kurwa nieźle. 
A nie, wróć. Najlepszym trenerem jest 
Maciej Kędzierski, muszę być jak on. 
No spoko rozgrywka, może jakby mnie 
obudzić po 3-dniowym melanżu w środku 
nocy bym tak nie zrobił. 
A tak serio to mega dobre, masz 
ogromny potencjał mordeczko, więcej 
wiary w siebie, więcej gry, więcej 
analizy porozdaniowej. 

\pagebreak
\section*{Rozdanie 26}
\handdiagramv{\vhand{JT53}{JT873}{5}{AT2}}
{\vhand{A74}{KQ2}{Q732}{J54}}
{\vhand{K98}{A65}{AJ86}{987}}
{\vhand{Q62}{94}{KT94}{KQ63}}
{NSEW}

\begin{table}[h!]
    \centering
    \begin{tabular}{cccc}
        \vul{W} & \vul{N} (B) & \vul{E} & \vul{S} (K)\\
        -- & -- & 1\clubs & \pass \\
        1\nt & all \pass & & \\
    \end{tabular}
\end{table}

[KG] Wist \xhearts J. O, burtówka. Jeszcze na naszą korzyść.\\
Na koniec 2x miałam szansę zabić karo asem (7ma lewa) i 2x puściłam.

[BS] Wist \xhearts J do K i A (1), 
kier przepuszczony(2), kier zabity. 
Trefl do Asa (3). 
Ściągnąłem 2 kiery (4,5), 
rozgrywający chyba wyrzucił jakiegoś 
pika blokując se kolor. 
\xspades J do K (6). 
Widać, że rozgrywający ma:\\
2 pikowe (akurat tu nie miał bo coś spierdolił z komunikacją)\\
1 kierową\\
3 treflowe (bo sam je grał, więc ma 4 a może i 5 w ręce, nie ma znaczenia, jak mam dychę to mi spada.)\\
Puszczenie raz mogło mieć sens bo nie pamiętam co wywalał. 2x na pewno nie.

[KK] Wyjście w pika i tak ma sporo sensu. 
I tak by sobie poradził z tymi karami. 
Wyrzucenie pika nie jest takie głupie, 
nie potrzebuje dojścia do lewy tylko podwójnego 
zatrzymania. Na ten moment zagrożeniem 
dla niego są piki. Kar potrzebuje, 
żeby mieć co wyrabiać.

\pagebreak
\section*{Rozdanie 27}
\handdiagramv{\vhand{KJ83}{AJ75}{QT4}{A2}}
{\vhand{952}{863}{J82}{Q764}}
{\vhand{QT64}{}{7653}{KJT98}}
{\vhand{A7}{KQT942}{AK9}{53}}
{}

\begin{table}[h!]
    \centering
    \begin{tabular}{cccc}
        \nvul{W} (K) & \nvul{N} & \nvul{E} (B) & \nvul{S}\\
        -- & -- & -- & \pass \\
        \pass & 1\hearts & 1\nt & \pass \\
        \alrts{2\clubs} & 2\hearts & 2\spades & all \pass \\
    \end{tabular}
\end{table}

[KG] Zapomniałam system :') 2\clubs powinno być \then\ \diams,
albo inwitem z krótkością,
na szczęście bez większych konsekwencji. Bartek dał 
bardzo fajne 2\spades = góra z 4 pikami na wypadek inwitu 
/ popraw na karo z karami. A ja się ucieszyłam i spasowałam,
chociaż właściwie dokład 4\spades mógł wcale nie być świrem? \\
Wist \xspades 2.

[BS] W rozgrywce przeciwnik po dojściu \xspades A ściągnął \xdiams AK 
i trzeba było trafić damę trefl co uczyniłem.

[KK] 2\spades też mi się całkiem podoba. 
Może nie jest to dość silna ręka, 
ale wygląda nieźle. \pass na 2\spades to trup, 
wypada coś ruszyć. Może nawet 4 od razu 
z dobrze zlokalizowanym renonsem.

\pagebreak
\section*{Rozdanie 28}
\handdiagramv{\vhand{K432}{QJ42}{KT5}{Q3}}
{\vhand{Q97}{T7}{986}{KJ864}}
{\vhand{J5}{A85}{AJ7432}{AT}}
{\vhand{AT86}{K963}{Q}{9752}}
{NS}

\begin{table}[h!]
    \centering
    \begin{tabular}{cccc}
        \nvul{W} (K) & \vul{N} & \nvul{E} (B) & \vul{S}\\
        \pass & \pass & 1\nt & \pass \\
        2\clubs & \pass & \pass$^{\text{XD}}$ & 2\diams \\
        \pass & 3\diams & all \pass \\
    \end{tabular}
\end{table}

[KG] O dziwo mimo bilansu tylko 3 pary zagrały końcówkę,
w tym jedna 5\diams. 3\nt z ręki \vul{N} idzie.\\
Ania miała na 2\nt (na \vul{N}), Michał miał na
dokład. Ale i tak zagrali 3\diams+1 za 64\%. Wist \xclubs 7.

[KK] Jestem absolutnym fanem odzywki 1\nt 
i passa na staymana. Idealne wskazanie wistu, 
dobre zrobienie słabej opozycji w ciula. 
Normalnie to nie zadziała, ale na maxy, 
na taką parę możemy się bawić. W żadnej innej 
sytuacji tak nie robimy, bardziej jako 
jednorazowa sztuczka. Pewnie spora satysfakcja.

\pagebreak
\section*{Rozdanie 29}
\handdiagramv{\vhand{JT73}{T832}{97}{A84}}
{\vhand{9}{97}{AK843}{KJT76}}
{\vhand{AKQ}{AJ64}{J652}{Q9}}
{\vhand{86542}{KQ5}{QT}{532}}
{NSEW}

\begin{table}[h!]
    \centering
    \begin{tabular}{cccc}
        \vul{W} (K) & \vul{N} & \vul{E} (B) & \vul{S}\\
        -- & \pass & 1\diams & \dbl \\
        \pass & 1\hearts & 2\clubs & \dbl \\
        2\diams & 2\hearts & 3\diams & 3\hearts \\
    \end{tabular}
\end{table}

[KG] Wydawało mi się, że zamiast 3\diams powinno się dać 3\clubs mimo wszystko.
Mogę mieć 3-3 w młodych z figurami w treflach i 3\clubs będzie lespsze,
(a na 2\diams musiałam poprawić). Natomiast z moją ręką
2\diams nie muszę mówić, nie ma opcji, że na 2\clubs\dbl\ spasują.
Więc ostatecznie po moim 2\diams na niewymuszonej zgadzam się z 3\diams.
Ale jak nie dam 2\diams to pewnie zagrają 2\hearts, nie 3. Do wistu \xdiams A
wyrzuciłam \xdiams T, wyrzucenie damy wypuszcza, bo nie ma komunikacji do przebitki.

Wzięliśmy 2 kara, przebitkę i 2 kiery.

[KK] Po prostu z 55 powtarzasz kolor, 
partner mógł mieć negatywny wybór koloru. 
Zawsze z 2-3 wynosimy w pierwszy kolor otwarcia. 

\pagebreak
\section*{Rozdanie 30}
\handdiagramv{\vhand{73}{Q9}{KT2}{KQJ984}}
{\vhand{KJ86}{AJT8}{8653}{5}}
{\vhand{AQT952}{4}{AQ7}{A73}}
{\vhand{4}{K76532}{J94}{T62}}
{}

\begin{table}[h!]
    \centering
    \begin{tabular}{cccc}
        \nvul{W} (K) & \nvul{N} & \nvul{E} (B) & \nvul{S}\\
        -- & -- &\pass & 1\spades \\
        \pass & 2\clubs & \pass & 4\hearts \\
        \pass & 4\spades & all \pass & \\
    \end{tabular}
\end{table}

[KG] Wist 5\clubs. gdzieś po drodze zrzuciłam 
\xhearts 6 zamiast \xclubs T (na kiera).
Jakoś nie chciałam tego trefla wyrzucić, kara spod wałka też nie,
lekko podświadomie uznałam, że środkowy kier jest, z grubsza, na kiera.

[KK] Z grubsza może być zrzutką neutralną. 
Nie wiem jaka była obrona, ale trefl 
wydaje się być kolorem wykluczonym. 
W takich sytuacjach można grać 
marką bezpośrednią, o której powiemy 
na prezentacji o sygnalizacji w obronie. 
Raczej wszystko git.

\end{document}