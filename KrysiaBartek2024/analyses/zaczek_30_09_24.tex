\documentclass[12pt, a4paper]{article}
\usepackage{import}

\import{../../lib/}{bridge.sty}

\title{Żaczek 25.09.24}
\author{Krysia \& Bartek}
\begin{document}
\maketitle

\section*{Rozdanie 1}
\handdiagramv{\vhand{T8542}{K3}{KJ3}{KJ7}}
{\vhand{Q7}{A2}{T9765}{Q963}}
{\vhand{93}{QJ985}{Q42}{852}}
{\vhand{AKJ6}{T764}{A8}{AT4}}
{}

\begin{table}[h!]
    \centering
    \begin{tabular}{cccc}
        \nvul{W} (K) & \nvul{N} & \nvul{E} (B) & \nvul{S}\\
        -- & \pass & \pass & \pass \\
        1\nt & all pass & & \\
    \end{tabular}
\end{table}

Bartek z 8pc zdecydował się nie inwitować,
3\nt po dość oczywistym wiście pikowym 
(u nas \xspades 4) niestety czapowe. Po innym też
idzie, chociaż nie tak bezproblemowo.

\hfill -K

\pagebreak
\section*{Rozdanie 2}
\handdiagramv{\vhand{J6532}{5}{KJ6}{KQ53}}
{\vhand{874}{AK843}{Q82}{97}}
{\vhand{KT9}{JT97}{T743}{84}}
{\vhand{AQ}{Q62}{A95}{AJT62}}
{NS}

\begin{table}[h!]
    \centering
    \begin{tabular}{cccc}
        \nvul{W} (K) & \vul{N} & \nvul{E} (B) & \vul{S}\\
        -- & -- & \pass & \pass \\
        1\clubs & 1\spades & \dbl & \pass \\
        2\nt & \pass & \alrts{3\clubs} & \pass \\
        3\hearts & \pass & 4\hearts & all pass \\
    \end{tabular}
\end{table}

Tu niestety zlagowałam w licytacji, chyba ze względu
na wejście przeciwnika -- pokazałam 4 kiery.
Na szczęście po systemowym 3\diams (3 kiery) też
kończymy w 4\hearts. Wist \xclubs K pomógł, bez niego
pewnie starałabym się impasować trefle zamiast liczyć
na wejście 1\spades z piątego wałka. Ściągnęłam 2
kiery i zagrałam trefla. \vul{N} wyszedł w pika, po raz kolejny
ułatwiając mi zadanie. Oddałam \xhearts J, \xclubs Q i \xdiams K.

\hfill -K

\pagebreak
\section*{Rozdanie 3}
\handdiagramv{\vhand{AQ65}{Q97}{K82}{J97}}
{\vhand{9842}{AKT54}{Q9}{64}}
{\vhand{3}{J632}{J654}{AQT5}}
{\vhand{KJT7}{8}{AT73}{K832}}
{EW}

\begin{table}[h!]
    \centering
    \begin{tabular}{cccc}
        \vul{W} (K) & \nvul{N} & \vul{E} (B) & \nvul{S}\\
        -- & -- & -- & \pass \\
        1\diams & \dbl & 1\hearts & \pass \\
        1\spades & \pass & 2\spades & all \pass \\
    \end{tabular}
\end{table}

Otwarłam z 11pc i po raz trzeci wylądowałam na rozgrywce.
Z pałą na \nvul{N} zupełnie się nie zgadzam.
Na 1\hearts niestety musiałam dać forsujące 1\spades 
(nie chciałam grać z punktami przed \nvul{N} 
z naciąganym otwarciem i
niedzielącymi się atutami i singlem w kolorze partnera)

Na szczęście po raz kolejny wszystko stało i
jeszcze wist wypuścił nadróbkę.

Wist \xspades 5 wzięty w stole, karo wzięte \xdiams K,
trefl do asa. Przebiłam karo i 2 trefle w stole, kiera w ręce,
na koniec oddając 2 piki.

\hfill -K

\pagebreak
\section*{Rozdanie 4}
\handdiagramv{\vhand{AKT5}{63}{Q73}{7543}}
{\vhand{Q64}{J98}{AJ6}{AJT6}}
{\vhand{8732}{QT54}{852}{K9}}
{\vhand{J9}{AK72}{KT94}{Q82}}
{NSEW}

\begin{table}[h!]
    \centering
    \begin{tabular}{cccc}
        \vul{W} & \vul{N} (B) & \vul{E} & \vul{S} (K) \\
        1\clubs & \pass & \alrts{2\spades} & \pass \\
        3\nt & all pass & & \\
    \end{tabular}
\end{table}

Wist \xspades A, dołożone \xspades 7 (demarka...? Może wsm
powinnam markę, jako że mam cztery piki?). Bartek zmienił atak na trefla,
wzięłam na króla i zagrałam w kiera. Rozgrywający
wziął na \xhearts 8 i na prawdę nie wiem dlaczego wziął tylko swoje...

\hfill -K

[pamiętasz co tam zrobił?]

\pagebreak
\section*{Rozdanie 5}
\handdiagramv{\vhand{K5}{A7}{987632}{J54}}
{\vhand{AQ82}{J4}{AKT}{KQT8}}
{\vhand{3}{K96532}{Q5}{9632}}
{\vhand{JT9764}{QT8}{J4}{A7}}
{NS}

\begin{table}[h!]
    \centering
    \begin{tabular}{cccc}
        \nvul{W} & \vul{N} (B) & \nvul{E} & \vul{S} (K) \\
        -- & \pass & 1\clubs & 2\hearts \\
        2\spades & \pass & 3\hearts & \pass \\
        3\nt & \pass & 4\spades & all pass \\
    \end{tabular}
\end{table}

Z 64 Mimo słabego koloru zdecydowałam się blokować 
(chociaż chyba nie mam na blok w tych założeniach?).
Wist w karo (z nadzieją na przebitkę?) wypuścił (+3).
Za zagraniem asa kier przemawia brak antykontry na 3\hearts,
ale jeśli faktycznie biorę przebitkę karową to po wiście asem nie będę
miała dojścia. Rozgrywający mógłby też mieć \xhearts K, w końcu pokazał trzymanie,
ale ja miałam do dyspozycji antykontrę. Chociaż może nie dałabym jej
z perspektywą 3\nt i szóstym QJT...?

Natomiast myślę, że z renonsem karo mogłabym dać na koniec pałę wistową,
licząc, że Bartek ma długość i się domyśli (chyba że pała jest jednoznacznie na trefla?).

\hfill -K

\pagebreak
\section*{Rozdanie 6}
\handdiagramv{\vhand{T74}{763}{76432}{65}}
{\vhand{9852}{AQ5}{Q95}{AKQ}}
{\vhand{AQ63}{KJ84}{J}{9843}}
{\vhand{KJ}{T92}{AKT8}{JT72}}
{EW}

\begin{table}[h!]
    \centering
    \begin{tabular}{cccc}
        \vul{W} & \nvul{N} (B) & \vul{E} & \nvul{S} (K)\\
        -- & -- & 1\nt & \alrts{2\clubs} \\
        3\nt & all pass & & \\
    \end{tabular}
\end{table}

Trochę mi się nudziło i w zielonych z przyzwoitymi punktami 
weszłam \\
2\clubs = \major z 44. Ja z ręką \vul{W} dałabym kontrę,
ale może założenia zachęciły przeciwnika do 3\nt.
Wist \xspades 3 wzięte królem. Rozgrywający ściągnął trefle i kara.
Wypuściło wyrzucenie na trefla \xspades T -- skończyłam
w przymusie wpustkowym, chyba mogłam jeszcze uratować sytuację singlując
króla kier (ale też mogło wypuścić?). Zostawienie \xspades T skończyłoby się tak samo,
bo żeby zerwać przymus musiałabym wyrzucić damę pik zostawiając \xspades A6.
Z ilościówek powinnam znać cały skład, ale skąd
mam wiedzieć o \xspades T...?

\hfill -K

\pagebreak
\section*{Rozdanie 7}
\handdiagramv{\vhand{T84}{J}{AKJ943}{J53}}
{\vhand{K532}{Q975}{7}{T742}}
{\vhand{AQ6}{AKT8}{852}{AKQ}}
{\vhand{J97}{6432}{QT6}{986}}
{NSEW}

\begin{table}[h!]
    \centering
    \begin{tabular}{cccc}
        \vul{W} & \vul{N} (B) & \vul{E} & \vul{S} (K) \\
        -- & -- & -- & \alrts{2\clubs} \\
        \pass & \alrts{2\diams} & \pass & 2\nt \\
        \pass & \alrts{3\spades} & \pass & 3\nt \\
        \pass & 4\diams & \pass & 4\hearts \\
        \pass & \alrts{5\hearts} & \pass & 5\nt \\
        \pass & 7\nt & all \pass & \\
    \end{tabular}
\end{table}

7\nt bez \xdiams Q ale trzeba było gonić Pawła. Na 6\nt się
bierze nadróbkę, a 7 się przegrywa :( bezpiecznie na -1 się nie opłaca
jak wiadomo że cała sala ugra 6\nt. Grałam kara w strone stołu,
ale babka na \vul{W} dokładała beznamiętnie.

\hfill -K

\pagebreak
\section*{Rozdanie 8}
\handdiagramv{\vhand{A9654}{542}{Q54}{K6}}
{\vhand{3}{QT63}{AKJT976}{5}}
{\vhand{QJT872}{AKJ}{}{QJT8}}
{\vhand{K}{987}{832}{A97432}}
{}

\begin{table}[h!]
    \centering
    \begin{tabular}{cccc}
        \nvul{W} & \nvul{N} (B) & \nvul{E} & \nvul{S} (K) \\
        3\clubs & \pass & \pass & 3\spades \\
        \pass & 4\spades & 5\diams & \pass \\
        \pass & 5\spades & all pass & \\
    \end{tabular}
\end{table}

W sumie jak pas jest forsujący, to ja powinnam wynieść.
Wist 3\diams, impas kier, +1. Nie znaleźli przebitki.

\pagebreak
\section*{Rozdanie 9}
\handdiagramv{\vhand{J7}{KQJ}{AT765}{AJT}}
{\vhand{T94}{T754}{J93}{952}}
{\vhand{KQ83}{93}{K84}{8763}}
{\vhand{A652}{A862}{Q2}{KQ4}}
{EW}

\begin{table}[h!]
    \centering
    \begin{tabular}{cccc}
        \vul{W} & \nvul{N} (B) & \vul{E} & \nvul{S} (K) \\
        -- & 1\nt & all \pass & \\
    \end{tabular}
\end{table}

Tym razem ja nie inwitowałam z 8... +2, wist \xhearts 7.

\pagebreak
\section*{Rozdanie 12}
\handdiagramv{\vhand{Q6}{953}{AQ92}{A864}}
{\vhand{J987}{AK762}{3}{QT7}}
{\vhand{AT5}{QJT8}{J84}{K32}}
{\vhand{K432}{4}{KT765}{J95}}
{NS}

\begin{table}[h!]
    \centering
    \begin{tabular}{cccc}
        \nvul{W} & \vul{N} (B) & \nvul{E} & \vul{S} (K) \\
        \pass & 1\clubs & 1\hearts & 1\nt \\
        \pass & \pass & \dbl & \rdbl \\
        2\diams & \dbl & 2\hearts & \pass \\
        2\spades & \pass$^*$ & \pass & \dbl \\
        all \pass & & & \\
    \end{tabular}
\end{table}

Pała na 1\nt jest co najmniej dziwna.\\
Został wezwany sędzia na namysł ($^*$), ale moja decyzja
była od niego niezależna. Bartek wiedział, że nie mam 4 pików,
a przeciwnicy mają fit i myślał nad wrzuceniem 2\nt. Wist \xclubs A, marka \xclubs 2,
Zmiana na 5\hearts, zabite asem. \xdiams do króla, wzięte asem.
Pod koniec Bartek wyszedł po podwójny renons w \clubs, co chyba nie miało znaczenia, 
ale było fajne i zmusiło rozgrywającego do nadbicia 
mojej \xspades 5 królem.\\
Finalnie -1, 2\nt okazało się być lepsze.

\hfill -K

[nie pamiętam co tam się działo w trakcie,
jak wyszedłeś w tego trefla to miałam wszystkie piki,
on jakoś wcześniej kiery przebijał, na co ty wziąłeś?]

\pagebreak
\section*{Rozdanie 13}
\handdiagramv{\vhand{KT5}{JT4}{K832}{862}}
{\vhand{AQ74}{A87}{A54}{KJ3}}
{\vhand{93}{Q932}{JT97}{QT7}}
{\vhand{J862}{K65}{Q6}{A954}}
{NSEW}

\begin{table}[h!]
    \centering
    \begin{tabular}{cccc}
        \vul{W} & \vul{N} (B) & \vul{E} & \vul{S} (K) \\
        -- & \pass & 1\clubs & \pass \\
        1\spades & \pass & \alrts{2\diams} & \pass \\
        \alrts{2\hearts} & \pass & 4\spades & all pass \\
    \end{tabular}
\end{table}

Wist \xclubs, -1.

[Ale czemu? To chodzi, nie
pamiętam co tam się stało xd]

\pagebreak
\section*{Rozdanie 14}
\handdiagramv{\vhand{532}{J2}{A94}{KJ863}}
{\vhand{T984}{AK53}{T3}{A72}}
{\vhand{AK6}{T9}{KQ7652}{94}}
{\vhand{QJ7}{Q8764}{J8}{QT5}}
{}

\begin{table}[h!]
    \centering
    \begin{tabular}{cccc}
        \nvul{W} & \nvul{N} (B) & \nvul{E} & \nvul{S} (K) \\
        -- & -- & 1\clubs & 1\diams \\
        1\hearts & 2\diams & 2\hearts & 3\diams \\
        3\hearts & all pass & & \\
    \end{tabular}
\end{table}

\xdiams A, \xdiams, \xspades A, dorzucone \xspades 5,
i właściwie widać, że jest duże, chyba że Bartek miałby
\xspades Q5, a rozgrywająca dołożyła \xspades 7
zamiast \xspades 2 lub 3, żeby mnie zmylić.
Uznałam odruchowo, że \xspades 5 jest mała, ściągnęłam króla,
i rozgrywająca oddała jeszcze tylko \xclubs K, -1
zamiast -2. A 4\diams było swoje (na trafieniu trefla), 
może mogłam wrzucać...?

\hfill -K

\pagebreak
\section*{Rozdanie 15}
\handdiagramv{\vhand{T984}{A82}{JT}{T642}}
{\vhand{Q63}{KJ7}{A74}{AJ87}}
{\vhand{K52}{Q653}{Q8532}{K}}
{\vhand{AJ7}{T94}{K96}{Q953}}
{NS}

\begin{table}[h!]
    \centering
    \begin{tabular}{cccc}
        \nvul{W} & \vul{N} (B) & \nvul{E} & \vul{S} (K) \\
        -- & -- & -- & 1\nt \\
        \pass & 3\nt \\ all pass & & \\
    \end{tabular}
\end{table}

Zdecydowałam się na wist kierowy, a nie karowy.
Wzięliśmy 4 kiery i \xclubs K. Po karowym 
wzięlibyśmy 3 kara, 2 kiery i (prawie na pewno) \xclubs K.

\hfill -K

\pagebreak
\section*{Rozdanie 16}
\handdiagramv{\vhand{JT5}{872}{9865}{A72}}
{\vhand{A4}{J95}{QJ7}{KJ965}}
{\vhand{KQ962}{AK64}{A3}{43}}
{\vhand{873}{QT3}{KT42}{QT8}}
{EW}

\begin{table}[h!]
    \centering
    \begin{tabular}{cccc}
        \vul{W} (K) & \nvul{N} & \vul{E} (B) & \nvul{S}\\
        \pass & \pass & \pass & 1\spades \\
        \pass & 2\spades & \pass & 4\spades \\
        all pass & & & \\
    \end{tabular}
\end{table}

Jeśli przeciwnicy mają w systemie drury, to wrzucenie
końcówki lekko świrowe. Tym razem udało mi się nie 
wypuścić wistem, chociaż pewnie logiczniej byłoby
wyciągnąć pika (ale niewiele tu brakuje żeby pik wypuszczał).
-1.

\hfill -K

\pagebreak
\section*{Rozdanie 17}
\handdiagramv{\vhand{T9}{AK64}{2}{AKQ864}}
{\vhand{Q762}{J3}{J976}{JT9}}
{\vhand{KJ54}{T987}{8543}{7}}
{\vhand{A83}{Q52}{AKQT}{532}}
{}

\begin{table}[h!]
    \centering
    \begin{tabular}{cccc}
        \nvul{W} (K) & \nvul{N} & \nvul{E} (B) & \nvul{S}\\
        -- & 1\clubs & \pass & 1\hearts \\
        \pass & 4\hearts & all pass & \\
    \end{tabular}
\end{table}

[sprawdź licytację bo nie pamiętam]\\
Wist \xdiams A, dorzucone 7? Uznałam to za Lavinthala na pika
i bojąc się, że Rozgrywający łatwo wyimpasuje moją damę kier
i zbierze dzielące się trefle, wyszłam \xspades A 
(do którego dostałam 2?) oszczędzając rozgrywającemu
konieczności zgadywania pika, swoje.

[co ty tam zrzucałeś...? xd]

\pagebreak
\section*{Rozdanie 18}
\handdiagramv{\vhand{AQ64}{54}{KQT65}{T3}}
{\vhand{T9}{KQJ}{A974}{KJ62}}
{\vhand{KJ752}{T9632}{J32}{}}
{\vhand{83}{A87}{8}{AQ98754}}
{NS}

\begin{table}[h!]
    \centering
    \begin{tabular}{cccc}
        \nvul{W} (K) & \vul{N} & \nvul{E} (B) & \vul{S}\\
        -- & -- & 1\clubs & 1\hearts (??) \\
        3\clubs & \pass & 3\diams & \pass \\
        3\nt & \pass & 5\clubs & all pass \\
    \end{tabular}
\end{table}

Powinnam dać 4\clubs zamiast 3\nt, ale na szczęście
partner uratował kontrakt bojąc się o piki. 5\clubs czapowe.
A przeciwnik chyba nie zauważył piątego pika i to mogło się dla nas
źle skończyć, bo po wejściu 1\spades nie wrzucalibyśmy 3\nt.

\hfill -K

\pagebreak
\section*{Rozdanie 19}
\handdiagramv{\vhand{J8532}{KJ}{KJ9}{J92}}
{\vhand{KQ6}{AQ653}{AQ42}{7}}
{\vhand{T74}{942}{853}{QT64}}
{\vhand{A9}{T87}{T76}{AK853}}
{EW}

\begin{table}[h!]
    \centering
    \begin{tabular}{cccc}
        \vul{W} (K) & \nvul{N} & \vul{E} (B) & \nvul{S}\\
        \pass & \pass & 1\hearts & \pass \\
        2\clubs & \pass & 4\clubs & \pass \\
        4\hearts & all pass & \\
    \end{tabular}
\end{table}

No otworzyć to trzeba było. Na 4\clubs też mogłam 4\diams,
niby last train ale jakoś głupio mi było bez cue \diams
i z \xclubs AK do krótkości. Raczej i tak kończymy w 5\hearts.
A szlemik bardzo mierny. Wist \xclubs 4, +2.

\hfill -K

\pagebreak
\section*{Rozdanie 20}
\handdiagramv{\vhand{T976}{9}{KJ8765}{T2}}
{\vhand{843}{QJ842}{43}{J85}}
{\vhand{QJ}{A7653}{AQ}{AK43}}
{\vhand{AK52}{KT}{T92}{Q976}}
{NSEW}

Rozdanie na Pluteckich, dociągnęli do 5\diams bez bilansu,
idzie. Ale 3\nt lepsze, na podziale kar, a jeszcze wist 
w pika wypuszcza nadróbkę.

\hfill -K

\pagebreak
\section*{Rozdanie 21}
\handdiagramv{\vhand{AJ4}{A85}{QJT7}{864}}
{\vhand{KQ8765}{KJ93}{}{T97}}
{\vhand{T93}{QT2}{K965}{KQJ}}
{\vhand{2}{764}{A8432}{A532}}
{NS}

\begin{table}[h!]
    \centering
    \begin{tabular}{cccc}
        \nvul{W} (K) & \vul{N} & \nvul{E} (B) & \vul{S}\\
        -- & \alrts{1\clubs} & 1\spades & \alrts{1\nt}\\
        \pass & 2\diams & 2\hearts & 3\diams \\
        all \pass & & & \\
    \end{tabular}
\end{table}

Myślałam nad pałą, mogło iść ale raczej widać, że ciężko.
Pluteccy pokazali 10-14 po obu stronach.\\
Wzięliśmy pika, przebitkę, dwa asy i kiera. Wist \xspades K.
Na asa trefl dostałam wysoką zrzutkę, żeby nie wychodzić w kiera,
wyszłam i tak, ale to chyba nic nie zmienia.\\

[czemu tu idzie 7?]

\pagebreak
\section*{Rozdanie 22}
\handdiagramv{\vhand{A5}{AJ87}{J842}{AT9}}
{\vhand{KQJ3}{92}{KT}{KQJ65}}
{\vhand{8}{KQT5}{AQ975}{743}}
{\vhand{T97642}{643}{63}{82}}
{EW}

\begin{table}[h!]
    \centering
    \begin{tabular}{cccc}
        \vul{W} (K) & \nvul{N} & \vul{E} (B) & \nvul{S}\\
        -- & -- & 1\clubs & 1\diams \\
        \pass & 1\hearts & 1\spades & 2\hearts \\
        3\spades &  \dbl & all \pass & \\
    \end{tabular}
\end{table}

Z 1\spades się zgadzam, nie mam chyba alternatywy
do tego 3\spades, na 2\spades od razu też nie mam.
Miałam idealną rękę na jednostronne multi, ale 
wejście popsuło plany. 3\spades jest -2,
więc not worth it. Wist \xhearts K.

\hfill -K

\pagebreak
\section*{Rozdanie 23}
\handdiagramv{\vhand{9}{KQT9}{Q65}{AQT96}}
{\vhand{QT62}{A865}{KJ84}{8}}
{\vhand{}{J2}{AT93}{KJ75432}}
{\vhand{AKJ87543}{743}{72}{}}
{NSEW}

\begin{table}[h!]
    \centering
    \begin{tabular}{cccc}
        \vul{W} (K) & \vul{N} & \vul{E} (B) & \vul{S}\\
        -- & -- & -- & 3\clubs \\
        4\spades & 5\clubs & 5\diams & \pass \\
        5\spades & all \pass & & \\
    \end{tabular}
\end{table}

Bartek zrespektował mój brak współpracy. Kara nie trafiłam,
-2. Wist \xhearts K. Co ciekawe, prawie cała sala decydowała się na
(co najmniej) 6\clubs, które nie idzie, ale było
zdarzało się =. A jedna para nawet odważyła się skontrować 5\spades (XD).
Na 6\spades z ręką \vul{S} powinno się dać pałę mając jedną lewę (czyli tak jak tu),
a \vul{N} mając też jedną powinien spasować (\xhearts KQ chyba
wystarczy).

\hfill -K

\pagebreak
\section*{Rozdanie 24}
\handdiagramv{\vhand{T943}{765}{J2}{A943}}
{\vhand{Q865}{A43}{T75}{K52}}
{\vhand{KJ2}{KJ9}{K3}{QJ876}}
{\vhand{A7}{QT82}{AQ9864}{T}}
{}

\begin{table}[h!]
    \centering
    \begin{tabular}{cccc}
        \nvul{W} (K) & \nvul{N} & \nvul{E} (B) & \nvul{S}\\
        1\diams & \pass & 1\spades & \pass \\
        2\diams & all \pass & & \\
    \end{tabular}
\end{table}

[S wchodził do licytacji 2\clubs? Tak mi się kojarzy]

[TODO dla mnie, przypomnę sobie]
Wist \xclubs A, \diams. Coś tu zrobiłam mocno nie tak,
nie pamiętam. Po tym wiście idzie +3 (?), wzięłam +2.
W ogóle tych kierów nie impasowałam...? A może on wyszedł w drugiej w kiera w ogóle,
bo to faktycznie postawiłoby mnie w sytuacji, że jak
przejdę karem 
nie trafię kiera to będę tylko +1.

\hfill -K

\pagebreak
\section*{Rozdanie 25}
\handdiagramv{\vhand{K752}{K3}{AQ5}{QJ93}}
{\vhand{93}{872}{T6}{AK7642}}
{\vhand{Q84}{QJT94}{J9}{T85}}
{\vhand{AJT6}{A65}{K87432}{}}
{EW}

\begin{table}[h!]
    \centering
    \begin{tabular}{cccc}
        \vul{W} & \nvul{N} (B) & \vul{E} & \nvul{S} (K) \\
        -- & 1\nt & \pass & 2\diams \\
        \pass & 2\hearts & all \pass & \\
    \end{tabular}
\end{table}

Wist \xclubs K, +2.

[Ty rozgrywałeś nwm co się działo]

\pagebreak
\section*{Rozdanie 26}
\handdiagramv{\vhand{JT53}{JT873}{5}{AT2}}
{\vhand{A74}{KQ2}{Q732}{J54}}
{\vhand{K98}{A65}{AJ86}{987}}
{\vhand{Q62}{94}{KT94}{KQ63}}
{NSEW}

\begin{table}[h!]
    \centering
    \begin{tabular}{cccc}
        \vul{W} & \vul{N} (B) & \vul{E} & \vul{S} (K)\\
        -- & -- & 1\clubs & \pass \\
        1\nt & all \pass & & \\
    \end{tabular}
\end{table}

Wist \xhearts J. O, burtówka. Jeszcze na naszą korzyść.\\
Wzięliśmy 6 lew, tyle się należało.

[Ale to chyba to co mogliśmy obłożyć bo nie wskoczyłam \xdiams A??
Jak to się toczyło???]

\hfill -K

\pagebreak
\section*{Rozdanie 27}
\handdiagramv{\vhand{KJ83}{AJ75}{QT4}{A2}}
{\vhand{952}{863}{J82}{Q764}}
{\vhand{QT64}{}{7653}{KJT98}}
{\vhand{A7}{KQT942}{AK9}{53}}
{}

\begin{table}[h!]
    \centering
    \begin{tabular}{cccc}
        \nvul{W} (K) & \nvul{N} & \nvul{E} (B) & \nvul{S}\\
        -- & -- & -- & \pass \\
        \pass & 1\hearts & 1\nt & \pass \\
        \alrts{2\clubs} & 2\hearts & 2\spades & all \pass \\
    \end{tabular}
\end{table}

Zapomniałam system :') 2\clubs powinno być \then\ \diams,
albo inwitem z krótkością,
na szczęście bez większych konsekwencji. Bartek dał 
bardzo fajne 2\spades = góra z 4 pikami na wypadek inwitu 
/ popraw na karo z karami. A ja się ucieszyłam i spasowałam,
chociaż właściwie dokład 4\spades mógł wcale nie być świrem?

\hfill -K

Wist \xspades 2.

[w rozgrywce nwm co się działo ale 10 idzie]

\pagebreak
\section*{Rozdanie 28}
\handdiagramv{\vhand{K432}{QJ42}{KT5}{Q3}}
{\vhand{Q97}{T7}{986}{KJ864}}
{\vhand{J5}{A85}{AJ7432}{AT}}
{\vhand{AT86}{K963}{Q}{9752}}
{NS}

\begin{table}[h!]
    \centering
    \begin{tabular}{cccc}
        \nvul{W} (K) & \vul{N} & \nvul{E} (B) & \vul{S}\\
        \pass & \pass & 1\nt & \pass \\
        2\clubs & \pass & \pass$^{\text{XD}}$ & 2\diams \\
        \pass & 3\diams & all \pass \\
    \end{tabular}
\end{table}

O dziwo mimo bilansu tylko 3 pary zagrały końcówkę,
w tym jedna 5\diams. 3\nt z ręki \vul{N} idzie.\\
Ania miała na 2\nt (na \vul{N}), Michał miał na
dokład. Ale i tak zagrali 3\diams+1 za 64\%. Wist \xclubs 7.

\hfill -K

\pagebreak
\section*{Rozdanie 29}
\handdiagramv{\vhand{JT73}{T832}{97}{A84}}
{\vhand{9}{97}{AK843}{KJT76}}
{\vhand{AKQ}{AJ64}{J652}{Q9}}
{\vhand{86542}{KQ5}{QT}{532}}
{NSEW}

\begin{table}[h!]
    \centering
    \begin{tabular}{cccc}
        \vul{W} (K) & \vul{N} & \vul{E} (B) & \vul{S}\\
        -- & \pass & 1\diams & \dbl \\
        \pass & 1\hearts & 2\clubs & \dbl \\
        2\diams & 2\hearts & 3\diams & 3\hearts \\
    \end{tabular}
\end{table}

Wydawało mi się, że zamiast 3\diams powinno się dać 3\clubs mimo wszystko.
Mogę mieć 3-3 w młodych z figurami w treflach i 3\clubs będzie lespsze,
(a na 2\diams musiałam poprawić). Natomiast z moją ręką
2\diams nie muszę mówić, nie ma opcji, że na 2\clubs\dbl\ spasują.
Więc ostatecznie po moim 2\diams na niewymuszonej zgadzam się z 3\diams.
Ale jak nie dam 2\diams to pewnie zagrają 2\hearts, nie 3. Do wistu \xdiams A
wyrzuciłam \xdiams T, wyrzucenie damy wypuszcza, bo nie ma komunikacji do przebitki.

\hfill -K

[teraz w kiera wyszedłeś?? czy pik]

Wzięliśmy 2 kara, przebitkę i 2 kiery.

\pagebreak
\section*{Rozdanie 30}
\handdiagramv{\vhand{73}{Q9}{KT2}{KQJ984}}
{\vhand{KJ86}{AJT8}{8653}{5}}
{\vhand{AQT952}{4}{AQ7}{A73}}
{\vhand{4}{K76532}{J94}{T62}}
{}

\begin{table}[h!]
    \centering
    \begin{tabular}{cccc}
        \nvul{W} (K) & \nvul{N} & \nvul{E} (B) & \nvul{S}\\
        -- & -- &\pass & 1\spades \\
        \pass & 2\clubs & \pass & 4\hearts \\
        \pass & 4\spades & all \pass & \\
    \end{tabular}
\end{table}

[2\clubs jest poprawne?]

Wist 5\clubs. gdzieś po drodze zrzuciłam 
\xhearts 6 zamiast \xclubs T (na kiera).
Jakoś nie chciałam tego trefla wyrzucić, kara spod wałka też nie,
lekko podświadomie uznałam, że środkowy kier jest, z grubsza, na kiera.

\hfill -K

[Ej nie pamiętam co tam się stało, na co myśmy wzięli 2 lewy
skoro nie wzięliśmy kierowej? Ty jeszcze miałeś pika po wzięciu na pika,
że moja zrzutka i twoje niewyjście w kiera puściło?]


\end{document}