\documentclass[12pt, a4paper]{article}
\usepackage{../../lib/bridgetex2}
\usepackage{polyglossia}
\usepackage{enumitem}
\usepackage[table]{xcolor}

\title{6\nt}
\author{Krystyna Gasińska}
\begin{document}
\maketitle

\fhandp{\hhand{A875}{AK62}{75}{K92}}{14}
        {}{}
        {\hhand{Q6}{873}{AKQJ4}{AQ4}}{18}
        {}{}
        {}

West leads 6\diams against 6\nt.
\begin{center}
    ***
\end{center}
We can make 12 tricks if hearts are 3-3 (36\%) or
if K\spades is with E (50\%), but we can only take
one of those chances. However, if hearts are 3-3 or if the defender 
holding K\spades has more hearts then the other, we 
can also win, and this time
we can try both. Play the small heart on the second trick
and then cash all your top tricks 
until you hold:

\fhandp{\hhand{--}{K2}{--}{2}}{}
        {xxx}{}
        {\hhand{Q}{8}{--}{A}}{}
        {xxx}{}
        {}

We hope that one of the defenders holds:

\chhand{K}{xx}{--}{--}

Now play the last club. Hopefully, one of the defenders
will have to discard his
K\spades or his heart guard. Of course, if anyone discards
a \hearts earlier, you don't have to hope for a squeeze.

\vspace{0.2cm}

What are the chances for making the contract now?
36\% that hearts are 3-3 and half of the remaining
64\% for the squeeze, so together we have 68\% --
significantly more then the initial 50\%.

\vspace{0.2cm}

The complete hand:

\fhandp{\hhand{A875}{AK62}{75}{K92}}{14}
        {\hhand{J943}{QT}{T832}{JT3}}{4}
        {\hhand{Q6}{873}{AKQJ4}{AQ4}}{18}
        {\hhand{KT2}{J954}{96}{8765}}{4}
        {}

\end{document}