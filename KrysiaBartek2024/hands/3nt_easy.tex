\documentclass[12pt, a4paper]{article}
\usepackage{../../lib/bridgetex2}
\usepackage{polyglossia}
\usepackage{enumitem}
\usepackage[table]{xcolor}

\title{3nt [easy]\\\vspace{0.3cm}\normalsize{from \textit{Planning the Play in Notrump}}}
\author{Krystyna Gasińska}
\begin{document}
\maketitle

\fhandp{\hhand{7}{KQ93}{Q982}{KJ64}}{11}
        {}{}
        {\hhand{AKT}{654}{K4}{AQT95}}{16}
        {}{}
        {}

West leads 4\spades against 3\nt, East puts J\spades.
\begin{center}
    ***
\end{center}
We take the lead with the king. 
If W has the A\hearts, we can easily take two \hearts
tricks, as defenders won't have time to get to the spades.
However, if ace is with E, playing hearts on the second trick
will doom the contract. Note, that W is now our `safe' opponent,
as we have double spade stopper against him.
Therefore, we should play the diamond from dummy before 
trying the hearts. If K\diams falls to W's ace, we
can simply take the next trick and play K\hearts.
Now A\hearts being with E is not a threat anymore.
There is still danger that E has JTxxx\diams together with A\hearts and W will
return a diamond after winning with A\diams, 
but the chances that this is not the case are much better
then simply hoping for A\hearts to be with W.

\vspace{0.2cm}

The complete hand:

\fhandp{\hhand{7}{KQ93}{Q982}{KJ64}}{11}
        {\hhand{J853}{AJ2}{J753}{82}}{7}
        {\hhand{AKT}{654}{K4}{AQT95}}{16}
        {\hhand{Q9642}{T87}{AT6}{73}}{6}
        {}


\end{document}