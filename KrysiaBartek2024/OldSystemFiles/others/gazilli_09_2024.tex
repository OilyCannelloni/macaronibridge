\documentclass[12pt, a4paper]{article}
\usepackage{import}

\usepackage{graphicx}
\usepackage{amssymb}
\usepackage{mathtools}  
\usepackage{amsthm}  
\usepackage{graphicx}
\usepackage{float}
\usepackage{caption}
\usepackage{enumerate}
\usepackage{hyperref}

\hypersetup{
    colorlinks=true,
    linkcolor=blue,
    urlcolor=cyan,
    }

\urlstyle{same}


\import{../../../lib/}{bridge.sty}
\setmainlanguage{english}

\title{\fontsize{30pt}{30pt}\selectfont Gazilli}
\author{Krysia Gasińska\\
        \small{na podstawie pomysłów Bartka Słupika}}
\begin{document}
\maketitle

\begin{formal}
Wykład: \href{https://www.youtube.com/@multi2diamonds}{nagranie}, 
\href{https://docs.google.com/presentation/d/1v0ixVhsnCNkZk1ckHag8bozFfjcQ0ZeFr_HqXATM9Xk/edit?usp=sharing}{slajdy}
\vspace{0.3cm}
\end{formal}

Poniżej wersja podstawowa. Niektóre odzywki zostały celowo pominięte. Dla lubiących ustalenia ich znaczenie
zostanie zaproponowane na końcu.

\subsubsection*{1\hearts\ -- ?}
\begin{itemize}
    \item 1\spades = 4+\spades, (3)4+ \hcp
    \item \alrts{1\ntx} = 8-11 bez fitu lub dowolne (3)4-7 (półforsujące)
\end{itemize}

\subsubsection*{1\spades\ -- ?}
\begin{itemize}
    \item \alrts{1\ntx} = 8-11 bez fitu lub dowolne (3)4-7 (półforsujące)
\end{itemize}

\subsubsection*{1\hearts\ -- 1\spades\ \\ ?}
\begin{itemize}
    \item \alrts{2\clubs} = 5\hearts4\clubs\ 11-15 \textbf{lub 16+} \fonce \imp
    \item 2\diams = 5\hearts4\diams\ 11-15
    \item 2\hearts = 6+\hearts 11-14
    \item 2\spades = 5\hearts4\spades\ 11-15
    \item 3\hearts = 6+\hearts\ \inv
    \item 3\spades = 5\hearts4\spades\ \inv
\end{itemize}

\subsubsection*{1\hearts\ -- 1\ntx\ \\ ?}
\begin{itemize}
    \item \pass = 12-14 (zwykle 5332)
    \item \alrts{2\clubs} = 5\hearts\clubs\ 11-15 \textbf{lub 16+} \fonce \imp
    \item 2\diams\ = 5\hearts4\diams\ 11-15
    \item 2\hearts\ = 6+\hearts 11-14
    \item 2\spades\ = rewers
    \item 3\clubs\ = 5\hearts5\clubs\ \gf
    \item 3\diams\ = 5\hearts5\diams\ \gf
    \item 3\hearts\ = 6+\hearts\ \inv
\end{itemize}

\subsubsection*{1\spades\ -- 1\ntx\ \\ ?}
\begin{itemize}
    \item \pass\ = 12-14 (zwykle 5332)
    \item \alrts{2\clubs} = 5\spades\clubs\ 11-15 \textbf{lub 16+} \fonce \imp
    \item 2\diams\ = 5\spades4\diams\ 11-15
    \item 2\hearts\ = 5\spades4\hearts\ 11-15
    \item 2\spades\ = 6+\spades 11-14
    \item 3\clubs\ = 5\spades5\clubs\ \gf
    \item 3\diams\ = 5\spades5\diams\ \gf
    \item 3\hearts\ = 5\spades5\hearts\ \gf
    \item 3\spades\ = 6+\spades\ \inv
\end{itemize}

\subsubsection*{1\hearts\ -- 1\spades\ \\ 2\clubs\ -- ?}
\begin{itemize}
    \item \alrts{2\diams} = dowolne 8+ \imp
    \item 2\hearts\ = 2-3\hearts\ 4-7
    \item 2\spades\ = dobre 5\spades, krótkość \hearts 4-7
    \item \alrts{2\ntx} = 1-\hearts\ 4-7
    \item 3\clubs\ = 6+\clubs\ 4-7
    \item 3\diams\ = 6+\diams\ 4-7
\end{itemize}

\subsubsection*{1\hearts\ -- 1\ntx\ \\ 2\clubs\ -- ?}
\begin{itemize}
    \item \alrts{2\diams} = dowolne 8+ \imp
    \item 2\hearts\ = 2-3\hearts\ 4-7
    \item \alrts{2\nt} = 1-\hearts\ 4-7
    \item 3\clubs\ = 6+\clubs\ 4-7
    \item 3\diams\ = 6+\diams\ 4-7
\end{itemize}

\subsubsection*{1\spades\ -- 1\ntx\ \\ 2\clubs\ -- ?}
\begin{itemize}
    \item \alrts{2\diams} = dowolne 8+ \imp
    \item 2\hearts\ = 5\hearts\ 4-7
    \item 2\spades\ = 2-3\spades\ 4-7
    \item \alrts{2\ntx} = 1-\spades\ 4-7
    \item 3\clubs\ = 6+\clubs\ 4-7
    \item 3\diams\ = 6+\diams\ 4-7
\end{itemize}

\subsubsection*{1\hearts\ -- 1\spades\ \\ 2\clubs\ -- 2\diams \\ ?}
\begin{itemize}
    \item 2\hearts\ = 5\hearts4\clubs\ 11-15
    \item 2\spades\ = 5\hearts =3\spades\ 16+
    \item 2\ntx\ = 18-20 \bal, w tym 6322 (3\clubs = ask)
    \item 3\clubs\ = 5\hearts4\clubs\ 16+
    \item 3\diams\ = 5\hearts4\diams\ 16+
    \item 3\hearts\ = 6\hearts\ \unbal, 16+
    \item 3\spades\ = 5\hearts4\spades\ \gf (ustala piki)
\end{itemize}

\subsubsection*{1\hearts\ -- 1\ntx\ \\ 2\clubs\ -- 2\diams \\ ?}
\begin{itemize}
    \item 2\hearts\ = 5\hearts4\clubs\ 11-15
    \item 2\spades\ = 5\hearts4\spades\ 16+
    \item 2\ntx\ = 18-20 \bal, w tym 6322 (3\clubs = ask)
    \item 3\clubs\ = 5\hearts4\clubs\ 16+
    \item 3\diams\ = 5\hearts4\diams\ 16+
    \item 3\hearts\ = 6\hearts\ \unbal, 16+
\end{itemize}

\subsubsection*{1\spades\ -- 1\ntx\ \\ 2\clubs\ -- 2\diams \\ ?}
\begin{itemize}
    \item 2\hearts\ = 5\spades4\hearts\ 16+
    \item 2\spades\ = 5\spades4\clubs\ 11-15
    \item 2\ntx\ = 18-20 \bal, w tym 6322 (3\clubs = ask)
    \item 3\clubs\ = 5\spades4\clubs\ 16+
    \item 3\diams\ = 5\spades4\diams\ 16+
    \item 3\spades\ = 6\spades\ \unbal, 16+
\end{itemize}

\subsubsection*{1\major\ -- 1\ntx\ \\ 
                2\clubs\ -- 2\major \\ ?}
\begin{itemize}
    \item 2\nt = silny inwit, \nf
    \item 3\major = silny inwit, 6+
    \item nowy kolor forsuje do 3\major
\end{itemize}

\subsubsection*{1\major\ -- 1\ntx\ \\ 
                2\clubs\ -- 2\major \\ 
                2\nt -- ?}
\begin{itemize}
    \item 3\major = \nf
\end{itemize}

\newpage
Dodatkowe ustalenia dla chętnych:

\subsubsection*{1\hearts\ -- 1\spades\ \\ ?}
\begin{itemize}
    \item 4\minor = słabszy splinter, około 16-18
\end{itemize}

\subsubsection*{1\hearts\ -- 1\spades\ \\ 
                2\clubs\ -- 2\diams\ \\ ?}
\begin{itemize}
    \item 4\minor = silny splinter, 18-19+
\end{itemize}

\subsubsection*{1\hearts\ -- 1\nt\ \\ 2\clubs -- ?}
\begin{itemize}
    \item 2\spades = 8-11 z długim młodym (2\nt = ask)
\end{itemize}

\subsubsection*{1\hearts\ -- 1\spades/1\ntx\ \\ ?}
\begin{itemize}
    \item 2\nt = 6\hearts 4\minor \gf
\end{itemize}

\subsubsection*{1\spades\ -- 1\ntx\ \\ ?}
\begin{itemize}
    \item 2\nt = 6\spades 4\minor \gf
\end{itemize}

\subsubsection*{1\major -- 1\spades/1\nt\\
                2\nt -- ?}
\begin{itemize}
    \item 3\clubs = \pass/correct
    \item 3\diams = ask \gf
    \item 3\major = uzgadnia \major
    \item 3\twosuit{\spades}{\hearts} = \nat
\end{itemize}

\subsubsection*{1\major -- 1\spades/1\nt\\
                2\nt -- 3\diams\\
                ?}
\begin{itemize}
    \item 3\hearts = \clubs (3\spades = ask o siłę \then 3\nt = słabsze)
    \item 3\spades = \diams dobra ręka
    \item 3\nt = \diams słabsza ręka
\end{itemize}

\end{document}
