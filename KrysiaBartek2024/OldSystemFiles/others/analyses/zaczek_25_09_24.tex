\documentclass[12pt, a4paper]{article}
\usepackage{import}

\import{../../../../lib/}{bridge.sty}

\title{Żaczek 25.09.24}
\author{Krysia \& Bartek}
\begin{document}
\maketitle

\vspace{-0.3cm}
\section*{Rozdanie 1}
\handdiagramv
        {\vhand{876}{A2}{A752}{J742}}
        {\vhand{AQ43}{KQJT4}{Q}{K83}}
        {\vhand{KJ5}{987653}{3}{Q65}}
        {\vhand{T92}{--}{KJT9864}{AT9}}
        {}

\vspace{-0.3cm}

\begin{table}[h!]
    \centering
    \begin{tabular}{cccc}
        \nvul{W} (K) & \nvul{N} & \nvul {E} (B) & \nvul{S} \\
        -- & \pass & 1\hearts & \pass \\
        \alrts{1\nt} & \pass & \alrts{2\clubs} & \pass \\
        \alrts{2\diams} & \pass & 2\spades & \pass \\
        \alrts{2\ntx} & \pass & \alrts{3\ntx} & all \pass \\
    \end{tabular}
\end{table}

Zamiast odzywki 2\nt lepsze byłoby 3\diams. 2\nt było pytaniem o skład,
3\nt pokazało 54 i krótkość karo. Wist 7\spades do damy i króla,
walet pik zabity asem. Wyszłam w karo obawiając się jedynie odwrotu w
trefla, który zabrałby mi dojście do lew kierowych i na koniec oddałabym trefla (=).
Odwrót nastąpił w pika i reszta lew była moja (+2), a obrońcy nie wzięli nawet na asa kier.

\pagebreak
\section*{Rozdanie 2}
\handdiagramv
        {\vhand{632}{A765}{Q9}{AK32}}
        {\vhand{KQT8}{J83}{754}{864}}
        {\vhand{J94}{QT42}{K86}{QT5}}
        {\vhand{A75}{K9}{AJT32}{J97}}
        {NS}

Po oczywistej licytacji znalazłam się (\nvul{W}) w kontrakcie 1\nt.
Obrońcy ściągnęli 4 trefle i wyszli w kiera, ze stołu małe, od S dama, z ręki król.
S na czwartego trefla wyrzucił pika, co jeszcze mocniej zasugerowało waleta po stronie N,
zatem zaimpasowałam go. Odwrót S w karo musiałam przejąć asem i pozostało mi jedynie zebrać pika (-2).

\pagebreak
\section*{Rozdanie 3}
\handdiagramv
        {\vhand{Q543}{AK5}{Q9}{KJ73}}
        {\vhand{AT72}{--}{T8542}{Q852}}
        {\vhand{J96}{JT97432}{J}{A4}}
        {\vhand{K8}{Q86}{AK763}{T96}}
        {WE}

\begin{table}[h!]
    \centering
    \begin{tabular}{cccc}
        \vul{W} (K) & \nvul{N} & \vul {E} (B) & \nvul{S} \\
        -- & -- & -- & \alrts{2\diams} \\
        \pass & \alrts{3\diams} & \pass & 4\hearts \\
        all \pass & & & \\
    \end{tabular}
\end{table}

Wist A\diams, K\diams przebity, od partnera 4\diams (marka) i T\diams (Lavinthal).
Rozgrywający zebrał atuty i z nieznanego powodu wyszedł ze stołu
Q\spades. Bartek wskoczył asem (inaczej wypuszczamy) i również
wyszedł w pika. \nvul{S} nie zgadł i podłożył waleta. 
Pozostało mu jedynie zaimpasować trefla,
a jako że impas nie stał, skończył bez 2.

\pagebreak
\section*{Rozdanie 4}
\handdiagramv{\vhand{K7}{QJ75}{75}{KJ542}}
{\vhand{T93}{T2}{AKJ42}{AQ9}}
{\vhand{AQ654}{AK963}{Q6}{T}}
{\vhand{J82}{84}{T983}{8763}}
{NSEW}

\begin{table}[h!]
    \centering
    \begin{tabular}{cccc}
        \vul{W} (K) & \vul{N} & \vul{E} (B) & \vul{S}\\
        \pass & \pass & 1\nt & 2\spades \\
        all \pass & & & \\
    \end{tabular}
\end{table}

Otwarcie 1\nt zablokowało przeciwników, którzy bez ustaleń nie znaleźli
końcówki kierowej. Zawistowałam w pika, rozgrywający ściągnął atuty i zaimpasował trefla,
Bartek ściągnął A\diams, do którego Krysia zrzucła \xdiams 3. Bartek (debil) nie ściągnął \xdiams K bo uznał trójkę za markę.
Nie ściągnęliśmy przez to K\diams, +3.

\pagebreak
\section*{Rozdanie 5}
\handdiagramv{\vhand{AK9832}{KQ96}{5}{JT}}
{\vhand{}{AT4}{AQ832}{A8543}}
{\vhand{JT64}{J82}{T64}{K72}}
{\vhand{Q75}{753}{KJ97}{Q96}}
{NS}

\begin{table}[h!]
    \centering
    \begin{tabular}{cccc}
        \nvul{W} & \vul{N} & \nvul{E} & \vul{S}\\
        -- & 1\spades & \alrts{2\nt} & 3\spades \\
        4\diams & 4\spades & 5\diams & all \pass \\
    \end{tabular}
\end{table}

Wist K\spades.
Coś głupiego tu się stało ale nie pamiętam.



\pagebreak
\section*{Rozdanie 6}
\handdiagramv{\vhand{KQT}{Q872}{KT}{A985}}
{\vhand{J8}{AK654}{97643}{T}}
{\vhand{9753}{JT93}{Q2}{KQ6}}
{\vhand{A642}{}{AJ85}{J7432}}
{EW}

\begin{table}[h!]
    \centering
    \begin{tabular}{cccc}
        \vul{W} & \nvul{N} & \vul{E} & \nvul{S}\\
         & & \pass & \pass \\
        1\diams & \dbl & 1\hearts & 1\spades \\
        \pass & \pass & 2\spades & \pass \\
        3\clubs & \pass & 3\diams \\  
    \end{tabular}
\end{table}

Uznałem, że ręka E jest za słaba na dorzucenie końcówki - możemy łatwo oddawać 3 topy.
Wist \xspades K przepuszczony, następnie \xspades T (??). Krysia wyrzuciła trefla a następnie wzięła 10 lew.
Należało jednak zauważyć, że kiery są prawie stuprocentowo 4-4, przebić 2 z nich a następnie zagrać na kara 2-2 biorąc 11.


\pagebreak
\section*{Rozdanie 7}
\handdiagramv{\vhand{KT9652}{T5}{986}{AK}}
{\vhand{QJ74}{}{KJT5}{QJ754}}
{\vhand{A3}{A8432}{AQ4}{962}}
{\vhand{8}{KQJ976}{732}{T83}}
{NSEW}

\begin{table}[h!]
    \centering
    \begin{tabular}{cccc}
        \vul{W} & \vul{N} & \vul{E} & \vul{S}\\
                                &&& 1\hearts \\
        \pass & 1\spades & 2\clubs & \pass \\
        \pass & 2\spades
    \end{tabular}
\end{table}

Wznowiłem "tylko" 2\spades oczekując maksymalnie dubla w stole oraz niedzielących się kolorów (wejście z QJ - pewnie ma układ)
W rozgrywce chyba nic się nie dzieje.




\pagebreak
\section*{Rozdanie 8}
\handdiagramv{\vhand{A82}{AKJT5}{K83}{K9}}
{\vhand{53}{82}{A75}{AT8643}}
{\vhand{KQJT764}{76}{QJ}{J2}}
{\vhand{9}{Q943}{T9642}{Q75}}
{}

\begin{table}[h!]
    \centering
    \begin{tabular}{cccc}
        \nvul{W} & \nvul{N} & \nvul{E} & \nvul{S}\\
        \pass & 1\hearts & \pass & 1\spades \\
        \pass & 2\clubs & \dbl & 2\diams \\
        \pass & 2\nt & \pass & 3\spades \\
        \pass & 4\clubs & \dbl & 4\spades
    \end{tabular}
\end{table}

Ja z ręką S dałbym 4\spades zamiast 2\diams ale to chyba kwestia stylu.
Pojawił się temat kontr wistowych na cue-bidy: przyjęliśmy ustalenie:
\begin{itemize}
    \item \rdbl = as lub renons
    \item \pass = brak kontroli (\rdbl partnera = as)
    \item inny cue = dobrze położony król lub singiel
\end{itemize}
Czy to dobre nie wiem ale chyba tak się gra



\pagebreak
\section*{Rozdanie 9}
\handdiagramv{\vhand{AJ}{KQJ872}{J4}{JT3}}
{\vhand{9}{T943}{QT73}{A962}}
{\vhand{K7654}{6}{AK62}{K75}}
{\vhand{QT832}{A5}{985}{Q84}}
{EW}

\begin{table}[h!]
    \centering
    \begin{tabular}{cccc}
        \vul{W} & \nvul{N} & \vul{E} & \nvul{S}\\
        &         1\hearts & \pass & 1\spades \\
        \pass & 2\hearts & \pass & 3\nt \\
    \end{tabular}
\end{table}

Zdecydowałem się na pokazanie 14+ mimo tych waletów, zostałem może i słusznie opierdolony.
Jednak jestem na pierwszej w korzystnych i otwarcie 2\hearts może być lekko
naciągane, aby kryć bardzo szeroki przedział multi. Na + jest jednak kolor KQJ, dojście oraz wygląda na to że przeciwnik może łatwo
dać lewę na wiście.


\pagebreak
\section*{Rozdanie 10}
\handdiagramv{\vhand{652}{A843}{K975}{A5}}
{\vhand{AQ8}{T652}{A863}{82}}
{\vhand{JT9}{J7}{QT42}{KT64}}
{\vhand{K743}{KQ9}{J}{QJ973}}
{NSEW}

\begin{table}[h!]
    \centering
    \begin{tabular}{cccc}
        \vul{W} & \vul{N} & \vul{E} & \vul{S}\\
        &       &           \pass & \pass \\
        2\clubs & \dbl & \pass & 2\diams \\
    \end{tabular}
\end{table}

Graliśmy sobie na pierwszym stole i zadziałał u mnie instynkt unikania gry w obronie Zapolowałem lekko świrową kontrą.

Krysia musiała wynieść śmieci. Wist \xclubs Q zabity asem (1). \xdiams 5 do T i J. W ściągnął 3 piki i zagrał \xhearts K do A (2).
Ze stołu blotka karo do \xdiams A i kontynuacja kiera do K i kiera przebitego w ręce (3). 
Ściągnięty \xclubs K (4). Teraz trefl przebity bez sensu siódemką, co oddawało lewę.

Czy kolor karowy gra się do 10? Krysia policz bo ja nie umiem ale chyba rzeczywiście tak.

\pagebreak
\section*{Rozdanie 11}
\handdiagramv{\vhand{Q643}{T}{A5432}{865}}
{\vhand{AJT85}{AK93}{97}{J4}}
{\vhand{9}{J8742}{JT86}{AQ9}}
{\vhand{K72}{Q65}{KQ}{KT732}}
{}

\begin{table}[h!]
    \centering
    \begin{tabular}{cccc}
        \nvul{W} & \nvul{N} & \nvul{E} & \nvul{S}\\

    \end{tabular}
\end{table}

\pagebreak
\section*{Rozdanie 12}
\handdiagramv{\vhand{Q643}{T}{A5432}{865}}
{\vhand{T7}{K83}{Q932}{9762}}
{\vhand{KJ964}{A4}{654}{A84}}
{\vhand{Q53}{JT965}{87}{QT3}}
{NS}

\begin{table}[h!]
    \centering
    \begin{tabular}{cccc}
        \nvul{W} & \vul{N} & \nvul{E} & \vul{S}\\

    \end{tabular}
\end{table}

\pagebreak
\section*{Rozdanie 13}
\handdiagramv{\vhand{T7654}{653}{J73}{82}}
{\vhand{KQ}{T742}{KT8}{AJT4}}
{\vhand{82}{AQ9}{A965}{K763}}
{\vhand{AJ93}{KJ8}{Q42}{Q95}}
{NSEW}

\begin{table}[h!]
    \centering
    \begin{tabular}{cccc}
        \vul{W} & \vul{N} & \vul{E} & \vul{S}\\

    \end{tabular}
\end{table}

\pagebreak
\section*{Rozdanie 14}
\handdiagramv{\vhand{52}{AQ}{T9643}{JT84}}
{\vhand{AT43}{J87643}{Q}{32}}
{\vhand{QJ87}{T92}{A82}{Q65}}
{\vhand{K96}{K5}{KJ75}{AK97}}
{}

\begin{table}[h!]
    \centering
    \begin{tabular}{cccc}
        \nvul{W} & \nvul{N} & \nvul{E} & \nvul{S}\\

    \end{tabular}
\end{table}

\pagebreak
\section*{Rozdanie 15}
\handdiagramv{\vhand{AJ96542}{K5}{2}{A87}}
{\vhand{7}{AQ9642}{QJ65}{65}}
{\vhand{QT8}{J8}{K94}{KQJ92}}
{\vhand{K3}{T73}{AT873}{T43}}
{NS}

\begin{table}[h!]
    \centering
    \begin{tabular}{cccc}
        \nvul{W} & \vul{N} & \nvul{E} & \vul{S}\\

    \end{tabular}
\end{table}

\pagebreak
\section*{Rozdanie 16}
\handdiagramv{\vhand{AJ32}{T7}{QT8}{8754}}
{\vhand{6}{K854}{AJ53}{AJT2}}
{\vhand{K95}{A96}{K97642}{3}}
{\vhand{QT874}{QJ32}{}{KQ96}}
{EW}

\begin{table}[h!]
    \centering
    \begin{tabular}{cccc}
        \vul{W} & \nvul{N} & \vul{E} & \nvul{S}\\

    \end{tabular}
\end{table}

\pagebreak
\section*{Rozdanie 17}
\handdiagramv{\vhand{A876}{3}{KQ65}{K853}}
{\vhand{43}{K8642}{T92}{JT7}}
{\vhand{KT95}{QT95}{AJ7}{64}}
{\vhand{QJ2}{AJ7}{843}{AQ92}}
{}

\begin{table}[h!]
    \centering
    \begin{tabular}{cccc}
        \nvul{W} & \nvul{N} & \nvul{E} & \nvul{S}\\

    \end{tabular}
\end{table}

\pagebreak
\section*{Rozdanie 18}
\handdiagramv{\vhand{A876}{3}{KQ65}{K853}}
{\vhand{43}{K8642}{T92}{JT7}}
{\vhand{KQT8}{J96}{Q964}{J7}}
{\vhand{A4}{KQ52}{JT}{T6432}}
{NS}

\begin{table}[h!]
    \centering
    \begin{tabular}{cccc}
        \nvul{W} & \vul{N} & \nvul{E} & \vul{S}\\

    \end{tabular}
\end{table}

\pagebreak
\section*{Rozdanie 19}
\handdiagramv{\vhand{96532}{A874}{A5}{95}}
{\vhand{J7}{T3}{K8732}{AKQ8}}
{\vhand{KQT8}{J96}{Q964}{J7}}
{\vhand{A4}{KQ52}{JT}{T6432}}
{EW}

\begin{table}[h!]
    \centering
    \begin{tabular}{cccc}
        \vul{W} & \nvul{N} & \vul{E} & \nvul{S}\\

    \end{tabular}
\end{table}

\pagebreak
\section*{Rozdanie 20}
\handdiagramv{\vhand{96532}{A874}{A5}{95}}
{\vhand{J7}{T3}{K8732}{AKQ8}}
{\vhand{KQT8}{J96}{Q964}{J7}}
{\vhand{A4}{KQ52}{JT}{T6432}}
{NSEW}

\begin{table}[h!]
    \centering
    \begin{tabular}{cccc}
        \vul{W} & \vul{N} & \vul{E} & \vul{S}\\

    \end{tabular}
\end{table}

\pagebreak
\section*{Rozdanie 21}
\handdiagramv{\vhand{96532}{A874}{A5}{95}}
{\vhand{J7}{T3}{K8732}{AKQ8}}
{\vhand{KQT8}{J96}{Q964}{J7}}
{\vhand{A4}{KQ52}{JT}{T6432}}
{NS}

\begin{table}[h!]
    \centering
    \begin{tabular}{cccc}
        \nvul{W} & \vul{N} & \nvul{E} & \vul{S}\\

    \end{tabular}
\end{table}

\pagebreak
\section*{Rozdanie 22}
\handdiagramv{\vhand{873}{953}{A543}{AK7}}
{\vhand{9}{K762}{KT862}{JT4}}
{\vhand{KJT64}{AQ8}{Q97}{92}}
{\vhand{AQ52}{JT4}{J}{Q8653}}
{EW}

\begin{table}[h!]
    \centering
    \begin{tabular}{cccc}
        \vul{W} & \nvul{N} & \vul{E} & \nvul{S}\\

    \end{tabular}
\end{table}

\pagebreak
\section*{Rozdanie 23}
\handdiagramv{\vhand{T83}{KJ762}{854}{97}}
{\vhand{AK652}{T5}{T62}{A86}}
{\vhand{QJ974}{93}{AKQ93}{Q}}
{\vhand{}{AQ84}{J7}{KJT5432}}
{NSEW}

\begin{table}[h!]
    \centering
    \begin{tabular}{cccc}
        \vul{W} & \vul{N} & \vul{E} & \vul{S}\\

    \end{tabular}
\end{table}

\pagebreak
\section*{Rozdanie 24}
\handdiagramv{\vhand{QJ}{J3}{AQ432}{JT98}}
{\vhand{KT96}{AK75}{KJ8}{K2}}
{\vhand{83}{Q8642}{6}{Q7654}}
{\vhand{A7542}{T9}{T975}{A3}}
{}

\begin{table}[h!]
    \centering
    \begin{tabular}{cccc}
        \nvul{W} & \nvul{N} & \nvul{E} & \nvul{S}\\

    \end{tabular}
\end{table}

\pagebreak
\section*{Rozdanie 25}
\handdiagramv{\vhand{AQ852}{A7}{T52}{754}}
{\vhand{T9}{KT643}{K43}{K92}}
{\vhand{J4}{95}{AQJ986}{AJT}}
{\vhand{K763}{QJ82}{7}{Q863}}
{EW}

\begin{table}[h!]
    \centering
    \begin{tabular}{cccc}
        \vul{W} & \nvul{N} & \vul{E} & \nvul{S}\\

    \end{tabular}
\end{table}

\pagebreak
\section*{Rozdanie 26}
\handdiagramv{\vhand{K6}{QT654}{A95}{QJ6}}
{\vhand{AJ73}{J9872}{2}{T94}}
{\vhand{QT852}{A3}{KJ843}{A}}
{\vhand{94}{K}{QT76}{K87532}}
{NSEW}

\begin{table}[h!]
    \centering
    \begin{tabular}{cccc}
        \vul{W} & \vul{N} & \vul{E} & \vul{S}\\

    \end{tabular}
\end{table}

\pagebreak
\section*{Rozdanie 27}
\handdiagramv{\vhand{Q743}{2}{AQ9532}{AT}}
{\vhand{62}{873}{J64}{98653}}
{\vhand{AT5}{KJ954}{T}{J742}}
{\vhand{KJ98}{AQT6}{K87}{KQ}}
{}

\begin{table}[h!]
    \centering
    \begin{tabular}{cccc}
        \nvul{W} & \nvul{N} & \nvul{E} & \nvul{S}\\

    \end{tabular}
\end{table}

\end{document}