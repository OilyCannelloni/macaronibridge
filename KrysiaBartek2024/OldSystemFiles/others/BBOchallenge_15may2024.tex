\documentclass[12pt, a4paper]{article}
\usepackage{import}

\import{../../lib/}{bridge.sty}

\title{Czalendż z dnia 14.05.2024}
\author{Krysia (vs Kacper)}
\begin{document}
\maketitle

1. 

\handdiagramv{\vhand{J4}{763}{AQ862}{643}}
        {\vhand{75}{AT54}{J95}{KT72}}
        {\vhand{AKQ9}{K98}{43}{AQ85}}
        {\vhand{T8632}{QJ2}{KT7}{J9}}
        {}

\begin{table}[h!]
    \centering
    \begin{tabular}{cccc}
        \nvul{W} & \nvul{N} & \nvul {E} & \nvul{S} \\
        - & \pass & \pass & 1\clubs \\
        \pass & 1\diams & \pass & 2\nt \\
        \pass & 3\nt & all \pass & \\
    \end{tabular}
\end{table}

Wist 3\spades. As kier stać musi. Przydałby się też impas karo. Wygrywamy przy podziale kar 3-3
lub impasie trefl. Podział trefli 3-3, jeśli impas trefl nie stoi i kara się nie dzielą, zwykle nie
wystarczy. Możemy zagrać w drugiej lewie impas karo i trefla do małego, lub małe karo z obu rąk,
przeciwnicy ściągną kiery, zaimpasujemy karo, sprawdzimy kara 3-3 i impas trefl.

\vspace{0.2cm}

\newpage
2.

\handdiagramv{\vhand{3}{A32}{T43}{AT9854}}
        {\vhand{QJ982}{KT654}{8}{KQ}}
        {\vhand{AKT75}{Q8}{QJ76}{32}}
        {\vhand{64}{J97}{AK952}{J76}}
        {NS}

\begin{table}[h!]
    \centering
    \begin{tabular}{cccc}
        \nvul{W} & \vul{N} & \nvul {E} & \vul{S} \\
        - & - & 1\spades & \pass \\
        1\nt & \pass & 2\hearts  & \pass \\
        2\spades & all \pass & & \\
    \end{tabular}
\end{table}

Po takiej licytacji wistujemy w atu, żeby rozgrywający nie mógł przebić
kierów w stole (np jeśli dziadek ma 2/2 w starych). Wistujemy A\spades.
W pierwszej lewie znamy rozkład pików, a rozgrywający nie przebije kierów, więc nie ma sensu
kontynuować wistu w atu.

Uwaga, \dbl na \vul{S}\ byłaby \textit{karna}, bo wywoławcze na młodych jest 2\nt.

\newpage
5.

\handdiagramv{\vhand{853}{J72}{AQ98}{764}}
        {\vhand{KJ96}{KQ8}{KJ3}{985}}
        {\vhand{T}{A63}{T62}{AKQJT3}}
        {\vhand{AQ742}{T954}{754}{2}}
        {NS}

\begin{table}[h!]
    \centering
    \begin{tabular}{cccc}
        \nvul{W} & \vul{N} & \nvul {E} & \vul{S} \\
        - & \pass & 1\clubs & \pass \\
        1\spades & \pass & 2\spades  & 3\clubs \\
        all \pass & & & \\
    \end{tabular}
\end{table}

Wist 2\clubs. Ściągamy atuty, gramy T\diams w koło.
\nvul {E}\ wychodzi 6\spades do damy partnera,
który gra T\hearts. Dokładamy blotkę z obu rąk i dopiero
kontynuację kierową bijemy asem. Gramy trefle wyrzucając
ze stołu piki.

Mamy następujący układ:

\handdiagramh{\hhand{-}{J}{AQ9}{-}}{}
        {\hhand{9}{K}{K3}{-}}{}
        {\hhand{-}{6}{62}{3}}{}
        {\hhand{A}{4}{75}{-}}{}
        {NS}

Gramy ostatniego trefla (wyrzucamy 9\diams).
E trzyma kiery i karo, musi więc pozbyć się pika (bezpiecznego wyjścia):
\textbf{{\color{red}p}{\color{orange}r}{\color{LimeGreen}z}{\color{cyan}y}{\color{blue}m}{\color{purple}u}{\color{red}s}}
\textbf{{\color{red}w}{\color{orange}p}{\color{LimeGreen}u}{\color{cyan}s}{\color{blue}t}{\color{purple}k}{\color{red}o}{\color{orange}w}{\color{LimeGreen}y}}.
Gramy kiera, wpuszczając \nvul{W}.

\newpage
8.

\handdiagramv{\vhand{K6}{QT854}{T5}{T962}}
        {\vhand{J}{962}{K8762}{Q873}}
        {\vhand{AT87542}{K3}{AQ}{A4}}
        {\vhand{Q93}{AJ7}{J943}{KJ5}}
        {}

\begin{table}[h!]
    \centering
    \begin{tabular}{cccc}
        \nvul{W} & \nvul{N} & \nvul {E} & \nvul{S} \\
        1\diams & \pass & 3\diams & 4\spades \\
        all \pass & & & \\
    \end{tabular}
\end{table}

Wist: 5\clubs, \nvul {E}\ zabija 9\clubs damą.
Będziemy potencjalnie potrzebować przejścia do stołu.
Ściągnięcie 2 górnych pików nas go pozbawi 
(obrońcy przepuszczą pierwszego kiera). 
W drugiej lewie wychodzimy K\hearts, \nvul{W}\
zabija asem i gra 2 górne trefle, wyrabiając nam
T\clubs. Ostatnie wyjście w trefla przebijamy, gramy 2 górne
piki kończąc w stole i wyrzucamy Q\diams na trefla.

Jeśli \nvul{W}\ zamiast wyrabiać nam trefle
wyjdzie 7\hearts, zabijemy damą,
następnie ściągmy 2 górne piki i zaimpasujemy karo.

Jeśli \nvul{W}\ nie weźmie K\hearts, wychodzimy 
ponownie w kiera. Później możemy wyrzucić
Q\diams na Q\hearts. Jeśli \nvul {E}\ ma AJxx\hearts
i tak nic nie dało się zrobić (ale prawdopodobnie
zalicytowałby wtedy 1\hearts zamiast 3\diams).

\newpage
10.

\handdiagramv{\vhand{A86}{952}{T9}{A9853}}
        {\vhand{T52}{KT76}{Q3}{QJT2}}
        {\vhand{J9}{AJ83}{AK75}{K74}}
        {\vhand{KQ743}{Q4}{J8642}{6}}
        {NSWE}

\begin{table}[h!]
    \centering
    \begin{tabular}{cccc}
        \vul{W} & \vul{N} & \vul {E} & \vul{S} \\
        - & - & \pass & 1\nt \\
        \pass & 2\clubs & \pass & 2\hearts \\
        \pass & 2\nt & \pass & 3\nt \\
        all \pass & & & \\
    \end{tabular}
\end{table}

Wist: 6\clubs. Na 6 kładziemy 9, aby Brydżowi Bogowie
pomogli nam ugrać kontrakt. 
Nie sprzyja nam jednak ich łaska, bo trefle 
postanawiają się nie dzielić. W drugiej lewie 
\vul {E}\ bierze trefla i wychodzi w pika do waleta, damy i asa.
To pozwoli nam później wyrobić 8\spades. Gramy kiera do waleta
licząc na... KQT\hearts u \vul {E}?. Na szczęście 
\vul{W}\ po wzięciu na Q\hearts wyciąga jedyną 
kartę wypuszczającą kontrakt: 8\diams. XD


\newpage
11.

\handdiagramv
        {\vhand{K954}{AT6}{AQJ87}{6}}
        {\vhand{AJ7}{Q432}{T653}{KT}}
        {\vhand{QT62}{9875}{K4}{A52}}
        {\vhand{83}{KJ}{92}{QJ98743}}
        {}

\begin{table}[h!]
    \centering
    \begin{tabular}{cccc}
        \nvul{W} & \nvul{N} & \nvul {E} & \nvul{S} \\
        - & 1\diams & \pass & 1\hearts \\
        3\clubs  & 3\hearts & \pass & 4\hearts \\
        all \pass & \\
    \end{tabular}
\end{table}

Wist 9\diams. Oddajemy kiera przeciwnikowi z lewej.
Wychodzi Q\clubs. Bierzemy asem, przebijamy trefla i ściągamy
A\hearts. Ściagamy kara. Dochodzi do końcówki:

\handdiagramh
        {\hhand{K954}{-}{8}{-}}{}
        {\hhand{AJ7}{Q4}{-}{-}}{}
        {\hhand{QT2}{98}{-}{-}}{}
        {nieistotne}{}
        {}

Gramy ostatnie karo. Niezależnie czy \nvul{E}\ 
przepuści je czy przebije (jeśli przebije nisko, nadbijamy!),
będzie finalnie wpuszczony i weźmie jedynie 2 lewy (Q\hearts i A\spades). 

\newpage
12.

Licytacja:

\begin{table}[h!]
    \centering
    \begin{tabular}{cccc}
        \nvul{W} & \vul{N} & \nvul {E} & \vul{S} \\
        1\clubs & 1\diams & 4\hearts & ? \\
    \end{tabular}
\end{table}

Trzymamy:

\vhand{AQT94}{-}{KQ6}{JT653}

Statystycznie trafimy na co najmniej dubla pik u partnera,
a gra 4\major na 7-karcie jest zwykle lepsza niż 5\minor,
dlatego lepiej zalicytować 4\spades. Nie zagramy też nigdy tego
kontraktu z \dbl, gdyż zawsze możemy uciec do 5\diams.

Pełne rozdanie:

\handdiagramv{\vhand{876}{3}{AJT42}{A742}}
        {\vhand{J}{KQJ98752}{9873}{-}}
        {\vhand{AQT94}{-}{KQ6}{JT653}}
        {\vhand{K532}{AT64}{5}{KQ98}}
        {NS}



\end{document}