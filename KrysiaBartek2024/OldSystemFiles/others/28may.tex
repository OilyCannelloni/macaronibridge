\documentclass[12pt, a4paper]{article}
\usepackage{import}

\import{../../lib/}{bridge.sty}

\title{27.05 poniedziałek, Żaczek}
\author{Krysia \& Kacper}
\begin{document}
\maketitle

1. 

\handdiagramv{\vhand{KJT7}{87}{652}{KT63}}
        {\vhand{A984}{KQ}{KJ743}{KJ743}}
        {\vhand{Q5}{AT9654}{A9}{852}}
        {\vhand{632}{J32}{QT8}{AJ97}}
        {}

\begin{table}[h!]
    \centering
    \begin{tabular}{cccc}
        \nvul{W} & \nvul{N} & \nvul {E} & \nvul{S} \\
        - & 1\diams & 1\hearts & 2\diams \\
        \pass & 2\nt & all \pass & \\
    \end{tabular}
\end{table}

\vspace{0.2cm}

Wist 6\hearts do K\hearts. Rozgrywający wyrabia kara,
które za drugim razem \nvul{S} (Krysia) bije asem. 
Należało teraz, zgodnie ze zrzutkami partnera (65\diams)
wyjść w pika. Weźmiemy: A\diams, A\hearts, 3 piki i K\clubs,
obkładając 2\nt.

\newpage
5. 

\handdiagramv{\vhand{K8754}{82}{AKQ}{J83}}
        {\vhand{A6}{AJ5}{J9653}{K94}}
        {\vhand{QJ92}{KT97}{874}{65}}
        {\vhand{T3}{Q643}{T2}{AQT72}}
        {NS}

Licytacja może przebiegać:

\begin{table}[h!]
    \centering
    \begin{tabular}{cccc}
        \nvul{W} & \vul{N} & \nvul {E} & \vul{S} \\
        1\spades & \dbl & 2\spades & 2\nt \\
        \pass & 3\diams & \pass & 3\hearts \\
        3\spades & all \pass & & \\
    \end{tabular}
\end{table}

\vspace{0.2cm}

2\nt = dowolna 2-kolorówka.

\newpage
11. 

\handdiagramv{\vhand{AQ73}{AQ}{93}{KQJ92}}
        {\vhand{8}{T54}{J7642}{A753}}
        {\vhand{JT54}{9732}{AQT8}{T}}
        {\vhand{K962}{KJ86}{K5}{864}}
        {}

\begin{table}[h!]
    \centering
    \begin{tabular}{cccc}
        \nvul{W} & \nvul{N} & \nvul {E} & \nvul{S} \\
        - & - & - & \pass \\
        \pass & 1\clubs & \pass & 1\hearts \\
        \pass & 2\spades & \pass & 2\nt \\
        \pass & 3\diams & \pass & 3\spades \\
        \pass & 4\spades & all \pass & \\
    \end{tabular}
\end{table}

\vspace{0.2cm}

2\nt = pytanie o skład, 3\diams = 5422. Rozgrywa \nvul{N} (Krysia).
Wist 4\diams do T\diams i K\diams. Gramy 3\diams do asa
i damę z nadzieją na zrzucenie kiera. Niestety 
\nvul{W} przebija, nadbijamy. Wychodzi w kiera.
Teraz musimy zaimpasować z nadzieją, że wyszedł spod króla.
Zagrany został jednak as, a następnie trefl do T\clubs,
z nadzieją, że \nvul{E} mając asa przepuści -- tak też się stało.

\newpage
13. 

\handdiagramv{\vhand{A862}{J43}{K}{A9853}}
        {\vhand{7}{AK9876}{T2}{QJT4}}
        {\vhand{KJ43}{Q52}{AQJ65}{6}}
        {\vhand{QT95}{T}{98743}{K72}}
        {NSWE}

\begin{table}[h!]
    \centering
    \begin{tabular}{cccc}
        \vul{W} & \vul{N} & \vul {E} & \vul{S} \\
        - & 1\clubs & \dbl & \rdbl \\
        1\diams & \pass & 2\hearts  & \dbl \\
        all \pass & & & \\
    \end{tabular}
\end{table}

\vspace{0.2cm}

Odzywki \dbl i 1\diams pokazują jedynie przedziały siły.
Kontra na \vul{E} (Kacper) -- ukarnia kolejną.
Wist 6\clubs do asa. Następnie, aby obłożyć bez 3,
należało wyjść najmniejszym treflem do przebitki,
aby dostać odwrót do K\diams (spod AQJ!). Wzięte zostaną:
A\clubs, przebitka \clubs, K\diams, przebitka \clubs, A\diams,
A\spades, przebitka \clubs i J\hearts na koniec (rozgrywający
nie ma dojścia do stołu aby go wyimpasować).

\newpage
15.

\handdiagramv{\vhand{AJ95}{AJT}{AT9732}{}}
        {\vhand{KQ4}{Q742}{6}{AQJT7}}
        {\vhand{T8}{9653}{J54}{9653}}
        {\vhand{7632}{K8}{KQ8}{K842}}
        {NS}

Para \nvul{EW} dolicytowała się do 3\nt (kontrakt
zagrany przez ok. połowę sali). Wist 7\diams
do J\diams i K\diams. Na \vul{N} (Krysia) jest problem
ze zrzutkami. Aby obłożyć kontrakt potrzebujemy 2x
dojść do ręki, żeby wyrobić kara. Ostatecznie i tak będziemy
w przymusie, ale jeśli zostawimy 
\xspades AJ\xhearts A\xdiams Axxx, 
jest szansa, że rozgrywający 
(po ściągnięciu trefli) zagra kiera do K\hearts.
Weźmiemy wtedy asem. Kluczowe jest, że partner ma 9\hearts,
co pokazał zrzutkami trefl: 3956 (,,długość''  \hearts).

\newpage
16.

\handdiagramv{\vhand{Q83}{93}{KQJ9}{QJT2}}
        {\vhand{T652}{52}{A53}{K543}}
        {\vhand{AK74}{KJ86}{7}{A987}}
        {\vhand{J9}{AQT74}{T8642}{6}}
        {EW}

\begin{table}[h!]
    \centering
    \begin{tabular}{cccc}
        \vul{W} & \nvul{N} & \vul {E} & \nvul{S} \\
        \pass & 1\clubs & \pass & 1\hearts \\
        \pass & 1\nt & \pass  & 2\diams \\
        \pass & 2\nt & \pass & 3\nt \\
        all \pass & & & \\
    \end{tabular}
\end{table}

Wist 6\spades. Zabicie Q\spades w ręce w pierwszej lewie
w nieoczywisty sposób wypuszcza lewę.

\end{document}