\documentclass[12pt, a4paper]{article}
\usepackage{import}

\import{../lib/}{bridge.sty}
\setmainlanguage{english}

\title{Negat}
\author{Krystyna Gasińska}
\begin{document}
\maketitle

Kiery \hearts \small{R.I.P.} trudno

W negacie:
\begin{itemize}
    \item GF na \diams - bal/krt OK
    \item 0-6 bez 6M
    \item 7-10 bez 4M
    \item młode słabe OK
    \item młode lepsze OK
    \item długie \clubs słabe OK
    \item długie \clubs lepsze OK
    \item inv z treflami OK
    \item długie \diams słabe OK
    \item długie \diams lepsze OK
    \item inv z karami OK
    \item inv+ młode z krt OK
\end{itemize}

\subsubsection*{1\clubs -- ?}
\begin{itemize}
    \item 1\diams = bardzo dużo różnych rzeczy
    \item 1\nt = inwit, może mieć 4M, 2\clubs = do gry, 2\diams = Stayman 
    \item 2\clubs = GF bal (nie trefle!) 
    \item 2\diams = multi 
    \item 2\spades = trefle, GF lub blok 
    \item 2\nt = inwit z długim młodym, dalej: 3\clubs = acc z \diams, do pasa z \clubs | 3\diams = przyjęcie z \clubs, pasuj z \diams | 3\hearts = GF pytanie o młody 
    \item 3\clubs = mixed? 
    \item 3\diams = GF \diams + 4\major $\leftarrow$ to mi się podoba 
    \item 3\major = splinter?
\end{itemize}

\subsubsection*{1\clubs -- 1\diams \\
                ?}
\begin{itemize}
    \item 1\hearts = [dowolne] bez 4\spades, forsuje!, może być 15-18 z treflami
    \item 1\spades = 4\spades
    \item 1\nt = 18-19 \bal 
    \item 2\clubs = 11-14 \clubs 
    \item 2\diams = acol cośtam
\end{itemize}

\subsubsection*{1\clubs -- 1\diams \\
                1\hearts -- ?}
\begin{itemize}
    \item 1\spades = relay 
    \item 1\nt = młode bardzo słabe lub \gf\ z krótkością lub też bez, 
    F1 -> 2\minor -> krótkość/2nt 
    \item 2\minor = lepsza ręka z młodym (jakieś 6-9) 
    \item 2\hearts = \inv\ młode z krótkością 
    \item 2\spades = \inv\ młode z krótkością 
    \item 2\nt = \inv\ na obu młodych 
    \item 3x = \gf\ \diams z krótkością x 
    \item 3\diams = \gf\ \diams bez krótkości
\end{itemize}

\subsubsection*{1\clubs -- 1\diams \\
                1\hearts -- 1\spades\\
                ?}
\begin{itemize}
    \item 1\nt = półautomat 
    \item 2\clubs = 15-18 trefle \nf -> ruszenie forsuje do 3\clubs (zatrzymania jak po inverted)
\end{itemize}

\subsubsection*{1\clubs -- 1\diams \\
                1\hearts -- 1\spades\\
                1\nt -- ?}
\begin{itemize}
    \item \pass = gówno gramy z dobrej ręki 
    \item 2\minor = młody do gry
\end{itemize}

\subsubsection*{1\clubs -- 1\diams \\
                1\spades -- ?}
\begin{itemize}
    \item 1\nt = gówno ze złej ręki trudno 
    \item 2\minor = do gry 0-9 
    \item 2\hearts = młode z krótkością \hearts, inv+ 
    \item 2\nt = [cośtam pewnie brakuje] 
    \item 3\clubs = \gf\ \diams z krótkością, 3\diams = pytanie
    \item 3\diams = \gf\ \diams bez krótkości 
    \item 3\spades = \gf\ młode z krótkością \spades
\end{itemize}

\end{document}