\documentclass[12pt, a4paper]{article}
\usepackage{import}

\import{../lib/}{bridge.sty}
\setmainlanguage{polish}

\title{\vspace{-2cm}Transfery po wejściach przeciwnika blokiem}
\date{}
\author{Krysia \& Bartek}


\begin{document}
\maketitle

{\Huge{\clubs}}\\
\begin{itemize}
    \item Po treflu gramy z grubsza jak po wejściu 
    na 1\nt, czyli ludzie z jakiegoś powodu grają transferami i bez Lebensohla,
    a my gramy to i to bo why not.
    \item Transfer na KP to pytanie o trzymanie \gf. Wiem, że
    to dokładnie na odwórt niż po 1\nt, ale nie da się odwrócić.
    Możemy po \nt grać na odwrót.
    \item Czasami (rzadko) cośtam wywalamy
    na korzyść czegoś (patrz {\color{BurntOrange}{\textbf{\large!}}}).
\end{itemize}

\sequence{{1\clubs}{(2\diams)}}
\begin{options}[1]
    \item[\dbl] takeout
    \item[2\major] \nat\ \nf
    \item[2\nt] Lebensohl
    \item[3\clubs] 54\major\ \gf \imp
    \item[3\diams] \then\ \hearts\ \invp
    \item[3\hearts] \then\ \spades\ \invp
    \item[3\spades] ask \diams stopper \gf\ $\leftarrow$ gdzieś to trzeba upchać, 3\clubs to co innego
\end{options}

\sequence{{1\clubs}{(2\hearts)}}
\begin{options}[1]
    \item[\dbl] takeout
    \item[2\spades] \nat\ \nf
    \item[2\nt] Lebensohl
    \item[3\clubs] \then\ \diams\ \invp
    \item[3\diams] ask \hearts stopper \gf\
    \item[3\hearts] \then\ \spades\ \invp
    \item[3\spades] \minor \imp
\end{options}

\sequence{{1\clubs}{(2\spades)}}
\begin{options}[1]
    \item[\dbl] takeout
    \item[2\nt] Lebensohl
    \item[3\clubs] \then\ \diams\ \invp
    \item[3\diams] \then\ \hearts\ \invp
    \item[3\hearts] ask \spades stopper \gf\
    \item[3\spades] \minor \imp
\end{options}

\sequence{{1\clubs}{(3\clubs)}}
\begin{options}[1]
    \item[\dbl] takeout \gf $\leftarrow$ niestety musi być \gf
    \item[3\diams] \then\ \hearts\ \invp
    \item[3\hearts] \then\ \spades\ \invp
    \item[3\spades] \then\ \diams\ \gf
\end{options}

\sequence{{1\clubs}{(3\diams)}}
\begin{options}[1]
    \item[\dbl] takeout \gf
    \item[3\hearts] \then\ \spades\ \invp
    \item[3\spades] \then\ \hearts\ \gf
\end{options}

\sequence{{1\clubs}{(3\hearts)}}
\begin{options}[1]
    \item[\dbl] \spades\ \invp \imp
    \item[3\spades] 4\spades, no \hearts\ stopper \gf $\leftarrow$ czyli jakby takeout
\end{options}

\sequence{{1\clubs}{(3\spades)}}
\begin{options}[1]
    \item[\dbl] takeout \gf
\end{options}

\newpage
{\Huge{\diams}}\\
\begin{itemize}
    \item Po karo tak samo jak po treflu, tylko fajnie czasem karo
    sfitować. 
    \item Tam gdzie jest Lebensohl to \diams competitive można dać przez Lebensohla.
    Tam gdzie nie ma, to trzeba sfitować naturalnie (tylko po 3\clubs).
    \item Nie mamy transferu na trefle, coś trzeba kombinować (np. Lebensohl i KP).
\end{itemize}

\sequence{{1\diams}{(2\hearts)}}
\begin{options}[1]
    \item[\dbl] takeout
    \item[2\spades] \nat\ \nf
    \item[2\nt] Lebensohl
    \item[3\clubs] \then\ \diams\ \invp
    \item[3\diams] ask \hearts stopper \gf
    \item[3\hearts] \then\ \spades\ \invp
    \item[3\spades] \then\ \nt $\leftarrow$ ale można wywalić, albo dać coś fajnego po 3\nt
\end{options}

\sequence{{1\diams}{(2\spades)}}
\begin{options}[1]
    \item[\dbl] takeout
    \item[2\nt] Lebensohl
    \item[3\clubs] \then\ \diams\ \invp
    \item[3\diams] \then\ \hearts\ \invp
    \item[3\hearts] ask \spades stopper \gf
    \item[3\spades] \then\ \nt
\end{options}

\sequence{{1\diams}{(3\clubs)}}
\begin{options}[1]
    \item[\dbl] takeout \gf
    \item[3\diams] competitive
    \item[3\hearts] \then\ \spades\ \invp
    \item[3\spades] \then\ \hearts\ \gf
\end{options}

\sequence{{1\diams}{(3\hearts)}}
\begin{options}[1]
    \item[\dbl] \spades\ \invp
    \item[3\spades] 4\spades, no \hearts\ stopper \gf
\end{options}

\sequence{{1\diams}{(3\spades)}}
\begin{options}[1]
    \item[\dbl] takeout \gf
\end{options}

\vspace{2cm}
{\Huge{\hearts}}\\
\begin{itemize}
    \item Dokładnie tak samo jak po \diams, jeden splinter bez przeskoku,
    ale to chyba i tak tak graliśmy.
\end{itemize}

\sequence{{1\hearts}{(2\spades)}}
\begin{options}[1]
    \item[\dbl] takeout
    \item[2\nt] Lebensohl
    \item[3\clubs] \then\ \diams\ \invp
    \item[3\diams] \then\ \hearts\ \invp
    \item[3\hearts] ask \spades stopper \gf
    \item[3\spades] splinter
    \item[4\minor] color + fit
\end{options}

\sequence{{1\hearts}{(3\clubs)}}
\begin{options}[1]
    \item[\dbl] takeout \gf
    \item[3\diams] \then\ \hearts\ \invp
    \item[3\hearts] competitive
    \item[3\spades] \spades\ \gf
    \item[4\clubs] slam try, no shortness
    \item[4\diams] slam try, shortness
\end{options}

\sequence{{1\hearts}{(3\diams)}}
\begin{options}[1]
    \item[\dbl] takeout \gf
    \item[3\hearts] competitive
    \item[3\spades] \spades\ \gf
    \item[4\clubs] slam try, no shortness
    \item[4\diams] slam try, shortness
\end{options}

\sequence{{1\hearts}{(3\spades)}}
\begin{options}[1]
    \item[\dbl] takeout \gf
    \item[4\clubs] slam try, no shortness
    \item[4\diams] slam try, shortness
    \item[4\spades] slam try, void 
\end{options}

\vspace{2cm}
{\Huge{\spades}}\\
\begin{itemize}
    \item Dokładnie tak jak po kierze, muszę iść na zajęcia, kiedyś dopiszę.
\end{itemize}

\end{document}