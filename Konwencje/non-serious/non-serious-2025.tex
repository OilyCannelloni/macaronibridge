\documentclass[12pt, a4paper]{article}
\usepackage{import}
\usepackage[left=70px,right=70px,top=100px,bottom=100px,paperheight=30in]{geometry}


\import{../../lib/}{bridge.sty}
\setmainlanguage{polish}

\title{Amerykańskie Two-Over-One (Non-Serious 3\spades/3\nt)}
\date{}
\author{Bartek Słupik }


\begin{document}
\maketitle

\section*{Co to jest?}
Jest to schemat licytacji po two-over-one, który pozwala pokazywać \textbf{kształt ręki bez ograniczeń}, a mimo to złapać bilans przed szlemikiem.


\subsubsection*{Dlaczego nikt oprócz mnie tak nie gra?}
To nie do końca prawda. System jest bardzo popularny w USA i jest tam używany nawet przez jedne z najlepszych na świecie par.
W Polsce też znajduje zastosowanie, choć w nieco innych sekwencjach i nie jest tak popularne.

Ja uczyłem się grać z materiałów amerykańskich, dlatego ``zaimportowałem'' ten schemat do naszego środowiska i gram nim z powodzeniem od kilku lat.


\subsubsection*{Skoro to jest takie dobre, to czemu Polska tego nie adoptowała?}
Porównajmy istniejące schematy:
\begin{itemize}
	\item 1\hearts -- 2\clubs -- 2\spades jest nadwyżkowe + brak dalszych ustaleń. Trudność: 4/10, precyzja: 4/10
	\item Układ bez nadwyżek. Trudność: 1/10, precyzja: 3/10 
	\item Non-Serious. Trudność 5/10, precyzja 8/10
	\item Rewersy nadwyżkowe + transfery + fragmenty. Trudność: 8/10, precyzja 9/10	
\end{itemize}

I teraz odpowiedź - bo istnieje bardziej precyzyjny system, który jednak jest \textbf{bardzo trudny}. Polacy lubią robić sobie niepotrzebne problemy. Taka tradycja.

W przyszłości prawdopodobnie nadejdzie u Was czas na zmianę, ale to jeszcze długa droga. A to i tak trzeba umieć.

\vspace{2cm}
\section*{Level 1 - Jak to działa?}

\subsubsection*{Do poziomu 3 włącznie pokazujemy tylko układ ręki nie zwracając uwagi na bilans.}

\sequence{{1\hearts}{2\clubs}}
\begin{options}[1]
	\item[2\diams] 4+\diams, siła do nieba
	\item[2\hearts] 6+\hearts, siła do nieba
	\item[2\spades] 4+\spades, siła od 11 do nieba
	\item[2\nt] 5\xhearts332, słabe lub silne - W każde 5M332 15-17 otwieramy 1\nt.
	\item[3\clubs] 4+\clubs, ale raczej nie 2524 (damy 2\nt)
	\br
	\item[3\diams] 4+\clubs, krótkość \diams (to już bardziej zaawansowane, ale to nie silne z karami!)
	\item[3\hearts] samoustalenie \hearts. Tworzy pozycję niepoważnego bilansowania!!!
\end{options}


\subsubsection*{Na kolorze starszym łapiemy bilans na wysokości 3\spades lub 3\nt}

Klasyfikujemy naszą rękę do jednego z 3 przedziałów:
\begin{itemize}
	\item \textbf{Zjazd.} Dolne 10\% rąk. Licytujemy 4 w nasz kolor.
	\item \textbf{Non-serious.} Ręka w przedziale 11-50\%. Licytujemy \textbf{odzywkę niepoważną}
	\item \textbf{Serious.} Ręka w przedziale 51\%+. Licytujemy od razu \textbf{cue-bid} informując partnera, że mamy rękę mocno nadwyżkową.
\end{itemize}


Poniższe ustalenia stosujemy w każdej sekwencji, w której ustalono kolor na poziomie 3\major lub niżej.
\sequence{{1\spades}{2\clubs}{2\spades}{3\spades}}
\begin{options}[1]
	\item[3\nt] Non-Serious (ręka 11-50\%). Nie mówi nic o składzie ani cue-bidach. \imp
	\item[4\clubs] Cue-bid (ręka 51+\%).
	\item[4\diams] Cue bid (ręka 51+\%).
	\item[4\spades] Zjazd (ręka 0-10\%).
\end{options}

\sequence{{1\hearts}{2\diams}{3\clubs}{3\hearts}}
\begin{options}[1]
	\item[3\spades] \textbf{Non-Serious} (ręka 11-50\%). Nie mówi nic o składzie ani cue-bidach. \vimp
	\item[3\nt] \textbf{Cue-bid \spades} (ręka 51+\%) \vimp
	\item[4\clubs] Cue-bid (ręka 51+\%).
	\item[4\diams] Cue bid (ręka 51+\%).
	\item[4\hearts] Zjazd (ręka 0-10\%).
\end{options}

Dlaczego tak dziwnie zamienione? \textbf{Tak jest prościej!} Cue-bid przed niepoważnym 3\nt powodowałby dziwne sytuacje z brakiem zbilansowania. Da się do tego bardzo szybko przyzwyczaić,
a zagranie kilka razy 3\nt zamiast 7\hearts juz na pewno utrwali tą podmiankę. 

\subsubsection*{I po co to wszystko?}

Na pocieszenie przeanalizujmy tą drugą sekwencję. Co ma partner?
\begin{itemize}
	\item Na pewno 5\hearts, 4+\clubs
	\item Ale mając 3 kara może dałby 3\diams? Czyli raczej nie ma.
	\item Mając 5422 dałby często 2\nt. Jeśli ma taki skład, to karta jest bardzo asowa.
	\item Czyli najczęściej ma 3514 lub 5-5. Bez większego systemu nie da się stwierdzić.
\end{itemize}

Ale co by się stało bez Non-Serious? Pewnie 1\hearts -- 2\diams -- 2\hearts -- 3\hearts. O składzie nie wiemy nic. 

\subsubsection*{Przykład}
\handdiagramv{}{\vhand{K84}{K75}{AT975}{AQ}}{}{\vhand{AT2}{AQ986}{9}{K953}}{}{}{}{}{}

\vspace*{-1cm}
\begin{center}
	\webidding{
		1\hearts & 2\diams \\
		3\clubs & 3\hearts \\
		3\spades\alrt & 3\nt\alrt \\
		4\clubs & 4\nt \\
		5\spades & 6\hearts 
	}
\end{center}

\begin{itemize}
	\item 3\spades = Niepoważne - mamy tylko 13 punktów, ale dobre, grające figury i aż 2 asy z damą. Kartę oceniam na około 35\%. Piąty trefl dawałby już okolice 45\%, ale nadal za mało na cue-bid.
	\item 3\nt = Cue-bid pik. Mamy duże nadwyżki i do większości niepoważnych rąk partnera chcemy jeszcze wymieniać informacje.
	\item 4\clubs = Współpracujemy. Ręka jest dużo lepsza niż jakieś tam losowe 13 pkt. To bliżej góry przedziału Non-Serious niż dołu.
	\item 4\nt = Liczymy lewy. Czwartego trefla przebijamy, w karach pewnie jest singiel - super! Szlemik będzie dobry. Sprawdźmy, czy nie oddajemy dwóch asów.
	\item 6\hearts = Na 7 szanse są marne. Partner pokazał już 13 punktów i cały skład (wiemy o AAQ i \xclubs K). Z dodatkową damą pik dałby raczej już serious. Teoretycznie szlema daje piąty trefl - ale nie ma jak sprawdzić. I tak by był na podziale 3-2 pewnie, więc zysk z grania go byłby statystycznie niewielki.
\end{itemize}


\vspace{2cm}
\section*{Level 2 - Pytanie o krótkość}

Kiedy umiemy już podstawy schematu, pora wycisnąć z niego zyski. Rozważmy następujące rozdanie:

\handdiagramv{}{\vhand{KQ6}{A9}{87432}{KJ6}}{}{\vhand{AJ84}{KQJ754}{-}{Q43}}{}{}{}{}{}
\vspace*{-1cm}

Na szlemika mamy 12 górnych lew. Ale jak go znaleźć? Normalnie poszłoby pewnie coś w rodzaju

\begin{center}
	\webidding{
		1\hearts & 2\clubs \\
		2\hearts & 2\nt \\
		3\hearts & 4\hearts
	}
\end{center}

E nie wie nic o ręce partnera. Nawet nie było blisko znalezienia dobrze sparowanej \textbf{krótkości z wyłączeniem}. Ale niewielkim nakładem pracy może być inaczej!

Jeśli partner pokazał 6 kierów lub pików, odzywka +1 \textbf{ustala kolor} i pyta o krótkość.

\sequence{{1\spades}{2\clubs}{2\spades}{2\ntx\alrt}}
\begin{options}[1]
	\item[3\clubs] Krótkość \clubs
	\item[3\diams] Krótkość \diams
	\item[3\hearts] Krótkość \hearts
	\item[3\spades] Brak krótkości
\end{options}

\sequence{{1\hearts}{2\clubs}{2\hearts}{2\spades\alrt}}
\begin{options}[1]
	\item[2\nt]  Brak krótkości
	\item[3\clubs] Krótkość \clubs
	\item[3\diams] Krótkość \diams
	\item[3\hearts] Krótkość \spades
\end{options}
 
Używając tego schematu licytacja układowych rąk staje się dużo prostsza:

\begin{center}
	\webidding{
		1\hearts & 2\clubs \\
		2\hearts & 2\spades\alrt \\
		3\diams\alrt & 3\nt\alrt \\
		4\diams\alrt & 6\hearts
	}
\end{center}

\begin{itemize}
	\item 2\spades = pytanie o krótkość
	\item 3\diams = krótkość \diams
	\item 3\nt = Cue-bid \spades. Do krótkości karo wartość naszej ręki urosła bardzo wysoko. Potrzebujemy \xspades A, \xhearts KQ, \xclubs A. To z jednej strony niewiele, z drugiej są to same asy - karty, które w układowych szlemikach są warte duzo więcej niż 4 punkty. Pokazaliśmy silną rękę, ale sami nie przekroczymy poziomu 4\hearts. Chyba że...
	\item 4\diams = A to co? Przecież partner pokazał już krótkość karo? Ma renons! Nigdy nie dajemy cue-bidu z singlowego Asa, jeśli wcześniej pokazaliśmy krótkość. Nawet jeśli krótkość była pokazana przez domysł.
	\item 6\hearts = Partner musi mieć \xspades A i \xhearts KQ - inaczej nie ma na otwarcie. Do tego jeszcze jakaś dama i mamy 12 lew.
\end{itemize}

\subsubsection*{Schemat LSF}
Odpowiadanie na pytanie o krótkość może powodować niemałe rozjazdy. Dlaczego bowiem na kierach 2\nt to brak a 3\hearts to krótkość \spades? Czemu nie na odwrót?
Okazuje się, że nieco bardziej sztuczny system może powodować znacznie mniej rozjazdów. Moglibyśmy odpowiadać schodkami, tak jak na pytanie o Asy:

\sequence{{1\spades}{2\clubs}{2\spades}{2\ntx\alrt}}
\begin{options}[1]
	\item[3\clubs] Brak krótkości
	\item[3\diams] Krótkość \clubs (niska)
	\item[3\hearts] Krótkość \diams (średnia)
	\item[3\spades] Krótkość \hearts (wysoka)
\end{options}

\sequence{{1\hearts}{2\clubs}{2\hearts}{2\spades\alrt}}
\begin{options}[1]
	\item[2\nt] Brak krótkości
	\item[3\clubs] Krótkość \clubs (niska)
	\item[3\diams] Krótkość \diams (średnia)
	\item[3\hearts] Krótkość \spades (wysoka)
\end{options}


Jest to Schemat LSF (Low Shortness First). Ja aktualnie gram pytaniem o krótkość w około 20 sekwencjach, w połowie z których pokazywanie krótkości naturalnie jest niewykonalne.
Dzięki LSF szansa na rozjazd jest bardzo mała, bo zawsze policzę schodki.

Pamiętajmy, że umieszczając LSF w innych pozycjach należy zwrócić uwagę na to, jakie krótkości są dostępne. Np po pytaniu 3\spades\alrt\ w poniższej sekwencji są tylko 3 możliwości - brak, niska lub wysoka.
\sequence{{1\clubs}{1\hearts}{3\hearts}{3\spades\alrt}}
\begin{options}
	\item[3\nt] Brak krótkości
	\item[4\clubs] Krótkość \diams (niska, bo \clubs nie może być!)
	\item[4\diams] Krótkość \spades (wysoka)
\end{options}

A czasami któtkość jest dostępna zawsze, na przykład w popularnym na AGH schemacie odpowiedzi na otwarcie 1\spades (2\nt = słaby splinter 9-11PC na jeszcze nieokreślonym kolorze, 3\clubs = pytanie)
\sequence{{1\spades}{2\ntx\alrt}{3\clubs\alrt}}
\begin{options}[2]
	\item[3\diams] = krótkość \clubs (niska, bo brak nie jest możliwy!)
	\item[3\hearts] = krótkość \diams (średnia)
	\item[3\spades] = krótkość \hearts (wysoka)
\end{options}

\subsubsection*{Układ a powaga}

W poprzednim rozdaniu dokonaliśmy upgrade'u karty ze względu na pasującą krótkość. A co jeśli krótkość nie pasuje?

\handdiagramv{}{\vhand{KQ6}{A9}{87432}{KJ6}}{}{\vhand{AJ84}{KQJ754}{Q43}{-}}{}{}{}{}{}
\vspace*{-1cm}

\begin{center}
	\webidding{
		1\hearts & 2\clubs \\
		2\hearts & 2\spades\alrt \\
		3\clubs\alrt & 4\hearts
	}
\end{center}

\begin{itemize}
	\item 2\spades = Pytanie o krótkość
	\item 3\clubs = Krótkośc \clubs
	\item 4\hearts = Zjazd. Ale partner może z tego dużo wyczytać! Nie zjechaliśmy od razu po 2\hearts, więc w karcie musi być coś interesującego, nieminimalnego, czego wartość dramatycznie spadła po informacji o krótkości \clubs. Czyli pewnie 2 zmarnowane figury trefl i łącznie ok 13-14PC, które do innej krótkości lub braku \textbf{oceniłoby swoją siłę na co najmniej niepoważną}
\end{itemize}

Teraz zaczynamy rozumieć, dlaczego ten system jest taki dobry. Bilansowanie powinno zostać przeprowadzone \textbf{w oparciu o dopasowanie układowe rąk}, a nie tylko na podstawie patrzenia w punkty i układ własnej ręki. Dlatego większa wolność w szybkim pokazaniu składu przynosi duze zyski.



\pagebreak
\vspace{2cm}
\section*{Level 3 - Dodatkowe mini-ustalenia}

\vspace{2cm}
\section*{Level 4 - Bilansowanie na kolorze młodszym}

\vspace{2cm}
\section*{Level 5 - Bilansowanie, jeśli nie znajdziemy fitu}

\vspace{2cm}
\section*{Level 6 - Pytanie o skład i silne ustalenia koloru}

\vspace{2cm}
\section*{Level 7 - Two-Way-Set}




\end{document}

