\documentclass[12pt, a4paper]{article}
\usepackage[paperwidth=21cm,paperheight=75cm,margin=1in]{geometry}
\usepackage{import}

\import{../../lib/}{bridge.sty}
\setmainlanguage{polish}

\title{Two Way Set}
\author{Bartek Słupik}
\begin{document}
\maketitle

 
\subsubsection*{Basic stuff}

\begin{itemize}
    \item Two Way Set is a slam convention which enables you to forcingly show a strong hand with support to a suit on
    the 4 level.
    \item It is most profitable in sequences, where other strength-showing conventions, such as Non-Serious, cannot 
    be used.
\end{itemize}

The algorithm follows a simple principle - the suit, in which the nearest game is the closest, is agreed via the cheapest
artificial bid. Let's see an example (4\clubs = \spades\ + \clubs):

\compsequence{{2\hearts}{4\clubs\alrt}{P}}
\begin{compoptions}[4]
    \item[4\diams] Strong with \spades
    \item[4\hearts] Strong with \clubs  
\end{compoptions}

Here, 2\nt is \gf asking for shape

\sequence{{1\diams}{1\spades}{2\clubs}{2N}{3\spades}}
\begin{options}[2]
    \item[3\nt] To play
    \item[4\clubs] Strong with \spades
    \item[4\diams] Strong with \clubs
    \item[4\hearts] Strong with \diams   
\end{options}

As you can see, this 2\nt with a natural responce scheme produces really silly auctions. 
But we can do better!

\sequence{{1X}{1Y}{1Z}{2N}}
\begin{options}
    \item[3\clubs] 3 of Y (everything agreed naturally)
    \item[3\diams] 5X 4Z 22
    \item[3\hearts] 5-5 or 6-4 in XZ (depending on the context) with doubleton Y
    \item[3\spades] 5-5 or 6-4 in XZ with Y shortness
    \item[3\nt] 5X-4Y-3-1Y    
\end{options}

It might seem complicated at first, but this arrangement of responses guarantees a forcing raise in every suit in every sequence!

\sequence{{1\hearts}{2\clubs}{2\diams}{2\ntx}{3\spades\alrt}}
\begin{options}[2]
    \item[4\clubs] Strong with \hearts
    \item[4\diams] Strong with \diams 
\end{options}


Here, we can set all 3 suits! (partner has 6\hearts4\spades)
\sequence{{1\hearts}{2\diams}{2\spades}{2\ntx}{3\hearts}}
\begin{options}[2]
    \item[3\spades] Sets \spades (non-serious ON)
    \item[3\nt] To play
    \item[4\clubs] Sets \hearts
    \item[4\diams] Sets \diams   
\end{options}

It works after reverses as well (partner has 6-4 with 2\spades):
\sequence{{1\clubs}{1\spades}{2\hearts}{2\ntx}{3\hearts\alrt}}
\begin{options}
    \item[3\spades] Sets \spades (non-serious ON)
    \item[3\nt] To play
    \item[4\clubs] Strong with \hearts \vimp
    \item[4\diams] Strong with \clubs
\end{options}

An absurd, 5-level case. 4\clubs = \diams\ + \major
\compsequence{{3\clubs}{4\clubs\alrt}{P}}
\begin{compoptions}[4]
    \item[4\diams] Show your major
    \item[4\hearts] Strong with \diams 
\end{compoptions}

\compsequence{{3\clubs}{4\clubs\alrt}{P}{4\diams}{P}{4\hearts}{P}}
\begin{compoptions}[4]
    \item[4\spades] Strong with \diams
    \item[4\nt] Strong with \hearts (5\hearts is higher than 5\diams!)
\end{compoptions}

\compsequence{{3\clubs}{4\clubs\alrt}{P}{4\diams}{P}{4\spades}{P}}
\begin{compoptions}[4]
    \item[4\nt] Strong with \diams (After all this, we gained some space for a much needed 5\clubs Last Train bid!)
    \item[5\clubs] Strong with \spades
\end{compoptions}


The absolute pinnacle of Two-Way Set in practice.
\handdiagramv{\vhand{KQJT76}{6}{Q9}{Q987}}
{\vhand{A2}{QJT74}{T843}{T2}}
{\vhand{}{A53}{AKJ652}{AKJ5}}
{\vhand{98543}{K982}{7}{643}}
{EW}

\begin{table}[h!]
    \centering
    \begin{tabular}{cccc}
        \vul{W} & \nvul{N} & \vul{E} & \nvul{S}\\
		  -  & 2\spades\alrt & \pass & 2\nt\alrt \\
		  \pass & 3\clubs\alrt & \pass & 3\diams\alrt \\
		  \pass & 3\spades\alrt & \pass & 4\diams\alrt \\
		  \pass & 4\spades\alrt & \pass & 5\diams\alrt \\
		  \pass & 5\spades\alrt & \pass & 5\nt\alrt \\
		  \pass & 7\clubs\alrt & \pass & \pass \\
		  \pass
    \end{tabular}
\end{table}

\begin{itemize}
    \item 2\nt = ASK gf
    \item 3\clubs = Any 6-4
    \item 3\diams = ASK
    \item 3\spades = 4\clubs
    \item 4\diams = Strong with \clubs (4\clubs sets \spades)
    \item 4\spades = \hearts shortness (lo/hi)
    \item 5\diams = keycard ask
    \item 5\spades = 0/3
    \item 5\nt = queen ask
\end{itemize}


\end{document}