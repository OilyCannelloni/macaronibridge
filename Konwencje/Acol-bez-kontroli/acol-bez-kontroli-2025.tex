\documentclass[12pt, a4paper]{report}
\usepackage{import}

\import{../../lib/}{bridge.sty}
\setmainlanguage{polish}

\title{Acol bez kontroli - wersja 2025}
\author{Bartek Słupik}
\begin{document}
\maketitle


\subsubsection{Podstawy}

\sequence{{2\clubs}}
\begin{options}[2]
    \item[2\diams] Automat
\end{options}

\sequence{{2\clubs}{2\diams}}
\begin{options}[1]
	\item[2\hearts] 5+\hearts \orr 25+ BAL \imp \qquad
	\item[2\spades] 5+\spades
	\item[2\nt] 23-24 BAL
	\item[3\clubs\alrt] 6+\diams bez starszej czwórki \vimp
	\item[3\diams\alrt] 5+\diams, 4\major \imp
	\item[3\hearts] Samoustalenie
	\item[3\spades] Samoustalenie
	\item[3\nt\alrt] 5+/5+ \minor \vimp
	\item[4\clubs] Samoustalenie
	\item[4\diams] Samoustalenie
\end{options}

\sequence{{2\clubs}{2\diams}{2\hearts}}
\begin{options}[2]
	\item[2\spades\alrt] Automat
\end{options}

\sequence{{2\clubs}{2\diams}{2\spades}}
\begin{options}[2]
	\item[2\nt\alrt] Semi-automatyczne pytanie
	\item[3\clubs\alrt] Słaba ręka bez fitu, która bardzo nie chce zajmować \nt
	\item[3\spades] W podstawowej wersji - ustala kolor.
\end{options}



\subsubsection{Transfery po pytaniu}
\sequence{{2\clubs}{2\diams}{2\hearts}{2\spades}}
\begin{options}[1]
	\item[2\nt] 23-24 BAL
	\item[3\clubs\alrt] 5+\hearts, 4+\diams
	\item[3\diams\alrt] 6+\hearts
	\item[3\hearts\alrt] 5+\hearts, 4\spades
	\item[3\spades\alrt] 5+\hearts, 4+\clubs
\end{options}

\sequence{{2\clubs}{2\diams}{2\spades}{2\ntx}}
\begin{options}[1]
	\item[3\clubs\alrt] 5+\spades, 4+\diams
	\item[3\diams\alrt] 5+\spades, 4+\hearts
	\item[3\hearts\alrt] 6+\spades
	\item[3\spades\alrt] 5+\spades, 4+\clubs
\end{options}



\subsubsection{Po transferze - przykłady (można grać już naturalnie)}
\sequence{{2\clubs}{2\diams}{2\hearts}{2\spades}{3\clubs}}
\begin{options}[2]
	\item[3\diams] Fit \diams
	\item[3\hearts] Dubel \hearts, w razie gdyby partner miał 6-4
	\item[3\spades\alrt] Silne ustalenie kierów
\end{options}

\sequence{{2\clubs}{2\diams}{2\spades}{2\ntx}{3\hearts}}
\begin{options}[2]
	\item[3\spades] 2+\spades, ustala
	\item[3\nt] Propozycja
\end{options}



\subsubsection{Acol na karach}
\sequence{{2\clubs}{2\diams}{3\clubs}}
\begin{options}[2]
	\item[3\diams\alrt] Pytanie o krótkosć, ustala \diams
	\item[3\hearts] 5+\hearts
	\item[3\spades] 5+\spades
\end{options}

\sequence{{2\clubs}{2\diams}{3\diams}}
\begin{options}[2]
	\item[3\hearts] 4+\hearts
	\item[3\spades] 4+\spades, brak 4\hearts
\end{options}

\sequence{{2\clubs}{2\diams}{3\diams}{3\hearts}}
\begin{options}[1]
	\item[3\spades] 4\spades
	\item[4\clubs] Ustalenie kierów
\end{options}


\sequence{{2\clubs}{2\diams}{3\diams}{3\spades}}
\begin{options}[1]
	\item[4\clubs] Ustalenie pików
\end{options}



\subsubsection{Acol na treflach - w otwarciu 1\clubs}
\sequence{{1\clubs}{1\hearts}}
\begin{options}[1]
	\item[2\spades\alrt] \gf na długich \clubs
\end{options}

\sequence{{1\clubs}{1\spades}}
\begin{options}[1]
	\item[3\diams\alrt] \gf na długich \clubs
\end{options}

\sequence{{1\clubs}{1\diams}}
\begin{options}[1]
	\item[2\diams\alrt] 5+\clubs, 4+\diams, ~19+PC \orr 6+\clubs, 19+PC (potem można nat)
\end{options}

\sequence{{1\clubs}{1\diams}{2\diams}}
\begin{options}[2]
	\item[2\hearts\alrt] Słaba ręka 0-4 PC
	\item[2\spades\alrt] 5+PC \gf (2\nt \then\ \clubs, 3\clubs \then\ \diams)
\end{options}

\sequence{{1\clubs}{1\diams}{2\diams}{2\hearts}}
\begin{options}[1]
	\item[2\spades\alrt] 19-22, 5\clubs 4\diams
	\item[2\nt\alrt] 19+, 6+\clubs (automat 3\clubs \nf)
	\item[3\clubs\alrt] 23+, 5\clubs 4\diams, \gf
\end{options}


\end{document}