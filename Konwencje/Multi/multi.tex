\documentclass[12pt, a4paper]{article}
\usepackage{import}

\import{../lib/}{bridge.sty}
\setmainlanguage{polish}

\title{Twoja stara otwiera multi i stawia 800}
\date{}
\author{Bartek Slupik}
\begin{document}
\maketitle
\section{Podstawowe odzywki}
\begin{itemize}
    \item \dbl\ = (13)14-16 \bal, wyklucza starszą piątkę. Młody singiel OK. \vimp
    \item 2\hearts\ = \hearts
    \item 2\spades\ = \spades
    \item 2\nt\ = 17-19 \bal, może zawierać starszą piątkę. \imp \br
    \item 3\clubs\ = \clubs, ale nie 5332/5422
    \item 3\diams\ = \diams, jw.
    \item 3\hearts, 3\spades\ = solidny kolor ale trochę za mało na objaśniaka
    \item 3\nt\ = młode \vimp
    \item 4\clubs\ = \clubs\ + \hearts\spades
    \item 4\diams\ = \diams\ + \hearts\spades
\end{itemize}

\pagebreak


\section{Wejście kontrą na bezpośredniej}
\subsubsection*{(2\diams) -- \dbl\ -- (\rdbl/\passx) -- ?}
\begin{itemize}
    \item \pass\ = chcę bronić, kontry są karne
    \item 2\hearts, 2\spades\ = do gry z piątki
    \item 2\nt\ = lebensohl, zawiera:
    \subitem \soff\ z \clubs\ (\pass)
    \subitem \gf\ bez starszych czwórek (3\diams) \vimp
    \subitem \inv z \hearts\ lub \spades\ (3\hearts, 3\spades)
    \item 3\clubs\ = \textbf{zwykły Stayman} \vimp
    \item 3\diams\ = transfer na \hearts, \gf\ + przyjęcia
    \item 3\hearts\ = transfer na \spades, \gf\ + przyjęcia \br
    \item 3\spades\ = transfer na \nt, brak trzymań w obydwu starych
    \item 4\diams, 4\hearts\ = transfer
\end{itemize}


\subsection*{(2\diams) -- \dbl\ -- (2\hearts/2\spades) -- ?}
\begin{itemize}
    \item \dbl\ = co najmniej inwit (9+), F do 2\nt. Wyklucza 5\hearts\spades\ i krótkość \hearts\spades.
    \item 2\nt\ = lebensohl, zawiera:
    \subitem \soff\ na \clubs\ lub \diams
    \subitem \inv\ na \hearts\ lub \spades
    \item 3\clubs\ = \textbf{zwykły Stayman} \vimp
    \item 3\diams\ = transfer na \hearts\ \gf
    \item 3\hearts\ = transfer na \spades\ \gf
    \item 3\spades\ = \gf\ ze splinterem w kolorze \textbf{licytowanym} przez przeciwnika
    \item 4\diams, 4\hearts\ = transfer
\end{itemize}


\section{Przeciwnik przedłuża blok}

\subsubsection*{(2\diams) -- \dbl\ -- (2\hearts/2\spades) -- \dbl \\ (P) -- ?}
\begin{itemize}
    \item \pass\ = do gry
    \item 2\spades\ = 4\spades, \fonce
    \item 2\nt\ = NAT, minimum
    \item 3\clubs\ = NAT, minimum
    \item 3\diams\ = NAT, minimum
    \item 3\hearts\ (po pikach) = NAT, minimum
    \item 3\hearts, 3\spades\ (KP) = maximum, brak trzymania, brak 4 pików
    \item 3\nt\ = maximum, trzymanie, brak 4 pików
\end{itemize}

\subsubsection*{(2\diams) -- \dbl\ -- (2\hearts) -- \dbl \\ (2\spades) -- ?}
\begin{itemize}
    \item \pass\ = \fonce, wznowienie 3\spades\ partnera = GF z singlem \spades
    \item \dbl\ = karna
    \item 2\nt\ = ``bardzo nie chcę bronić'', \gf
    \item 3\clubs\ = NAT, \gf
    \item 3\diams\ = NAT, \gf
    \item 3\hearts\ = NAT, \gf
    \item 3\spades\ = maximum, brak trzymania
    \item 3\nt\ = maximum, trzymanie, brak 4 pików
\end{itemize}

\subsubsection*{(2\diams) -- \dbl\ -- (2\spades) -- \dbl \\ (3\hearts) -- ?}
\begin{itemize}
    \item \pass\ = 14-16 bez 4 pików lub objaśniak \fonce
    \item \dbl\ = 14-16, 4\spades, defensywne
    \item 3\spades\ = 14-16, 4\spades, ofensywne
    \item 3\nt\ = 14-16, stopper
\end{itemize}

\subsubsection*{(2\diams) -- \dbl\ -- (2\spades) -- \dbl \\ (3\hearts) -- P -- (P) -- ?}
\begin{itemize}
    \item \dbl\ = karna
    \item 3\spades\ = GF, brak trzymania \hearts
    \item 3\nt\ = młode 4+/4+, nie chcemy pałować
    \item 4\hearts\ = \gf, splinter \hearts
\end{itemize}

\subsubsection*{(2\diams) -- \dbl\ -- (3\hearts) -- ?}
\begin{itemize}
    \item \pass\ = gówno lub takeout z krótkością \spades
    \item \dbl\ = \gf\
    \item 3\spades\ = \gf
\end{itemize}



\pagebreak
\section{Wejście na czwartej ręce}

\subsubsection*{(2\diams) -- P -- (2\hearts) -- ?}
\begin{itemize}
    \item \pass\ = gówno lub takeout z krótkością pik
    \item \dbl\ = 14-16 BAL. Na wszystkie wyższe odzywki \dbl\ są wywoławcze.
    \item 2\spades\ = może być z ładnej czwórki jeśli ma krótkość kier
    \item 2\nt\ = 17-19 BAL
\end{itemize}


\subsubsection*{(2\diams) -- P -- (2\spades) -- ?}
\begin{itemize}
    \item \pass\ = gówno lub takeout z krótkością kier
    \item \dbl\ = \textbf{takeout z krótkością pik} \imp
    \item 2\nt\ = 16-19 BAL \imp
\end{itemize}


\section*{Druga ręka licytuje w drugim kółku}
\subsubsection*{(2\diams) -- P -- (2\hearts) -- P \\ (P/2\spades) -- ?}
\begin{itemize}
    \item \pass\ = gówno lub takeout z krótkością kier
    \item \dbl\ = Takeout, może mieć silną siłę. Nie jest to wznówka.
    \item 2\nt\ = młode, po 2\spades\ może być dowolna z kierami
\end{itemize}
\end{document}