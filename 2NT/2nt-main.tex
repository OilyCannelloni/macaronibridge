\documentclass[12pt, a4paper]{article}
\usepackage{../lib/bridgetex2}
\usepackage{polyglossia}
\setmainlanguage{polish}

\title{Otwarcie 2\ntx}
\author{Bartek Słupik}

\begin{document}
    \section{Streszczenie}
    \begin{itemize}
        \item Poniższe ustalenia występują również w sekwencjach silniejszych:
        \subitem 2\nt
        \subitem 2\clubs\ - 2\diams\ --- 2\nt\
        \subitem 2\clubs\ - 2\diams\ --- 2\hearts\ - 2\spades\ --- 2\nt\
        \item Wykorzystujemy następujące konwencje:
        \subitem Puppet Stayman
        \subitem Transfery z fitującymi przyjęciami
        \subitem Minor Puppet Stayman
    \end{itemize}

    \section{2\ntx\ --- ?}
    \begin{itemize}
        \item 3\clubs\ - Puppet Stayman
        \item 3\diams\ - Transfer na 5\hearts, \gf
        \item 3\hearts\ - Transfer na 5\spades, \gf 
        \item 3\spades\ - 6+\minor\ lub 5/5 \minor
        \item 3\nt\ - 5\spades, 4\hearts. \textbf{4\clubs\ ustala \hearts, 4\diams\ ustala \spades}
        \item 4\clubs\ - 5/5 \major\ \st{Bez aspiracji szlemikowych}
        Bo nie ma jak pokazać 5/5 z aspiracjami sensownie i tak xd
        \item 4\diams\ - Transfer na 6+\hearts\ \st{Bez aspiracji szlemikowych}
        \item 4\hearts\ - Transfer na 6+\spades\ \st{Bez aspiracji szlemikowych}
        Bo po zwykłym transferze i jego odrzuceniu i tak się nie ma jak wygrzebać
    \end{itemize}

    \pagebreak
    \section{Sekwnecje Puppeta}
    \subsection{2\ntx\ --- 3\clubs \\ 3\diams\ --- ?}
    \begin{itemize}
        \item 3\hearts\ - 4\spades, otwierający ustala cue bidem
        \item 3\spades\ - 4\hearts, otwierający ustala cue bidem
        \item 3\nt\ - brak starszej czwórki \br
        \item 4\clubs\ - \st{obie czwórki, ręka szlemikowa, 4c ustala kiery, 4nt ustala piki}
        \textbf{Minor Puppet Stayman} \\
        Zaletą szlemikowego 4\clubs\ na starych jest tylko silne ustalenie kierów przez 4\diams.
        Za malo żeby mnie przekonać. A jak OTW sobie ustali piki przez 4\nt\ to wgl się syf robi
        \item 4\diams\ - obie czwórki, do koloru. \st{Bez aspiracji szlemikowych}
    \end{itemize}

    \subsection{2\ntx\ --- 3\clubs \\ 3\hearts\ --- ?}
    \begin{itemize}
        \item 3\spades\ - silne ustalenie kierów
        \item 3\nt\ - do gry
        \item 4\clubs\ - \textbf{5+\clubs\ naturalne}
        \item 4\diams\ - \textbf{5+\diams\ naturalne}\\
        Po pokazaniu starszej piątki MPS nie ma sensu, bo OTW ma rzadko 52(4/2).
        \item 4\hearts\ - do gry
    \end{itemize}

    \pagebreak
    \subsection{2\ntx\ --- 3\clubs \\ 3\spades\ --- ?}
    \begin{itemize}
        \item 3\nt\ - do gry
        \item 4\clubs\ - \textbf{5+\clubs\ naturalne}
        \item 4\diams\ - \textbf{5+\diams\ naturalne}\\
        Po pokazaniu starszej piątki MPS nie ma sensu, bo OTW ma rzadko 52(4/2).
        \item 4\hearts\ - silne ustalenie pików
        \item 4\spades\ - do gry
    \end{itemize}

    \subsection{2\ntx\ --- 3\clubs\ \\ 3\ntx\ --- ?}
    \begin{itemize}
        \item 4\clubs\ - Minor Puppet Stayman
        \item 4\diams\ - 5+\clubs\ (MPS)
        \item 4\hearts\ - 5+\diams\ (MPS)
    \end{itemize}


    \pagebreak
    \section{Sekwencje transferowe}
    \subsection{2\ntx\ --- 3\diams \\ ?}
    \begin{itemize}
        \item 3\hearts\ - dubel kier
        \subitem 3\nt\ - do gry
        \subitem 4\clubs\ - \textbf{Minor Puppet Stayman}
        \subitem 4\diams\ - 5+\clubs\ (MPS)
        \subitem 4\hearts\ - 5+\diams\ (MPS)\\
        Bo jak otwierający ma dubla \hearts\ to jest duża szansa na fit w młodym
        \item 3\spades\ - 5\hearts, 4\spades
        \item 3\nt\ - 3-kartowy fit, 4\diams retransfer
        \item 3\spades, 4\clubs, 4\diams\ - 4-kartowy fit i cue
    \end{itemize}

    \subsection{2\ntx\ --- 3\hearts \\ ?}
    \begin{itemize}
        \item 3\spades\ - dubel kier
        \subitem 3\nt\ - do gry
        \subitem 4\clubs\ - \textbf{Minor Puppet Stayman}
        \subitem 4\diams\ - 5+\clubs (MPS)
        \subitem 4\hearts\ - 5+\diams (MPS)\\
        Bo jak otwierający ma dubla \spades\ to jest duża szansa na fit w młodym
        \item 3\nt\ - 3-kartowy fit, 4\hearts\ retransfer
        \item 3\spades, 4\clubs, 4\diams\ - 4-kartowy fit i cue
    \end{itemize}

    \pagebreak
    \section{Inne}
    \subsection{2\ntx\ --- 3\spades \\ 3\ntx\ --- ?}
    \begin{itemize}
        \item 4\clubs\ - 6+\clubs
        \item 4\diams\ - 6+\diams
        \item 4\hearts\ - 5/5 \minor\, krótkość \hearts
        \item 4\spades\ - 5/5 \minor\, krótkość \spades
    \end{itemize}

\end{document}