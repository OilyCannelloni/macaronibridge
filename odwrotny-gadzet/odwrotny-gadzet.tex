\documentclass[12pt, a4paper]{article}
\usepackage{import}
\import{../lib}{bridge.sty}

\title{\vspace{-2.5cm}Odwrócony Gadżet 3\clubs}
\author{}
\date{}

\begin{document}
\maketitle

\subsection*{Schemat}
W każdej sekwencji po rebidzie 2NT odpowiadający wybiera \textbf{2 interesujące kolory}, które eksplorujemy.
Dalsza licytacja jest tylko w kontekście tych kolorów.

\noindent
\begin{minipage}{.3\textwidth}
    \sequence{{1\clubs}{1\hearts}{2\ntx}}
    \begin{itemize}[align=left, itemindent=4em, labelwidth=3.2em, itemsep=0em]
        \vspace*{-2.3em}
        \item[3\clubs] \hearts i \clubs
        \item[3\diams] \hearts i \diams
        \item[3\hearts] \hearts i \spades \vimp
        \item[3\spades] \hearts i \spades 
    \end{itemize}
\end{minipage}\hspace{2cm}%
\begin{minipage}{.6\textwidth}
    \sequence{{1\clubs}{1\spades}{2\ntx}}
    \begin{itemize}[align=left, itemindent=4em, labelwidth=3.2em, itemsep=0em]
        \vspace*{-2.3em}
        \item[3\clubs] \spades i \clubs
        \item[3\diams] \spades i \diams
        \item[3\hearts] \spades i \hearts
        \item[3\spades] \spades
    \end{itemize}
\end{minipage}
\vspace*{1em}

Dodatkowo:
\begin{itemize}
    \item Odpowiedź na gadżet \textbf{w kolor partnera} zawsze ustala go z \textbf{czwórki}
    \item Odpowiedź na gadżet 3\diams pokazuje \textbf{3 karty} w kolrze partnera (i na razie nic o drugim kolorze)
    \item Jeśli kolor da się ustalić naturalnie, wszystkie inne odzywki poza 
    ustalającą dotyczą \textbf{drugiego z interesujących kolorów}.
    \item Należy uważać na pasowanie na 3\nt jako otwierający! W wielu sekwencjach takie 3\nt zawiera już 
    inwit do gry kolorowej (bo zostało zalicytowane przez Gadżet).
\end{itemize}

Nieoczywiste zalety:
\begin{itemize}
    \item W sekwencjach typu 1\clubs - 1\spades --- 2\nt - 3\clubs --- 3\spades można
    wsadzić +1 \lsf\ i wszystko wiadomo o ręce. 
    \item Dzięki temu można rozdzielić splintery na bezpośrednie (GF bo splinter) i przez 2NT (GF + splinter)
    czy tam na odwrót
\end{itemize}

\pagebreak
\subsection*{Rebid 2\nt po otwarciu 1\clubs i odwrócony gadżet}

\sequence{{1\clubs}{1\hearts}{2\ntx}}
\begin{options}[2]
    \item[3\clubs] \textbf{Sztuczne pytanie} o trefle lub fit \hearts \imp
    \item[3\diams] 4+\diams, zachęta do gry w kolor
    \item[3\hearts] 6+\hearts \orr 5\hearts4\spades (nie ustala kierów!)  \vimp
    \item[3\spades] 4\hearts, 4\spades (Ustalenie niższy/wyższy)
    \item[4\clubs] Auto-splinter na 6\hearts
    \item[4\diams] Auto-splinter na 6\hearts
\end{options}

\sequence{{1\clubs}{1\spades}{2\ntx}}
\begin{options}[2]
    \item[3\clubs] \textbf{Sztuczne pytanie} o trefle lub fit \spades \imp
    \item[3\diams] 4+\diams, zachęta do gry w kolor
    \item[3\hearts] 4\hearts 5\spades
    \item[3\spades] 6+\spades (ustala)
    \item[4\clubs] Auto-splinter na 6\spades
    \item[4\diams] Auto-splinter na 6\spades
\end{options}

\subsubsection*{Pytanie 3\clubs}

\sequence{{1\clubs}{1\hearts}{2\ntx}{3\clubs}}
\begin{options}[1]
    \item[3\diams] 3\hearts
    \item[3\hearts] 4\hearts
    \item[3\spades] 4+\clubs i 2\hearts \vimp
    \item[3\nt] Nic z powyższych
\end{options}

\sequence{{1\clubs}{1\spades}{2\ntx}{3\clubs}}
\begin{options}[1]
    \item[3\diams] 3\spades
    \item[3\hearts] 4+\clubs i 2\spades \vimp
    \item[3\spades] 4\spades
    \item[3\nt] Nic z powyższych
\end{options}

\sequence{{1\clubs}{1\hearts}{2\ntx}{3\clubs}{3\diams}}
\begin{options}[2]
    \item[3\hearts] Ustalenie koloru
    \item[3\spades] Jeśli masz 4+\clubs, daj 4\clubs
\end{options}

\sequence{{1\clubs}{1\spades}{2\ntx}{3\clubs}{3\diams}}
\begin{options}[2]
    \item[3\hearts] Jeśli masz 4+\clubs, daj 4\clubs
    \item[3\spades] Ustalenie koloru
\end{options}

\subsubsection*{Pytanie 3\diams}
Jako otwierający, jeśli mamy 3\hearts i 4\diams, pokazujemy 3 kiery a następnie wynosimy 3\nt partnera w 4\diams!
Użycie 3\diams zaprasza do przekroczenia 3\nt.

\sequence{{1\clubs}{1\hearts}{2\ntx}{3\diams}}
\begin{options}[1]
    \item[3\hearts] 4\hearts
    \item[3\spades] 3\hearts (4\clubs ustala, 4\diams jest NAT) \vimp
    \item[3\nt] Nic ciekawego
    \item[4\diams] 4+\diams i dubel \hearts 
\end{options}

\sequence{{1\clubs}{1\spades}{2\ntx}{3\diams}}
\begin{options}[1]
    \item[3\hearts] 3\spades
    \item[3\spades] 4\spades
    \item[3\nt] Nic ciekawego
    \item[4\diams] 4+\diams i dubel \spades 
\end{options}


\subsubsection*{1\hearts \then 3\hearts --- 6 kierów (slam try) lub 5\hearts4\spades (może byc słabe)}

\sequence{{1\clubs}{1\hearts}{2\ntx}{3\hearts}}
\begin{options}[1]
    \item[3\spades] 4\spades
    \item[3\nt] Brak fitu w obu starych
    \item[4\clubs+] 3+\hearts, cue-bid na \hearts
\end{options}

\sequence{{1\clubs}{1\hearts}{2\ntx}{3\hearts}{3\spades}}
\begin{options}[2]
    \item[3\nt] 6\hearts, brak 4\spades \imp
    \item[4\clubs+] Cue na \spades
\end{options}

\sequence{{1\clubs}{1\hearts}{2\ntx}{3\hearts}{3\spades}{3\ntx}}
\begin{options}[1]
    \item[4\clubs+] Cue na \hearts
\end{options}

\sequence{{1\clubs}{1\hearts}{2\ntx}{3\hearts}{3\ntx}}
\begin{options}[2]
    \item[4\clubs+] Cue na 6+\hearts
\end{options}

\subsubsection*{Inne sekwencje}

\sequence{{1\clubs}{1\hearts}{2\ntx}{3\spades}}
\begin{options}[1]
    \item[3\nt] Brak fitu w obu starych
    \item[4\clubs] Nadzwyczaj silna karta z fitem \hearts
    \item[4\diams] Nadzwyczaj silna karta z fitem \spades 
    \item[4\hearts] Słabe
    \item[4\spades] Słabe  
\end{options}

\sequence{{1\clubs}{1\spades}{2\ntx}{3\hearts}}
\begin{options}[1]
    \item[3\nt] Brak fitu w obu starych
    \item[4\clubs] Nadzwyczaj silna karta z fitem \hearts
    \item[4\diams] Nadzwyczaj silna karta z fitem \spades 
    \item[4\hearts] Słabe
    \item[4\spades] Słabe  
\end{options}

\sequence{{1\clubs}{1\spades}{2\ntx}{3\spades}}
\vspace*{-0.7cm}Piki są ustalone.









\pagebreak
\subsection*{Rebid 2\nt po otwarciu 1\diams i odwrócony gadżet}
\begin{itemize}
    \item 1\diams \then 2\nt może zawierać rękę UnBAL z długimi \diams
    \item 1\diams \then 2\nt \textbf{nie może} zawierać drugiej starszej czwóki \vimp
\end{itemize}
\vspace{1em}

\sequence{{1\diams}{1\hearts}{2\ntx}}
\begin{options}[2]
    \item[3\clubs] \textbf{Sztuczne pytanie} o trefle lub fit \hearts \imp
    \item[3\diams] 3+\diams, zachęta do gry w kolor
    \item[3\hearts] 6+\hearts (nie ustala kierów! OTW może mieć krótkość)  \vimp
    \item[3\spades] \st{4H 4S}
    \item[4\clubs] Auto-splinter na 7\hearts
    \item[4\diams] Auto-splinter na 7\hearts
\end{options}

\sequence{{1\diams}{1\spades}{2\ntx}}
\begin{options}[2]
    \item[3\clubs] \textbf{Sztuczne pytanie} o trefle lub fit \spades \imp
    \item[3\diams] 3+\diams, zachęta do gry w kolor
    \item[3\hearts] \st{4H 5S}
    \item[3\spades] 6+\spades (nie ustala pików! OTW może mieć krótkość) 
    \item[4\clubs] Auto-splinter na 7\spades
    \item[4\diams] Auto-splinter na 7\spades
\end{options}

\subsubsection*{Pytanie 3\clubs}

\sequence{{1\diams}{1\hearts}{2\ntx}{3\clubs}}
\begin{options}[1]
    \item[3\diams] 3\hearts
    \item[3\hearts] 4\hearts
    \item[3\spades] UnBAL z 6+\diams i max 2\hearts \vimp
    \item[3\nt] Nic z powyższych
\end{options}

\sequence{{1\diams}{1\spades}{2\ntx}{3\clubs}}
\begin{options}[1]
    \item[3\diams] 3\spades
    \item[3\hearts] UnBAL z 6+\diams i max 2\hearts \vimp
    \item[3\spades] 4\spades
    \item[3\nt] Nic z powyższych
\end{options}

\sequence{{1\diams}{1\hearts}{2\ntx}{3\clubs}{3\diams}}
\begin{options}[2]
    \item[3\hearts] Ustalenie koloru
    \item[3\spades] Jeśli masz 3\clubs, daj 4\clubs
\end{options}

\sequence{{1\diams}{1\spades}{2\ntx}{3\clubs}{3\diams}}
\begin{options}[2]
    \item[3\hearts] Jeśli masz 3\clubs, daj 4\clubs
    \item[3\spades] Ustalenie koloru
\end{options}

\subsubsection*{Pytanie 3\diams}
Jako otwierający, jeśli mamy 3\hearts i 6+\diams, pokazujemy 3 kiery a następnie wynosimy 3\nt partnera w 4\diams!
Użycie 3\diams zaprasza do przekroczenia 3\nt.

\sequence{{1\diams}{1\hearts}{2\ntx}{3\diams}}
\begin{options}[1]
    \item[3\hearts] 4\hearts
    \item[3\spades] 3\hearts (4\clubs ustala, 4\diams jest NAT) \vimp
    \item[3\nt] Nic ciekawego
    \item[4\diams] 6+\diams i max dubel \hearts 
\end{options}

\sequence{{1\diams}{1\spades}{2\ntx}{3\diams}}
\begin{options}[1]
    \item[3\hearts] 3\spades
    \item[3\spades] 4\spades
    \item[3\nt] Nic ciekawego
    \item[4\diams] 6+\diams i max dubel \spades 
\end{options}


\subsubsection*{1\major \then 3\major --- 6M (może byc słabe)}
\sequence{{1\diams}{1\hearts}{2\ntx}{3\hearts}}
\begin{options}[1]
    \item[3\spades] cue na \hearts
    \item[3\nt] UnBAL z 6+\diams i max singlem \hearts
    \item[4\clubs+] cue na \hearts
\end{options}

\sequence{{1\diams}{1\spades}{2\ntx}{3\spades}}
\begin{options}[1]
    \item[3\nt] UnBAL z 6+\diams i max singlem \spades
    \item[4\clubs+] cue na \spades
\end{options}

\end{document}