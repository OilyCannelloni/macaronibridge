\documentclass[12pt, a4paper]{article}
\usepackage{../lib/bridgetex2}
\usepackage{polyglossia}
\setmainlanguage{polish}

\title{1NT Nieforsujące - Advanced}
\author{Bartek Słupik}

\begin{document}
    \maketitle
    \section{Motywacja}
    Ta stosunkowo prosta wersja systemu korzysta z kilku twardych założeń,
    może trochę zbyt optymistycznych. Ale dzięki temu jest prosta i 
    co najważniejsze, \textbf{przyjemna w grze} i \textbf{naturalna}

    \subsection{Założenia}
    Tak na prawdę jedynym założeniem jest to, że
    sekwencje 1\major\ - 1\nt\ --- 2\minor zawsze pokazują kolor z czwórki. 
    To \textbf{upraszcza} system dużo bardziej, niż mogłoby się wydawać!

    \section{Wykorzystywane konwencje}
    \subsection{Courtesy Raise}
    Rebidy otwierającego 2\clubs, 2\diams\ i 2\hearts\ są w bardzo szerokim przedziale siły.
    Stąd wprowadzamy rozróżnienie:
    \begin{itemize}
        \item Bezpośrednie podniesienie rebidowanego koloru jest 7-10 z fitem \emph{(Courtesy Raise)}
        \item Zalicytowanie go inaczej (o czym za chwilę) jest 11-12
    \end{itemize}
    Musimy podnosić z 7\hcp, bo partner może mieć 18!

    \subsection{Transfer Bart'a}
    Rebid odpowiadającego o stopień niżej od koloru otwarcia jest transferem na kolor otwarcia.
    Może zawierać:
    \begin{itemize}
        \item Słabą rękę do pasa - pasujemy transfer
        \item Dowolny inwit - wynosimy transfer
    \end{itemize}
    Konwencja zachodzi tylko, \textbf{jeśli mamy miejsce} na wykonanie transferu!

    \subsection{Niemożliwe 2\spades}
    W licytacji 1\hearts\ --- 1\nt\ --- 2\minor\ --- 2\spades\ ostatnia odzywka nie może być naturalna.
    Wykorzystuje się ją do pokazania karty, która była za słaba na bezpośredni \gf, jednak
    po rebidzie partnera szanse na końcówkę wzrosły.
    Obiecuje 11-12\hcp\ z 4-kartowym fitem w kolorze rebidu.

    \pagebreak
    \section{System}
    \subsection{1\hearts\ --- 1\ntx \\ ?}
    \begin{itemize}
        \item \pass\ - 5332, 11-14
        \item 2\clubs\ - 11-18, 4+\clubs
        \item 2\diams\ - 11-18, 4+\diams 
        \item 2\hearts\ - 11-15, 6+\hearts 
        \item 2\spades\ - \emph{Rewers.} 16+, 4\spades, \fonce 
        \item 2\nt\ - Dowolny GF nie zawarty w innych odzywkach \gf 
    \end{itemize}

    \subsubsection{1\hearts\ --- 1\ntx \\ 2\clubs\ --- ?}
    \begin{itemize}
        \item 2\diams\ - \emph{Transfer Bart'a} na 2\hearts.
        Może zawierać słabą rękę z kierami lub dowolny inwit.
        \subitem \pass\ - do gry, słaba preferencja
        \subitem 2\spades\ - 
        \subitem 2\nt\ - mocny równy inwit
        \subitem 3\clubs\ - 10-11 z fitem \clubs
        \subitem 3\diams\ - 10-11 z własnym kolorem 6+
        \item 2\hearts\ \textbf{9-11} z dwukartowym fitem
        \item 2\spades\ \emph{Niemożliwe 2\spades.} Prawie GFowa karta z 4+\clubs:
        \item 2\nt\ - Równy inwit
        \item 3\clubs\ - \emph{Courtesy Raise} 7-9, 5+\clubs
    \end{itemize}

    \subsubsection{1\hearts\ --- 1\ntx \\ 2\diams\ --- ?}
    \begin{itemize}
        \item 2\hearts\ \soff
        \item 2\spades\ \emph{Niemożliwe 2\spades.} Prawie GFowa karta z 4+\diams, pytanie o skład:
        \subitem 2\nt\ - 5332 13-14
        \subitem 3\diams\ - 5-4, 11-12
        \subitem Inne - licytowane trzymanie, 5-4, maximum \gf
        \item 2\nt\ - Inwit na równym
        \item 3\clubs\ - \soff
        \item 3\diams\ - \emph{Courtesy Raise} 6-8, 5+\diams
        \item 3\hearts\ - Inwit z 6+\hearts
    \end{itemize}
    

    \subsection{1\spades\ --- 1\ntx \\ ?}
    \begin{itemize}
        \item \pass\ - 5332, 11-12
        \item 2\clubs\ - 13+ 5332 lub 11+ z 4\clubs
        \item 2\diams\ - 13+ 5332 lub 11+ z 4\diams 
        \item 2\hearts\ - 11-18, 4+\hearts 
        \item 2\spades\ - 11-15, 6+\spades
        \item 2\nt\ - Dowolny GF nie zawarty w innych odzywkach \gf 
    \end{itemize}

    \subsubsection{1\spades\ --- 1\ntx \\ 2\clubs\ --- ?}
    \begin{itemize}
        \item 2\diams\ - \emph{Transfer Bart'a} na 2\hearts.
        Inwit lub do pasa z 6+\hearts
        \item 2\hearts\ - \emph{Transfer Bart'a} na 2\spades.
        Może zawierać słabą rękę z pikami lub dowolny inwit.
        \item 2\spades\ \textbf{9-11} z dwukartowym fitem
        \item 2\nt\ - Transfer na \diams
        \item 3\clubs\ - \emph{Courtesy Raise} 7-9, 5+\clubs
    \end{itemize}

    \subsubsection{1\spades\ --- 1\ntx \\ 2\diams\ --- ?}
    \begin{itemize}
        \item 2\hearts\ - \emph{Transfer Bart'a} na 2\spades.
        Może zawierać słabą rękę z pikami lub dowolny inwit.
        \item 2\spades\ \textbf{9-11} z dwukartowym fitem
        \item 2\nt\ - Transfer na \clubs
        \item 3\diams\ - \emph{Courtesy Raise} 7-9, 5+\diams
    \end{itemize}

    \subsubsection{1\spades\ --- 1\ntx \\ 2\hearts\ --- ?}
    \begin{itemize}
        \item 2\spades\ \soff\ z \spades
        \item 2\nt\ - Bilansowy inwit do 3\nt
        \item 3\clubs\ - Kolor + fit, 11-12
        \item 3\diams\ - Kolor + fit, 11-12
        \item 3\hearts\ \emph{Courtesy Raise}, 8-10, 4+\hearts
    \end{itemize}


\end{document}