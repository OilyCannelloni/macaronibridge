
\documentclass[12pt, a4paper]{article}
\usepackage{import}

\import{../../lib/}{bridge.sty}

\title{Dlaczego przegraliśmy na Sośnika?}
\author{Bartek Słupik}

\begin{document}
\maketitle

Poniżej prezentuję analizę wybranych rozdań z mojego stołu z pierwszego dnia ligi.
W sumie \textbf{tylko na naszym stole} wypuszczone zostały \textbf{144 impy}, któych wypuszczenia dało się uniknąć. I to w przeciągu 60 rozdań!
Dawno nie zagrałem tak słabego turnieju, za co przepraszam. Ale może dzięki tej analizie się czegoś nauczycie.

Impy liczę względem wyniku na drugim stole, którego pochodzenie mnie nie interesuje.
Oznacza to, że przegrana chodząca końcówka, kiedy na drugim stole nasi postawili 1100, jest warta tylko 
4 impy.
    
\pagebreak
\section*{Rozdanie 5}
\handdiagramv{\vhand{AT95}{KJ63}{QJT}{AJ}}
{\vhand{KQJ72}{Q542}{A2}{Q2}}
{\vhand{}{T987}{K943}{KT764}}
{\vhand{8643}{A}{8765}{9853}}
{NS}

\begin{table}[h!]
    \centering
    \begin{tabular}{cccc}
        \vul{W} & \nvul{N} & \vul{E} & \nvul{S}\\
        - & 1\nt & 2\clubs & \pass\\
        2\spades & \pass & 3\spades & \pass \\
        \pass & \pass \\
    \end{tabular}
\end{table}

[BS]
Rozważałem wist atutowy, który na takie sekwencje jest zwykle dobry,
jednak konfiguracja z 10 i 9 mnie do niego zniechęciła. Niestety karowy wypuszcza lewę.
Po dojściu Asem trefl miałem jeszcze szansę na poprawę błędu, jednak brnąłem w zaparte, że wolę piki zostawić
do rozgrywania przeciwnikowi.

Postawione IMPy: Świrowanie: 0 | Umiejętności: 7 (+6) | Pech: 0



\pagebreak
\section*{Rozdanie 6}
\handdiagramv{\vhand{T5}{T65}{6}{AQJT876}}
{\vhand{Q93}{AK3}{K9842}{52}}
{\vhand{AKJ}{9874}{A3}{K943}}
{\vhand{87642}{QJ2}{QJT75}{}}
{EW}

\begin{table}[h!]
    \centering
    \begin{tabular}{cccc}
        \vul{W} & \nvul{N} & \vul{E} & \nvul{S}\\
		  -  &  -  & 1\diams & \dbl \\
        1\spades & 5\clubs & \pass & \pass \\
        5\diams & \pass & \pass & 6\clubs \\
        \dbl & \pass & \pass & \pass \\
    \end{tabular}
\end{table}

[BS]
Partner myślał, że pas na 5\clubs jest forsujący i z asową kartą zwęszył okazję do wzięcia 920.
Niestety moja odzywka 5\clubs była ukierunkowana obronnie. Należało walnąć 5\diams i zawistować
\textbf{Królem \spades} w celu uzyskania ilościówki. Bez dwóch!

Postawione IMPy: Świrowanie: 15 (+15) | Umiejętności: 7 | Pech: 0

Na drugim stole też jakaś katastrofa.




\pagebreak
\section*{Rozdanie 9}
\handdiagramv{\vhand{QT8}{}{AKQ64}{AK953}}
{\vhand{AK654}{KT9865}{J5}{}}
{\vhand{J2}{QJ743}{T87}{QT4}}
{\vhand{973}{A2}{932}{J8762}}
{EW}

\begin{table}[h!]
    \centering
    \begin{tabular}{cccc}
        \vul{W} & \nvul{N} & \vul{E} & \nvul{S}\\
		  -  & 1\diams & 1\hearts & \pass \\
        \pass & 5\clubs & \pass & 5\diams \\
        \pass & \pass & \pass

    \end{tabular}
\end{table}

[BS]
Kolejny przykład, że mini-maxi stare to niewypał. Tutaj na szczęście dla przeciwników - bez konsekwencji.
Wist \xspades AK, \xhearts 8 - przebita. Ściągnąłem 2 atuty, spadł walet.

Teraz zagrałem jak idiota - dociągnąłem atuta i zargałem Asa trefl, planując ogarnąć podział 4-1.
Czy dało się zabezpieczyć przed 5-0? Oczywiście!

Należy przed dociągnięciem atuta zagrać trefla do Damy, co daje zachowanie komunikacji w takim przypadku.
Rozgrywka level kolba. Przepraszam. No chyba że Wzorek jest na tyle cwany, żeby z J95 dołożyć w drugiej lewie waleta\dots

Postawione IMPy: Świrowanie: 15 | Umiejętności: 19 (+12) | Pech: 0



\pagebreak
\section*{Rozdanie 11}
\handdiagramv{\vhand{874}{Q}{AT3}{K86542}}
{\vhand{T952}{6542}{K}{AQJ7}}
{\vhand{AQJ}{T973}{9765}{93}}
{\vhand{K63}{AKJ8}{QJ842}{T}}
{}

\begin{table}[h!]
    \centering
    \begin{tabular}{cccc}
        \nvul{W} & \nvul{N} & \nvul{E} & \nvul{S}\\
		  -  &  -  &  -  & \pass \\
        1\diams & 3\clubs & \dbl & \pass \\
        3\hearts & \pass & \pass & \pass \\

    \end{tabular}
\end{table}

Agresywny blok wybił przeciwnika z systemu - kontra na wejścia na poziomie 3 powinna forsować do końcówki!
Ale mógł też spasować i wziąć 800.


\pagebreak
\section*{Rozdanie 14}
\handdiagramv{\vhand{KT974}{8}{AKQT92}{6}}
{\vhand{QJ52}{J65}{875}{KQJ}}
{\vhand{A3}{T732}{J64}{8432}}
{\vhand{86}{AKQ94}{3}{AT975}}
{}

\begin{table}[h!]
    \centering
    \begin{tabular}{cccc}
        \nvul{W} & \nvul{N} & \nvul{E} & \nvul{S}\\
		  -  &  -  & \pass & \pass \\
        1\hearts & 2\hearts & 3\hearts & \pass \\
        4\hearts & 5\diams & \dbl & \pass \\
        \pass & \pass \\

    \end{tabular}
\end{table}

[BS]
Mogę się cieszyć, że rodzina nie wyrzuciła mnie za to z domu. Przepraszam.

Postawione IMPy: Świrowanie: 30 (+15) | Umiejętności: 19 | Pech: 0


\pagebreak
\section*{Rozdanie 21}
\handdiagramv{\vhand{Q953}{A9}{Q964}{T53}}
{\vhand{AT86}{4}{AJ8}{AQJ42}}
{\vhand{KJ742}{KQ765}{5}{K8}}
{\vhand{}{JT832}{KT732}{976}}
{}

\begin{table}[h!]
    \centering
    \begin{tabular}{cccc}
        \nvul{W} & \nvul{N} & \nvul{E} & \nvul{S}\\
		  -  & \pass & 1\clubs & 1\spades \\
        \dbl & 2\spades & \dbl\alrts & \pass \\
        3\diams & \pass & 3\spades\alrts & \pass \\
        4\diams & \pass & 5\diams \\

    \end{tabular}
\end{table}

[BS]
Licytacja może nieidealna, ale jako jedyni na sali znaleźliśmy fit karowy i zagraliśmy 
całkiem niezłą końcówkę, która wymaga kar 3-2 i trafienia trzeciej damy karo lub impasu trefl. 
Na słabą opozycję - kontrakt marzenie. Niestety kara się nie podzieliły.

Postawione IMPy: Świrowanie: 30 | Umiejętności: 19 | Pech: 12 (+12)



\pagebreak
\section*{Rozdanie 23}
\handdiagramv{\vhand{KJ4}{2}{J987653}{85}}
{\vhand{Q9}{QT74}{K4}{AJT32}}
{\vhand{T85}{K853}{QT2}{KQ7}}
{\vhand{A7632}{AJ96}{A}{964}}
{EW}

\begin{table}[h!]
    \centering
    \begin{tabular}{cccc}
        \vul{W} & \nvul{N} & \vul{E} & \nvul{S}\\
		  -  &  -  &  -  & \pass \\
        1\spades & 3\diams & \dbl & 4\diams \\
        4\hearts & \pass & \pass & \pass \\
        
    \end{tabular}
\end{table}

[BS]
Nie wiem, co tu się wydarzyło na rozgrywce, ale ja bym zagrał tak:
Po zabiciu wistu karowego Asem gram trefla do dziesiątki. Jeśli S weźmie, nie może wyjść w pika (gram przez piki).
Gra zatem karo, które biję, wyrzucając trefla. Teraz As trefl i trefl przebity dziewiątką.
Następnie \xhearts A, \xhearts J do Króla. S nie może zagrać w karo więc oddam jeszcze pika.

Dlaczego taka linia? Trefle się na 99\% dzielą, bo N zawistowałby w singla (no chyba że ma 7-4)
Przegrywamy tylko, jeśli trefle są Fxxx - F, a \xhearts K nie stoi.

Postawione IMPy: Świrowanie: 30 | Umiejętności: 31 (+12) | Pech: 12


\pagebreak
\section*{Rozdanie 28}
\handdiagramv{\vhand{J632}{Q8763}{K6}{K2}}
{\vhand{KT}{542}{QJT3}{AT76}}
{\vhand{94}{AKJT9}{A987}{94}}
{\vhand{AQ875}{}{542}{QJ853}}
{}

\begin{table}[h!]
    \centering
    \begin{tabular}{cccc}
        \nvul{W} & \nvul{N} & \nvul{E} & \nvul{S}\\
		\pass & \pass & 1\diams & 1\hearts \\
        1\spades & 2\hearts & \pass & \pass \\
        3\diams & 3\hearts & \pass & \pass \\
        \pass \\

    \end{tabular}
\end{table}

Przeciwnik nie zagrał dobrej, acz niechodzącej końcówki.

Postawione IMPy: Świrowanie: 30 | Umiejętności: 31 | Pech: 23 (+11)

\pagebreak
\section*{Rozdanie 32}
\handdiagramv{\vhand{KJ7}{J43}{KQ742}{J3}}
{\vhand{QT986}{T96}{A}{Q865}}
{\vhand{A3}{KQ52}{T93}{AK74}}
{\vhand{542}{A87}{J865}{T92}}
{NS}

\begin{table}[h!]
    \centering
    \begin{tabular}{cccc}
        \nvul{W} & \vul{N} & \nvul{E} & \vul{S}\\
		\pass & \pass & 2\diams\alrts & \dbl \\
        4\hearts & \pass & 4\spades & \pass \\
        \pass & \pass \\
    \end{tabular}
\end{table}

[BS]
Uwaga uwaga, kontra atakująco-obronna! A raczej jej brak u przeciwnika, który dał mi zagrać 4\spades 
bez kontry.


\pagebreak
\section*{Rozdanie 34}
\handdiagramv{\vhand{T52}{K3}{KT42}{J876}}
{\vhand{A973}{9}{Q985}{AKQ5}}
{\vhand{J}{AQT654}{73}{T932}}
{\vhand{KQ864}{J872}{AJ6}{4}}
{}

\begin{table}[h!]
    \centering
    \begin{tabular}{cccc}
        \nvul{W} & \nvul{N} & \nvul{E} & \nvul{S}\\
		  -  &  -  & 1\diams & 2\hearts \\
        3\diams\alrts & \pass & 4\hearts & \pass \\
        4\nt & \pass & 5\hearts & \pass \\
        6\spades & \pass & \pass & \pass \\
    \end{tabular}
\end{table}

[BS] 
Tak właśnie się kończy robienie wykładów - Adrian słuchał na ostatniej Stasikówce i na licytację ze 
splinterem wyciągnął jedyny obkładający wist w atu! Impas karo nie stał i szlemik się wywalił.
Po drodze przeciwnik, który nie słuchał na stasikówce wypuścił, nie dając partnerowi połączyć atu po raz drugi,
przejmując Króla kier Asem. Niestety mimo to wygrać było i tak bardzo ciężko.
 
Postawione IMPy: Świrowanie: 30 | Umiejętności: 31 | Pech: 44 (+21)


\pagebreak
\section*{Rozdanie 38}
\handdiagramv{\vhand{A6}{A86}{QT92}{AQ84}}
{\vhand{K8}{J73}{KJ764}{T63}}
{\vhand{732}{Q4}{A853}{KJ72}}
{\vhand{QJT954}{KT952}{}{95}}
{NS}

\begin{table}[h!]
    \centering
    \begin{tabular}{cccc}
        \nvul{W} & \vul{N} & \nvul{E} & \vul{S}\\
		  -  &  -  & \pass & \pass \\
        2\spades\alrts & \pass & \pass & \pass \\
    \end{tabular}
\end{table}

[BS]
Otwarcie 2\spades = 9-12 6+\spades całkowicie zmiotło przeciwnika z planszy.

Postawione IMPy: Świrowanie: 30 | Umiejętności: 35 (+4) | Pech: 44


\pagebreak
\section*{Rozdanie 42}
\handdiagramv{\vhand{AKQ94}{T6}{32}{AT52}}
{\vhand{JT2}{KQJ9}{AT9}{973}}
{\vhand{}{A842}{Q8754}{KQJ6}}
{\vhand{87653}{753}{KJ6}{84}}
{NSEW}

\begin{table}[h!]
    \centering
    \begin{tabular}{cccc}
        \vul{W} & \vul{N} & \vul{E} & \vul{S}\\
		  -  &  -  & \pass & 1\diams \\
        \pass & 1\spades & \pass & 2\clubs \\
        \pass & 2\hearts & \pass & 2\nt \\
        \pass & 3\spades & \pass & 3\nt \\
        \pass & \pass & \pass \\
    \end{tabular}
\end{table}

[BS]
Niestety bardzo źle to policytowałem. Z perspektywy N bardzo nie chcę grać 3\nt,
więc postanowiłem poszukać końcówki pikowej odzywką 3\spades. Należało zacząć od 3\clubs,
po którym jeszcze dowiem się, czy partnerka ma dubla pik, a zachęcę do gry w trefle w innym przypadku.

5\clubs jest czapowe, 3\nt chodzi na farcie, bo spada \xspades JTx, co w rozgrywce zostało przeoczone.


\pagebreak
\section*{Rozdanie 44}
\handdiagramv{\vhand{AKQ5}{KQJT}{KJ832}{}}
{\vhand{J}{876}{75}{KJ98742}}
{\vhand{T42}{A94}{T964}{AT6}}
{\vhand{98763}{532}{AQ}{Q53}}
{NSEW}

\begin{table}[h!]
    \centering
    \begin{tabular}{cccc}
        \vul{W} & \vul{N} & \vul{E} & \vul{S}\\
		\pass & 1\diams & 3\clubs & \pass \\
        4\clubs & \dbl & \pass & 5\diams \\
        \pass & 6\diams & \pass & \pass \\
        \pass \\

    \end{tabular}
\end{table}

[BS]
Nie jsetem fanem pasa na 3\clubs, który bardzo ułatwiłby mi życie. Nie jestem też fanem rozkładu,
który miał miejsce w rozdaniu. Kolejny dobry szlemik przegrany.

Postawione IMPy: Świrowanie: 30 | Umiejętności: 35 | Pech: 69 (+25)







\end{document}        
