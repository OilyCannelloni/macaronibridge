
\documentclass[12pt, a4paper]{article}
\usepackage{import}
\usepackage{hyperref}
\hypersetup{
	colorlinks=true,
	urlcolor=blue,    
}
\usepackage{emoji}
\usepackage{polyglossia}
\setmainlanguage{polish}



\import{../../lib/}{bridge.sty}

\title{KWB 2025 - Otwarte Mistrzostwa Małopolski Par Mikstowych}
\author{Bartek Słupik \\ using \href{https://github.com/OilyCannelloni/macaronibridge/tree/master/tc-results-parser}{TC Results Parser}}

\begin{document}
\maketitle

    
    
\pagebreak
\section*{Rozdanie 1}
\handdiagramv{\vhand{JT42}{A985}{T5}{Q64}}
{\vhand{AK}{KT74}{KQ43}{987}}
{\vhand{Q753}{Q3}{986}{KJ32}}
{\vhand{986}{J62}{AJ72}{AT5}}
{}

\begin{table}[h!]
    \centering
    \begin{tabular}{cccc}
        \nvul{W} & \nvul{N} & \nvul{E} & \nvul{S}\\
		  -  & \pass & 1\nt & \pass \\
		  3\nt & \pass & \pass & \pass
    \end{tabular}
\end{table}

Wist \xspades 3.

Mamy 7 lew z góry, potrzebujemy wziąć 2 kierowe. Trzeba trafić. Ale można się najpierw czegoś dowiedzieć.
Wziąłem i ściągnąłem 4 kara, do których N dołożył 2 kiery, a S trefla. Po takim rozwoju sytuacji widać, że N nie ma \xhearts Q.
Zagrałem zatem kiera do Króla i wziąłem. Tak się podnieciłem, że ograłem Marksa, że teraz nastąpił moment brydżowego zezwieręcenia, bo zapomniałęm, że wypadałoby odblokować kolor.
Dycha w koło. \emoji{skull}

Wynik: =, 70\%. 

Wypuszczone przez bycie idiotą: 30\%

Reputacja Marksa: -10.




\pagebreak
\section*{Rozdanie 2}
\handdiagramv{\vhand{952}{AQJT}{A64}{KJ5}}
{\vhand{743}{K8632}{K7}{942}}
{\vhand{8}{974}{QT853}{T763}}
{\vhand{AKQJT6}{5}{J92}{AQ8}}
{NS}

\begin{table}[h!]
    \centering
    \begin{tabular}{cccc}
        \nvul{W} & \vul{N} & \nvul{E} & \vul{S}\\
		  -  &  -  & \pass & \pass \\
		  1\spades & \dbl & 2\spades\alrt & 2\nt\alrt \\
		  3\spades & \pass & \pass & \pass
    \end{tabular}
\end{table}

3\spades !!! top forma, nie ma na końcówkę, w tych pudłach 2\spades od zera.

Wist nie wiem.

Rozgrywka też nie wiem, wystarczy wykonać impas karo.

Wynik: -1 (37\%)



\pagebreak
\section*{Rozdanie 3}
\handdiagramv{\vhand{J}{Q82}{Q97}{QJT763}}
{\vhand{9843}{AK763}{AKT8}{}}
{\vhand{T6}{J94}{J542}{AK95}}
{\vhand{AKQ752}{T5}{63}{842}}
{EW}

\begin{table}[h!]
    \centering
    \begin{tabular}{cccc}
        \vul{W} & \nvul{N} & \vul{E} & \nvul{S}\\
		  -  &  -  &  -  & \pass  \\
		  2\spades\alrt & \pass & 2\nt & \pass \\
		  3\spades\alrt & \pass & 4\clubs\alrt & \pass \\
		  4\spades & \pass &  5\clubs & \pass \\
		  5\spades & \pass & \pass & \pass 
    \end{tabular}
\end{table}

No ześwirowałem no. Zapomniałem, że 5\clubs od razu to exclusion (jebac tom konwencje), a potem już nie było jak. 
Niestety w drugim kółku 5\clubs jest do gry. 

Nie dołożyłem, bo po zjazdach spodziewałem się oddania pika, a 5\spades sugeruje brak czerwonej damy.
W sekwencjach silnie zlimitowanych (9-12) silne ustalenie 4\clubs wymusza cuebid. Przestraszyłem się zwrotu 1 lub nawet 2 pikowych
i połączenia atutów, po czym nie będzie co zrobić z 3 treflem. Ponadto kara też muszą jakiś tam współpracować.


Wynik: +2, 25\%. Postawione przez zapomnienie systemu: 45\%


\pagebreak
\section*{Rozdanie 4}
\handdiagramv{\vhand{75}{A962}{A862}{T94}}
{\vhand{T6}{J875}{KQ7}{J732}}
{\vhand{KQ9843}{K}{543}{A85}}
{\vhand{AJ2}{QT43}{JT9}{KQ6}}
{NSEW}

\begin{table}[h!]
    \centering
    \begin{tabular}{cccc}
        \vul{W} & \vul{N} & \vul{E} & \vul{S}\\
		1\clubs & \pass & 1\hearts & 1\spades \\
		2\hearts & \pass & \pass & 2\spades \\
		\pass & \pass & \pass 
    \end{tabular}
\end{table}

Wist \xdiams J do Asa (Jedyny kładący!), wyrzuciłem Króla. Kier do Króla i karo 2 razy, wzięte u E.

Teraz techniczny \xclubs J (nie robił) do Asa, 2 trefle oddane.

Wynik: -1 (75\%)




\pagebreak
\section*{Rozdanie 5}
\handdiagramv{\vhand{Q76}{72}{KT942}{963}}
{\vhand{T82}{A854}{Q3}{A752}}
{\vhand{AJ53}{T9}{J765}{KT4}}
{\vhand{K94}{KQJ63}{A8}{QJ8}}
{NS}

\begin{table}[h!]
    \centering
    \begin{tabular}{cccc}
        \nvul{W} & \vul{N} & \nvul{E} & \vul{S}\\
		  -  & \pass & \pass & \pass \\
		  1\nt & \pass & 2\clubs & \pass \\
		  2\hearts & \pass & 4\hearts & \pass \\
		  \pass & \pass 

    \end{tabular}
\end{table}

Wist nie wiem.

Dziękuję, że zostałem potraktowany lepiej niż bot na bbo.


\pagebreak
\section*{Rozdanie 6}
\handdiagramv{\vhand{984}{9754}{865}{764}}
{\vhand{AQJT}{}{T97}{AKJT32}}
{\vhand{75}{AKJ63}{AKQJ4}{5}}
{\vhand{K632}{QT82}{32}{Q98}}
{EW}

\begin{table}[h!]
    \centering
    \begin{tabular}{cccc}
        \vul{W} & \nvul{N} & \vul{E} & \nvul{S}\\
		  -  &  -  & 1\clubs & 1\hearts \\
		  1\spades & 3\hearts & 4\hearts & 5\diams \\
		  \dbl & 5\hearts & 6\spades & \pass \\
		  \pass & \pass
    \end{tabular}
\end{table}

Wydaje mi się, że \dbl na wistowe 5\diams nie sugeruje zamiaru skontrowania 5\hearts. Raczej bym powiedział, że właśnie coś w karach.
Ale ja nie umiem grać.





\pagebreak
\section*{Rozdanie 7}
\handdiagramv{\vhand{AQ6}{AT9732}{2}{T54}}
{\vhand{K85}{QJ6}{AJT986}{2}}
{\vhand{742}{K54}{753}{AK86}}
{\vhand{JT93}{8}{KQ4}{QJ973}}
{NSEW}

\begin{table}[h!]
    \centering
    \begin{tabular}{cccc}
        \vul{W} & \vul{N} & \vul{E} & \vul{S}\\
		  -  &  -  &  -  & \pass \\
		  \pass & 2\diams\alrt & \pass & 2\hearts \\
		  \pass & \pass & 3\diams & 3\hearts \\
		  \pass & \pass & \pass  

    \end{tabular}
\end{table}

Nuda.


\pagebreak
\section*{Rozdanie 8}
\handdiagramv{\vhand{QJ5}{Q43}{J973}{AT4}}
{\vhand{A98762}{JT}{Q}{J875}}
{\vhand{KT43}{A6}{A8652}{92}}
{\vhand{}{K98752}{KT4}{KQ63}}
{}



\begin{table}[h!]
    \centering
    \begin{tabular}{cccc}
        \nvul{W} & \nvul{N} & \nvul{E} & \nvul{S}\\
		2\hearts\alrt & \pass & 3\hearts & \pass \\
		\pass & \pass 

    \end{tabular}
\end{table}

Kierowy kładzie ale jest niewyciągalny. Nuda.



\pagebreak
\section*{Rozdanie 9}
\handdiagramv{\vhand{KQJT76}{6}{Q9}{Q987}}
{\vhand{A2}{QJT74}{T843}{T2}}
{\vhand{}{A53}{AKJ652}{AKJ5}}
{\vhand{98543}{K982}{7}{643}}
{EW}

\begin{table}[h!]
    \centering
    \begin{tabular}{cccc}
        \vul{W} & \nvul{N} & \vul{E} & \nvul{S}\\
		  -  & 2\spades & \pass & 3\nt \\
		  \pass & \pass & \pass 

    \end{tabular}
\end{table}
Jakby obrócić pudełko to by nie dość że nie było afery, to *legendarna* licytacja karo-pik-karo-pik-karo-pik do szlema w trefle. Odgadnięcie sensu zostawiam czytelnikowi.
\begin{table}[h!]
    \centering
    \begin{tabular}{cccc}
        \vul{W} & \nvul{N} & \vul{E} & \nvul{S}\\
		  -  & 2\spades\alrt & \pass & 2\nt\alrt \\
		  \pass & 3\clubs\alrt & \pass & 3\diams\alrt \\
		  \pass & 3\spades\alrt & \pass & 4\diams\alrt \\
		  \pass & 4\spades\alrt & \pass & 5\diams\alrt \\
		  \pass & 5\spades\alrt & \pass & 5\nt\alrt \\
		  \pass & 7\clubs\alrt & \pass & \pass \\
		  \pass
		  

    \end{tabular}
\end{table}


\pagebreak
\section*{Rozdanie 10}
\handdiagramv{\vhand{}{AJT9862}{Q853}{AQ}}
{\vhand{KJ64}{Q7}{A4}{76543}}
{\vhand{875}{54}{K762}{JT92}}
{\vhand{AQT932}{K3}{JT9}{K8}}
{NSEW}

\begin{table}[h!]
    \centering
    \begin{tabular}{cccc}
        \vul{W} & \vul{N} & \vul{E} & \vul{S}\\
		  -  &  -  & \pass & \pass \\
		  1\spades & 4\hearts & 4\spades & \pass \\
		  \pass & \pass
    \end{tabular}
\end{table}

Biedny przeciwnik dostaje kolejny opierdol jeszcze przed wistem. Kier, karo, 2 trefle dziękuję dobranoc. Ale 5\hearts chodzi.





\pagebreak
\section*{Rozdanie 11}
\handdiagramv{\vhand{A5}{KJ83}{J832}{A93}}
{\vhand{T976}{AQ94}{4}{8642}}
{\vhand{KQJ8}{75}{KT976}{KT}}
{\vhand{432}{T62}{AQ5}{QJ75}}
{}

\begin{table}[h!]
    \centering
    \begin{tabular}{cccc}
        \nvul{W} & \nvul{N} & \nvul{E} & \nvul{S}\\
		  -  &  -  &  -  & 1\diams \\
		  \pass & 1\hearts & \pass & 1\spades \\
		  \pass & 2\diams & \pass & 2\nt \\
		  \pass & 3\diams & \pass & 3\nt \\
		  \pass & \pass & \pass

    \end{tabular}
\end{table}

Aaa dziewiątka trefl się wyrobiła lol.



\pagebreak
\section*{Rozdanie 12}
\handdiagramv{\vhand{KJ8743}{85}{65}{QJ4}}
{\vhand{Q96}{KJ6}{732}{T732}}
{\vhand{T2}{AT942}{AQJT}{A6}}
{\vhand{A5}{Q73}{K984}{K985}}
{NS}

\begin{table}[h!]
    \centering
    \begin{tabular}{cccc}
        \nvul{W} & \vul{N} & \nvul{E} & \vul{S}\\


    \end{tabular}
\end{table}
SPALONE

\pagebreak
\section*{Rozdanie 13}
\handdiagramv{\vhand{QJ432}{843}{5}{AJ75}}
{\vhand{86}{KT62}{AT94}{Q64}}
{\vhand{AKT95}{Q}{J76}{9832}}
{\vhand{7}{AJ975}{KQ832}{KT}}
{NSEW}

\begin{table}[h!]
    \centering
    \begin{tabular}{cccc}
        \vul{W} & \vul{N} & \vul{E} & \vul{S}\\
		  -  & \pass & \pass & \pass (?) \\
		  1\hearts & 1\spades & 2\hearts & 3\spades \\
		  4\hearts & \pass & \pass & 4\spades \\
		  \pass\alrt & \pass & 5\hearts & \pass \\
		  \pass & \pass 
    \end{tabular}
\end{table}

Passive player został bezlitośnie ograny.


\pagebreak
\section*{Rozdanie 14}
\handdiagramv{\vhand{T954}{9}{Q87}{A9864}}
{\vhand{QJ}{AKQ762}{KT2}{Q5}}
{\vhand{K873}{85}{A963}{KT7}}
{\vhand{A62}{JT43}{J54}{J32}}
{}

\begin{table}[h!]
    \centering
    \begin{tabular}{cccc}
        \nvul{W} & \nvul{N} & \nvul{E} & \nvul{S}\\
		  -  &  -  & 1\hearts & \pass \\
		  2\hearts & \pass & 3\nt & \pass \\
		  \pass & \pass
    \end{tabular}
\end{table}

Nuda. 


\pagebreak
\section*{Rozdanie 15}
\handdiagramv{\vhand{J543}{Q853}{KT43}{4}}
{\vhand{A8}{K62}{AQJ5}{AKT8}}
{\vhand{Q96}{AJT4}{9876}{73}}
{\vhand{KT72}{97}{2}{QJ9652}}
{NS}

\begin{table}[h!]
    \centering
    \begin{tabular}{cccc}
        \nvul{W} & \vul{N} & \nvul{E} & \vul{S}\\
		  -  &  -  &  -  & \pass \\
		  \pass & \pass & 2\nt & \pass \\
		  3\clubs & \pass & 3\nt & \pass \\
		  5\clubs & \pass & \pass & \pass
    \end{tabular}
\end{table}

Wrongsided więc nie dołożyłem choć nie otworzywszy 22-23 ugryzłem się w język.

\pagebreak
\section*{Rozdanie 16}
\handdiagramv{\vhand{852}{J52}{QJ9743}{J}}
{\vhand{KQT9}{A6}{A2}{Q8754}}
{\vhand{J643}{KT9}{5}{AKT92}}
{\vhand{A7}{Q8743}{KT86}{63}}
{EW}

\begin{table}[h!]
    \centering
    \begin{tabular}{cccc}
        \vul{W} & \nvul{N} & \vul{E} & \nvul{S}\\
		\pass & \pass & 1\clubs & \dbl \\
		1\hearts & 2\diams & 2\spades & \pass \\
		2\nt & \pass & \pass & \pass
    \end{tabular}
\end{table}



\pagebreak
\section*{Rozdanie 17}
\handdiagramv{\vhand{J8}{QJ9763}{AK8}{84}}
{\vhand{A7}{T5}{76432}{A752}}
{\vhand{QT632}{82}{QT95}{J6}}
{\vhand{K954}{AK4}{J}{KQT93}}
{}

\begin{table}[h!]
    \centering
    \begin{tabular}{cccc}
        \nvul{W} & \nvul{N} & \nvul{E} & \nvul{S}\\
		  -  & 1\hearts & \pass & 1\spades \\
		  2\clubs & 2\hearts & 3\clubs & \pass \\
		  \pass & \pass
    \end{tabular}
\end{table}

Zrolowałem lewkę za 4\%.

\pagebreak
\section*{Rozdanie 18}
\handdiagramv{\vhand{82}{AKQJ43}{T3}{KJT}}
{\vhand{J6543}{76}{J972}{Q7}}
{\vhand{K9}{85}{AK8654}{A54}}
{\vhand{AQT7}{T92}{Q}{98632}}
{NS}

\begin{table}[h!]
    \centering
    \begin{tabular}{cccc}
        \nvul{W} & \vul{N} & \nvul{E} & \vul{S}\\
		  -  &  -  & & \\

    \end{tabular}
\end{table}

\pagebreak
\section*{Rozdanie 19}
\handdiagramv{\vhand{J65}{AK8}{AJ4}{KJ75}}
{\vhand{Q9}{Q764}{76}{AQ862}}
{\vhand{AK732}{J3}{QT852}{4}}
{\vhand{T84}{T952}{K93}{T93}}
{EW}

\begin{table}[h!]
    \centering
    \begin{tabular}{cccc}
        \vul{W} & \nvul{N} & \vul{E} & \nvul{S}\\
		  -  &  -  &  -  & 2\spades \\
		  \pass & 2\nt & \pass & 3\spades\alrt \\
		  \pass & 4\spades & \pass & \pass \\
		  \pass
		      \end{tabular}
\end{table}

panie sędzio żółta kartka za grę pasywną


\pagebreak
\section*{Rozdanie 20}
\handdiagramv{\vhand{AQJ62}{Q54}{94}{AJ7}}
{\vhand{T4}{K98632}{K753}{T}}
{\vhand{K5}{JT}{QT862}{9852}}
{\vhand{9873}{A7}{AJ}{KQ643}}
{NSEW}

\begin{table}[h!]
    \centering
    \begin{tabular}{cccc}
        \vul{W} & \vul{N} & \vul{E} & \vul{S}\\
		1\clubs & 1\nt & \pass & \pass \\
		\pass
    \end{tabular}
\end{table}

No 2\hearts się daje

\pagebreak
\section*{Rozdanie 21}
\handdiagramv{\vhand{Q86}{AK94}{QT6}{A96}}
{\vhand{KJ542}{J6}{KJ7}{T32}}
{\vhand{A9}{T853}{A83}{KJ85}}
{\vhand{T73}{Q72}{9542}{Q74}}
{NS}

\begin{table}[h!]
    \centering
    \begin{tabular}{cccc}
        \nvul{W} & \vul{N} & \nvul{E} & \vul{S}\\
		  -  & 1\nt & \pass & 2\clubs \\
		  \pass & 2\hearts & \pass & 4\hearts \\
		  \pass & \pass & \pass
    \end{tabular}
\end{table}

Wist \xhearts J.

Zdezorientowany zabiłem Asem i zagrałem 2 razy kiera z góry. W zdziwiony, że nie wyciąłem mu damy, wyszedł w trefla (podziękowałem w myśli).
Ściągnąłem 3 trefle, wyrzuciłem karo i zagrałem karo do Q i \xdiams K. E wyszła w pika i sclaimowałem 11 lew.

Ale zaraz zaraz, ten wpust nie działa! E może wyjść \textbf{waletem karo} przygważdżając moją singlową dziesiątkę!

Zgrywając ostatniego trefla ustawiłem swoją rękę w przymusie wy-pustkowym. A forta mogła spokojnie poczekać!

\pagebreak
\section*{Rozdanie 22}
\handdiagramv{\vhand{K76}{97}{A52}{KQT96}}
{\vhand{52}{Q8542}{QJT3}{A2}}
{\vhand{JT3}{AKJT63}{96}{73}}
{\vhand{AQ984}{}{K874}{J854}}
{EW}

\begin{table}[h!]
    \centering
    \begin{tabular}{cccc}
        \vul{W} & \nvul{N} & \vul{E} & \nvul{S}\\
		  -  &  -  & \pass & 2\hearts\alrt \\
		  2\spades & \dbl & \pass & \pass \\
		  \rdbl & \pass & \pass & \pass
    \end{tabular}
\end{table}

Dlaczego zawsze na tych gości się dzieje jakaś obora. Nie wiem czy mam na kontrę ale widząc, że mamy 21-24 punkty pewspektywa wzięcia 200 jest 
nadwyraz kusząca. Szczególnie, że wejście nie zostanie powtórzone przez salę. Uznałem, że kontra to 80\% na 100\% i 20\% na zero. +EV. Akurat co ciekawe 200 dawało tylko 73\% bo 
ludzie grali 3\spades\dbl


\pagebreak
\section*{Rozdanie 23}
\handdiagramv{\vhand{K765}{AJ5}{986532}{}}
{\vhand{T932}{QT83}{A74}{74}}
{\vhand{AQ4}{K2}{QJ}{AT9632}}
{\vhand{J8}{9764}{KT}{KQJ85}}
{NSEW}

\begin{table}[h!]
    \centering
    \begin{tabular}{cccc}
        \vul{W} & \vul{N} & \vul{E} & \vul{S}\\
		  -  &  -  &  -  & 1\nt \\
		  \pass & 3\diams & \pass & 3\nt \\
		  \pass & \pass & \pass

    \end{tabular}
\end{table}

3\diams jakieś sus.

\pagebreak
\section*{Rozdanie 24}
\handdiagramv{\vhand{Q5}{9643}{AKT}{AQ92}}
{\vhand{A73}{AT872}{85}{764}}
{\vhand{KT98}{}{QJ742}{KJ53}}
{\vhand{J642}{KQJ5}{963}{T8}}
{}

\begin{table}[h!]
    \centering
    \begin{tabular}{cccc}
        \nvul{W} & \nvul{N} & \nvul{E} & \nvul{S}\\
		1\nt & \pass & 2\clubs & \pass \\
		2\hearts & \pass & 3\diams & \pass \\
		4\clubs & \pass & 5\clubs & \pass \\
		5\diams & \pass & \pass & \pass 
    \end{tabular}
\end{table}

Rozjazd zakończony idiotycznym zrolowaniem dwóch lew w czapowym 5\diams.


\pagebreak
\section*{Rozdanie 25}
\handdiagramv{\vhand{AKQ}{T8542}{53}{QT5}}
{\vhand{J54}{Q96}{T9872}{K2}}
{\vhand{T97}{KJ3}{KQJ4}{963}}
{\vhand{8632}{A7}{A6}{AJ874}}
{EW}

\begin{table}[h!]
    \centering
    \begin{tabular}{cccc}
        \vul{W} & \nvul{N} & \vul{E} & \nvul{S}\\
		  -  & 1\hearts & \pass & 2\hearts \\
		  \dbl & \pass & 2\spades & \pass \\
		  \pass & \pass
    \end{tabular}
\end{table}
rozgrywka nienajlepsza, obrona fachowa. Bez dwóch.

\pagebreak
\section*{Rozdanie 26}
\handdiagramv{\vhand{82}{QJ}{Q8643}{AKQ2}}
{\vhand{K9743}{AT4}{9}{JT54}}
{\vhand{AT}{K752}{AJT72}{83}}
{\vhand{QJ65}{9863}{K5}{976}}
{NSEW}

\begin{table}[h!]
    \centering
    \begin{tabular}{cccc}
        \vul{W} & \vul{N} & \vul{E} & \vul{S}\\
		  -  &  -  & 1\clubs\alrt & \pass  \\
		  2\clubs\alrt & \pass & 2\diams\alrt & \pass \\
		  2\hearts\alrt & \pass & 3\clubs & \pass \\
		  3\diams & \pass & 4\diams & \pass \\
		  5\clubs & \pass & 5\diams & \pass \\
		  \pass & \pass
    \end{tabular}
\end{table}

Znowu rozjazd, ale przynajmniej nie zagraliśmy 3\nt.

\begin{itemize}
	\item 2\clubs = \gf bez starszej czwórki, może mieć dużo kar
	\item 2\diams = skład równy 
	\item 2\hearts = relay
	\item 3\clubs = 5\clubs
\end{itemize}

\pagebreak
\section*{Rozdanie 27}
\handdiagramv{\vhand{T63}{T8653}{JT}{AK6}}
{\vhand{K7542}{KJ2}{Q3}{QT3}}
{\vhand{Q8}{A74}{A652}{9854}}
{\vhand{AJ9}{Q9}{K9874}{J72}}
{}

\begin{table}[h!]
    \centering
    \begin{tabular}{cccc}
        \nvul{W} & \nvul{N} & \nvul{E} & \nvul{S}\\
		  -  &  -  &  -  & \pass \\
		 1\diams & 1\hearts & \dbl\alrt & 2\hearts \\
		 \dbl & \pass & 3\spades & \pass \\
		 \pass & \pass 

    \end{tabular}
\end{table}

Wist \xhearts A i zmiana na karo, wzięte w ręce. E zagrał \xclubs Q do Króla, odepchnąłem się kierem - wzięte w ręce. Teraz karo do Asa. Trefl 2 razy i... stół wpuszczony! Musi trafiać, a że z bilansu ja mam więcej to został zagrany impas. Noga.

\pagebreak
\section*{Rozdanie 28}
\handdiagramv{\vhand{AK32}{AQT5}{AJT8}{2}}
{\vhand{754}{K7}{6432}{QT83}}
{\vhand{J9}{J94}{KQ97}{J754}}
{\vhand{QT86}{8632}{5}{AK96}}
{NS}

\begin{table}[h!]
    \centering
    \begin{tabular}{cccc}
        \nvul{W} & \vul{N} & \nvul{E} & \vul{S}\\
		\pass & 1\diams & \pass & 3\clubs\alrt \\
		\pass & 3\diams & \pass & \pass \\
		\pass
    \end{tabular}
\end{table}

Instynkt z pokazaniem 0-6 nie zawiódł.

\pagebreak
\section*{Rozdanie 29}
\handdiagramv{\vhand{A7}{AKQJ}{986}{J732}}
{\vhand{QT85}{T875}{J53}{A8}}
{\vhand{K642}{42}{AQT}{K964}}
{\vhand{J93}{963}{K742}{QT5}}
{NSEW}

\begin{table}[h!]
    \centering
    \begin{tabular}{cccc}
        \vul{W} & \vul{N} & \vul{E} & \vul{S}\\
		  -  & & & \\

    \end{tabular}
\end{table}

szkoda gadać skompromitowałem się, czas na kolbe. A taki ładny wpust by był, gdybym policzył ile już oddałem.


\pagebreak
\section*{Rozdanie 30}
\handdiagramv{\vhand{QJ86}{AQ8}{QT82}{J7}}
{\vhand{K732}{T932}{K96}{63}}
{\vhand{5}{KJ74}{743}{AKT94}}
{\vhand{AT94}{65}{AJ5}{Q852}}
{}

\begin{table}[h!]
    \centering
    \begin{tabular}{cccc}
        \nvul{W} & \nvul{N} & \nvul{E} & \nvul{S}\\
		  -  &  -  & \pass  & 1\clubs \\	
		\pass & 1\spades & \pass & 1\nt \\
		\pass & 3\nt & \pass & \pass \\
		\pass 

    \end{tabular}
\end{table}

Energia niewykonanego wpustu przeszła na to rozdanie, w którym pan Janusz poczuł się wpuszczony na wiście i wyszedł w trelfa.

\end{document}        
