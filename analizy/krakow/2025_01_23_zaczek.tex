
\documentclass[12pt, a4paper]{article}
\usepackage{import}

\import{../../lib/}{bridge.sty}

\title{Żaczek 22/01/2025 - Analiza}
\author{using MacaroniBridge/TCResultsParser}

\begin{document}
\maketitle

    
    
\pagebreak
\section*{Rozdanie 1}
\handdiagramv{\vhand{KJ4}{AQ732}{Q42}{T7}}
{\vhand{2}{J986}{AK73}{K965}}
{\vhand{T98753}{T54}{T86}{8}}
{\vhand{AQ6}{K}{J95}{AQJ432}}
{}

\begin{table}[h!]
    \centering
    \begin{tabular}{cccc}
        \nvul{W} & \nvul{N} & \nvul{E} & \nvul{S}\\
		  -  & 1\hearts & \pass  & \pass  \\
		  \dbl & \pass & 1\nt & \pass \\
		  2\clubs & \pass & 3\nt

    \end{tabular}
\end{table}

[Bartek]
Licytacja na maksy jest prosta - byleby zagrać 3\nt, mamy spore nadwyżki siłowe.

Natomiast na impy dzieje się więcej. Siedząc na E, odzywka 3\nt jest słaba. Zauważmy, że z dużym prawdopodobieństwem w rozdaniu są podwójne fity - mamy singla pik, a partner 
statystycznie ma około czterech. Często dostaniemy też krótkosć kierową. Należałoby na pewno rozważyć grę w kolor młodszy. Już karta typu \hhand{Axxx}{x}{Qxx}{AQJxx} jest blisko szlemika treflowego, a partner może mieć dużo więcej. 3\nt może być także obkładane na wiscie przez piki.

W takich rozdaniach duży komfort daje ustalenie, że na kolorze młodszym 4\nt jest do gry.



\pagebreak
\section*{Rozdanie 2}
\handdiagramv{\vhand{JT5}{KT63}{T8652}{K}}
{\vhand{AKQ83}{Q95}{AKJ93}{}}
{\vhand{72}{72}{Q4}{JT97542}}
{\vhand{964}{AJ84}{7}{AQ863}}
{NS}
\vspace*{-0.5cm}
\begin{table}[h!]
    \centering
    \begin{tabular}{cccc}
        \nvul{W} & \vul{N} & \nvul{E} & \vul{S}\\
		  -  &  -  & 2\clubs & \pass \\
		  2\diams & \pass & 2\spades & \pass \\
		  3\spades & \pass & 4\clubs & \pass \\
		  4\diams & \pass & 5\clubs & \pass \\
		  5\hearts & \pass & 6\clubs & \pass \\
		  7\spades
    \end{tabular}
\end{table}

[Bartek]
Rozdanie przegrałem w licytacji. Ideą odzywki 6\clubs było zachęcenie do dołożenia z czwartym pikiem (to nie jest żadne ustalenie!), jednak ta sytuacja nie może mieć miejsca.
Wiedząc, że partnerka ma krótkosć karo, z czterokrotnym fitem raczej dałaby splintera. Bez dłuższych atutów szlem nie wygląda sensownie.

Lepszym pomysłem byłaby odzywka 6\hearts. Sekwencja nie umożliwiła spytania o króle, a 5\hearts pokazało tylko Asa (nie Króla, gdyż zostawia decyzję przed szlemem). Jako że partnerka nie ma \xhearts Q, a dubel jest niestatystyczny, dołoży tylko z królem. Nie ma co grać naciąganych szlemów - statystycznie potrzebne jest 70\%, a na maksy na takiej sali koło 85\% szans.

Jesli chodzi o otwarcie - w każdych innych założeniach otworzę 1\spades. Po acolu mamy system umożliwiający pokazanie takiej ręki.

\textbf{Wypuszczone przez bartka: 60\%}


\pagebreak
\section*{Rozdanie 3}
\handdiagramv{\vhand{Q8763}{Q83}{JT}{QT2}}
{\vhand{AK}{JT6}{93}{AJ8643}}
{\vhand{J92}{K7}{AKQ76542}{}}
{\vhand{T54}{A9542}{8}{K975}}
{EW}

\begin{table}[h!]
    \centering
    \begin{tabular}{cccc}
        \vul{W} & \nvul{N} & \vul{E} & \nvul{S}\\
		  -  &  -  &  -  & 1\diams \\
		  1\hearts & 1\spades & 2\diams\alrt & \dbl \\
		  2\hearts\alrt & 2\spades & 3\clubs & 4\spades \\
		  \pass & \pass & \dbl
    \end{tabular}
\end{table}

Zacznę od tego, że wejcie 1\hearts to moim zdaniem odjazd. Wist kierowy wcale nie musi być najlepszy, a przepych z taką kartą skończy się raczej na -200.

Rozgrywkę zrolowałem, gdyż byłem przekonany, że kontra S na sztuczne 2\diams była wydłużeniem kar, a N ma 6 pików. Kontra, jak się później okazało, była fit.
Zawistowałem w \xclubs A licząc na przebitkę u partnerki. Przebitka w stole (\xclubs 9 od partnerki) i zagrany pik do damy. Widząc potencjalne wyrzucenie kiera ze stołu na \xclubs K zagrałem \xhearts J do K i A.
Teraz nastąpił odwrót w karo do waleta i zagrany kolejny pik.

Teraz wypuciłem kontrakt chcąc ciągnąć kiera. Wyobraziłem sobie kartę partnerki jako \hhand{xx}{AQxxxx}{x}{Qxxx}, bo mając \xclubs K weszłaby 2\hearts = 9-12. Nie zauważyłem jednak, że karo musi być singlowe! 
Przecież z T8 i J8 partnerka wyszłaby honorem! A po co grać singlowe karo, jak wtedy rozgrywająca ma 2 dojcia do stołu? Tylko pod przebitkę! Partnerka musi mieć trzeciego atuta.

\textbf{Wypuszczone przez bartka: 93\%}


\pagebreak
\section*{Rozdania 4-6}
Przeciwnicy licytują jakies odjechane końcówki bez bilansu donejtując nam 3x 90\% za darmo.

\section*{Rozdanie 7}
\handdiagramv{\vhand{AQ86}{Q972}{64}{JT8}}
{\vhand{JT7}{8}{KQ985}{AK73}}
{\vhand{94}{KJT543}{JT7}{Q2}}
{\vhand{K532}{A6}{A32}{9654}}
{NSEW}

\begin{table}[h!]
    \centering
    \begin{tabular}{cccc}
        \vul{W} & \vul{N} & \vul{E} & \vul{S}\\
		  -  &  -  &  -  & 2\diams \\
		4\diams & \dbl & 4\hearts 
    \end{tabular}
\end{table}

Trochę mnie poniosło z blokiem z tak silną kartą. Zachęcony faktem, że otwieramy multi w tej pozycji w sile ok 3-8PC, przeciwnik ma często bilans na końcówkę.
Jednak na maksy chodzi o to, żeby pokonać salę. A sala albo nie otworzy multi, albo da tylko 3\hearts, bo karta do standardowego przedziału otwarcia jest za silna na 4\diams.

\textbf{Wypuszczone: 50\%}


\pagebreak
\section*{Rozdanie 8}
\handdiagramv{\vhand{A7}{AK764}{Q864}{KT}}
{\vhand{98432}{J95}{K3}{985}}
{\vhand{T65}{QT2}{AJT5}{AQJ}}
{\vhand{KQJ}{83}{972}{76432}}
{}

\begin{table}[h!]
    \centering
    \begin{tabular}{cccc}
        \nvul{W} & \nvul{N} & \nvul{E} & \nvul{S}\\
	  \pass & 1\hearts & \pass & 2\clubs \\
	  \pass & 2\diams & \pass & 2\hearts \\
	  \pass & 2\nt & \pass & 3\diams \\
	  \pass & 3\hearts & \pass & 3\spades\alrt \\
	  \pass & 3\nt\alrt & \pass & 4\clubs \\
	  \pass & 4\nt & \pass & 5\spades \\
	  \pass & 6\hearts

    \end{tabular}
\end{table}

[Bartek]
Bardzo nieoczywisty w licytacji szlemik. Kluczowe było tutaj wyobrażenie sobie karty partnerki. Po odzywce 3\diams istnieje bardzo dużo rąk będących blisko szlemika.
Problemem jest lewa pikowa, którą przeciwnik wyrobi na wiscie, a potem sciągnie po dojsciu np \xhearts Q. Natomiast jeli partnerka ma \xclubs AQ, możemy pika wyrzucić.

Po 4\clubs wiemy juz o asie. Przykładową ręką może być \hhand{xxx}{xxx}{AKJx}{Axx}, ale wtedy raczej dostalibysmy zjazd w 4\hearts zamiast niepoważnego 3\spades. Karta musi być o damę silniejsza.
Jesli jest to dama trefl - szlemik górny. Jeli dama pik - może nie zawistują. Jesli dama kier - przepuszczenie wistu da jakies szanse przymusowe w końcówce.

Możemy też mieć mniej w karach, a więcej w pozostałych kolorach - to działa na naszą korzysć.


\pagebreak
\section*{Rozdania 9-11}
W 9 jakie 1\nt, trochę źle zagrałem ale to nie jest nic istotnego
10-11 = dwa donejty po 90\% za nic.


\section*{Rozdanie 12}
\handdiagramv{\vhand{K}{KQ}{JT986}{KT742}}
{\vhand{QJ43}{AJ875}{K}{A83}}
{\vhand{AT}{T9642}{A54}{QJ9}}
{\vhand{987652}{3}{Q732}{65}}
{NS}

\begin{table}[h!]
    \centering
    \begin{tabular}{cccc}
        \nvul{W} & \vul{N} & \nvul{E} & \vul{S}\\
	  2\diams & \pass & 4\diams & \pass \\
	  4\spades 
    \end{tabular}
\end{table}

Otwarcie multi jak najbardziej poprawne. Trzeba maksymalnie utrudnić przeciwnikowi wejscie kierami i znalezienie końcówki.
Dałem 4\diams, bo tak na prawdę przy karcie typu \hhand{Axxxxx}{x}{xxx}{xxx} końcówka ma grę. Może na maksy była to jednak przesada, bo sala na pewno końcówki nie zagra, a 3\spades ma szansę zbiec.


\pagebreak
\section*{Rozdanie 13}
\handdiagramv{\vhand{Q9842}{A}{7642}{Q95}}
{\vhand{A653}{Q642}{3}{AK63}}
{\vhand{KJ}{J98}{KQJT95}{84}}
{\vhand{T7}{KT753}{A8}{JT72}}
{NSEW}

\begin{table}[h!]
    \centering
    \begin{tabular}{cccc}
        \vul{W} & \vul{N} & \vul{E} & \vul{S}\\
		  -  & \pass & 1\clubs & 3\diams \\
		  \pass & 3\hearts & \pass & \pass \\
		  \pass 

    \end{tabular}
\end{table}

Wynik: bez szesciu, 30\%. Niestety końcówka kierowa nie była grana przez salę, choć jest dosc oczywista, jesli nie pójdzie wejscie 3\diams lecz 2.

Odzywka 3\hearts po pasie nie była blefem - chciałem wskazać wist, by skomunikować się karem i wziąć przebitkę. Jest to myslenie na poziomie obory, bo przecież przeciwnik zagra w kiery. Albo i nie?


\pagebreak
\section*{Rozdanie 15}
\handdiagramv{\vhand{}{A874}{AJT942}{AJ7}}
{\vhand{KQ65}{K95}{K87}{KQ5}}
{\vhand{AJ9}{Q}{Q653}{T9643}}
{\vhand{T87432}{JT632}{}{82}}
{NS}

\begin{table}[h!]
    \centering
    \begin{tabular}{cccc}
        \nvul{W} & \vul{N} & \nvul{E} & \vul{S}\\
		  -  &  -  &  -  & \pass \\
		  \pass & 1\diams & 1\nt & 2\diams \\
		  2\spades & 4\diams & 4\spades & 5\diams \\
		  \pass & \pass & \dbl
    \end{tabular}
\end{table}

Bardzo nie podoba mi się 2\diams. Mamy czterokrotny fit, dużo punktów i przeciwnik ma fit kierowy. Koniecznie należy dać 3\diams.
5\diams torchę życzeniowe, mamy 1,5 lewy pikowej.

Ciekawą sprawą jest, co należy dać z W w drugim kółku. 2\spades jest fajną opcją, bo powoduje, że przeciwnikom ciężej będzie znaleźć bilans do końcówki, która z dużym prawdopodobieństwem im wychodzi.

\pagebreak
\section*{Rozdanie 16}
\handdiagramv{\vhand{AQ9}{T963}{T7}{AT87}}
{\vhand{K82}{AJ85}{A92}{K92}}
{\vhand{J6543}{K72}{KQ85}{Q}}
{\vhand{T7}{Q4}{J643}{J6543}}
{EW}

\begin{table}[h!]
    \centering
    \begin{tabular}{cccc}
        \vul{W} & \nvul{N} & \vul{E} & \nvul{S}\\
		\pass & \pass & 1\nt\\
    \end{tabular}
\end{table}

Wist pikowy należy w ręce N pobić damą! Teraz rozgrywający musi zgadywać, czy zabić królem, zostawiając komunikację do długiego koloru, czy przepuscić oddając lewę, jesli as jest u S.

\pagebreak
\section*{Rozdanie 17}
\handdiagramv{\vhand{KQ75}{754}{AQJ}{J52}}
{\vhand{JT42}{J93}{K542}{T9}}
{\vhand{A9}{Q86}{T9863}{874}}
{\vhand{863}{AKT2}{7}{AKQ63}}
{}

\begin{table}[h!]
    \centering
    \begin{tabular}{cccc}
        \nvul{W} & \nvul{N} & \nvul{E} & \nvul{S}\\
		  -  & & & \\

    \end{tabular}
\end{table}
nuda

\pagebreak
\section*{Rozdanie 18}
\handdiagramv{\vhand{}{K653}{AT7432}{983}}
{\vhand{K42}{JT4}{KJ6}{AKT6}}
{\vhand{AJ6}{AQ87}{95}{QJ74}}
{\vhand{QT98753}{92}{Q8}{52}}
{NS}

\begin{table}[h!]
    \centering
    \begin{tabular}{cccc}
        \nvul{W} & \vul{N} & \nvul{E} & \vul{S}\\
		  -  &  -  & & \\

    \end{tabular}
\end{table}
nuda

\pagebreak
\section*{Rozdanie 19}
\handdiagramv{\vhand{AK65}{87}{852}{9732}}
{\vhand{J974}{AQJ92}{A9}{64}}
{\vhand{T3}{K64}{KQ74}{QT85}}
{\vhand{Q82}{T53}{JT63}{AKJ}}
{EW}

\begin{table}[h!]
    \centering
    \begin{tabular}{cccc}
        \vul{W} & \nvul{N} & \vul{E} & \nvul{S}\\
		  -  &  -  &  -  & \pass \\
		  \pass & 1\spades & 2\hearts & \dbl \\
		  \rdbl & 3\clubs & \pass & \pass \\
		  3\hearts

    \end{tabular}
\end{table}

Przeciwnik przegrał sobie górny kontrakt


\pagebreak
\section*{Rozdanie 20}
\handdiagramv{\vhand{Q2}{QJ}{KQJ74}{Q853}}
{\vhand{AJT84}{A6543}{}{A42}}
{\vhand{953}{K72}{T852}{JT9}}
{\vhand{K76}{T98}{A963}{K76}}
{NSEW}

\begin{table}[h!]
    \centering
    \begin{tabular}{cccc}
        \vul{W} & \vul{N} & \vul{E} & \vul{S}\\
		\\

    \end{tabular}
\end{table}

Przeciwnik nie zagrał sobie górnej końcówki


\pagebreak
\section*{Rozdanie 21}
\handdiagramv{\vhand{T54}{K5}{JT75}{KJ98}}
{\vhand{J632}{AT932}{8}{542}}
{\vhand{K97}{86}{AQ63}{QT73}}
{\vhand{AQ8}{QJ74}{K942}{A6}}
{NS}

\begin{table}[h!]
    \centering
    \begin{tabular}{cccc}
        \nvul{W} & \vul{N} & \nvul{E} & \vul{S}\\
		  -  & \pass  & \pass & 1\clubs \\
		  1\nt & \pass & 2\clubs & \pass \\
		  2\hearts & \dbl & \pass & 2\nt \\
		  \pass & 3\clubs & 3\hearts 

    \end{tabular}
\end{table}



\pagebreak
\section*{Rozdanie 22}
\handdiagramv{\vhand{KT}{A643}{KQ52}{AQ5}}
{\vhand{A832}{KJ952}{}{KJ42}}
{\vhand{QJ965}{Q8}{JT73}{86}}
{\vhand{74}{T7}{A9864}{T973}}
{EW}

\begin{table}[h!]
    \centering
    \begin{tabular}{cccc}
        \vul{W} & \nvul{N} & \vul{E} & \nvul{S}\\
		  -  &  -  & & \\

    \end{tabular}
\end{table}

\pagebreak
\section*{Rozdanie 23}
\handdiagramv{\vhand{}{AQT87}{KT65}{J962}}
{\vhand{A9876}{J64}{94}{KT5}}
{\vhand{KQ53}{953}{J872}{A4}}
{\vhand{JT42}{K2}{AQ3}{Q873}}
{NSEW}

\begin{table}[h!]
    \centering
    \begin{tabular}{cccc}
        \vul{W} & \vul{N} & \vul{E} & \vul{S}\\
		  -  &  -  &  -  & \pass \\
		  1\clubs & 1\hearts & 1\spades & 2\clubs\alrt \\
		  2\spades & 3\hearts & 3\spades 

    \end{tabular}
\end{table}

Drury z takim gównem w ewidentnie równym punktowo rozdaniu to swir.
Mogłem dać 2\nt zamiast 3\hearts, ale wtedy pewnie przeciwnik by się nie przepchał.


\pagebreak
\section*{Rozdanie 24}
\handdiagramv{\vhand{J872}{QT7}{876}{T64}}
{\vhand{K93}{J965}{AJ}{J872}}
{\vhand{A65}{AK832}{952}{Q5}}
{\vhand{QT4}{4}{KQT43}{AK93}}
{}

\begin{table}[h!]
    \centering
    \begin{tabular}{cccc}
        \nvul{W} & \nvul{N} & \nvul{E} & \nvul{S}\\
		\\

    \end{tabular}
\end{table}

\pagebreak
\section*{Rozdanie 25}
\handdiagramv{\vhand{T862}{AT53}{AKQ7}{7}}
{\vhand{KQ9753}{KJ94}{4}{A3}}
{\vhand{}{Q7}{J98632}{T9654}}
{\vhand{AJ4}{862}{T5}{KQJ82}}
{EW}

\begin{table}[h!]
    \centering
    \begin{tabular}{cccc}
        \vul{W} & \nvul{N} & \vul{E} & \nvul{S}\\
		  -  & & & \\

    \end{tabular}
\end{table}

\pagebreak
\section*{Rozdanie 26}
\handdiagramv{\vhand{4}{QT43}{AT4}{KQT72}}
{\vhand{AJ53}{A7}{QJ963}{J6}}
{\vhand{92}{KJ96}{K5}{A9543}}
{\vhand{KQT876}{852}{872}{8}}
{NSEW}

\begin{table}[h!]
    \centering
    \begin{tabular}{cccc}
        \vul{W} & \vul{N} & \vul{E} & \vul{S}\\
		  -  &  -  & & \\

    \end{tabular}
\end{table}

\pagebreak
\section*{Rozdanie 27}
\handdiagramv{\vhand{T54}{762}{Q72}{K642}}
{\vhand{962}{AJ3}{954}{T987}}
{\vhand{KQJ3}{KT95}{KJ86}{3}}
{\vhand{A87}{Q84}{AT3}{AQJ5}}
{}

\begin{table}[h!]
    \centering
    \begin{tabular}{cccc}
        \nvul{W} & \nvul{N} & \nvul{E} & \nvul{S}\\
		  -  &  -  &  -  & \\

    \end{tabular}
\end{table}

\end{document}        
