\documentclass[12pt, a4paper]{article}
\usepackage{../lib/bridgetex2}
\usepackage{polyglossia}
\usepackage{enumitem}
\setmainlanguage{polish}


\title{\vspace{-2cm}MiniStrefa}
\author{}
\date{}

\begin{document}
\section{Założenia}

Ten system został stworzony, by sprostać oczekiwaniom najbardziej wymagających brydżystów, a mianowicie:
\begin{itemize}
    \item Maksymalna prostota i minimalna ilość rzeczy do zapamiętania
    \item Możliwość rozlicytowania > 98\% rąk (99.9\% jeśli umie się trochę oszukiwać)
    \item Łatwa i naturalna licytacja nawet po kilku piwach (czego nie można powiedzieć o WJ)
    \item Nowoczesne narzędzia i luźny styl
\end{itemize}

W systemie będzie dużo dziur i nieopisanych odzywek - jest to intencjonalne, by dać Wam
możliwość dodawania własnych ustaleń według Waszego stylu

\pagebreak
\section{Otwarcie 1\major}
\subsection*{Z czym otwierać?}
\begin{formal}
    Jeśli mamy \textbf{starszą piątkę} i 11 punktów, otwieramy.
\end{formal}

\subsection*{Podnoszenie otwarcia z fitem}
\begin{itemize}
    \item 1\major\ -- 2\major\ = 7-10 z 3-kartowym fitem
    \item 1\major\ -- 3\major\ = 7-11 z 4-kartowym fitem i odrobiną układu (inwit)
    \item 1\major\ -- 2\nt\ = 11-12 z 3-kartowym fitem (inwit)
    \item 1\hearts\ -- 3\spades, 1\major\ - 4\clubs, 1\spades\ -- 4\diams\ = splinter, 4-kartowy fit i 13+ PC
    \item 1\major\ -- 3\nt\ = \textbf{najwyższy splinter}, żeby pokazać go niżej
\end{itemize}

\subsection*{Podnoszenie otwarcia bez fitu}
\begin{itemize}
    \item 1\nt\ -- \textbf{nieforsujące.} Każda ręka z przedziału 4-11 bez fitu.
    \item 2\clubs\ = \gf\ z treflami, z fitem bez koloru, lub na równym
    \item 2\diams, 2\hearts, 2\spades\ = 13+ \gf\ NAT
\end{itemize}

\subsection*{Odzywki niewykorzystywane}
\begin{itemize}
    \item 1\nt\ - można grać, że może być też bardzo słabe z fitem (3-5 PC)
    \item 3\clubs, 3\diams\ - można grać jako:
    \begin{itemize}
        \item naturalne inwity 6+
        \item 5+Kolor, 4+Fit (inwitujące lub bardzo silne)
        \item \textbf{Bergeny}
    \end{itemize}
\end{itemize}



\pagebreak
\section{Otwarcie 1\diams}
\subsection*{Z czym otwieramy?}
\begin{formal}
    Jeśli mamy 10-11 punktów, otwieramy tylko, jeśli punkty są ładne i mamy \textbf{łatwy rebid}
\end{formal}

\subsection*{Odpowiedzi}
\begin{itemize}
    \item 1\hearts, 1\spades\ = 4+, \fonce.
    (np z \gf\ na 5\clubs\ i 4\hearts, licytujemy 2\clubs!)
    \item 1\nt\ = 7-10, brak 4+\major, brak 4+\diams. Zatem na taką odpowiedź mamy 4+\clubs!
    \item 2\clubs\ = \gf\ z treflami lub na równym
    \item 2\hearts, 2\spades\ = 6+, \gf
    \item 2\diams\ = 4+\diams\ i co najmniej inwit (10+), forsuje do 3\diams, a ten kto przekroczy 3\diams\
    ustawia \gf
    \item 3\diams\ = 7-9, 4+\diams
    \item 2\nt\ = \inv\ bez starszych czwórek
    \item 3\hearts, 3\spades, 4\clubs\ = potężny splinter, bo wyklucza 3\nt! (13+) 
\end{itemize}


\pagebreak
\section{Otwarcie 1\clubs}
\subsection*{Z czym otwierać?}
\begin{formal}
    Jeśli mamy 12 punktów, \textbf{otwieramy.}
\end{formal}

\subsection*{Odpowiedzi}
\begin{itemize}
    \item \pass\ = jakaś ręka typu \hhand{xx}{xxx}{xxx}{Qxxxx}, z którą może być problem z licytacją po negacie
    \item 1\diams\ = dowolne 0-6 \st{lub 7-11 z 5+\diams} (to tylko robi problemy, walić kara! Z karami 1\nt)
    \item 1\hearts, 1\spades\ = 4+, \fonce
    \item 1\nt\ = 7-10 równe, bez starszych czwórek
    \item 2\nt\ = 11-12, bez starszych czwórek
    \item 3\nt\ = do gry
    \item 2\clubs\ = \gf\ z treflami lub na równym
    \item 2\diams\ = 5+, \gf
    \item 2\hearts, 2\spades\ = 6+, \gf
\end{itemize}

\subsection*{Licytacja po 1\clubs\ --- 1\diams}
\begin{itemize}
    \item \pass\ = wiem co robię, a przynajmniej mi się tak wydaje
    \item 1\hearts\ = 12-18, może być z \textbf{trójki}, stay low!
    \item 1\spades\ = 12-18, 4+\spades\ (w skrajnym przypadku z trójki, w składzie 3244)
    \item 1\nt\ = 18-20 na równym (planowaliśmy rebid 2\nt) - licytacja jak po otwarciu 1\nt
    \item 2\clubs\ = 5+\clubs
    \item 2\diams, 2\hearts, 2\spades\ = 19+, 5+\clubs\ i czwórka (silny rewers)
\end{itemize}


\pagebreak
\section{Licytacja po nieforsującym 1\ntx}
\subsection*{Z czym dawać 1\ntx?}
\begin{formal}
    Z każdą kartą, która ma jakiś sensowny kontrakt, np:
    \begin{itemize}
        \item \hhand{x}{xx}{KTxxxx}{Jxxx}
        \item \hhand{x}{Qxxxxxx}{Jxx}{xx}
    \end{itemize}
\end{formal}
\subsection*{Rebid otwierającego po 1\spades\ --- 1\ntx}
\begin{itemize}
    \item \pass\ = 12-14, 5332
    \item 2\clubs, 2\diams, 2\hearts = 12-17, 4+\clubs\diams\hearts
    \item 2\spades\ = 12-15, 6+\spades
    \item 2\nt\ = \gf\ na w miarę równym składzie 
    \item 3\clubs, 3\diams, 3\hearts\ = \gf\ (19+) na 5-4+
    \item 3\spades\ = 16-18, 6+\spades
\end{itemize}

\subsection*{1\spades\ -- 1\ntx\ \\ 2\clubs\ -- ?}
\begin{itemize}
    \item 2\diams, 2\hearts\ = 5+\diams\hearts, konstrutywne! (7-8+)
    \item 2\spades\ = 5-9 z dublem \spades
    \item 2\nt\ = 10-11
    \item 3\clubs\ = 9-11, 4+\clubs
\end{itemize}

\subsection*{Niemożliwe 2\spades}
\begin{formal}
    W sekwencji 1\hearts\ - 1\nt\ --- 2\minor\ - 2\spades\ ostatnia odzywka nie może pokazywać pików,
    więc jest niemożliwa. Wskazuje kartę, której wartość urosła do \gf\ po rebidzie partnera,
    np dzięki dużemu fitowi \minor.  
\end{formal}


\pagebreak
\section{Licytacja po rebidzie 1\ntx}

\section{Licytacja po rebidzie 2\ntx}

\section{Rewersy}

\section{Otwarcie 1\ntx}

\section{Otwarcie 2\ntx}

\section{Bloki i licytacja po blokach}

\section{Otwarcia trzecio- i czwartoręczne}

\section{Wejścia kolorem i licytacja po nich}

\section{Wejście przeciwnika na otwarcie w kolor}

\section{Wejście przeciwnika na nasze 1\ntx}



\end{document}