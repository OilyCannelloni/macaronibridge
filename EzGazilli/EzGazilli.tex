\documentclass[12pt, a4paper]{article}
\usepackage{../lib/bridgetex2}
\usepackage{polyglossia}
\usepackage{graphicx}
\usepackage{enumitem}
\usepackage{parskip}
\setmainlanguage{polish}



\title{Gazilli, czyli system łatwiejszy niż jego brak}
\date{\vspace*{-2cm}}
\begin{document}

\maketitle

\begin{figure}[h!]
    \centering
    \includegraphics[width=0.5\linewidth]{gazilli-1.jpg}
\end{figure}
\section*{A po co to komu}
Licytacja biegnie do nas: 1\spades\ -- 1\nt\ --- 2\diams. Trzymamy:
\begin{center}
    \hhand{Q4}{K9732}{QJ75}{72}
\end{center}
Jakie mamy opcje?
\begin{itemize}
    \item Decydujemy się na \pass. Rozgrywający posiadał akurat 17 punktów i 3\nt\ było z góry. Och jak przykro!
    \item Wybieramy 2\hearts. Rozgrywający ma 11 punktów w składzie 5242 i przegrywamy 2\nt.
    \item Nieco agresywnie próbujemy 3\diams. Rozgrywający dokłada końcówkę ze swoim ładnym piętnastakiem,
    ale to za mało.
    \item Wywalamy karty przez okno
\end{itemize}

\pagebreak
\section*{Koncepcja}
Czego nam brakowało? Przede wszystkim poznania \textbf{bilansu otwierającego} na poziomie 2. A teraz jak temu zaradzić:
W sekwencjach 1\hearts\ -- 1\spades, 1\hearts\ -- 1\nt\ oraz 1\spades\ -- 1\nt, mamy do dyspozycji następujące odzywki: 
\begin{itemize}
    \item \pass\ 
    \subitem 12-14 w składzie 5332 - tak, czasami zagramy to na 26 PC. Ale i wtedy końcóka może być słaba.
    A zysk z grania 1\nt\ zamiast 2\nt\ kiedy nie mamy bilansu jest ogromny.

    \item 2\clubs\ \fonce
    \subitem Dowolna \emph{mięsna} ręka \textbf{16+} (silne Gazilli) lub
    \subitem 4+\clubs, 11-15

    \item 2\diams, 2\hearts\ = 11-15, 4+
\end{itemize}

\subsection*{Odpowiedzi na 2\clubs}
Po rebidzie 2\diams, 2\hearts\ licytujemy naturalnie - znamy bilans!
Tutaj zamieszczam ogólny schemat, odzywki różnią się minimalnie w różnych sekwencjach.
\begin{itemize}
    \item 2\diams\ = Dowolna ręka \textbf{8+}. \emph{"Jeśli masz silne Gazilli, chcę ustalić GF"}
    \item 2\hearts, 2\spades\ = 5-7 PC
    \item 2\nt\ = oba młode z singlem lub renonsem w kolorze otwarcia, 5-7 PC
    \item 3\clubs, 3\diams\ = 6+, 5-7 PC
\end{itemize}

\subsection*{Powrót na kolor otwarcia}
Jeśli otwierający zalicytował 2\clubs, a następnie po 2\diams\ wrócił na kolor otwarcia,
oznacza to że ma \textbf{słabą rękę} z naturalnymi treflami! Tu nie ma GF. Np:
\begin{center}
\webidding{1\spades\ & 1\nt \\ 2\clubs & 2\diams \\ 2\spades}
11-15 PC 5\spades4\clubs.
\\[1em] \webidding{1\spades\ & 1\nt \\ 2\clubs & 2\diams \\ 2\hearts}
16+ PC 5\spades4\hearts.
\end{center}


\subsection*{Inne intuicyjne schematy}
\begin{itemize}
    \item Rebid odpowiadającego 2\nt\ to słaba ręka na młodych
    \item Ręce układowe sprzedaje się bezpośrednio, a mięsne przez Gazilli. Np:
    \subitem 1\spades\ -- 1\nt\ --- 3\clubs\ = ściśle 5/5 GF
    \subitem 1\spades\ -- 1\nt\ --- 2\clubs\ -- 2\spades\ --- 3\clubs\ = 5/4 19+PC
    \item Jeśli odpowiadający pokaże 5-7, licytacja jest naturalna. Raczej od razu widać co trzeba grać.
    Główny system jest po 2\clubs\ -- 2\diams.
\end{itemize}

\subsection*{Co jeszcze zyskujemy?}
Otwieramy 1\hearts, partner odpowiada 1\nt.
\begin{center}
    \hhand{K5}{AKQJ86}{AJ3}{J6}
\end{center}
Normalnie pokazujemy to przez 2\nt, lecz mało partnerów pamięta o obowiązkowym 3\clubs, żeby sprawdzić 
właśnie ten przypadek... A nawet jeśli, to i tak jest mało miejsca!




\pagebreak
\section*{1\hearts\ --- 1\ntx}

\subsection*{1\hearts\ --- 1\ntx \\ 2\clubs\ --- ?}
\begin{itemize}
    \item 2\diams\ = dowolna ręka \textbf{8+}
    \item 2\hearts\ = 2-3 \hearts, 4-7
    \item 2\spades\ = 5\clubs5\diams \vimp
    \item 2\nt\ = Krótsza niż 5-5 karta na młodych z max. singlem \hearts \imp
    \item 3\clubs\ = \nat 6+ 5-7
    \item 3\diams\ = \nat 6+ 5-7
\end{itemize}

\subsection*{1\hearts\ --- 1\ntx \\ 2\clubs\ --- 2\diams \\ ?}
\begin{itemize}
    \item 2\hearts\ = 5\hearts, 4+\clubs, 11-15
    \item 2\spades\ = 4\spades, 16+ (Jak zwykły rewers - bezpośredni rewers to co innego!)
    \item 2\nt\ = 18+, 5332 (Bezpośrednie 2\nt\ to co innego!)
    \item 3\clubs\ = 5\hearts, 4\clubs\ 16+
    \item 3\diams\ = 5\hearts, 4\diams\ 16+
    \item 3\hearts\ = 6+\hearts\ 16+
\end{itemize}



\pagebreak
\section*{1\spades\ --- 1\ntx}
\subsection*{1\spades\ --- 1\ntx \\ 2\clubs\ --- ?}
\begin{itemize}
    \item 2\diams\ = dowolna ręka \textbf{8+}
    \item 2\hearts\ = 5+\hearts, 5-7 
    \item 2\spades\ = 2-3\spades, 4-7
    \item 2\nt\ = Młode z max. singlem \spades \imp
    \item 3\clubs\ = NAT 6+ 5-7
    \item 3\diams\ = NAT 6+ 5-7
\end{itemize}

\subsection*{1\spades\ --- 1\ntx \\ 2\clubs\ --- 2\diams \\ ?}
\begin{itemize}
    \item 2\hearts\ = 5\spades, 4+\hearts, 16+
    \item 2\spades\ = 5\spades, 4\clubs, 11-15
    \item 2\nt\ = 5332, 18+ (Bezpośrednie 2\nt\ to co innego!)
    \item 3\clubs\ = 5\spades, 4\clubs\ 16+
    \item 3\diams\ = 5\spades, 4\diams\ 16+
    \item 3\hearts\ = 5\spades, 4\hearts\ 16+
    \item 3\spades\ = 6+\spades, 16+
\end{itemize}


\pagebreak
\section*{1\hearts\ --- 1\spades}
\subsection*{1\hearts\ --- 1\spades \\ 2\clubs\ --- ?}
\begin{itemize}
    \item 2\diams\ = dowolna ręka \textbf{8+}
    \item 2\hearts\ = 2\hearts, 5-7 
    \item 2\spades\ = 5+ dobrych \spades, 5-7
    \item 2\nt\ = 4\spades, singiel \hearts\ i jakieś młode
    \item 3\clubs\ = NAT 6+ 5-7
    \item 3\diams\ = NAT 6+ 5-7
\end{itemize}

\subsection*{1\hearts\ --- 1\spades \\ 2\clubs\ --- 2\diams \\ ?}
\begin{itemize}
    \item 2\hearts\ = 11-15, 5\hearts, 4+\clubs
    \item 2\spades\ = 16+, \textbf{dokładnie} 3\spades\ \imp
    \item 2\nt\ = 18+, 5332 (Bezpośrednie 2\nt\ to co innego!)
    \item 3\clubs\ = 5\hearts, 4\clubs, 16+
    \item 3\diams\ = 5\hearts, 4\diams, 16+
    \item 3\hearts\ = 6+\hearts, 16+
    \item 3\spades\ = 4\spades\ \gf
\end{itemize}


\pagebreak
\section*{Inne bezsensowne ustalenia}
\subsection*{1\hearts\ --- 1\ntx}
\begin{itemize}
    \item 2\spades\ = 6\hearts, 5\spades, \gf \imp
    \item 2\nt\ = 6\hearts, 5\clubs\ lub 5\diams, \gf
    \subitem 3\clubs\ pytanie, 3\diams\ = \clubs, 3\hearts\ = \diams
    \item 3\clubs\ = 5\hearts, 5+\clubs
    \item 3\diams\ = 5\hearts, 5+\diams
\end{itemize}

\subsection*{GF z fitem przez 1\spades}
\begin{center}
    \webidding{1\hearts & 1\spades \\ 2\clubs & 3\hearts\ }
    \gf\ z fitem kier
\end{center}

\subsection*{GF samoustalenie}
\begin{center}
    \webidding{1\hearts & 1\spades \\ 3\hearts} 
    \webidding{1\hearts & 1\nt \\ 3\hearts} 
    \webidding{1\spades & 1\nt \\ 3\spades} 
    \gf\ samoustalenie 
\end{center}

\end{document}